\documentclass[12pt]{article}
%\usepackage[left=3cm, right=2.5cm, top=2.5cm, bottom=2.5cm]{geometry}e}
\usepackage[utf8]{inputenc}
\usepackage[spanish,english]{babel}
\usepackage{apacite}
\usepackage[round]{natbib}
\usepackage{hyperref}
\usepackage{float}
\usepackage{svg}
\usepackage[margin = 1in, top=2cm]{geometry}% Margins
\setlength{\parindent}{2em}
\setlength{\parskip}{0.2em}
\usepackage{setspace} % Setting the spacing between lines
\usepackage{amsthm, amsmath, amsfonts, mathtools, amssymb, bm} % Math packages 
\usepackage{svg}
\usepackage{graphicx}
\usepackage{pgfplots}
\usepackage{epstopdf}
\usepackage{subfig} % Manipulation and reference of small or sub figures and tables
\usepackage{hyperref} % To create hyperlinks within the document
\spacing{1.15}
\usepackage{tikz}
\usepackage{appendix}
\usepackage{xcolor}
\usepackage{cancel}
\usepackage{enumerate}
\usepackage[shortlabels]{enumitem}
\usepackage{optidef}
\pgfplotsset{compat=1.17}
\usepackage[round]{natbib}
%\bibliographystyle{plainnat}
\bibliographystyle{apacite}


\newtheorem{defin}{Definition.}
\newtheorem{teo}{Theorem. }
\newtheorem{lema}{Lemma. }
\newtheorem{coro}{Corolary. }
\newtheorem{prop}{Proposition. }
\theoremstyle{definition}
\newtheorem{examp}{Example. }
\newtheorem{problem}{Problem}
\newtheorem{subproblem}{}[problem]
% \numberwithin{problem}{subsection} 

\newcommand{\card}{\operatorname{card}}
\newcommand{\qiq}{\qquad \implies \qquad}
\newcommand{\qiffq}{\qquad \iff \qquad}
\newcommand{\qaq}{\qquad \textbf{and} \qquad}
\newcommand{\qoq}{\qquad \textbf{or} \qquad}
\newcommand{\settf}{\text{ \emph{:} }}
\newcommand{\chbox}{\makebox[0pt][l]{$\square$}\raisebox{.15ex}{\hspace{.9em}}}
\newcommand{\cchbox}{\makebox[0pt][l]{$\square$}\raisebox{.15ex}{\hspace{0.1em}$\checkmark$}}

\title{Problem Set 1}
\author{Mitchell Valdés-Bobes}
\date{\today}

\begin{document}

\maketitle

\begin{problem}
Assume that firm $k=1,2, \ldots$ in a competitive industry has $\operatorname{cost} k^{2}+q+q^{2}$ of output level
q. (Here, $k^{2}$ is an escapable cost in the long run.) Then the long run industry saw tooth supply curve has positive output only for prices at least what level?
\end{problem}

\begin{proof}[Answer]
Supply of firm $k$ is 
$$MC_k = P \qiq P = 1+2q \qiq q = \frac{p-1}{2}$$
In the long run the fixed cost is escapable thus included in costs therefore firm $k$ has positive supply if:
$$AC_k \leq p \qiq \frac{p^2+4k^2-1}{2(p-1)}\leq p \qiq 4k^2-1\leq p^2-2p$$

The industry has positive output if at least one firm has positive output. We know that firms with higher costs are less likely to produce. The minimum price for which there is at least a firm willing to produce is the price for which $k=1$ is the marginal firm:

$$AC_1 = p \qiq 3 = p^2-2p \qiq \boxed{p=3}$$

\end{proof}

\begin{problem}
Assume a continuum of potential iPhone developers indexed by the quality of their idea. Each developer has a fixed cost of 1 can produce software code according to the function $q=2 \theta x,$ where $x$ is quantity of variable input used by the developer. A unit of software code sells for a price of $1,$ the cost of using a quantity $x$ of the input is $x^{2} .$ The "number" (i.e mass) of firms with ideas above $\theta>0$ is given by $M(\theta)=\theta^{-\beta},$ where $\beta>2$. Apple taxes developers' revenues at a percentage rate $0<\tau<1$.
\begin{subproblem}
    Derive the aggregate supply curve of developer code.
\end{subproblem}
\begin{proof}[Answer]
    Firm $\theta$ produces $q$ units at a cost of 
$$c(q(x)) = x^2 + 1 = \frac{q^2}{4\theta^2} + 1$$

Taking Apple's tax in to account the profits of firm $\theta$ are

$$\pi_\theta (q|\tau) = (1-\tau)q - \frac{q^2}{4\theta^2} - 1$$ 

The optimal production level is:

$$q(\theta|\tau) = 2(1-\tau)\theta^2$$

We know that in the long run firms only produce if they make non-negative profits, this is:

$$(1-\tau)q \geq  \frac{q^2}{4\theta^2} + 1 \qiq 2(1-\tau)^2\theta^2 \geq (1-\tau)^2\theta^2  + 1 \qiq \theta \geq \frac{1}{1-\tau}$$ 

Aggregate supply is given by:

\begin{align*}
2(1-\tau)\int_{\frac{1}{1-\tau}}^{\infty}{\theta^2d(1-\theta^{-\beta})} &= 2\beta(1-\tau)\int_{\frac{1}{1-\tau}}^{\infty}{\theta^{2-\beta-1}d\theta}\\
&= 2\beta(1-\tau) \left(\frac{\theta^{2-\beta}}{2-\beta} \Big|_{\frac{1}{1-\tau}}^\infty \right)\\
  Q(\tau,\beta)&=\frac{2\beta(1-\tau)^{\beta-1}}{\beta-2}
\end{align*}

\end{proof}

\begin{subproblem}
 When Apple raises its tax rate, what happens to the mass of developer firms, and the amount of code each produces?
\end{subproblem}

\begin{proof}[Answer]
If Apple raises it's tax rate from $\tau$ to $\tau'$ where $\tau'>\tau$ then:
\begin{enumerate}
    \item $\frac{1}{1-\tau} < \frac{1}{1-\tau'} \quad \implies$ the mass of developer firms shrink, i.e. there are firms that where producing before the tax hike that are not producing now.
    \item Form any firm in the market $q(\theta|\tau')=2(1-\tau')\theta^2 < 2(1-\tau)\theta^2 q(\theta|\tau') \quad \implies$ the amount of code produced is lower.
\end{enumerate}


\end{proof}

\begin{subproblem}
What is Apple's revenue maximizing tax?
\end{subproblem}

\begin{proof}[Answer]
Apple tax revenues are given by:
$$R(\tau) = \tau Q(\tau, \beta) = \tau\frac{2\beta(1-\tau)^{\beta-1}}{\beta-2} $$

We know that any increasing transformation of this function will be maximized at the same tax level, then it is easier to take logs and solve:

$$\max_{\tau}\quad \log{(\tau)} + (\beta-1)\log{(1-\tau)} \qiq \boxed{\tau = \frac{1}{\beta}}$$

\end{proof}

\begin{subproblem}
What happens if $\beta$ rises (while $\tau$ is fixed)? Interpret this in terms of firm heterogeneity.
\end{subproblem}
\begin{proof}[Answer]
Recall that the aggregate supply in is:
$$Q(\tau, \beta) = 2\beta(1-\tau)\int_{\frac{1}{1-\tau}}^{\infty}{\theta^{2-\beta-1}d\theta}$$

Suppose that $\beta$ increases to $\beta'$ while $\tau$ remains constant, then:

\begin{align*}
    Q(\tau, \beta) - Q(\tau, \beta') &= 2(1-\tau)\int_{\frac{1}{1-\tau}}^{\infty}{\theta(\beta\theta^{-\beta}-\beta'\theta^{-\beta'})d\theta} \\
    & \geq 2\beta'(1-\tau)\int_{\frac{1}{1-\tau}}^{\infty}{\frac{1}{1-\tau}(\theta^{-\beta}-\theta^{-\beta'})d\theta} \\
    &= 2\beta'\int_{\frac{1}{1-\tau}}^{\infty}{(\theta^{-\beta}-\theta^{-\beta'})d\theta}
\end{align*}

Since $\tau < 1$ then for all $\theta \geq \frac{1}{1-\tau}$ we have that  $\theta^{-\beta}>\theta^{-\beta'}$ therefore:

$$Q(\tau, \beta) > Q(\tau, \beta')$$

As $\beta$ increases firms become more homogeneous (i.e. the mass of "lower" type firms increases disproportionately).

\begin{tikzpicture}[/tikz/background rectangle/.style={fill={rgb,1:red,1.0;green,1.0;blue,1.0}, draw opacity={1.0}}, show background rectangle]
\begin{axis}[point meta max={nan}, point meta min={nan}, legend cell align={left}, title={Firm density function}, title style={at={{(0.5,1)}}, anchor={south}, font={{\fontsize{14 pt}{18.2 pt}\selectfont}}, color={rgb,1:red,0.0;green,0.0;blue,0.0}, draw opacity={1.0}, rotate={0.0}}, legend style={color={rgb,1:red,0.1333;green,0.1333;blue,0.3333}, draw opacity={0.1}, line width={1}, solid, fill={rgb,1:red,1.0;green,1.0;blue,1.0}, fill opacity={0.9}, text opacity={1.0}, font={{\fontsize{8 pt}{10.4 pt}\selectfont}}, text={rgb,1:red,0.0;green,0.0;blue,0.0}, at={(1.02, 1)}, anchor={north west}}, axis background/.style={fill={rgb,1:red,1.0;green,1.0;blue,1.0}, opacity={1.0}}, anchor={north west}, xshift={1.0mm}, yshift={-1.0mm}, width={145.4mm}, height={99.6mm}, scaled x ticks={false}, xlabel={$\theta$}, x tick style={draw={none}}, x tick label style={color={rgb,1:red,0.0;green,0.0;blue,0.0}, opacity={1.0}, rotate={0}}, xlabel style={at={(ticklabel cs:0.5)}, anchor=near ticklabel, font={{\fontsize{11 pt}{14.3 pt}\selectfont}}, color={rgb,1:red,0.0;green,0.0;blue,0.0}, draw opacity={1.0}, rotate={0.0}}, xmajorgrids={true}, xmin={0.97}, xmax={2.03}, xtick={{1.0,1.25,1.5,1.75,2.0}}, xticklabels={{$1.00$,$1.25$,$1.50$,$1.75$,$2.00$}}, xtick align={inside}, xticklabel style={font={{\fontsize{8 pt}{10.4 pt}\selectfont}}, color={rgb,1:red,0.0;green,0.0;blue,0.0}, draw opacity={1.0}, rotate={0.0}}, x grid style={color={rgb,1:red,0.1333;green,0.1333;blue,0.3333}, draw opacity={0.1}, line width={0.5}, solid}, extra x ticks={{1.05,1.1,1.15,1.2,1.3,1.35,1.4,1.45,1.55,1.6,1.65,1.7,1.8,1.85,1.9,1.95}}, extra x tick labels={}, extra x tick style={grid={major}, x grid style={color={rgb,1:red,0.1333;green,0.1333;blue,0.3333}, draw opacity={0.05}, line width={0.5}, solid}, major tick length={0}}, x axis line style={{draw opacity = 0}}, scaled y ticks={false}, ylabel={$m(\theta)$}, y tick style={draw={none}}, y tick label style={color={rgb,1:red,0.0;green,0.0;blue,0.0}, opacity={1.0}, rotate={0}}, ylabel style={at={(ticklabel cs:0.5)}, anchor=near ticklabel, font={{\fontsize{11 pt}{14.3 pt}\selectfont}}, color={rgb,1:red,0.0;green,0.0;blue,0.0}, draw opacity={1.0}, rotate={0.0}}, ymajorgrids={true}, ymin={-0.294970703125}, ymax={10.299853515625}, ytick={{0.0,2.5,5.0,7.5,10.0}}, yticklabels={{$0.0$,$2.5$,$5.0$,$7.5$,$10.0$}}, ytick align={inside}, yticklabel style={font={{\fontsize{8 pt}{10.4 pt}\selectfont}}, color={rgb,1:red,0.0;green,0.0;blue,0.0}, draw opacity={1.0}, rotate={0.0}}, y grid style={color={rgb,1:red,0.1333;green,0.1333;blue,0.3333}, draw opacity={0.1}, line width={0.5}, solid}, extra y ticks={{0.5,1.0,1.5,2.0,3.0,3.5,4.0,4.5,5.5,6.0,6.5,7.0,8.0,8.5,9.0,9.5}}, extra y tick labels={}, extra y tick style={grid={major}, y grid style={color={rgb,1:red,0.1333;green,0.1333;blue,0.3333}, draw opacity={0.05}, line width={0.5}, solid}, major tick length={0}}, y axis line style={{draw opacity = 0}}]
    \addplot[color={rgb,1:red,0.9333;green,0.4667;blue,0.2}, name path={69481a18-1737-4d43-8393-a880e1b3322b}, draw opacity={1.0}, line width={1.2}, solid]
        table[row sep={\\}]
        {
            \\
            1.0  2.5  \\
            1.001  2.4912696514648154  \\
            1.002  2.48257846218563  \\
            1.003  2.4739262176983026  \\
            1.004  2.465312704925418  \\
            1.005  2.4567377121659675  \\
            1.006  2.4482010290850846  \\
            1.007  2.4397024467038984  \\
            1.008  2.4312417573894414  \\
            1.009  2.422818754844672  \\
            1.01  2.4144332340985524  \\
            1.011  2.4060849914962392  \\
            1.012  2.3977738246893248  \\
            1.013  2.389499532626193  \\
            1.014  2.3812619155424244  \\
            1.015  2.37306077495131  \\
            1.016  2.3648959136344176  \\
            1.017  2.3567671356322646  \\
            1.018  2.34867424623504  \\
            1.019  2.340617051973431  \\
            1.02  2.332595360609499  \\
            1.021  2.3246089811276627  \\
            1.022  2.3166577237257187  \\
            1.023  2.308741399805976  \\
            1.024  2.3008598219664265  \\
            1.025  2.2930128039920255  \\
            1.026  2.285200160846007  \\
            1.027  2.277421708661305  \\
            1.028  2.269677264732019  \\
            1.029  2.26196664750497  \\
            1.03  2.254289676571305  \\
            1.031  2.2466461726582003  \\
            1.032  2.239035957620596  \\
            1.033  2.231458854433034  \\
            1.034  2.223914687181535  \\
            1.035  2.2164032810555643  \\
            1.036  2.2089244623400424  \\
            1.037  2.201478058407444  \\
            1.038  2.1940638977099365  \\
            1.039  2.1866818097716076  \\
            1.04  2.179331625180733  \\
            1.041  2.1720131755821313  \\
            1.042  2.1647262936695544  \\
            1.043  2.1574708131781675  \\
            1.044  2.150246568877064  \\
            1.045  2.143053396561866  \\
            1.046  2.135891133047361  \\
            1.047  2.128759616160221  \\
            1.048  2.1216586847317593  \\
            1.049  2.1145881785907665  \\
            1.05  2.1075479385563827  \\
            1.051  2.1005378064310514  \\
            1.052  2.0935576249935046  \\
            1.053  2.0866072379918306  \\
            1.054  2.079686490136571  \\
            1.055  2.0727952270938994  \\
            1.056  2.0659332954788314  \\
            1.057  2.05910054284851  \\
            1.058  2.0522968176955243  \\
            1.059  2.045521969441304  \\
            1.06  2.038775848429543  \\
            1.061  2.0320583059196986  \\
            1.062  2.0253691940805236  \\
            1.063  2.018708365983666  \\
            1.064  2.012075675597303  \\
            1.065  2.0054709777798445  \\
            1.066  1.9988941282736674  \\
            1.067  1.9923449836989198  \\
            1.068  1.985823401547353  \\
            1.069  1.9793292401762246  \\
            1.07  1.9728623588022287  \\
            1.071  1.9664226174954937  \\
            1.072  1.9600098771736099  \\
            1.073  1.9536239995957214  \\
            1.074  1.9472648473566467  \\
            1.075  1.9409322838810659  \\
            1.076  1.9346261734177315  \\
            1.077  1.928346381033748  \\
            1.078  1.9220927726088746  \\
            1.079  1.9158652148298916  \\
            1.08  1.9096635751849946  \\
            1.081  1.90348772195825  \\
            1.082  1.8973375242240738  \\
            1.083  1.8912128518417757  \\
            1.084  1.8851135754501245  \\
            1.085  1.8790395664619761  \\
            1.086  1.8729906970589245  \\
            1.087  1.8669668401860124  \\
            1.088  1.860967869546466  \\
            1.089  1.8549936595964884  \\
            1.09  1.849044085540075  \\
            1.091  1.8431190233238888  \\
            1.092  1.8372183496321544  \\
            1.093  1.8313419418816141  \\
            1.094  1.825489678216502  \\
            1.095  1.8196614375035751  \\
            1.096  1.8138570993271665  \\
            1.097  1.808076543984294  \\
            1.098  1.8023196524797882  \\
            1.099  1.7965863065214767  \\
            1.1  1.7908763885153869  \\
            1.101  1.785189781561007  \\
            1.102  1.7795263694465575  \\
            1.103  1.7738860366443285  \\
            1.104  1.7682686683060236  \\
            1.105  1.7626741502581666  \\
            1.106  1.7571023689975183  \\
            1.107  1.7515532116865522  \\
            1.108  1.7460265661489436  \\
            1.109  1.7405223208651117  \\
            1.11  1.7350403649677775  \\
            1.111  1.7295805882375772  \\
            1.112  1.7241428810986845  \\
            1.113  1.7187271346144897  \\
            1.114  1.7133332404832926  \\
            1.115  1.7079610910340446  \\
            1.116  1.702610579222106  \\
            1.117  1.6972815986250556  \\
            1.118  1.6919740434385093  \\
            1.119  1.686687808471994  \\
            1.12  1.681422789144831  \\
            1.121  1.67617888148207  \\
            1.122  1.6709559821104347  \\
            1.123  1.6657539882543204  \\
            1.124  1.6605727977317983  \\
            1.125  1.6554123089506736  \\
            1.126  1.6502724209045527  \\
            1.127  1.6451530331689521  \\
            1.128  1.6400540458974386  \\
            1.129  1.6349753598177839  \\
            1.13  1.6299168762281675  \\
            1.131  1.6248784969933896  \\
            1.132  1.61986012454113  \\
            1.133  1.6148616618582161  \\
            1.134  1.60988301248694  \\
            1.135  1.6049240805213816  \\
            1.136  1.5999847706037778  \\
            1.137  1.5950649879209045  \\
            1.138  1.590164638200498  \\
            1.139  1.585283627707689  \\
            1.14  1.5804218632414802  \\
            1.141  1.5755792521312315  \\
            1.142  1.5707557022331895  \\
            1.143  1.5659511219270255  \\
            1.144  1.5611654201124172  \\
            1.145  1.5563985062056354  \\
            1.146  1.5516502901361782  \\
            1.147  1.5469206823434076  \\
            1.148  1.542209593773233  \\
            1.149  1.5375169358747984  \\
            1.15  1.532842620597212  \\
            1.151  1.528186560386286  \\
            1.152  1.5235486681813117  \\
            1.153  1.5189288574118467  \\
            1.154  1.5143270419945396  \\
            1.155  1.5097431363299618  \\
            1.156  1.5051770552994812  \\
            1.157  1.5006287142621386  \\
            1.158  1.4960980290515664  \\
            1.159  1.491584915972913  \\
            1.16  1.4870892917998053  \\
            1.161  1.4826110737713172  \\
            1.162  1.4781501795889775  \\
            1.163  1.4737065274137815  \\
            1.164  1.469280035863243  \\
            1.165  1.4648706240084501  \\
            1.166  1.4604782113711585  \\
            1.167  1.4561027179208899  \\
            1.168  1.4517440640720694  \\
            1.169  1.447402170681165  \\
            1.17  1.443076959043865  \\
            1.171  1.4387683508922602  \\
            1.172  1.434476268392062  \\
            1.173  1.4302006341398252  \\
            1.174  1.4259413711602063  \\
            1.175  1.4216984029032256  \\
            1.176  1.417471653241566  \\
            1.177  1.4132610464678734  \\
            1.178  1.4090665072920958  \\
            1.179  1.4048879608388207  \\
            1.18  1.4007253326446525  \\
            1.181  1.3965785486555913  \\
            1.182  1.3924475352244432  \\
            1.183  1.3883322191082388  \\
            1.184  1.3842325274656813  \\
            1.185  1.3801483878546015  \\
            1.186  1.3760797282294406  \\
            1.187  1.3720264769387427  \\
            1.188  1.367988562722675  \\
            1.189  1.3639659147105538  \\
            1.19  1.3599584624184  \\
            1.191  1.3559661357465016  \\
            1.192  1.351988864977004  \\
            1.193  1.348026580771505  \\
            1.194  1.3440792141686817  \\
            1.195  1.3401466965819184  \\
            1.196  1.3362289597969665  \\
            1.197  1.3323259359696071  \\
            1.198  1.3284375576233443  \\
            1.199  1.324563757647099  \\
            1.2  1.3207044692929353  \\
            1.201  1.3168596261737895  \\
            1.202  1.3130291622612242  \\
            1.203  1.3092130118831913  \\
            1.204  1.30541110972182  \\
            1.205  1.3016233908112067  \\
            1.206  1.297849790535237  \\
            1.207  1.2940902446254066  \\
            1.208  1.290344689158672  \\
            1.209  1.286613060555304  \\
            1.21  1.282895295576767  \\
            1.211  1.2791913313236034  \\
            1.212  1.2755011052333418  \\
            1.213  1.2718245550784115  \\
            1.214  1.2681616189640805  \\
            1.215  1.2645122353263962  \\
            1.216  1.260876342930156  \\
            1.217  1.257253880866874  \\
            1.218  1.2536447885527802  \\
            1.219  1.2500490057268174  \\
            1.22  1.246466472448665  \\
            1.221  1.2428971290967654  \\
            1.222  1.2393409163663762  \\
            1.223  1.2357977752676215  \\
            1.224  1.232267647123573  \\
            1.225  1.2287504735683275  \\
            1.226  1.2252461965451142  \\
            1.227  1.2217547583044  \\
            1.228  1.2182761014020205  \\
            1.229  1.2148101686973138  \\
            1.23  1.211356903351277  \\
            1.231  1.2079162488247235  \\
            1.232  1.2044881488764665  \\
            1.233  1.2010725475615025  \\
            1.234  1.197669389229218  \\
            1.235  1.1942786185215983  \\
            1.236  1.1909001803714594  \\
            1.237  1.1875340200006805  \\
            1.238  1.184180082918461  \\
            1.239  1.180838314919576  \\
            1.24  1.1775086620826583  \\
            1.241  1.1741910707684773  \\
            1.242  1.1708854876182433  \\
            1.243  1.1675918595519121  \\
            1.244  1.1643101337665105  \\
            1.245  1.1610402577344638  \\
            1.246  1.1577821792019458  \\
            1.247  1.1545358461872284  \\
            1.248  1.1513012069790522  \\
            1.249  1.1480782101350004  \\
            1.25  1.1448668044798924  \\
            1.251  1.1416669391041767  \\
            1.252  1.1384785633623458  \\
            1.253  1.135301626871358  \\
            1.254  1.1321360795090645  \\
            1.255  1.1289818714126574  \\
            1.256  1.1258389529771178  \\
            1.257  1.122707274853683  \\
            1.258  1.1195867879483175  \\
            1.259  1.1164774434201992  \\
            1.26  1.113379192680211  \\
            1.261  1.1102919873894495  \\
            1.262  1.1072157794577335  \\
            1.263  1.1041505210421354  \\
            1.264  1.1010961645455088  \\
            1.265  1.098052662615039  \\
            1.266  1.0950199681407908  \\
            1.267  1.0919980342542783  \\
            1.268  1.0889868143270325  \\
            1.269  1.0859862619691891  \\
            1.27  1.0829963310280764  \\
            1.271  1.0800169755868219  \\
            1.272  1.0770481499629572  \\
            1.273  1.0740898087070454  \\
            1.274  1.0711419066013037  \\
            1.275  1.0682043986582481  \\
            1.276  1.0652772401193358  \\
            1.277  1.062360386453627  \\
            1.278  1.0594537933564452  \\
            1.279  1.0565574167480571  \\
            1.28  1.0536712127723509  \\
            1.281  1.0507951377955325  \\
            1.282  1.0479291484048223  \\
            1.283  1.0450732014071693  \\
            1.284  1.0422272538279624  \\
            1.285  1.039391262909764  \\
            1.286  1.0365651861110379  \\
            1.287  1.0337489811048979  \\
            1.288  1.0309426057778546  \\
            1.289  1.0281460182285793  \\
            1.29  1.0253591767666668  \\
            1.291  1.0225820399114167  \\
            1.292  1.0198145663906115  \\
            1.293  1.0170567151393146  \\
            1.294  1.0143084452986635  \\
            1.295  1.0115697162146837  \\
            1.296  1.0088404874370993  \\
            1.297  1.00612071871816  \\
            1.298  1.0034103700114678  \\
            1.299  1.000709401470821  \\
            1.3  0.9980177734490545  \\
            1.301  0.9953354464968989  \\
            1.302  0.9926623813618363  \\
            1.303  0.9899985389869748  \\
            1.304  0.9873438805099182  \\
            1.305  0.984698367261655  \\
            1.306  0.9820619607654444  \\
            1.307  0.9794346227357174  \\
            1.308  0.9768163150769783  \\
            1.309  0.9742069998827214  \\
            1.31  0.9716066394343443  \\
            1.311  0.9690151962000794  \\
            1.312  0.9664326328339221  \\
            1.313  0.9638589121745751  \\
            1.314  0.9612939972443914  \\
            1.315  0.9587378512483309  \\
            1.316  0.9561904375729178  \\
            1.317  0.9536517197852107  \\
            1.318  0.9511216616317729  \\
            1.319  0.9486002270376566  \\
            1.32  0.9460873801053853  \\
            1.321  0.9435830851139514  \\
            1.322  0.941087306517812  \\
            1.323  0.9386000089458997  \\
            1.324  0.9361211572006298  \\
            1.325  0.933650716256925  \\
            1.326  0.9311886512612348  \\
            1.327  0.9287349275305725  \\
            1.328  0.9262895105515483  \\
            1.329  0.9238523659794163  \\
            1.33  0.9214234596371219  \\
            1.331  0.9190027575143604  \\
            1.332  0.916590225766636  \\
            1.333  0.9141858307143327  \\
            1.334  0.9117895388417857  \\
            1.335  0.9094013167963634  \\
            1.336  0.9070211313875507  \\
            1.337  0.9046489495860428  \\
            1.338  0.90228473852284  \\
            1.339  0.8999284654883535  \\
            1.34  0.8975800979315107  \\
            1.341  0.895239603458873  \\
            1.342  0.8929069498337517  \\
            1.343  0.8905821049753377  \\
            1.344  0.8882650369578287  \\
            1.345  0.8859557140095684  \\
            1.346  0.883654104512186  \\
            1.347  0.8813601769997466  \\
            1.348  0.8790739001578998  \\
            1.349  0.8767952428230423  \\
            1.35  0.8745241739814766  \\
            1.351  0.8722606627685838  \\
            1.352  0.8700046784679939  \\
            1.353  0.8677561905107674  \\
            1.354  0.8655151684745768  \\
            1.355  0.8632815820828996  \\
            1.356  0.8610554012042078  \\
            1.357  0.8588365958511721  \\
            1.358  0.8566251361798618  \\
            1.359  0.8544209924889585  \\
            1.36  0.8522241352189662  \\
            1.361  0.850034534951434  \\
            1.362  0.8478521624081771  \\
            1.363  0.8456769884505088  \\
            1.364  0.8435089840784705  \\
            1.365  0.8413481204300746  \\
            1.366  0.8391943687805428  \\
            1.367  0.8370477005415587  \\
            1.368  0.8349080872605159  \\
            1.369  0.8327755006197796  \\
            1.37  0.830649912435944  \\
            1.371  0.8285312946591027  \\
            1.372  0.826419619372116  \\
            1.373  0.8243148587898899  \\
            1.374  0.8222169852586523  \\
            1.375  0.8201259712552407  \\
            1.376  0.8180417893863887  \\
            1.377  0.815964412388019  \\
            1.378  0.8138938131245441  \\
            1.379  0.8118299645881633  \\
            1.38  0.8097728398981732  \\
            1.381  0.8077224123002749  \\
            1.382  0.8056786551658912  \\
            1.383  0.8036415419914826  \\
            1.384  0.8016110463978736  \\
            1.385  0.7995871421295764  \\
            1.386  0.7975698030541258  \\
            1.387  0.795559003161411  \\
            1.388  0.7935547165630183  \\
            1.389  0.7915569174915715  \\
            1.39  0.7895655803000827  \\
            1.391  0.7875806794613003  \\
            1.392  0.7856021895670675  \\
            1.393  0.7836300853276794  \\
            1.394  0.7816643415712485  \\
            1.395  0.7797049332430688  \\
            1.396  0.7777518354049904  \\
            1.397  0.7758050232347907  \\
            1.398  0.7738644720255566  \\
            1.399  0.7719301571850635  \\
            1.4  0.7700020542351648  \\
            1.401  0.7680801388111775  \\
            1.402  0.7661643866612811  \\
            1.403  0.7642547736459094  \\
            1.404  0.7623512757371563  \\
            1.405  0.7604538690181756  \\
            1.406  0.7585625296825937  \\
            1.407  0.7566772340339168  \\
            1.408  0.7547979584849506  \\
            1.409  0.7529246795572143  \\
            1.41  0.7510573738803676  \\
            1.411  0.749196018191632  \\
            1.412  0.7473405893352238  \\
            1.413  0.7454910642617838  \\
            1.414  0.7436474200278151  \\
            1.415  0.7418096337951212  \\
            1.416  0.7399776828302504  \\
            1.417  0.7381515445039398  \\
            1.418  0.7363311962905668  \\
            1.419  0.7345166157675997  \\
            1.42  0.7327077806150561  \\
            1.421  0.7309046686149598  \\
            1.422  0.729107257650805  \\
            1.423  0.7273155257070201  \\
            1.424  0.7255294508684396  \\
            1.425  0.7237490113197721  \\
            1.426  0.7219741853450795  \\
            1.427  0.7202049513272519  \\
            1.428  0.7184412877474917  \\
            1.429  0.7166831731847956  \\
            1.43  0.7149305863154445  \\
            1.431  0.7131835059124922  \\
            1.432  0.7114419108452603  \\
            1.433  0.7097057800788337  \\
            1.434  0.7079750926735613  \\
            1.435  0.7062498277845574  \\
            1.436  0.7045299646612085  \\
            1.437  0.7028154826466799  \\
            1.438  0.7011063611774294  \\
            1.439  0.6994025797827188  \\
            1.44  0.6977041180841336  \\
            1.441  0.6960109557951004  \\
            1.442  0.694323072720412  \\
            1.443  0.6926404487557508  \\
            1.444  0.6909630638872191  \\
            1.445  0.6892908981908683  \\
            1.446  0.6876239318322346  \\
            1.447  0.6859621450658736  \\
            1.448  0.684305518234902  \\
            1.449  0.682654031770537  \\
            1.45  0.6810076661916437  \\
            1.451  0.6793664021042793  \\
            1.452  0.677730220201247  \\
            1.453  0.6760991012616444  \\
            1.454  0.6744730261504226  \\
            1.455  0.6728519758179408  \\
            1.456  0.6712359312995297  \\
            1.457  0.6696248737150516  \\
            1.458  0.6680187842684687  \\
            1.459  0.6664176442474085  \\
            1.46  0.6648214350227367  \\
            1.461  0.6632301380481286  \\
            1.462  0.6616437348596459  \\
            1.463  0.6600622070753132  \\
            1.464  0.6584855363947005  \\
            1.465  0.656913704598504  \\
            1.466  0.6553466935481331  \\
            1.467  0.6537844851852965  \\
            1.468  0.6522270615315948  \\
            1.469  0.6506744046881092  \\
            1.47  0.6491264968350006  \\
            1.471  0.6475833202311027  \\
            1.472  0.6460448572135251  \\
            1.473  0.6445110901972518  \\
            1.474  0.642982001674749  \\
            1.475  0.6414575742155668  \\
            1.476  0.6399377904659527  \\
            1.477  0.6384226331484586  \\
            1.478  0.6369120850615566  \\
            1.479  0.6354061290792521  \\
            1.48  0.6339047481507037  \\
            1.481  0.6324079252998407  \\
            1.482  0.6309156436249873  \\
            1.483  0.6294278862984836  \\
            1.484  0.6279446365663154  \\
            1.485  0.6264658777477397  \\
            1.486  0.6249915932349174  \\
            1.487  0.6235217664925434  \\
            1.488  0.6220563810574846  \\
            1.489  0.6205954205384129  \\
            1.49  0.6191388686154476  \\
            1.491  0.6176867090397935  \\
            1.492  0.6162389256333862  \\
            1.493  0.6147955022885347  \\
            1.494  0.6133564229675713  \\
            1.495  0.6119216717024971  \\
            1.496  0.6104912325946368  \\
            1.497  0.609065089814288  \\
            1.498  0.6076432276003794  \\
            1.499  0.6062256302601252  \\
            1.5  0.604812282168686  \\
            1.501  0.6034031677688276  \\
            1.502  0.6019982715705849  \\
            1.503  0.6005975781509273  \\
            1.504  0.5992010721534224  \\
            1.505  0.5978087382879081  \\
            1.506  0.5964205613301592  \\
            1.507  0.5950365261215628  \\
            1.508  0.5936566175687892  \\
            1.509  0.5922808206434703  \\
            1.51  0.5909091203818755  \\
            1.511  0.5895415018845922  \\
            1.512  0.5881779503162059  \\
            1.513  0.5868184509049847  \\
            1.514  0.5854629889425624  \\
            1.515  0.5841115497836268  \\
            1.516  0.5827641188456062  \\
            1.517  0.5814206816083615  \\
            1.518  0.5800812236138757  \\
            1.519  0.5787457304659499  \\
            1.52  0.5774141878298961  \\
            1.521  0.5760865814322363  \\
            1.522  0.5747628970603993  \\
            1.523  0.5734431205624225  \\
            1.524  0.5721272378466526  \\
            1.525  0.5708152348814508  \\
            1.526  0.5695070976948957  \\
            1.527  0.5682028123744932  \\
            1.528  0.5669023650668823  \\
            1.529  0.5656057419775473  \\
            1.53  0.5643129293705281  \\
            1.531  0.5630239135681352  \\
            1.532  0.5617386809506624  \\
            1.533  0.5604572179561061  \\
            1.534  0.559179511079881  \\
            1.535  0.5579055468745411  \\
            1.536  0.5566353119495003  \\
            1.537  0.5553687929707565  \\
            1.538  0.5541059766606132  \\
            1.539  0.5528468497974087  \\
            1.54  0.5515913992152404  \\
            1.541  0.5503396118036958  \\
            1.542  0.5490914745075813  \\
            1.543  0.5478469743266555  \\
            1.544  0.5466060983153609  \\
            1.545  0.5453688335825602  \\
            1.546  0.5441351672912708  \\
            1.547  0.5429050866584046  \\
            1.548  0.5416785789545047  \\
            1.549  0.5404556315034883  \\
            1.55  0.5392362316823864  \\
            1.551  0.5380203669210895  \\
            1.552  0.5368080247020897  \\
            1.553  0.5355991925602296  \\
            1.554  0.5343938580824474  \\
            1.555  0.5331920089075284  \\
            1.556  0.5319936327258522  \\
            1.557  0.5307987172791477  \\
            1.558  0.529607250360243  \\
            1.559  0.5284192198128225  \\
            1.56  0.5272346135311802  \\
            1.561  0.5260534194599789  \\
            1.562  0.5248756255940066  \\
            1.563  0.5237012199779378  \\
            1.564  0.5225301907060932  \\
            1.565  0.521362525922203  \\
            1.566  0.5201982138191694  \\
            1.567  0.5190372426388328  \\
            1.568  0.5178796006717361  \\
            1.569  0.516725276256894  \\
            1.57  0.5155742577815602  \\
            1.571  0.5144265336809981  \\
            1.572  0.5132820924382512  \\
            1.573  0.5121409225839167  \\
            1.574  0.5110030126959174  \\
            1.575  0.5098683513992783  \\
            1.576  0.5087369273659007  \\
            1.577  0.5076087293143418  \\
            1.578  0.5064837460095903  \\
            1.579  0.505361966262849  \\
            1.58  0.5042433789313129  \\
            1.581  0.5031279729179536  \\
            1.582  0.5020157371713003  \\
            1.583  0.5009066606852257  \\
            1.584  0.4998007324987305  \\
            1.585  0.498697941695731  \\
            1.586  0.49759827740484547  \\
            1.587  0.49650172879918486  \\
            1.588  0.49540828509614065  \\
            1.589  0.49431793555717857  \\
            1.59  0.4932306694876282  \\
            1.591  0.49214647623647884  \\
            1.592  0.4910653451961717  \\
            1.593  0.4899872658023976  \\
            1.594  0.48891222753389174  \\
            1.595  0.4878402199122333  \\
            1.596  0.48677123250164217  \\
            1.597  0.4857052549087812  \\
            1.598  0.48464227678255484  \\
            1.599  0.4835822878139133  \\
            1.6  0.48252527773565346  \\
            1.601  0.4814712363222254  \\
            1.602  0.480420153389535  \\
            1.603  0.47937201879475233  \\
            1.604  0.47832682243611735  \\
            1.605  0.47728455425274924  \\
            1.606  0.4762452042244544  \\
            1.607  0.4752087623715382  \\
            1.608  0.4741752187546147  \\
            1.609  0.4731445634744205  \\
            1.61  0.4721167866716263  \\
            1.611  0.4710918785266527  \\
            1.612  0.4700698292594838  \\
            1.613  0.4690506291294855  \\
            1.614  0.46803426843522006  \\
            1.615  0.46702073751426665  \\
            1.616  0.46601002674303843  \\
            1.617  0.46500212653660433  \\
            1.618  0.46399702734850823  \\
            1.619  0.46299471967059247  \\
            1.62  0.4619951940328194  \\
            1.621  0.4609984410030965  \\
            1.622  0.46000445118709926  \\
            1.623  0.45901321522809874  \\
            1.624  0.45802472380678616  \\
            1.625  0.4570389676411025  \\
            1.626  0.45605593748606466  \\
            1.627  0.4550756241335957  \\
            1.628  0.45409801841235553  \\
            1.629  0.4531231111875705  \\
            1.63  0.4521508933608668  \\
            1.631  0.4511813558701016  \\
            1.632  0.4502144896891984  \\
            1.633  0.44925028582797977  \\
            1.634  0.4482887353320044  \\
            1.635  0.4473298292824016  \\
            1.636  0.4463735587957101  \\
            1.637  0.4454199150237142  \\
            1.638  0.4444688891532842  \\
            1.639  0.4435204724062143  \\
            1.64  0.4425746560390647  \\
            1.641  0.4416314313430012  \\
            1.642  0.4406907896436387  \\
            1.643  0.43975272230088286  \\
            1.644  0.4388172207087748  \\
            1.645  0.43788427629533455  \\
            1.646  0.43695388052240763  \\
            1.647  0.43602602488550957  \\
            1.648  0.43510070091367437  \\
            1.649  0.4341779001693009  \\
            1.65  0.4332576142480024  \\
            1.651  0.4323398347784545  \\
            1.652  0.43142455342224667  \\
            1.653  0.4305117618737315  \\
            1.654  0.42960145185987797  \\
            1.655  0.42869361514012166  \\
            1.656  0.42778824350621997  \\
            1.657  0.426885328782104  \\
            1.658  0.42598486282373493  \\
            1.659  0.42508683751895754  \\
            1.66  0.424191244787358  \\
            1.661  0.423298076580119  \\
            1.662  0.42240732487987886  \\
            1.663  0.42151898170058827  \\
            1.664  0.42063303908737076  \\
            1.665  0.41974948911638066  \\
            1.666  0.41886832389466555  \\
            1.667  0.4179895355600252  \\
            1.668  0.4171131162808756  \\
            1.669  0.41623905825610935  \\
            1.67  0.4153673537149608  \\
            1.671  0.4144979949168682  \\
            1.672  0.41363097415134004  \\
            1.673  0.41276628373781876  \\
            1.674  0.4119039160255483  \\
            1.675  0.41104386339343907  \\
            1.676  0.41018611824993695  \\
            1.677  0.40933067303288984  \\
            1.678  0.4084775202094176  \\
            1.679  0.4076266522757802  \\
            1.68  0.40677806175724923  \\
            1.681  0.40593174120797654  \\
            1.682  0.4050876832108677  \\
            1.683  0.4042458803774521  \\
            1.684  0.4034063253477572  \\
            1.685  0.4025690107901803  \\
            1.686  0.40173392940136365  \\
            1.687  0.4009010739060679  \\
            1.688  0.4000704370570484  \\
            1.689  0.39924201163492967  \\
            1.69  0.39841579044808345  \\
            1.691  0.39759176633250376  \\
            1.692  0.39676993215168654  \\
            1.693  0.39595028079650607  \\
            1.694  0.39513280518509564  \\
            1.695  0.3943174982627252  \\
            1.696  0.39350435300168324  \\
            1.697  0.3926933624011557  \\
            1.698  0.39188451948710895  \\
            1.699  0.3910778173121702  \\
            1.7  0.39027324895551113  \\
            1.701  0.38947080752272994  \\
            1.702  0.38867048614573607  \\
            1.703  0.38787227798263313  \\
            1.704  0.38707617621760493  \\
            1.705  0.3862821740607996  \\
            1.706  0.38549026474821724  \\
            1.707  0.3847004415415942  \\
            1.708  0.38391269772829245  \\
            1.709  0.38312702662118514  \\
            1.71  0.3823434215585465  \\
            1.711  0.3815618759039392  \\
            1.712  0.3807823830461049  \\
            1.713  0.38000493639885274  \\
            1.714  0.37922952940095117  \\
            1.715  0.3784561555160174  \\
            1.716  0.37768480823241046  \\
            1.717  0.37691548106312167  \\
            1.718  0.3761481675456684  \\
            1.719  0.37538286124198605  \\
            1.72  0.37461955573832306  \\
            1.721  0.3738582446451333  \\
            1.722  0.3730989215969724  \\
            1.723  0.3723415802523915  \\
            1.724  0.3715862142938342  \\
            1.725  0.3708328174275315  \\
            1.726  0.3700813833833998  \\
            1.727  0.3693319059149367  \\
            1.728  0.3685843787991204  \\
            1.729  0.3678387958363061  \\
            1.73  0.3670951508501264  \\
            1.731  0.366353437687389  \\
            1.732  0.365613650217978  \\
            1.733  0.3648757823347525  \\
            1.734  0.36413982795344857  \\
            1.735  0.36340578101257964  \\
            1.736  0.36267363547333886  \\
            1.737  0.3619433853195002  \\
            1.738  0.36121502455732235  \\
            1.739  0.3604885472154507  \\
            1.74  0.359763947344822  \\
            1.741  0.359041219018567  \\
            1.742  0.35832035633191645  \\
            1.743  0.35760135340210486  \\
            1.744  0.35688420436827706  \\
            1.745  0.35616890339139284  \\
            1.746  0.35545544465413464  \\
            1.747  0.35474382236081303  \\
            1.748  0.35403403073727546  \\
            1.749  0.3533260640308122  \\
            1.75  0.35261991651006624  \\
            1.751  0.3519155824649407  \\
            1.752  0.3512130562065085  \\
            1.753  0.3505123320669217  \\
            1.754  0.3498134043993209  \\
            1.755  0.34911626757774694  \\
            1.756  0.34842091599705005  \\
            1.757  0.3477273440728026  \\
            1.758  0.34703554624120936  \\
            1.759  0.34634551695902077  \\
            1.76  0.34565725070344466  \\
            1.761  0.34497074197205985  \\
            1.762  0.3442859852827286  \\
            1.763  0.3436029751735115  \\
            1.764  0.34292170620258067  \\
            1.765  0.34224217294813486  \\
            1.766  0.34156437000831436  \\
            1.767  0.3408882920011167  \\
            1.768  0.3402139335643116  \\
            1.769  0.3395412893553584  \\
            1.77  0.3388703540513216  \\
            1.771  0.3382011223487888  \\
            1.772  0.3375335889637873  \\
            1.773  0.3368677486317029  \\
            1.774  0.33620359610719663  \\
            1.775  0.33554112616412535  \\
            1.776  0.33488033359545877  \\
            1.777  0.3342212132132004  \\
            1.778  0.33356375984830605  \\
            1.779  0.33290796835060527  \\
            1.78  0.33225383358872046  \\
            1.781  0.33160135044998934  \\
            1.782  0.3309505138403846  \\
            1.783  0.33030131868443724  \\
            1.784  0.3296537599251569  \\
            1.785  0.3290078325239559  \\
            1.786  0.3283635314605706  \\
            1.787  0.3277208517329858  \\
            1.788  0.32707978835735724  \\
            1.789  0.32644033636793607  \\
            1.79  0.3258024908169926  \\
            1.791  0.32516624677474176  \\
            1.792  0.3245315993292669  \\
            1.793  0.3238985435864461  \\
            1.794  0.3232670746698769  \\
            1.795  0.3226371877208037  \\
            1.796  0.3220088778980422  \\
            1.797  0.321382140377908  \\
            1.798  0.3207569703541423  \\
            1.799  0.32013336303784007  \\
            1.8  0.31951131365737734  \\
            1.801  0.31889081745833986  \\
            1.802  0.31827186970345056  \\
            1.803  0.3176544656724995  \\
            1.804  0.3170386006622719  \\
            1.805  0.31642426998647855  \\
            1.806  0.31581146897568424  \\
            1.807  0.31520019297723967  \\
            1.808  0.31459043735521003  \\
            1.809  0.3139821974903071  \\
            1.81  0.31337546877981953  \\
            1.811  0.3127702466375447  \\
            1.812  0.31216652649372  \\
            1.813  0.31156430379495553  \\
            1.814  0.3109635740041656  \\
            1.815  0.31036433260050217  \\
            1.816  0.30976657507928723  \\
            1.817  0.3091702969519467  \\
            1.818  0.30857549374594345  \\
            1.819  0.3079821610047117  \\
            1.82  0.3073902942875908  \\
            1.821  0.30679988916976036  \\
            1.822  0.30621094124217435  \\
            1.823  0.30562344611149705  \\
            1.824  0.3050373994000376  \\
            1.825  0.3044527967456868  \\
            1.826  0.30386963380185206  \\
            1.827  0.3032879062373947  \\
            1.828  0.3027076097365658  \\
            1.829  0.3021287399989439  \\
            1.83  0.30155129273937137  \\
            1.831  0.3009752636878929  \\
            1.832  0.30040064858969223  \\
            1.833  0.2998274432050314  \\
            1.834  0.29925564330918797  \\
            1.835  0.2986852446923948  \\
            1.836  0.29811624315977797  \\
            1.837  0.297548634531297  \\
            1.838  0.2969824146416833  \\
            1.839  0.29641757934038115  \\
            1.84  0.2958541244914866  \\
            1.841  0.29529204597368897  \\
            1.842  0.29473133968021015  \\
            1.843  0.294172001518747  \\
            1.844  0.29361402741141107  \\
            1.845  0.2930574132946713  \\
            1.846  0.2925021551192946  \\
            1.847  0.29194824885028897  \\
            1.848  0.2913956904668446  \\
            1.849  0.2908444759622774  \\
            1.85  0.2902946013439708  \\
            1.851  0.28974606263331987  \\
            1.852  0.28919885586567345  \\
            1.853  0.28865297709027893  \\
            1.854  0.2881084223702248  \\
            1.855  0.28756518778238593  \\
            1.856  0.28702326941736683  \\
            1.857  0.2864826633794473  \\
            1.858  0.2859433657865259  \\
            1.859  0.28540537277006683  \\
            1.86  0.2848686804750434  \\
            1.861  0.2843332850598851  \\
            1.862  0.28379918269642246  \\
            1.863  0.2832663695698338  \\
            1.864  0.282734841878591  \\
            1.865  0.2822045958344067  \\
            1.866  0.2816756276621804  \\
            1.867  0.2811479335999463  \\
            1.868  0.2806215098988196  \\
            1.869  0.28009635282294537  \\
            1.87  0.2795724586494449  \\
            1.871  0.27904982366836506  \\
            1.872  0.27852844418262523  \\
            1.873  0.2780083165079671  \\
            1.874  0.2774894369729021  \\
            1.875  0.2769718019186618  \\
            1.876  0.2764554076991455  \\
            1.877  0.27594025068087075  \\
            1.878  0.275426327242923  \\
            1.879  0.2749136337769045  \\
            1.88  0.2744021666868857  \\
            1.881  0.27389192238935445  \\
            1.882  0.27338289731316734  \\
            1.883  0.27287508789949977  \\
            1.884  0.27236849060179785  \\
            1.885  0.27186310188572843  \\
            1.886  0.27135891822913183  \\
            1.887  0.27085593612197223  \\
            1.888  0.2703541520662907  \\
            1.889  0.2698535625761562  \\
            1.89  0.2693541641776191  \\
            1.891  0.26885595340866214  \\
            1.892  0.2683589268191548  \\
            1.893  0.2678630809708049  \\
            1.894  0.2673684124371128  \\
            1.895  0.26687491780332373  \\
            1.896  0.2663825936663825  \\
            1.897  0.2658914366348862  \\
            1.898  0.26540144332903887  \\
            1.899  0.2649126103806054  \\
            1.9  0.26442493443286624  \\
            1.901  0.26393841214057134  \\
            1.902  0.26345304016989585  \\
            1.903  0.26296881519839427  \\
            1.904  0.26248573391495633  \\
            1.905  0.26200379301976184  \\
            1.906  0.2615229892242368  \\
            1.907  0.26104331925100865  \\
            1.908  0.26056477983386295  \\
            1.909  0.2600873677176988  \\
            1.91  0.259611079658486  \\
            1.911  0.25913591242322076  \\
            1.912  0.2586618627898836  \\
            1.913  0.2581889275473951  \\
            1.914  0.257717103495574  \\
            1.915  0.25724638744509387  \\
            1.916  0.2567767762174414  \\
            1.917  0.2563082666448732  \\
            1.918  0.2558408555703746  \\
            1.919  0.2553745398476171  \\
            1.92  0.25490931634091707  \\
            1.921  0.2544451819251937  \\
            1.922  0.2539821334859286  \\
            1.923  0.2535201679191233  \\
            1.924  0.25305928213125967  \\
            1.925  0.2525994730392579  \\
            1.926  0.252140737570437  \\
            1.927  0.25168307266247303  \\
            1.928  0.2512264752633604  \\
            1.929  0.2507709423313702  \\
            1.93  0.2503164708350117  \\
            1.931  0.2498630577529914  \\
            1.932  0.2494107000741745  \\
            1.933  0.2489593947975447  \\
            1.934  0.24850913893216564  \\
            1.935  0.24805992949714115  \\
            1.936  0.24761176352157716  \\
            1.937  0.24716463804454225  \\
            1.938  0.2467185501150299  \\
            1.939  0.2462734967919193  \\
            1.94  0.24582947514393785  \\
            1.941  0.24538648224962253  \\
            1.942  0.24494451519728283  \\
            1.943  0.244503571084962  \\
            1.944  0.24406364702040068  \\
            1.945  0.24362474012099838  \\
            1.946  0.24318684751377734  \\
            1.947  0.24274996633534457  \\
            1.948  0.2423140937318558  \\
            1.949  0.24187922685897795  \\
            1.95  0.24144536288185334  \\
            1.951  0.24101249897506258  \\
            1.952  0.240580632322589  \\
            1.953  0.24014976011778186  \\
            1.954  0.23971987956332128  \\
            1.955  0.23929098787118144  \\
            1.956  0.23886308226259603  \\
            1.957  0.2384361599680218  \\
            1.958  0.23801021822710416  \\
            1.959  0.23758525428864113  \\
            1.96  0.2371612654105493  \\
            1.961  0.236738248859828  \\
            1.962  0.23631620191252545  \\
            1.963  0.23589512185370348  \\
            1.964  0.23547500597740387  \\
            1.965  0.23505585158661332  \\
            1.966  0.2346376559932302  \\
            1.967  0.23422041651802955  \\
            1.968  0.23380413049063048  \\
            1.969  0.23338879524946135  \\
            1.97  0.23297440814172718  \\
            1.971  0.23256096652337552  \\
            1.972  0.232148467759064  \\
            1.973  0.23173690922212636  \\
            1.974  0.23132628829454036  \\
            1.975  0.2309166023668943  \\
            1.976  0.23050784883835485  \\
            1.977  0.2301000251166341  \\
            1.978  0.2296931286179577  \\
            1.979  0.22928715676703199  \\
            1.98  0.22888210699701247  \\
            1.981  0.22847797674947135  \\
            1.982  0.22807476347436625  \\
            1.983  0.22767246463000781  \\
            1.984  0.22727107768302904  \\
            1.985  0.22687060010835283  \\
            1.986  0.2264710293891617  \\
            1.987  0.2260723630168658  \\
            1.988  0.22567459849107258  \\
            1.989  0.22527773331955522  \\
            1.99  0.22488176501822255  \\
            1.991  0.22448669111108788  \\
            1.992  0.22409250913023884  \\
            1.993  0.2236992166158068  \\
            1.994  0.22330681111593692  \\
            1.995  0.22291529018675754  \\
            1.996  0.22252465139235084  \\
            1.997  0.22213489230472244  \\
            1.998  0.22174601050377213  \\
            1.999  0.22135800357726365  \\
            2.0  0.2209708691207961  \\
        }
        ;
    \addlegendentry {$\beta = 2.5$}
    \addplot[color={rgb,1:red,0.0;green,0.4667;blue,0.7333}, name path={0c7eec19-1b72-468c-9cae-38cecb44378a}, draw opacity={1.0}, line width={1.2}, dashed]
        table[row sep={\\}]
        {
            \\
            1.0  5.0  \\
            1.001  4.970104720628746  \\
            1.002  4.9404177700398275  \\
            1.003  4.910937490725499  \\
            1.004  4.881662239999157  \\
            1.005  4.85259038984829  \\
            1.006  4.8237203267889175  \\
            1.007  4.795050451721727  \\
            1.008  4.766579179789675  \\
            1.009  4.738304940237224  \\
            1.01  4.710226176271034  \\
            1.011  4.682341344922236  \\
            1.012  4.6546489169101255  \\
            1.013  4.627147376507413  \\
            1.014  4.5998352214068445  \\
            1.015  4.572710962589359  \\
            1.016  4.545773124193571  \\
            1.017  4.519020243386757  \\
            1.018  4.492450870237146  \\
            1.019  4.466063567587664  \\
            1.02  4.43985691093096  \\
            1.021  4.413829488285853  \\
            1.022  4.387979900075028  \\
            1.023  4.362306759004121  \\
            1.024  4.336808689942017  \\
            1.025  4.311484329802523  \\
            1.026  4.286332327427209  \\
            1.027  4.261351343469589  \\
            1.028  4.236540050280463  \\
            1.029  4.2118971317945535  \\
            1.03  4.187421283418271  \\
            1.031  4.163111211918762  \\
            1.032  4.138965635314043  \\
            1.033  4.114983282764399  \\
            1.034  4.091162894464835  \\
            1.035  4.0675032215387645  \\
            1.036  4.044003025932735  \\
            1.037  4.02066108031235  \\
            1.038  3.9974761679592117  \\
            1.039  3.97444708266903  \\
            1.04  3.9515726286507293  \\
            1.041  3.928851620426697  \\
            1.042  3.906282882734006  \\
            1.043  3.883865250426749  \\
            1.044  3.861597568379328  \\
            1.045  3.839478691390833  \\
            1.046  3.8175074840903482  \\
            1.047  3.7956828208433273  \\
            1.048  3.7740035856588716  \\
            1.049  3.752468672098056  \\
            1.05  3.7310769831831374  \\
            1.051  3.709827431307782  \\
            1.052  3.6887189381481633  \\
            1.053  3.6677504345750482  \\
            1.054  3.64692086056674  \\
            1.055  3.626229165122983  \\
            1.056  3.6056743061797074  \\
            1.057  3.585255250524715  \\
            1.058  3.564970973714177  \\
            1.059  3.5448204599900683  \\
            1.06  3.524802702198382  \\
            1.061  3.504916701708261  \\
            1.062  3.485161468331899  \\
            1.063  3.465536020245334  \\
            1.064  3.4460393839099885  \\
            1.065  3.426670593995093  \\
            1.066  3.407428693300835  \\
            1.067  3.388312732682364  \\
            1.068  3.3693217709745205  \\
            1.069  3.3504548749173884  \\
            1.07  3.3317111190825615  \\
            1.071  3.3130895858002196  \\
            1.072  3.2945893650868907  \\
            1.073  3.276209554574015  \\
            1.074  3.2579492594371775  \\
            1.075  3.2398075923261223  \\
            1.076  3.221783673295422  \\
            1.077  3.2038766297359134  \\
            1.078  3.1860855963067696  \\
            1.079  3.1684097148683255  \\
            1.08  3.1508481344155226  \\
            1.081  3.1334000110120948  \\
            1.082  3.1160645077253553  \\
            1.083  3.098840794561709  \\
            1.084  3.0817280484027565  \\
            1.085  3.0647254529421026  \\
            1.086  3.047832198622747  \\
            1.087  3.031047482575163  \\
            1.088  3.014370508555946  \\
            1.089  2.997800486887132  \\
            1.09  2.981336634396079  \\
            1.091  2.964978174356  \\
            1.092  2.9487243364270443  \\
            1.093  2.9325743565980247  \\
            1.094  2.9165274771286698  \\
            1.095  2.900582946492518  \\
            1.096  2.884740019320321  \\
            1.097  2.8689979563440717  \\
            1.098  2.8533560243415366  \\
            1.099  2.8378134960813983  \\
            1.1  2.8223696502688855  \\
            1.101  2.8070237714920063  \\
            1.102  2.7917751501682577  \\
            1.103  2.7766230824919207  \\
            1.104  2.7615668703818286  \\
            1.105  2.7466058214297013  \\
            1.106  2.7317392488489443  \\
            1.107  2.7169664714240085  \\
            1.108  2.7022868134601934  \\
            1.109  2.6876996047340067  \\
            1.11  2.6732041804439572  \\
            1.111  2.6587998811618876  \\
            1.112  2.6444860527847354  \\
            1.113  2.6302620464868234  \\
            1.114  2.6161272186725664  \\
            1.115  2.6020809309296937  \\
            1.116  2.5881225499828835  \\
            1.117  2.574251447647902  \\
            1.118  2.5604670007861428  \\
            1.119  2.5467685912596623  \\
            1.12  2.533155605886603  \\
            1.121  2.51962743639711  \\
            1.122  2.5061834793896214  \\
            1.123  2.4928231362876407  \\
            1.124  2.4795458132968777  \\
            1.125  2.4663509213628605  \\
            1.126  2.453237876128914  \\
            1.127  2.440206097894575  \\
            1.128  2.4272550115744167  \\
            1.129  2.414384046657238  \\
            1.13  2.401592637165702  \\
            1.131  2.388880221616306  \\
            1.132  2.376246242979803  \\
            1.133  2.363690148641944  \\
            1.134  2.35121139036466  \\
            1.135  2.338809424247561  \\
            1.136  2.326483710689865  \\
            1.137  2.3142337143526386  \\
            1.138  2.3020589041214548  \\
            1.139  2.28995875306936  \\
            1.14  2.277932738420245  \\
            1.141  2.265980341512524  \\
            1.142  2.2541010477632057  \\
            1.143  2.2422943466322685  \\
            1.144  2.2305597315874146  \\
            1.145  2.2188967000691266  \\
            1.146  2.207304753456096  \\
            1.147  2.1957833970309424  \\
            1.148  2.1843321399463025  \\
            1.149  2.1729504951912006  \\
            1.15  2.1616379795577827  \\
            1.151  2.150394113608323  \\
            1.152  2.139218421642592  \\
            1.153  2.12811043166549  \\
            1.154  2.117069675355028  \\
            1.155  2.1060956880305772  \\
            1.156  2.0951880086214563  \\
            1.157  2.084346179635777  \\
            1.158  2.073569747129627  \\
            1.159  2.062858260676505  \\
            1.16  2.05221127333708  \\
            1.161  2.0416283416292016  \\
            1.162  2.031109025498234  \\
            1.163  2.0206528882876245  \\
            1.164  2.010259496709796  \\
            1.165  1.9999284208172685  \\
            1.166  1.9896592339740966  \\
            1.167  1.9794515128275323  \\
            1.168  1.969304837279996  \\
            1.169  1.9592187904612715  \\
            1.17  1.9491929587009984  \\
            1.171  1.9392269315013864  \\
            1.172  1.929320301510222  \\
            1.173  1.9194726644940905  \\
            1.174  1.909683619311885  \\
            1.175  1.8999527678885275  \\
            1.176  1.8902797151889699  \\
            1.177  1.8806640691924026  \\
            1.178  1.8711054408667382  \\
            1.179  1.8616034441432994  \\
            1.18  1.852157695891774  \\
            1.181  1.842767815895372  \\
            1.182  1.8334334268262436  \\
            1.183  1.8241541542210957  \\
            1.184  1.8149296264570638  \\
            1.185  1.8057594747277774  \\
            1.186  1.7966433330196776  \\
            1.187  1.7875808380885232  \\
            1.188  1.778571629436144  \\
            1.189  1.7696153492873785  \\
            1.19  1.7607116425672544  \\
            1.191  1.751860156878351  \\
            1.192  1.7430605424783954  \\
            1.193  1.7343124522580422  \\
            1.194  1.7256155417188794  \\
            1.195  1.7169694689516135  \\
            1.196  1.7083738946144793  \\
            1.197  1.6998284819118208  \\
            1.198  1.691332896572898  \\
            1.199  1.6828868068308567  \\
            1.2  1.6744898834019208  \\
            1.201  1.6661417994647465  \\
            1.202  1.657842230639993  \\
            1.203  1.6495908549700493  \\
            1.204  1.6413873528989797  \\
            1.205  1.6332314072526166  \\
            1.206  1.6251227032188678  \\
            1.207  1.6170609283281732  \\
            1.208  1.6090457724341665  \\
            1.209  1.6010769276944856  \\
            1.21  1.5931540885517845  \\
            1.211  1.5852769517148888  \\
            1.212  1.5774452161401518  \\
            1.213  1.5696585830129477  \\
            1.214  1.5619167557293656  \\
            1.215  1.554219439878035  \\
            1.216  1.5465663432221466  \\
            1.217  1.5389571756816083  \\
            1.218  1.5313916493153874  \\
            1.219  1.5238694783039834  \\
            1.22  1.5163903789320918  \\
            1.221  1.508954069571388  \\
            1.222  1.5015602706635003  \\
            1.223  1.4942087047031047  \\
            1.224  1.486899096221199  \\
            1.225  1.4796311717685016  \\
            1.226  1.4724046598990221  \\
            1.227  1.465219291153757  \\
            1.228  1.4580747980445534  \\
            1.229  1.450970915038094  \\
            1.23  1.4439073785400462  \\
            1.231  1.4368839268793343  \\
            1.232  1.4299003002925694  \\
            1.233  1.422956240908598  \\
            1.234  1.4160514927332106  \\
            1.235  1.4091858016339633  \\
            1.236  1.4023589153251566  \\
            1.237  1.3955705833529235  \\
            1.238  1.3888205570804777  \\
            1.239  1.3821085896734613  \\
            1.24  1.3754344360854545  \\
            1.241  1.3687978530435818  \\
            1.242  1.3621985990342755  \\
            1.243  1.3556364342891378  \\
            1.244  1.3491111207709507  \\
            1.245  1.3426224221597898  \\
            1.246  1.3361701038392784  \\
            1.247  1.3297539328829444  \\
            1.248  1.3233736780407162  \\
            1.249  1.3170291097255173  \\
            1.25  1.3107199999999999  \\
            1.251  1.3044461225633743  \\
            1.252  1.2982072527383677  \\
            1.253  1.292003167458296  \\
            1.254  1.2858336452542383  \\
            1.255  1.2796984662423396  \\
            1.256  1.2735974121112075  \\
            1.257  1.267530266109436  \\
            1.258  1.2614968130332207  \\
            1.259  1.2554968392141  \\
            1.26  1.2495301325067845  \\
            1.261  1.2435964822771117  \\
            1.262  1.2376956793900862  \\
            1.263  1.2318275161980423  \\
            1.264  1.225991786528891  \\
            1.265  1.2201882856744886  \\
            1.266  1.214416810379085  \\
            1.267  1.2086771588278948  \\
            1.268  1.2029691306357437  \\
            1.269  1.1972925268358363  \\
            1.27  1.1916471498685994  \\
            1.271  1.1860328035706413  \\
            1.272  1.1804492931637887  \\
            1.273  1.1748964252442349  \\
            1.274  1.1693740077717636  \\
            1.275  1.1638818500590862  \\
            1.276  1.1584197627612476  \\
            1.277  1.1529875578651472  \\
            1.278  1.1475850486791261  \\
            1.279  1.142212049822667  \\
            1.28  1.13686837721616  \\
            1.281  1.1315538480707772  \\
            1.282  1.1262682808784135  \\
            1.283  1.1210114954017345  \\
            1.284  1.1157833126642882  \\
            1.285  1.1105835549407224  \\
            1.286  1.105412045747064  \\
            1.287  1.1002686098311028  \\
            1.288  1.0951530731628374  \\
            1.289  1.0900652629250198  \\
            1.29  1.0850050075037645  \\
            1.291  1.0799721364792543  \\
            1.292  1.0749664806165062  \\
            1.293  1.069987871856236  \\
            1.294  1.0650361433057793  \\
            1.295  1.0601111292301115  \\
            1.296  1.0552126650429217  \\
            1.297  1.050340587297783  \\
            1.298  1.0454947336793783  \\
            1.299  1.0406749429948188  \\
            1.3  1.0358810551650166  \\
            1.301  1.0311129112161506  \\
            1.302  1.0263703532711834  \\
            1.303  1.0216532245414696  \\
            1.304  1.016961369318416  \\
            1.305  1.0122946329652311  \\
            1.306  1.0076528619087237  \\
            1.307  1.0030359036311904  \\
            1.308  0.998443606662353  \\
            1.309  0.99387582057138  \\
            1.31  0.9893323959589593  \\
            1.311  0.9848131844494522  \\
            1.312  0.9803180386830991  \\
            1.313  0.9758468123083044  \\
            1.314  0.9713993599739706  \\
            1.315  0.9669755373219125  \\
            1.316  0.9625752009793188  \\
            1.317  0.9581982085512919  \\
            1.318  0.9538444186134348  \\
            1.319  0.949513690704515  \\
            1.32  0.9452058853191734  \\
            1.321  0.9409208639007098  \\
            1.322  0.9366584888339138  \\
            1.323  0.9324186234379682  \\
            1.324  0.9282011319593984  \\
            1.325  0.9240058795650931  \\
            1.326  0.9198327323353708  \\
            1.327  0.9156815572571154  \\
            1.328  0.9115522222169553  \\
            1.329  0.9074445959945139  \\
            1.33  0.9033585482557001  \\
            1.331  0.8992939495460694  \\
            1.332  0.8952506712842258  \\
            1.333  0.8912285857552906  \\
            1.334  0.8872275661044127  \\
            1.335  0.8832474863303451  \\
            1.336  0.8792882212790608  \\
            1.337  0.8753496466374349  \\
            1.338  0.8714316389269643  \\
            1.339  0.8675340754975538  \\
            1.34  0.8636568345213379  \\
            1.341  0.8597997949865676  \\
            1.342  0.8559628366915341  \\
            1.343  0.852145840238555  \\
            1.344  0.8483486870279959  \\
            1.345  0.8445712592523555  \\
            1.346  0.8408134398903842  \\
            1.347  0.8370751127012641  \\
            1.348  0.8333561622188248  \\
            1.349  0.8296564737458174  \\
            1.35  0.8259759333482228  \\
            1.351  0.8223144278496193  \\
            1.352  0.8186718448255831  \\
            1.353  0.8150480725981473  \\
            1.354  0.8114430002302918  \\
            1.355  0.8078565175204926  \\
            1.356  0.8042885149973005  \\
            1.357  0.8007388839139779  \\
            1.358  0.7972075162431653  \\
            1.359  0.7936943046716043  \\
            1.36  0.7901991425948898  \\
            1.361  0.7867219241122778  \\
            1.362  0.7832625440215218  \\
            1.363  0.7798208978137646  \\
            1.364  0.7763968816684573  \\
            1.365  0.7729903924483316  \\
            1.366  0.7696013276944018  \\
            1.367  0.7662295856210177  \\
            1.368  0.7628750651109459  \\
            1.369  0.7595376657105009  \\
            1.37  0.7562172876247062  \\
            1.371  0.7529138317125034  \\
            1.372  0.749627199481989  \\
            1.373  0.7463572930857009  \\
            1.374  0.7431040153159313  \\
            1.375  0.7398672696000872  \\
            1.376  0.736646959996079  \\
            1.377  0.7334429911877517  \\
            1.378  0.730255268480353  \\
            1.379  0.7270836977960305  \\
            1.38  0.7239281856693748  \\
            1.381  0.7207886392429873  \\
            1.382  0.7176649662630943  \\
            1.383  0.7145570750751851  \\
            1.384  0.7114648746196941  \\
            1.385  0.7083882744277092  \\
            1.386  0.7053271846167222  \\
            1.387  0.7022815158864032  \\
            1.388  0.6992511795144195  \\
            1.389  0.6962360873522766  \\
            1.39  0.6932361518212022  \\
            1.391  0.6902512859080528  \\
            1.392  0.6872814031612627  \\
            1.393  0.6843264176868145  \\
            1.394  0.6813862441442518  \\
            1.395  0.6784607977427133  \\
            1.396  0.6755499942370081  \\
            1.397  0.6726537499237111  \\
            1.398  0.6697719816372993  \\
            1.399  0.6669046067463105  \\
            1.4  0.6640515431495384  \\
            1.401  0.6612127092722504  \\
            1.402  0.6583880240624447  \\
            1.403  0.655577406987125  \\
            1.404  0.6527807780286159  \\
            1.405  0.649998057680897  \\
            1.406  0.6472291669459751  \\
            1.407  0.6444740273302746  \\
            1.408  0.6417325608410664  \\
            1.409  0.6390046899829155  \\
            1.41  0.6362903377541638  \\
            1.411  0.6335894276434326  \\
            1.412  0.6309018836261621  \\
            1.413  0.6282276301611662  \\
            1.414  0.6255665921872258  \\
            1.415  0.622918695119697  \\
            1.416  0.6202838648471574  \\
            1.417  0.6176620277280662  \\
            1.418  0.6150531105874628  \\
            1.419  0.6124570407136782  \\
            1.42  0.6098737458550838  \\
            1.421  0.6073031542168542  \\
            1.422  0.6047451944577662  \\
            1.423  0.6021997956870109  \\
            1.424  0.5996668874610417  \\
            1.425  0.5971463997804363  \\
            1.426  0.5946382630867918  \\
            1.427  0.5921424082596358  \\
            1.428  0.5896587666133692  \\
            1.429  0.5871872698942234  \\
            1.43  0.584727850277251  \\
            1.431  0.5822804403633295  \\
            1.432  0.5798449731761972  \\
            1.433  0.5774213821595033  \\
            1.434  0.5750096011738893  \\
            1.435  0.5726095644940831  \\
            1.436  0.5702212068060256  \\
            1.437  0.5678444632040094  \\
            1.438  0.565479269187848  \\
            1.439  0.5631255606600587  \\
            1.44  0.5607832739230755  \\
            1.441  0.5584523456764736  \\
            1.442  0.5561327130142246  \\
            1.443  0.5538243134219643  \\
            1.444  0.5515270847742886  \\
            1.445  0.5492409653320627  \\
            1.446  0.546965893739758  \\
            1.447  0.5447018090228024  \\
            1.448  0.542448650584957  \\
            1.449  0.5402063582057064  \\
            1.45  0.5379748720376752  \\
            1.451  0.535754132604058  \\
            1.452  0.5335440807960741  \\
            1.453  0.531344657870436  \\
            1.454  0.5291558054468444  \\
            1.455  0.5269774655054924  \\
            1.456  0.5248095803845988  \\
            1.457  0.522652092777951  \\
            1.458  0.5205049457324741  \\
            1.459  0.5183680826458107  \\
            1.46  0.5162414472639272  \\
            1.461  0.514124983678729  \\
            1.462  0.5120186363257033  \\
            1.463  0.5099223499815699  \\
            1.464  0.5078360697619584  \\
            1.465  0.5057597411190953  \\
            1.466  0.5036933098395151  \\
            1.467  0.5016367220417823  \\
            1.468  0.49958992417423653  \\
            1.469  0.49755286301274804  \\
            1.47  0.49552548565849724  \\
            1.471  0.4935077395357629  \\
            1.472  0.49149957238973496  \\
            1.473  0.48950093228433567  \\
            1.474  0.48751176760006476  \\
            1.475  0.48553202703185294  \\
            1.476  0.483561659586939  \\
            1.477  0.48160061458275527  \\
            1.478  0.479648841644836  \\
            1.479  0.4777062907047341  \\
            1.48  0.47577291199796046  \\
            1.481  0.4738486560619317  \\
            1.482  0.47193347373393973  \\
            1.483  0.4700273161491296  \\
            1.484  0.4681301347384996  \\
            1.485  0.4662418812269082  \\
            1.486  0.46436250763110415  \\
            1.487  0.46249196625776257  \\
            1.488  0.4606302097015438  \\
            1.489  0.4587771908431588  \\
            1.49  0.45693286284745643  \\
            1.491  0.4550971791615168  \\
            1.492  0.45327009351276704  \\
            1.493  0.45145155990710256  \\
            1.494  0.4496415326270303  \\
            1.495  0.44783996622981775  \\
            1.496  0.44604681554566294  \\
            1.497  0.44426203567587086  \\
            1.498  0.44248558199105004  \\
            1.499  0.44071741012931587  \\
            1.5  0.438957475994513  \\
            1.501  0.4372057357544459  \\
            1.502  0.43546214583912446  \\
            1.503  0.43372666293902484  \\
            1.504  0.43199924400335654  \\
            1.505  0.43027984623835025  \\
            1.506  0.42856842710555015  \\
            1.507  0.42686494432012606  \\
            1.508  0.4251693558491912  \\
            1.509  0.42348161991013833  \\
            1.51  0.421801694968982  \\
            1.511  0.42012953973871897  \\
            1.512  0.4184651131776943  \\
            1.513  0.41680837448798547  \\
            1.514  0.41515928311379224  \\
            1.515  0.41351779873984396  \\
            1.516  0.4118838812898129  \\
            1.517  0.41025749092474423  \\
            1.518  0.40863858804149245  \\
            1.519  0.4070271332711741  \\
            1.52  0.40542308747762634  \\
            1.521  0.4038264117558823  \\
            1.522  0.4022370674306516  \\
            1.523  0.40065501605481774  \\
            1.524  0.39908021940794036  \\
            1.525  0.3975126394947743  \\
            1.526  0.39595223854379336  \\
            1.527  0.3943989790057304  \\
            1.528  0.39285282355212225  \\
            1.529  0.39131373507387107  \\
            1.53  0.3897816766798099  \\
            1.531  0.38825661169528425  \\
            1.532  0.3867385036607386  \\
            1.533  0.3852273163303182  \\
            1.534  0.3837230136704753  \\
            1.535  0.3822255598585916  \\
            1.536  0.3807349192816037  \\
            1.537  0.37925105653464547  \\
            1.538  0.37777393641969326  \\
            1.539  0.376303523944227  \\
            1.54  0.37483978431989523  \\
            1.541  0.37338268296119503  \\
            1.542  0.3719321854841559  \\
            1.543  0.37048825770503857  \\
            1.544  0.36905086563903744  \\
            1.545  0.3676199754989979  \\
            1.546  0.36619555369413714  \\
            1.547  0.36477756682877954  \\
            1.548  0.3633659817010957  \\
            1.549  0.36196076530185556  \\
            1.55  0.3605618848131853  \\
            1.551  0.3591693076073382  \\
            1.552  0.35778300124546886  \\
            1.553  0.3564029334764214  \\
            1.554  0.35502907223552127  \\
            1.555  0.3536613856433802  \\
            1.556  0.35229984200470477  \\
            1.557  0.35094440980711894  \\
            1.558  0.3495950577199885  \\
            1.559  0.34825175459326074  \\
            1.56  0.3469144694563054  \\
            1.561  0.34558317151677  \\
            1.562  0.3442578301594376  \\
            1.563  0.3429384149450981  \\
            1.564  0.3416248956094219  \\
            1.565  0.3403172420618467  \\
            1.566  0.33901542438446775  \\
            1.567  0.3377194128309393  \\
            1.568  0.3364291778253804  \\
            1.569  0.33514468996129265  \\
            1.57  0.3338659200004803  \\
            1.571  0.33259283887198315  \\
            1.572  0.33132541767101203  \\
            1.573  0.33006362765789665  \\
            1.574  0.32880744025703557  \\
            1.575  0.32755682705585853  \\
            1.576  0.32631175980379173  \\
            1.577  0.3250722104112338  \\
            1.578  0.3238381509485356  \\
            1.579  0.32260955364499083  \\
            1.58  0.3213863908878295  \\
            1.581  0.32016863522122285  \\
            1.582  0.31895625934529026  \\
            1.583  0.3177492361151185  \\
            1.584  0.3165475385397823  \\
            1.585  0.3153511397813765  \\
            1.586  0.31416001315405084  \\
            1.587  0.31297413212305547  \\
            1.588  0.3117934703037885  \\
            1.589  0.3106180014608551  \\
            1.59  0.3094476995071282  \\
            1.591  0.30828253850282  \\
            1.592  0.30712249265455627  \\
            1.593  0.30596753631446044  \\
            1.594  0.30481764397924016  \\
            1.595  0.30367279028928457  \\
            1.596  0.30253295002776304  \\
            1.597  0.3013980981197349  \\
            1.598  0.30026820963126055  \\
            1.599  0.29914325976852335  \\
            1.6  0.298023223876953  \\
            1.601  0.29690807744035935  \\
            1.602  0.2957977960800676  \\
            1.603  0.29469235555406426  \\
            1.604  0.29359173175614406  \\
            1.605  0.29249590071506726  \\
            1.606  0.2914048385937187  \\
            1.607  0.29031852168827627  \\
            1.608  0.28923692642738136  \\
            1.609  0.28816002937131885  \\
            1.61  0.28708780721119875  \\
            1.611  0.28602023676814736  \\
            1.612  0.2849572949925  \\
            1.613  0.28389895896300305  \\
            1.614  0.28284520588601786  \\
            1.615  0.28179601309473346  \\
            1.616  0.2807513580483813  \\
            1.617  0.2797112183314586  \\
            1.618  0.2786755716529537  \\
            1.619  0.27764439584558  \\
            1.62  0.2766176688650116  \\
            1.621  0.2755953687891278  \\
            1.622  0.27457747381725905  \\
            1.623  0.27356396226944146  \\
            1.624  0.272554812585673  \\
            1.625  0.2715500033251782  \\
            1.626  0.2705495131656743  \\
            1.627  0.26955332090264505  \\
            1.628  0.26856140544861884  \\
            1.629  0.26757374583244975  \\
            1.63  0.26659032119860704  \\
            1.631  0.2656111108064652  \\
            1.632  0.26463609402960314  \\
            1.633  0.26366525035510374  \\
            1.634  0.26269855938286235  \\
            1.635  0.26173600082489595  \\
            1.636  0.26077755450466056  \\
            1.637  0.25982320035637  \\
            1.638  0.25887291842432236  \\
            1.639  0.25792668886222736  \\
            1.64  0.2569844919325425  \\
            1.641  0.25604630800580924  \\
            1.642  0.2551121175599976  \\
            1.643  0.2541819011798515  \\
            1.644  0.25325563955624236  \\
            1.645  0.2523333134855226  \\
            1.646  0.25141490386888793  \\
            1.647  0.25050039171174  \\
            1.648  0.24958975812305648  \\
            1.649  0.2486829843147626  \\
            1.65  0.2477800516011095  \\
            1.651  0.24688094139805394  \\
            1.652  0.24598563522264505  \\
            1.653  0.24509411469241232  \\
            1.654  0.24420636152476025  \\
            1.655  0.24332235753636453  \\
            1.656  0.242442084642575  \\
            1.657  0.2415655248568193  \\
            1.658  0.24069266029001396  \\
            1.659  0.23982347314997599  \\
            1.66  0.23895794574084167  \\
            1.661  0.23809606046248588  \\
            1.662  0.23723779980994858  \\
            1.663  0.23638314637286173  \\
            1.664  0.2355320828348834  \\
            1.665  0.23468459197313216  \\
            1.666  0.23384065665762865  \\
            1.667  0.23300025985073752  \\
            1.668  0.23216338460661617  \\
            1.669  0.23133001407066392  \\
            1.67  0.23050013147897824  \\
            1.671  0.22967372015781112  \\
            1.672  0.2288507635230322  \\
            1.673  0.2280312450795924  \\
            1.674  0.22721514842099405  \\
            1.675  0.22640245722876165  \\
            1.676  0.22559315527191864  \\
            1.677  0.22478722640646537  \\
            1.678  0.22398465457486252  \\
            1.679  0.223185423805516  \\
            1.68  0.2223895182122671  \\
            1.681  0.22159692199388378  \\
            1.682  0.22080761943355792  \\
            1.683  0.22002159489840312  \\
            1.684  0.21923883283895815  \\
            1.685  0.21845931778869165  \\
            1.686  0.21768303436351208  \\
            1.687  0.21690996726127837  \\
            1.688  0.21614010126131672  \\
            1.689  0.21537342122393738  \\
            1.69  0.21460991208995778  \\
            1.691  0.21384955888022553  \\
            1.692  0.2130923466951476  \\
            1.693  0.21233826071421974  \\
            1.694  0.21158728619556166  \\
            1.695  0.21083940847545238  \\
            1.696  0.21009461296787163  \\
            1.697  0.2093528851640411  \\
            1.698  0.20861421063197164  \\
            1.699  0.20787857501601062  \\
            1.7  0.2071459640363949  \\
            1.701  0.206416363488804  \\
            1.702  0.2056897592439187  \\
            1.703  0.20496613724698018  \\
            1.704  0.2042454835173543  \\
            1.705  0.20352778414809608  \\
            1.706  0.20281302530551992  \\
            1.707  0.20210119322876957  \\
            1.708  0.20139227422939363  \\
            1.709  0.2006862546909215  \\
            1.71  0.1999831210684439  \\
            1.711  0.19928285988819439  \\
            1.712  0.19858545774713535  \\
            1.713  0.19789090131254466  \\
            1.714  0.1971991773216072  \\
            1.715  0.19651027258100653  \\
            1.716  0.1958241739665219  \\
            1.717  0.19514086842262499  \\
            1.718  0.19446034296208198  \\
            1.719  0.19378258466555556  \\
            1.72  0.19310758068121203  \\
            1.721  0.1924353182243284  \\
            1.722  0.1917657845769044  \\
            1.723  0.19109896708727458  \\
            1.724  0.19043485316972553  \\
            1.725  0.18977343030411242  \\
            1.726  0.18911468603548143  \\
            1.727  0.18845860797369118  \\
            1.728  0.18780518379303957  \\
            1.729  0.18715440123189045  \\
            1.73  0.18650624809230504  \\
            1.731  0.1858607122396731  \\
            1.732  0.18521778160234928  \\
            1.733  0.1845774441712889  \\
            1.734  0.1839396879996888  \\
            1.735  0.18330450120262784  \\
            1.736  0.18267187195671214  \\
            1.737  0.18204178849972022  \\
            1.738  0.1814142391302527  \\
            1.739  0.18078921220738192  \\
            1.74  0.18016669615030598  \\
            1.741  0.17954667943800273  \\
            1.742  0.17892915060888845  \\
            1.743  0.1783140982604758  \\
            1.744  0.17770151104903698  \\
            1.745  0.17709137768926614  \\
            1.746  0.17648368695394634  \\
            1.747  0.17587842767361656  \\
            1.748  0.1752755887362428  \\
            1.749  0.17467515908688896  \\
            1.75  0.1740771277273925  \\
            1.751  0.17348148371603922  \\
            1.752  0.17288821616724237  \\
            1.753  0.17229731425122308  \\
            1.754  0.17170876719369177  \\
            1.755  0.17112256427553346  \\
            1.756  0.17053869483249307  \\
            1.757  0.16995714825486452  \\
            1.758  0.16937791398717988  \\
            1.759  0.16880098152790265  \\
            1.76  0.1682263404291205  \\
            1.761  0.16765398029624232  \\
            1.762  0.16708389078769506  \\
            1.763  0.16651606161462443  \\
            1.764  0.16595048254059538  \\
            1.765  0.16538714338129662  \\
            1.766  0.16482603400424484  \\
            1.767  0.1642671443284928  \\
            1.768  0.16371046432433728  \\
            1.769  0.16315598401303077  \\
            1.77  0.16260369346649314  \\
            1.771  0.16205358280702664  \\
            1.772  0.16150564220703134  \\
            1.773  0.16095986188872372  \\
            1.774  0.1604162321238553  \\
            1.775  0.15987474323343506  \\
            1.776  0.15933538558745136  \\
            1.777  0.15879814960459776  \\
            1.778  0.15826302575199844  \\
            1.779  0.1577300045449374  \\
            1.78  0.15719907654658724  \\
            1.781  0.15667023236774164  \\
            1.782  0.15614346266654758  \\
            1.783  0.15561875814824092  \\
            1.784  0.15509610956488212  \\
            1.785  0.15457550771509507  \\
            1.786  0.1540569434438058  \\
            1.787  0.15354040764198473  \\
            1.788  0.15302589124638857  \\
            1.789  0.15251338523930566  \\
            1.79  0.15200288064830098  \\
            1.791  0.15149436854596474  \\
            1.792  0.15098784004966045  \\
            1.793  0.15048328632127653  \\
            1.794  0.14998069856697754  \\
            1.795  0.1494800680369588  \\
            1.796  0.1489813860252005  \\
            1.797  0.1484846438692256  \\
            1.798  0.14798983294985685  \\
            1.799  0.14749694469097727  \\
            1.8  0.14700597055929066  \\
            1.801  0.1465169020640848  \\
            1.802  0.1460297307569947  \\
            1.803  0.1455444482317689  \\
            1.804  0.14506104612403545  \\
            1.805  0.14457951611107112  \\
            1.806  0.14409984991157018  \\
            1.807  0.14362203928541653  \\
            1.808  0.14314607603345514  \\
            1.809  0.14267195199726687  \\
            1.81  0.1421996590589429  \\
            1.811  0.14172918914086222  \\
            1.812  0.1412605342054686  \\
            1.813  0.14079368625505095  \\
            1.814  0.14032863733152295  \\
            1.815  0.13986537951620603  \\
            1.816  0.13940390492961172  \\
            1.817  0.1389442057312271  \\
            1.818  0.13848627411929995  \\
            1.819  0.13803010233062674  \\
            1.82  0.1375756826403402  \\
            1.821  0.13712300736170013  \\
            1.822  0.1366720688458834  \\
            1.823  0.13622285948177712  \\
            1.824  0.1357753716957714  \\
            1.825  0.13532959795155491  \\
            1.826  0.13488553074991014  \\
            1.827  0.13444316262851136  \\
            1.828  0.13400248616172236  \\
            1.829  0.1335634939603969  \\
            1.83  0.13312617867167878  \\
            1.831  0.1326905329788047  \\
            1.832  0.13225654960090674  \\
            1.833  0.13182422129281757  \\
            1.834  0.1313935408448753  \\
            1.835  0.13096450108273106  \\
            1.836  0.13053709486715598  \\
            1.837  0.13011131509385124  \\
            1.838  0.12968715469325748  \\
            1.839  0.12926460663036682  \\
            1.84  0.1288436639045346  \\
            1.841  0.12842431954929392  \\
            1.842  0.12800656663216925  \\
            1.843  0.12759039825449334  \\
            1.844  0.1271758075512231  \\
            1.845  0.12676278769075852  \\
            1.846  0.12635133187476102  \\
            1.847  0.1259414333379742  \\
            1.848  0.12553308534804447  \\
            1.849  0.12512628120534397  \\
            1.85  0.1247210142427933  \\
            1.851  0.12431727782568654  \\
            1.852  0.123915065351516  \\
            1.853  0.1235143702497994  \\
            1.854  0.1231151859819067  \\
            1.855  0.12271750604088923  \\
            1.856  0.12232132395130862  \\
            1.857  0.121926633269068  \\
            1.858  0.12153342758124282  \\
            1.859  0.12114170050591413  \\
            1.86  0.12075144569200144  \\
            1.861  0.12036265681909783  \\
            1.862  0.1199753275973048  \\
            1.863  0.11958945176706946  \\
            1.864  0.1192050230990212  \\
            1.865  0.11882203539381078  \\
            1.866  0.118440482481949  \\
            1.867  0.11806035822364758  \\
            1.868  0.11768165650865978  \\
            1.869  0.11730437125612321  \\
            1.87  0.11692849641440223  \\
            1.871  0.11655402596093253  \\
            1.872  0.11618095390206556  \\
            1.873  0.11580927427291485  \\
            1.874  0.11543898113720212  \\
            1.875  0.11507006858710563  \\
            1.876  0.114702530743108  \\
            1.877  0.11433636175384601  \\
            1.878  0.113971555795961  \\
            1.879  0.11360810707394949  \\
            1.88  0.11324600982001595  \\
            1.881  0.1128852582939249  \\
            1.882  0.1125258467828554  \\
            1.883  0.11216776960125503  \\
            1.884  0.1118110210906959  \\
            1.885  0.11145559561973038  \\
            1.886  0.11110148758374883  \\
            1.887  0.11074869140483694  \\
            1.888  0.11039720153163518  \\
            1.889  0.1100470124391977  \\
            1.89  0.10969811862885355  \\
            1.891  0.10935051462806704  \\
            1.892  0.1090041949903007  \\
            1.893  0.10865915429487726  \\
            1.894  0.10831538714684402  \\
            1.895  0.10797288817683671  \\
            1.896  0.10763165204094519  \\
            1.897  0.10729167342057896  \\
            1.898  0.10695294702233447  \\
            1.899  0.10661546757786206  \\
            1.9  0.10627922984373492  \\
            1.901  0.1059442286013175  \\
            1.902  0.10561045865663599  \\
            1.903  0.10527791484024823  \\
            1.904  0.10494659200711576  \\
            1.905  0.10461648503647511  \\
            1.906  0.10428758883171137  \\
            1.907  0.10395989832023099  \\
            1.908  0.10363340845333677  \\
            1.909  0.10330811420610216  \\
            1.91  0.10298401057724757  \\
            1.911  0.10266109258901629  \\
            1.912  0.10233935528705207  \\
            1.913  0.10201879374027648  \\
            1.914  0.10169940304076801  \\
            1.915  0.10138117830364066  \\
            1.916  0.10106411466692457  \\
            1.917  0.10074820729144592  \\
            1.918  0.10043345136070891  \\
            1.919  0.10011984208077712  \\
            1.92  0.09980737468015674  \\
            1.921  0.09949604440967937  \\
            1.922  0.09918584654238645  \\
            1.923  0.09887677637341352  \\
            1.924  0.09856882921987602  \\
            1.925  0.09826200042075461  \\
            1.926  0.09795628533678255  \\
            1.927  0.0976516793503322  \\
            1.928  0.0973481778653036  \\
            1.929  0.09704577630701247  \\
            1.93  0.09674447012207987  \\
            1.931  0.09644425477832148  \\
            1.932  0.0961451257646387  \\
            1.933  0.09584707859090895  \\
            1.934  0.09555010878787808  \\
            1.935  0.09525421190705202  \\
            1.936  0.09495938352059032  \\
            1.937  0.094665619221199  \\
            1.938  0.09437291462202527  \\
            1.939  0.09408126535655174  \\
            1.94  0.09379066707849221  \\
            1.941  0.09350111546168707  \\
            1.942  0.0932126062000003  \\
            1.943  0.09292513500721608  \\
            1.944  0.09263869761693691  \\
            1.945  0.09235328978248135  \\
            1.946  0.09206890727678332  \\
            1.947  0.0917855458922911  \\
            1.948  0.0915032014408676  \\
            1.949  0.09122186975369048  \\
            1.95  0.09094154668115376  \\
            1.951  0.09066222809276883  \\
            1.952  0.0903839098770673  \\
            1.953  0.09010658794150307  \\
            1.954  0.08983025821235621  \\
            1.955  0.08955491663463627  \\
            1.956  0.08928055917198717  \\
            1.957  0.08900718180659155  \\
            1.958  0.08873478053907673  \\
            1.959  0.08846335138842018  \\
            1.96  0.08819289039185656  \\
            1.961  0.08792339360478418  \\
            1.962  0.087654857100673  \\
            1.963  0.08738727697097232  \\
            1.964  0.08712064932501973  \\
            1.965  0.08685497028994979  \\
            1.966  0.08659023601060396  \\
            1.967  0.08632644264944041  \\
            1.968  0.08606358638644494  \\
            1.969  0.08580166341904172  \\
            1.97  0.08554066996200521  \\
            1.971  0.08528060224737191  \\
            1.972  0.08502145652435333  \\
            1.973  0.08476322905924864  \\
            1.974  0.08450591613535861  \\
            1.975  0.08424951405289918  \\
            1.976  0.08399401912891652  \\
            1.977  0.08373942769720143  \\
            1.978  0.0834857361082053  \\
            1.979  0.08323294072895561  \\
            1.98  0.08298103794297271  \\
            1.981  0.08273002415018627  \\
            1.982  0.08247989576685297  \\
            1.983  0.0822306492254739  \\
            1.984  0.08198228097471323  \\
            1.985  0.08173478747931633  \\
            1.986  0.08148816522002951  \\
            1.987  0.08124241069351898  \\
            1.988  0.08099752041229148  \\
            1.989  0.08075349090461417  \\
            1.99  0.08051031871443604  \\
            1.991  0.08026800040130873  \\
            1.992  0.08002653254030886  \\
            1.993  0.07978591172195972  \\
            1.994  0.07954613455215426  \\
            1.995  0.07930719765207792  \\
            1.996  0.07906909765813233  \\
            1.997  0.07883183122185887  \\
            1.998  0.07859539500986347  \\
            1.999  0.07835978570374082  \\
            2.0  0.078125  \\
        }
        ;
    \addlegendentry {$\beta = 5.0$}
    \addplot[color={rgb,1:red,0.0;green,0.6;blue,0.5333}, name path={98c48795-8364-4759-907f-2eb16f5c5eaf}, draw opacity={1.0}, line width={1.2}, dashdotted]
        table[row sep={\\}]
        {
            \\
            1.0  10.0  \\
            1.001  9.890657149980061  \\
            1.002  9.782617279204139  \\
            1.003  9.675863583570678  \\
            1.004  9.57037949213413  \\
            1.005  9.466148663642485  \\
            1.006  9.363154983129217  \\
            1.007  9.261382558559456  \\
            1.008  9.160815717528816  \\
            1.009  9.061439004014629  \\
            1.01  8.963237175178051  \\
            1.011  8.866195198216888  \\
            1.012  8.770298247267563  \\
            1.013  8.675531700356153  \\
            1.014  8.581881136396914  \\
            1.015  8.48933233223827  \\
            1.016  8.39787125975473  \\
            1.017  8.30748408298467  \\
            1.018  8.218157155312555  \\
            1.019  8.129877016695518  \\
            1.02  8.042630390932894  \\
            1.021  7.9564041829786625  \\
            1.022  7.8711854762954525  \\
            1.023  7.7869615302500215  \\
            1.024  7.703719777548942  \\
            1.025  7.621447821714414  \\
            1.026  7.540133434598956  \\
            1.027  7.459764553938922  \\
            1.028  7.380329280945617  \\
            1.029  7.3018158779339775  \\
            1.03  7.22421276598762  \\
            1.031  7.147508522660245  \\
            1.032  7.071691879712208  \\
            1.033  6.996751720882269  \\
            1.034  6.922677079693377  \\
            1.035  6.849457137292485  \\
            1.036  6.777081220323291  \\
            1.037  6.7055387988319195  \\
            1.038  6.6348194842044546  \\
            1.039  6.5649130271363525  \\
            1.04  6.495809315632679  \\
            1.041  6.427498373039197  \\
            1.042  6.359970356103282  \\
            1.043  6.2932155530647  \\
            1.044  6.2272243817752475  \\
            1.045  6.161987387847302  \\
            1.046  6.097495242830307  \\
            1.047  6.033738742415231  \\
            1.048  5.970708804666075  \\
            1.049  5.908396468278455  \\
            1.05  5.846792890864371  \\
            1.051  5.785889347263186  \\
            1.052  5.725677227877945  \\
            1.053  5.6661480370370825  \\
            1.054  5.607293391380655  \\
            1.055  5.549105018271159  \\
            1.056  5.4915747542280835  \\
            1.057  5.434694543386277  \\
            1.058  5.378456435977295  \\
            1.059  5.322852586833795  \\
            1.06  5.267875253916206  \\
            1.061  5.213516796861696  \\
            1.062  5.15976967555471  \\
            1.063  5.106626448719117  \\
            1.064  5.054079772531237  \\
            1.065  5.002122399253793  \\
            1.066  4.950747175890082  \\
            1.067  4.8999470428584315  \\
            1.068  4.8497150326862215  \\
            1.069  4.80004426872355  \\
            1.07  4.750927963875865  \\
            1.071  4.7023594193556235  \\
            1.072  4.654332023452313  \\
            1.073  4.606839250320903  \\
            1.074  4.559874658788091  \\
            1.075  4.513431891176413  \\
            1.076  4.467504672145571  \\
            1.077  4.422086807551075  \\
            1.078  4.377172183319582  \\
            1.079  4.332754764340987  \\
            1.08  4.288828593376705  \\
            1.081  4.245387789984182  \\
            1.082  4.2024265494570905  \\
            1.083  4.159939141781261  \\
            1.084  4.1179199106057975  \\
            1.085  4.07636327222945  \\
            1.086  4.035263714601705  \\
            1.087  3.9946157963386515  \\
            1.088  3.954414145753117  \\
            1.089  3.914653459899124  \\
            1.09  3.875328503630173  \\
            1.091  3.8364341086714058  \\
            1.092  3.797965172705171  \\
            1.093  3.7599166584700456  \\
            1.094  3.722283592872839  \\
            1.095  3.685061066113649  \\
            1.096  3.6482442308235012  \\
            1.097  3.6118283012146346  \\
            1.098  3.5758085522429854  \\
            1.099  3.5401803187829444  \\
            1.1  3.504938994813922  \\
            1.101  3.4700800326188204  \\
            1.102  3.435598941993957  \\
            1.103  3.4014912894705245  \\
            1.104  3.3677526975471594  \\
            1.105  3.3343788439336928  \\
            1.106  3.301365460805674  \\
            1.107  3.2687083340697374  \\
            1.108  3.236403302639416  \\
            1.109  3.2044462577214627  \\
            1.11  3.172833142112314  \\
            1.111  3.141559949504737  \\
            1.112  3.1106227238043056  \\
            1.113  3.080017558455766  \\
            1.114  3.049740595778927  \\
            1.115  3.019788026314142  \\
            1.116  2.9901560881770295  \\
            1.117  2.960841066422499  \\
            1.118  2.9318392924177332  \\
            1.119  2.903147143224202  \\
            1.12  2.874761040988356  \\
            1.121  2.8466774523410905  \\
            1.122  2.8188928878056236  \\
            1.123  2.7914039012138785  \\
            1.124  2.7642070891310384  \\
            1.125  2.737299090288344  \\
            1.126  2.710676585023826  \\
            1.127  2.6843362947310023  \\
            1.128  2.658274981315359  \\
            1.129  2.6324894466584463  \\
            1.13  2.606976532089606  \\
            1.131  2.581733117865058  \\
            1.132  2.5567561226544044  \\
            1.133  2.532042503034258  \\
            1.134  2.507589252989082  \\
            1.135  2.48339340341894  \\
            1.136  2.45945202165424  \\
            1.137  2.4357622109771877  \\
            1.138  2.4123211101500264  \\
            1.139  2.3891258929497905  \\
            1.14  2.36617376770964  \\
            1.141  2.343461976866522  \\
            1.142  2.3209877965152166  \\
            1.143  2.2987485359685254  \\
            1.144  2.276741537323656  \\
            1.145  2.254964175034568  \\
            1.146  2.233413855490335  \\
            1.147  2.2120880165992904  \\
            1.148  2.1909841273790187  \\
            1.149  2.1700996875519545  \\
            1.15  2.1494322271466597  \\
            1.151  2.128979306104546  \\
            1.152  2.1087385138921064  \\
            1.153  2.0887074691184364  \\
            1.154  2.068883819158101  \\
            1.155  2.049265239779138  \\
            1.156  2.0298494347762563  \\
            1.157  2.010634135609013  \\
            1.158  1.9916171010450399  \\
            1.159  1.9727961168081047  \\
            1.16  1.9541689952310741  \\
            1.161  1.9357335749135698  \\
            1.162  1.9174877203843868  \\
            1.163  1.8994293217684697  \\
            1.164  1.8815562944585107  \\
            1.165  1.8638665787909767  \\
            1.166  1.846358139726624  \\
            1.167  1.829028966535314  \\
            1.168  1.8118770724851876  \\
            1.169  1.7949004945360163  \\
            1.17  1.7780972930367902  \\
            1.171  1.7614655514273565  \\
            1.172  1.7450033759441785  \\
            1.173  1.728708895330028  \\
            1.174  1.7125802605476785  \\
            1.175  1.6966156444974203  \\
            1.176  1.6808132417384618  \\
            1.177  1.6651712682140443  \\
            1.178  1.6496879609803317  \\
            1.179  1.6343615779389054  \\
            1.18  1.6191903975729312  \\
            1.181  1.604172718686825  \\
            1.182  1.5893068601494875  \\
            1.183  1.5745911606409368  \\
            1.184  1.5600239784024112  \\
            1.185  1.5456036909897717  \\
            1.186  1.5313286950302762  \\
            1.187  1.5171974059825617  \\
            1.188  1.5032082578998986  \\
            1.189  1.489359703196568  \\
            1.19  1.4756502124174147  \\
            1.191  1.4620782740104352  \\
            1.192  1.4486423941024533  \\
            1.193  1.4353410962777455  \\
            1.194  1.4221729213596639  \\
            1.195  1.4091364271951297  \\
            1.196  1.3962301884420367  \\
            1.197  1.3834527963594476  \\
            1.198  1.3708028586006176  \\
            1.199  1.3582789990087294  \\
            1.2  1.3458798574153816  \\
            1.201  1.333604089441709  \\
            1.202  1.3214503663021804  \\
            1.203  1.3094173746109539  \\
            1.204  1.2975038161908365  \\
            1.205  1.2857084078847267  \\
            1.206  1.2740298813695938  \\
            1.207  1.262466982972867  \\
            1.208  1.2510184734912888  \\
            1.209  1.2396831280121186  \\
            1.21  1.228459735736725  \\
            1.211  1.2173470998064653  \\
            1.212  1.2063440371308884  \\
            1.213  1.1954493782181599  \\
            1.214  1.1846619670077483  \\
            1.215  1.1739806607052692  \\
            1.216  1.163404329619531  \\
            1.217  1.1529318570016753  \\
            1.218  1.1425621388864617  \\
            1.219  1.1322940839355875  \\
            1.22  1.1221266132830925  \\
            1.221  1.1120586603827445  \\
            1.222  1.1020891708574496  \\
            1.223  1.0922171023505913  \\
            1.224  1.0824414243793377  \\
            1.225  1.072761118189824  \\
            1.226  1.0631751766142508  \\
            1.227  1.0536826039298033  \\
            1.228  1.0442824157194375  \\
            1.229  1.0349736387344353  \\
            1.23  1.0257553107587754  \\
            1.231  1.0166264804752245  \\
            1.232  1.0075862073331974  \\
            1.233  0.9986335614182871  \\
            1.234  0.9897676233235166  \\
            1.235  0.9809874840222176  \\
            1.236  0.9722922447425798  \\
            1.237  0.9636810168437858  \\
            1.238  0.9551529216937714  \\
            1.239  0.9467070905485262  \\
            1.24  0.9383426644329772  \\
            1.241  0.9300587940233713  \\
            1.242  0.9218546395311966  \\
            1.243  0.9137293705885619  \\
            1.244  0.9056821661350746  \\
            1.245  0.897712214306138  \\
            1.246  0.8898187123227039  \\
            1.247  0.8820008663824079  \\
            1.248  0.8742578915521216  \\
            1.249  0.8665890116618487  \\
            1.25  0.8589934591999999  \\
            1.251  0.8514704752099793  \\
            1.252  0.8440193091881  \\
            1.253  0.8366392189828016  \\
            1.254  0.8293294706951295  \\
            1.255  0.8220893385805044  \\
            1.256  0.8149181049517098  \\
            1.257  0.8078150600831386  \\
            1.258  0.800779502116224  \\
            1.259  0.7938107369660935  \\
            1.26  0.7869080782293808  \\
            1.261  0.780070847093224  \\
            1.262  0.7732983722453918  \\
            1.263  0.7665899897855646  \\
            1.264  0.7599450431377142  \\
            1.265  0.7533628829636069  \\
            1.266  0.7468428670773757  \\
            1.267  0.7403843603611875  \\
            1.268  0.7339867346819486  \\
            1.269  0.7276493688090797  \\
            1.27  0.7213716483332977  \\
            1.271  0.715152965586441  \\
            1.272  0.708992719562276  \\
            1.273  0.7028903158383164  \\
            1.274  0.6968451664985995  \\
            1.275  0.69085669005745  \\
            1.276  0.6849243113841732  \\
            1.277  0.6790474616287097  \\
            1.278  0.6732255781481972  \\
            1.279  0.6674581044344665  \\
            1.28  0.661744490042422  \\
            1.281  0.6560841905193305  \\
            1.282  0.6504766673349726  \\
            1.283  0.6449213878126778  \\
            1.284  0.6394178250611998  \\
            1.285  0.633965457907453  \\
            1.286  0.6285637708300655  \\
            1.287  0.6232122538937728  \\
            1.288  0.6179104026846052  \\
            1.289  0.6126577182458945  \\
            1.29  0.6074537070150541  \\
            1.291  0.6022978807611562  \\
            1.292  0.5971897565232627  \\
            1.293  0.5921288565495325  \\
            1.294  0.5871147082370628  \\
            1.295  0.5821468440724866  \\
            1.296  0.5772248015732852  \\
            1.297  0.5723481232298367  \\
            1.298  0.5675163564481623  \\
            1.299  0.5627290534933891  \\
            1.3  0.5579857714338899  \\
            1.301  0.5532860720861262  \\
            1.302  0.5486295219601465  \\
            1.303  0.54401569220577  \\
            1.304  0.5394441585594113  \\
            1.305  0.5349145012915706  \\
            1.306  0.5304263051549483  \\
            1.307  0.5259791593332092  \\
            1.308  0.5215726573903541  \\
            1.309  0.5172063972207248  \\
            1.31  0.5128799809996009  \\
            1.311  0.5085930151344128  \\
            1.312  0.5043451102165326  \\
            1.313  0.500135880973665  \\
            1.314  0.49596494622280063  \\
            1.315  0.4918319288237548  \\
            1.316  0.4877364556332539  \\
            1.317  0.4836781574595928  \\
            1.318  0.4796566690178249  \\
            1.319  0.47567162888550923  \\
            1.32  0.47172267945897717  \\
            1.321  0.46780946691014097  \\
            1.322  0.4639316411438089  \\
            1.323  0.4600888557555293  \\
            1.324  0.4562807679899271  \\
            1.325  0.45250703869955644  \\
            1.326  0.4487673323042338  \\
            1.327  0.4450613167508731  \\
            1.328  0.44138866347378924  \\
            1.329  0.4377490473554921  \\
            1.33  0.4341421466879356  \\
            1.331  0.43056764313424567  \\
            1.332  0.4270252216908918  \\
            1.333  0.4235145706503246  \\
            1.334  0.42003538156404685  \\
            1.335  0.41658734920613827  \\
            1.336  0.4131701715372026  \\
            1.337  0.4097835496687572  \\
            1.338  0.4064271878280349  \\
            1.339  0.4031007933232162  \\
            1.34  0.39980407650906385  \\
            1.341  0.3965367507529774  \\
            1.342  0.39329853240143925  \\
            1.343  0.3900891407468709  \\
            1.344  0.3869082979948702  \\
            1.345  0.3837557292318489  \\
            1.346  0.38063116239304184  \\
            1.347  0.37753432823090566  \\
            1.348  0.3744649602838812  \\
            1.349  0.37142279484553453  \\
            1.35  0.3684075709340526  \\
            1.351  0.36541903026210903  \\
            1.352  0.3624569172070749  \\
            1.353  0.35952097878159067  \\
            1.354  0.3566109646044746  \\
            1.355  0.3537266268719832  \\
            1.356  0.3508677203293997  \\
            1.357  0.34803400224296666  \\
            1.358  0.3452252323721369  \\
            1.359  0.3424411729421617  \\
            1.36  0.3396815886169883  \\
            1.361  0.3369462464724865  \\
            1.362  0.33423491596997784  \\
            1.363  0.3315473689300844  \\
            1.364  0.3288833795068736  \\
            1.365  0.32624272416231437  \\
            1.366  0.3236251816410219  \\
            1.367  0.32103053294530715  \\
            1.368  0.318458561310506  \\
            1.369  0.31590905218060705  \\
            1.37  0.31338179318415227  \\
            1.371  0.3108765741104277  \\
            1.372  0.30839318688592143  \\
            1.373  0.3059314255510644  \\
            1.374  0.3034910862372314  \\
            1.375  0.3010719671440184  \\
            1.376  0.2986738685167742  \\
            1.377  0.296296592624398  \\
            1.378  0.29393994373739385  \\
            1.379  0.2916037281061734  \\
            1.38  0.2892877539396171  \\
            1.381  0.28699183138387474  \\
            1.382  0.28471577250141855  \\
            1.383  0.28245939125033004  \\
            1.384  0.2802225034638328  \\
            1.385  0.27800492683005384  \\
            1.386  0.2758064808720246  \\
            1.387  0.27362698692790477  \\
            1.388  0.27146626813144076  \\
            1.389  0.2693241493926408  \\
            1.39  0.26720045737867915  \\
            1.391  0.26509502049501166  \\
            1.392  0.26300766886671567  \\
            1.393  0.2609382343200348  \\
            1.394  0.2588865503641438  \\
            1.395  0.2568524521731129  \\
            1.396  0.25483577656808637  \\
            1.397  0.25283636199965726  \\
            1.398  0.2508540485304496  \\
            1.399  0.24888867781789278  \\
            1.4  0.24694009309719853  \\
            1.401  0.24500813916452502  \\
            1.402  0.24309266236033922  \\
            1.403  0.2411935105529613  \\
            1.404  0.23931053312230302  \\
            1.405  0.23744358094378357  \\
            1.406  0.23559250637243467  \\
            1.407  0.2337571632271792  \\
            1.408  0.23193740677529417  \\
            1.409  0.2301330937170433  \\
            1.41  0.22834408217048952  \\
            1.411  0.22657023165647294  \\
            1.412  0.22481140308376463  \\
            1.413  0.2230674587343816  \\
            1.414  0.22133826224907388  \\
            1.415  0.21962367861296836  \\
            1.416  0.21792357414138108  \\
            1.417  0.21623781646578266  \\
            1.418  0.21456627451992757  \\
            1.419  0.21290881852613358  \\
            1.42  0.2112653199817208  \\
            1.421  0.20963565164559717  \\
            1.422  0.20801968752500027  \\
            1.423  0.2064173028623819  \\
            1.424  0.2048283741224449  \\
            1.425  0.20325277897931984  \\
            1.426  0.2016903963038904  \\
            1.427  0.20014110615125477  \\
            1.428  0.19860478974833268  \\
            1.429  0.19708132948160537  \\
            1.43  0.19557060888499722  \\
            1.431  0.19407251262788772  \\
            1.432  0.19258692650326134  \\
            1.433  0.19111373741598497  \\
            1.434  0.18965283337122021  \\
            1.435  0.18820410346295943  \\
            1.436  0.18676743786269434  \\
            1.437  0.1853427278082052  \\
            1.438  0.18392986559247923  \\
            1.439  0.18252874455274692  \\
            1.44  0.1811392590596447  \\
            1.441  0.1797613045064926  \\
            1.442  0.1783947772986953  \\
            1.443  0.17703957484325542  \\
            1.444  0.17569559553840758  \\
            1.445  0.1743627387633619  \\
            1.446  0.17304090486816542  \\
            1.447  0.17172999516367055  \\
            1.448  0.17042991191161846  \\
            1.449  0.16914055831482738  \\
            1.45  0.1678618385074928  \\
            1.451  0.16659365754559036  \\
            1.452  0.165335921397388  \\
            1.453  0.1640885369340584  \\
            1.454  0.16285141192039818  \\
            1.455  0.16162445500564476  \\
            1.456  0.16040757571439831  \\
            1.457  0.15920068443763818  \\
            1.458  0.15800369242384243  \\
            1.459  0.15681651177019976  \\
            1.46  0.15563905541392198  \\
            1.461  0.1544712371236467  \\
            1.462  0.15331297149093828  \\
            1.463  0.15216417392187725  \\
            1.464  0.15102476062874529  \\
            1.465  0.1498946486217966  \\
            1.466  0.1487737557011231  \\
            1.467  0.14766200044860375  \\
            1.468  0.14655930221994548  \\
            1.469  0.1454655811368064  \\
            1.47  0.14438075807900866  \\
            1.471  0.14330475467683135  \\
            1.472  0.1422374933033913  \\
            1.473  0.14117889706710215  \\
            1.474  0.1401288898042189  \\
            1.475  0.13908739607145937  \\
            1.476  0.138054341138709  \\
            1.477  0.1370296509818009  \\
            1.478  0.13601325227537692  \\
            1.479  0.135005072385823  \\
            1.48  0.13400503936428326  \\
            1.481  0.1330130819397464  \\
            1.482  0.1320291295122097  \\
            1.483  0.13105311214591311  \\
            1.484  0.13008496056264962  \\
            1.485  0.12912460613514382  \\
            1.486  0.12817198088050502  \\
            1.487  0.12722701745374737  \\
            1.488  0.1262896491413824  \\
            1.489  0.1253598098550772  \\
            1.49  0.12443743412538356  \\
            1.491  0.12352245709553109  \\
            1.492  0.1226148145152897  \\
            1.493  0.12171444273489458  \\
            1.494  0.12082127869903922  \\
            1.495  0.11993525994092916  \\
            1.496  0.1190563245764025  \\
            1.497  0.1181844112981098  \\
            1.498  0.11731945936975899  \\
            1.499  0.11646140862041851  \\
            1.5  0.11561019943888409  \\
            1.501  0.11476577276810242  \\
            1.502  0.11392807009965601  \\
            1.503  0.11309703346830652  \\
            1.504  0.11227260544659412  \\
            1.505  0.1114547291394968  \\
            1.506  0.11064334817914329  \\
            1.507  0.10983840671958497  \\
            1.508  0.10903984943162  \\
            1.509  0.10824762149767508  \\
            1.51  0.10746166860673852  \\
            1.511  0.10668193694934941  \\
            1.512  0.10590837321263705  \\
            1.513  0.1051409245754153  \\
            1.514  0.10437953870332586  \\
            1.515  0.10362416374403555  \\
            1.516  0.10287474832248111  \\
            1.517  0.10213124153616705  \\
            1.518  0.10139359295050987  \\
            1.519  0.10066175259423409  \\
            1.52  0.09993567095481376  \\
            1.521  0.09921529897396453  \\
            1.522  0.09850058804318017  \\
            1.523  0.09779148999931868  \\
            1.524  0.0970879571202317  \\
            1.525  0.09638994212044247  \\
            1.526  0.09569739814686619  \\
            1.527  0.09501027877457777  \\
            1.528  0.09432853800262114  \\
            1.529  0.09365213024986484  \\
            1.53  0.09298101035089822  \\
            1.531  0.09231513355197292  \\
            1.532  0.09165445550698426  \\
            1.533  0.09099893227349658  \\
            1.534  0.09034852030880786  \\
            1.535  0.08970317646605724  \\
            1.536  0.08906285799037089  \\
            1.537  0.08842752251505005  \\
            1.538  0.08779712805779633  \\
            1.539  0.0871716330169784  \\
            1.54  0.08655099616793512  \\
            1.541  0.08593517665931909  \\
            1.542  0.08532413400947589  \\
            1.543  0.08471782810286285  \\
            1.544  0.08411621918650276  \\
            1.545  0.08351926786647619  \\
            1.546  0.08292693510444796  \\
            1.547  0.0823391822142315  \\
            1.548  0.08175597085838655  \\
            1.549  0.0811772630448539  \\
            1.55  0.08060302112362276  \\
            1.551  0.08003320778343437  \\
            1.552  0.07946778604851756  \\
            1.553  0.07890671927535978  \\
            1.554  0.07834997114950917  \\
            1.555  0.07779750568241171  \\
            1.556  0.07724928720827849  \\
            1.557  0.07670528038098716  \\
            1.558  0.07616545017101328  \\
            1.559  0.07562976186239474  \\
            1.56  0.07509818104972552  \\
            1.561  0.07457067363518192  \\
            1.562  0.0740472058255774  \\
            1.563  0.07352774412944918  \\
            1.564  0.07301225535417283  \\
            1.565  0.07250070660310806  \\
            1.566  0.0719930652727718  \\
            1.567  0.07148929905004166  \\
            1.568  0.07098937590938639  \\
            1.569  0.07049326411012594  \\
            1.57  0.07000093219371777  \\
            1.571  0.06951234898107243  \\
            1.572  0.06902748356989462  \\
            1.573  0.06854630533205294  \\
            1.574  0.06806878391097457  \\
            1.575  0.06759488921906803  \\
            1.576  0.06712459143517041  \\
            1.577  0.06665786100202195  \\
            1.578  0.06619466862376464  \\
            1.579  0.06573498526346756  \\
            1.58  0.06527878214067581  \\
            1.581  0.06482603072898549  \\
            1.582  0.06437670275364177  \\
            1.583  0.06393077018916263  \\
            1.584  0.06348820525698497  \\
            1.585  0.06304898042313599  \\
            1.586  0.06261306839592644  \\
            1.587  0.062180442123668515  \\
            1.588  0.061751074792415275  \\
            1.589  0.06132493982372415  \\
            1.59  0.06090201087244147  \\
            1.591  0.06048226182451065  \\
            1.592  0.0600656667948008  \\
            1.593  0.059652200124958664  \\
            1.594  0.059241836381280555  \\
            1.595  0.058834550352606915  \\
            1.596  0.058430317048236625  \\
            1.597  0.0580291116958635  \\
            1.598  0.05763090973953195  \\
            1.599  0.05723568683761449  \\
            1.6  0.05684341886080798  \\
            1.601  0.0564540818901512  \\
            1.602  0.05606765221506084  \\
            1.603  0.055684106331388344  \\
            1.604  0.055303420939494766  \\
            1.605  0.05492557294234607  \\
            1.606  0.054550539443626  \\
            1.607  0.05417829774586907  \\
            1.608  0.053808825348610693  \\
            1.609  0.05344209994655698  \\
            1.61  0.05307809942777137  \\
            1.611  0.05271680187188052  \\
            1.612  0.05235818554829663  \\
            1.613  0.05200222891445865  \\
            1.614  0.05164891061408959  \\
            1.615  0.05129820947547234  \\
            1.616  0.05095010450974113  \\
            1.617  0.0506045749091913  \\
            1.618  0.050261600045604274  \\
            1.619  0.04992115946859044  \\
            1.62  0.04958323290394696  \\
            1.621  0.04924780025203316  \\
            1.622  0.04891484158616049  \\
            1.623  0.04858433715099961  \\
            1.624  0.048256267361001935  \\
            1.625  0.047930612798837785  \\
            1.626  0.04760735421384887  \\
            1.627  0.04728647252051654  \\
            1.628  0.046967948796945165  \\
            1.629  0.04665176428335957  \\
            1.63  0.04633790038061825  \\
            1.631  0.04602633864874005  \\
            1.632  0.0457170608054463  \\
            1.633  0.045410048724716115  \\
            1.634  0.0451052844353569  \\
            1.635  0.04480275011958769  \\
            1.636  0.04450242811163739  \\
            1.637  0.04420430089635562  \\
            1.638  0.04390835110783814  \\
            1.639  0.04361456152806458  \\
            1.64  0.04332291508555051  \\
            1.641  0.043033394854011524  \\
            1.642  0.042745984051041346  \\
            1.643  0.042460666036801785  \\
            1.644  0.04217742431272639  \\
            1.645  0.04189624252023563  \\
            1.646  0.041617104439465584  \\
            1.647  0.04133999398800793  \\
            1.648  0.04106489521966311  \\
            1.649  0.04079179232320458  \\
            1.65  0.04052066962115601  \\
            1.651  0.0402515115685792  \\
            1.652  0.03998430275187492  \\
            1.653  0.039719027887594086  \\
            1.654  0.03945567182126151  \\
            1.655  0.03919421952621002  \\
            1.656  0.03893465610242676  \\
            1.657  0.03867696677540957  \\
            1.658  0.03842113689503543  \\
            1.659  0.03816715193443877  \\
            1.66  0.037914997488901536  \\
            1.661  0.03766465927475292  \\
            1.662  0.03741612312828065  \\
            1.663  0.03716937500465179  \\
            1.664  0.03692440097684473  \\
            1.665  0.03668118723459062  \\
            1.666  0.03643972008332569  \\
            1.667  0.03619998594315288  \\
            1.668  0.03596197134781411  \\
            1.669  0.035725662943671646  \\
            1.67  0.03549104748869994  \\
            1.671  0.035258111851486314  \\
            1.672  0.03502684301024203  \\
            1.673  0.03479722805182189  \\
            1.674  0.03456925417075403  \\
            1.675  0.03434290866827824  \\
            1.676  0.03411817895139401  \\
            1.677  0.03389505253191712  \\
            1.678  0.03367351702554575  \\
            1.679  0.03345356015093482  \\
            1.68  0.03323516972877984  \\
            1.681  0.03301833368090868  \\
            1.682  0.03280304002938278  \\
            1.683  0.032589276895606034  \\
            1.684  0.032377032499442904  \\
            1.685  0.03216629515834418  \\
            1.686  0.031957053286481706  \\
            1.687  0.03174929539389052  \\
            1.688  0.03154301008561991  \\
            1.689  0.03133818606089173  \\
            1.69  0.03113481211226736  \\
            1.691  0.030932877124821827  \\
            1.692  0.030732370075326438  \\
            1.693  0.030533280031438376  \\
            1.694  0.030335596150898658  \\
            1.695  0.030139307680736933  \\
            1.696  0.029944403956484456  \\
            1.697  0.029750874401393755  \\
            1.698  0.029558708525666347  \\
            1.699  0.029367895925686994  \\
            1.7  0.02917842628326584  \\
            1.701  0.02899028936488697  \\
            1.702  0.028803475020964697  \\
            1.703  0.028617973185106126  \\
            1.704  0.028433773873381314  \\
            1.705  0.02825086718359959  \\
            1.706  0.028069243294593264  \\
            1.707  0.027888892465507444  \\
            1.708  0.027709805035097075  \\
            1.709  0.02753197142102993  \\
            1.71  0.027355382119196713  \\
            1.711  0.027180027703027007  \\
            1.712  0.02700589882281218  \\
            1.713  0.026832986205033993  \\
            1.714  0.026661280651700082  \\
            1.715  0.026490773039684987  \\
            1.716  0.026321454320077927  \\
            1.717  0.026153315517536052  \\
            1.718  0.025986347729644275  \\
            1.719  0.025820542126280432  \\
            1.72  0.025655889948986967  \\
            1.721  0.025492382510347838  \\
            1.722  0.02533001119337176  \\
            1.723  0.02516876745088059  \\
            1.724  0.02500864280490398  \\
            1.725  0.024849628846078976  \\
            1.726  0.024691717233055833  \\
            1.727  0.024534899691908682  \\
            1.728  0.024379168015552233  \\
            1.729  0.024224514063163286  \\
            1.73  0.024070929759608155  \\
            1.731  0.023918407094874788  \\
            1.732  0.02376693812351068  \\
            1.733  0.023616514964065416  \\
            1.734  0.023467129798538898  \\
            1.735  0.02331877487183407  \\
            1.736  0.023171442491215253  \\
            1.737  0.02302512502577089  \\
            1.738  0.022879814905881757  \\
            1.739  0.022735504622693553  \\
            1.74  0.022592186727594804  \\
            1.741  0.022449853831699076  \\
            1.742  0.02230849860533242  \\
            1.743  0.022168113777524985  \\
            1.744  0.022028692135507844  \\
            1.745  0.02189022652421385  \\
            1.746  0.021752709845783598  \\
            1.747  0.021616135059075375  \\
            1.748  0.021480495179180093  \\
            1.749  0.021345783276940172  \\
            1.75  0.021211992478473246  \\
            1.751  0.021079115964699795  \\
            1.752  0.020947146970875383  \\
            1.753  0.020816078786127123  \\
            1.754  0.02068590475299409  \\
            1.755  0.02055661826697232  \\
            1.756  0.020428212776063548  \\
            1.757  0.020300681780328386  \\
            1.758  0.020174018831443236  \\
            1.759  0.020048217532261554  \\
            1.76  0.019923271536378732  \\
            1.761  0.01979917454770132  \\
            1.762  0.019675920320019708  \\
            1.763  0.019553502656585134  \\
            1.764  0.019431915409690075  \\
            1.765  0.019311152480252853  \\
            1.766  0.019191207817405572  \\
            1.767  0.019072075418086216  \\
            1.768  0.018953749326633965  \\
            1.769  0.01883622363438863  \\
            1.77  0.018719492479293245  \\
            1.771  0.01860355004550068  \\
            1.772  0.01848839056298335  \\
            1.773  0.018374008307146892  \\
            1.774  0.018260397598446863  \\
            1.775  0.018147552802009324  \\
            1.776  0.018035468327254403  \\
            1.777  0.017924138627523664  \\
            1.778  0.017813558199710398  \\
            1.779  0.017703721583893637  \\
            1.78  0.017594623362975056  \\
            1.781  0.017486258162319527  \\
            1.782  0.01737862064939847  \\
            1.783  0.017271705533436817  \\
            1.784  0.017165507565062747  \\
            1.785  0.0170600215359609  \\
            1.786  0.016955242278528355  \\
            1.787  0.01685116466553402  \\
            1.788  0.016747783609780704  \\
            1.789  0.01664509406377058  \\
            1.79  0.01654309101937325  \\
            1.791  0.01644176950749712  \\
            1.792  0.016341124597763372  \\
            1.793  0.016241151398183187  \\
            1.794  0.01614184505483745  \\
            1.795  0.016043200751559687  \\
            1.796  0.015945213709621483  \\
            1.797  0.015847879187420984  \\
            1.798  0.01575119248017388  \\
            1.799  0.01565514891960744  \\
            1.8  0.015559743873656905  \\
            1.801  0.01546497274616495  \\
            1.802  0.015370830976583434  \\
            1.803  0.015277314039678152  \\
            1.804  0.01518441744523583  \\
            1.805  0.01509213673777408  \\
            1.806  0.01500046749625357  \\
            1.807  0.01490940533379305  \\
            1.808  0.014818945897386599  \\
            1.809  0.014729084867623659  \\
            1.81  0.014639817958411233  \\
            1.811  0.014551140916698848  \\
            1.812  0.014463049522205614  \\
            1.813  0.014375539587150005  \\
            1.814  0.014288606955981714  \\
            1.815  0.014202247505116162  \\
            1.816  0.014116457142671034  \\
            1.817  0.014031231808205438  \\
            1.818  0.01394656747246105  \\
            1.819  0.01386246013710583  \\
            1.82  0.013778905834479683  \\
            1.821  0.01369590062734265  \\
            1.822  0.013613440608625006  \\
            1.823  0.013531521901179844  \\
            1.824  0.013450140657537515  \\
            1.825  0.013369293059662533  \\
            1.826  0.01328897531871229  \\
            1.827  0.013209183674798182  \\
            1.828  0.013129914396748519  \\
            1.829  0.01305116378187381  \\
            1.83  0.01297292815573378  \\
            1.831  0.012895203871906735  \\
            1.832  0.01281798731176063  \\
            1.833  0.012741274884226415  \\
            1.834  0.012665063025573077  \\
            1.835  0.01258934819918492  \\
            1.836  0.012514126895340491  \\
            1.837  0.012439395630993721  \\
            1.838  0.012365150949556668  \\
            1.839  0.012291389420684433  \\
            1.84  0.012218107640061685  \\
            1.841  0.01214530222919126  \\
            1.842  0.012072969835184376  \\
            1.843  0.012001107130552888  \\
            1.844  0.011929710813003109  \\
            1.845  0.011858777605231654  \\
            1.846  0.011788304254722798  \\
            1.847  0.011718287533547847  \\
            1.848  0.011648724238165969  \\
            1.849  0.011579611189227013  \\
            1.85  0.011510945231375776  \\
            1.851  0.011442723233058207  \\
            1.852  0.011374942086329012  \\
            1.853  0.0113075987066612  \\
            1.854  0.01124069003275699  \\
            1.855  0.011174213026360598  \\
            1.856  0.011108164672072414  \\
            1.857  0.011042541977164975  \\
            1.858  0.010977341971400309  \\
            1.859  0.010912561706849075  \\
            1.86  0.010848198257711031  \\
            1.861  0.010784248720137246  \\
            1.862  0.010720710212053636  \\
            1.863  0.010657579872986225  \\
            1.864  0.010594854863887667  \\
            1.865  0.010532532366965507  \\
            1.866  0.010470609585511647  \\
            1.867  0.010409083743733526  \\
            1.868  0.010347952086586499  \\
            1.869  0.010287211879607882  \\
            1.87  0.01022686040875219  \\
            1.871  0.01016689498022798  \\
            1.872  0.010107312920335901  \\
            1.873  0.010048111575308319  \\
            1.874  0.009989288311150084  \\
            1.875  0.009930840513480893  \\
            1.876  0.009872765587378776  \\
            1.877  0.00981506095722504  \\
            1.878  0.009757724066550549  \\
            1.879  0.00970075237788316  \\
            1.88  0.00964414337259667  \\
            1.881  0.009587894550760812  \\
            1.882  0.009532003430992744  \\
            1.883  0.009476467550309599  \\
            1.884  0.00942128446398247  \\
            1.885  0.009366451745391443  \\
            1.886  0.009311966985882037  \\
            1.887  0.009257827794622694  \\
            1.888  0.009204031798463639  \\
            1.889  0.009150576641796741  \\
            1.89  0.009097459986416777  \\
            1.891  0.009044679511383637  \\
            1.892  0.008992232912885909  \\
            1.893  0.008940117904105416  \\
            1.894  0.008888332215083084  \\
            1.895  0.008836873592585745  \\
            1.896  0.008785739799974256  \\
            1.897  0.008734928617072545  \\
            1.898  0.008684437840037926  \\
            1.899  0.008634265281232342  \\
            1.9  0.008584408769094849  \\
            1.901  0.008534866148015019  \\
            1.902  0.008485635278207544  \\
            1.903  0.008436714035587758  \\
            1.904  0.008388100311648342  \\
            1.905  0.00833979201333691  \\
            1.906  0.008291787062934784  \\
            1.907  0.00824408339793661  \\
            1.908  0.00819667897093116  \\
            1.909  0.008149571749482951  \\
            1.91  0.008102759716015026  \\
            1.911  0.008056240867692547  \\
            1.912  0.008010013216307533  \\
            1.913  0.007964074788164367  \\
            1.914  0.007918423623966468  \\
            1.915  0.00787305777870369  \\
            1.916  0.00782797532154087  \\
            1.917  0.007783174335707112  \\
            1.918  0.007738652918386162  \\
            1.919  0.007694409180607537  \\
            1.92  0.0076504412471387106  \\
            1.921  0.007606747256378048  \\
            1.922  0.007563325360248776  \\
            1.923  0.007520173724093683  \\
            1.924  0.0074772905265708435  \\
            1.925  0.007434673959550053  \\
            1.926  0.007392322228010286  \\
            1.927  0.007350233549937826  \\
            1.928  0.00730840615622542  \\
            1.929  0.00726683829057209  \\
            1.93  0.007225528209383946  \\
            1.931  0.0071844741816756465  \\
            1.932  0.007143674488972842  \\
            1.933  0.007103127425215263  \\
            1.934  0.007062831296660759  \\
            1.935  0.0070227844217899855  \\
            1.936  0.006982985131212018  \\
            1.937  0.006943431767570601  \\
            1.938  0.006904122685451306  \\
            1.939  0.006865056251289329  \\
            1.94  0.006826230843278165  \\
            1.941  0.006787644851278907  \\
            1.942  0.006749296676730434  \\
            1.943  0.006711184732560177  \\
            1.944  0.006673307443095783  \\
            1.945  0.006635663243977344  \\
            1.946  0.006598250582070497  \\
            1.947  0.006561067915380088  \\
            1.948  0.006524113712964692  \\
            1.949  0.006487386454851687  \\
            1.95  0.006450884631953165  \\
            1.951  0.0064146067459823885  \\
            1.952  0.006378551309371076  \\
            1.953  0.006342716845187217  \\
            1.954  0.006307101887053699  \\
            1.955  0.006271704979067464  \\
            1.956  0.006236524675719458  \\
            1.957  0.006201559541815105  \\
            1.958  0.0061668081523955426  \\
            1.959  0.006132269092659397  \\
            1.96  0.0060979409578853  \\
            1.961  0.006063822353354925  \\
            1.962  0.0060299118942767635  \\
            1.963  0.005996208205710403  \\
            1.964  0.0059627099224915416  \\
            1.965  0.005929415689157494  \\
            1.966  0.0058963241598734145  \\
            1.967  0.005863433998359005  \\
            1.968  0.005830743877815936  \\
            1.969  0.005798252480855733  \\
            1.97  0.005765958499428375  \\
            1.971  0.00573386063475134  \\
            1.972  0.005701957597239352  \\
            1.973  0.005670248106434577  \\
            1.974  0.005638730890937496  \\
            1.975  0.005607404688338229  \\
            1.976  0.0055762682451485175  \\
            1.977  0.005545320316734155  \\
            1.978  0.005514559667248061  \\
            1.979  0.005483985069563796  \\
            1.98  0.005453595305209718  \\
            1.981  0.005423389164303556  \\
            1.982  0.005393365445487626  \\
            1.983  0.005363522955864455  \\
            1.984  0.005333860510933034  \\
            1.985  0.005304376934525471  \\
            1.986  0.005275071058744271  \\
            1.987  0.005245941723899996  \\
            1.988  0.005216987778449551  \\
            1.989  0.0051882080789348445  \\
            1.99  0.005159601489922054  \\
            1.991  0.00513116688394127  \\
            1.992  0.0051029031414267295  \\
            1.993  0.005074809150657425  \\
            1.994  0.00504688380769829  \\
            1.995  0.0050191260163417505  \\
            1.996  0.004991534688049858  \\
            1.997  0.004964108741896765  \\
            1.998  0.004936847104511773  \\
            1.999  0.0049097487100227336  \\
            2.0  0.0048828125  \\
        }
        ;
    \addlegendentry {$\beta = 10.0$}
\end{axis}
\end{tikzpicture}


Then the conclusion from this problem would be markets with heterogeneous firms have a higher aggregate output. 

\end{proof}
\end{problem}

\begin{problem}
The only Ben and Jerry's in town faces different linear inverse demand curves $P=a-b Q$ for triple-chocolate-chunk ice cream and fruit-bowl-punch ice cream. It buys each at the same constant unit cost from its supplier. The inverse demand curves cross at an interior point, above marginal cost, with Fruit-bowl-punch having a higher price intercept than Triple-chunk. Does it follow that its price charged is higher on the Fruit-Bowl Punch?
\end{problem}

\begin{proof}[Answer]

Monopolist faces demands:

$$P_i = a_i - b_iQ_i\quad i = 1,2 \qquad a_1 > a_2$$

Monopolist solve:

$$\max_{Q_i}\: P(Q_i)Q_i - cQ_i \qiq Q_i^* = \frac{a_i-c}{2b_i}$$

Substituting in the demand:

$$P_i = \frac{a_i+c}{2} \qiq P_1 > P_2$$

The monopolist charges a higher price to the product with the higher demand intercept.

\end{proof}

\begin{problem}
\begin{subproblem}
Draw a negatively sloped demand curve with elasticity of magnitude greater than 1 , and a corresponding marginal revenue curve.
\end{subproblem}


\begin{tikzpicture}[/tikz/background rectangle/.style={fill={rgb,1:red,1.0;green,1.0;blue,1.0}, draw opacity={1.0}}, show background rectangle]
\begin{axis}[point meta max={nan}, point meta min={nan}, legend cell align={left}, title={}, title style={at={{(0.5,1)}}, anchor={south}, font={{\fontsize{14 pt}{18.2 pt}\selectfont}}, color={rgb,1:red,0.0;green,0.0;blue,0.0}, draw opacity={1.0}, rotate={0.0}}, legend style={color={rgb,1:red,0.1333;green,0.1333;blue,0.3333}, draw opacity={0.1}, line width={1}, solid, fill={rgb,1:red,1.0;green,1.0;blue,1.0}, fill opacity={0.9}, text opacity={1.0}, font={{\fontsize{8 pt}{10.4 pt}\selectfont}}, text={rgb,1:red,0.0;green,0.0;blue,0.0}, at={(1.02, 1)}, anchor={north west}}, axis background/.style={fill={rgb,1:red,1.0;green,1.0;blue,1.0}, opacity={1.0}}, anchor={north west}, xshift={1.0mm}, yshift={-1.0mm}, width={145.4mm}, height={99.6mm}, scaled x ticks={false}, xlabel={}, x tick style={draw={none}}, x tick label style={color={rgb,1:red,0.0;green,0.0;blue,0.0}, opacity={1.0}, rotate={0}}, xlabel style={at={(ticklabel cs:0.5)}, anchor=near ticklabel, font={{\fontsize{11 pt}{14.3 pt}\selectfont}}, color={rgb,1:red,0.0;green,0.0;blue,0.0}, draw opacity={1.0}, rotate={0.0}}, xmajorgrids={true}, xmin={-0.09}, xmax={3.09}, xtick={{0.0,1.0,2.0,3.0}}, xticklabels={{$0$,$1$,$2$,$3$}}, xtick align={inside}, xticklabel style={font={{\fontsize{8 pt}{10.4 pt}\selectfont}}, color={rgb,1:red,0.0;green,0.0;blue,0.0}, draw opacity={1.0}, rotate={0.0}}, x grid style={color={rgb,1:red,0.1333;green,0.1333;blue,0.3333}, draw opacity={0.1}, line width={0.5}, solid}, extra x ticks={{0.2,0.4,0.6,0.8,1.2,1.4,1.6,1.8,2.2,2.4,2.6,2.8}}, extra x tick labels={}, extra x tick style={grid={major}, x grid style={color={rgb,1:red,0.1333;green,0.1333;blue,0.3333}, draw opacity={0.05}, line width={0.5}, solid}, major tick length={0}}, x axis line style={{draw opacity = 0}}, scaled y ticks={false}, ylabel={}, y tick style={draw={none}}, y tick label style={color={rgb,1:red,0.0;green,0.0;blue,0.0}, opacity={1.0}, rotate={0}}, ylabel style={at={(ticklabel cs:0.5)}, anchor=near ticklabel, font={{\fontsize{11 pt}{14.3 pt}\selectfont}}, color={rgb,1:red,0.0;green,0.0;blue,0.0}, draw opacity={1.0}, rotate={0.0}}, ymajorgrids={true}, ymin={0}, ymax={3}, ytick={{0.0,1.0,2.0,3.0}}, yticklabels={{$0$,$1$,$2$,$3$}}, ytick align={inside}, yticklabel style={font={{\fontsize{8 pt}{10.4 pt}\selectfont}}, color={rgb,1:red,0.0;green,0.0;blue,0.0}, draw opacity={1.0}, rotate={0.0}}, y grid style={color={rgb,1:red,0.1333;green,0.1333;blue,0.3333}, draw opacity={0.1}, line width={0.5}, solid}, extra y ticks={{0.2,0.4,0.6,0.8,1.2,1.4,1.6,1.8,2.2,2.4,2.6,2.8}}, extra y tick labels={}, extra y tick style={grid={major}, y grid style={color={rgb,1:red,0.1333;green,0.1333;blue,0.3333}, draw opacity={0.05}, line width={0.5}, solid}, major tick length={0}}, y axis line style={{draw opacity = 0}}]
    \addplot[color={rgb,1:red,0.9333;green,0.4667;blue,0.2}, name path={c921738f-6636-40bd-8731-8645b09da82c}, draw opacity={1.0}, line width={1.2}, solid]
        table[row sep={\\}]
        {
            \\
            0.0  3.0  \\
            0.01  2.993  \\
            0.02  2.986  \\
            0.03  2.979  \\
            0.04  2.972  \\
            0.05  2.965  \\
            0.06  2.958  \\
            0.07  2.951  \\
            0.08  2.944  \\
            0.09  2.937  \\
            0.1  2.93  \\
            0.11  2.923  \\
            0.12  2.916  \\
            0.13  2.909  \\
            0.14  2.902  \\
            0.15  2.895  \\
            0.16  2.888  \\
            0.17  2.881  \\
            0.18  2.874  \\
            0.19  2.867  \\
            0.2  2.86  \\
            0.21  2.853  \\
            0.22  2.846  \\
            0.23  2.839  \\
            0.24  2.832  \\
            0.25  2.825  \\
            0.26  2.818  \\
            0.27  2.811  \\
            0.28  2.804  \\
            0.29  2.797  \\
            0.3  2.79  \\
            0.31  2.783  \\
            0.32  2.776  \\
            0.33  2.769  \\
            0.34  2.762  \\
            0.35  2.755  \\
            0.36  2.7479999999999998  \\
            0.37  2.741  \\
            0.38  2.734  \\
            0.39  2.727  \\
            0.4  2.72  \\
            0.41  2.713  \\
            0.42  2.706  \\
            0.43  2.699  \\
            0.44  2.692  \\
            0.45  2.685  \\
            0.46  2.678  \\
            0.47  2.671  \\
            0.48  2.664  \\
            0.49  2.657  \\
            0.5  2.65  \\
            0.51  2.643  \\
            0.52  2.636  \\
            0.53  2.629  \\
            0.54  2.622  \\
            0.55  2.615  \\
            0.56  2.608  \\
            0.57  2.601  \\
            0.58  2.594  \\
            0.59  2.587  \\
            0.6  2.58  \\
            0.61  2.573  \\
            0.62  2.566  \\
            0.63  2.559  \\
            0.64  2.552  \\
            0.65  2.545  \\
            0.66  2.538  \\
            0.67  2.531  \\
            0.68  2.524  \\
            0.69  2.517  \\
            0.7  2.5100000000000002  \\
            0.71  2.503  \\
            0.72  2.496  \\
            0.73  2.489  \\
            0.74  2.482  \\
            0.75  2.475  \\
            0.76  2.468  \\
            0.77  2.461  \\
            0.78  2.454  \\
            0.79  2.447  \\
            0.8  2.44  \\
            0.81  2.433  \\
            0.82  2.426  \\
            0.83  2.419  \\
            0.84  2.412  \\
            0.85  2.405  \\
            0.86  2.398  \\
            0.87  2.391  \\
            0.88  2.384  \\
            0.89  2.3770000000000002  \\
            0.9  2.37  \\
            0.91  2.363  \\
            0.92  2.356  \\
            0.93  2.349  \\
            0.94  2.342  \\
            0.95  2.335  \\
            0.96  2.328  \\
            0.97  2.321  \\
            0.98  2.314  \\
            0.99  2.307  \\
            1.0  2.3  \\
            1.01  2.293  \\
            1.02  2.286  \\
            1.03  2.279  \\
            1.04  2.272  \\
            1.05  2.265  \\
            1.06  2.258  \\
            1.07  2.251  \\
            1.08  2.244  \\
            1.09  2.237  \\
            1.1  2.23  \\
            1.11  2.223  \\
            1.12  2.216  \\
            1.13  2.209  \\
            1.14  2.202  \\
            1.15  2.195  \\
            1.16  2.188  \\
            1.17  2.181  \\
            1.18  2.174  \\
            1.19  2.167  \\
            1.2  2.16  \\
            1.21  2.153  \\
            1.22  2.146  \\
            1.23  2.1390000000000002  \\
            1.24  2.132  \\
            1.25  2.125  \\
            1.26  2.118  \\
            1.27  2.111  \\
            1.28  2.104  \\
            1.29  2.097  \\
            1.3  2.09  \\
            1.31  2.083  \\
            1.32  2.076  \\
            1.33  2.069  \\
            1.34  2.062  \\
            1.35  2.055  \\
            1.36  2.048  \\
            1.37  2.041  \\
            1.38  2.0340000000000003  \\
            1.39  2.027  \\
            1.4  2.02  \\
            1.41  2.013  \\
            1.42  2.0060000000000002  \\
            1.43  1.999  \\
            1.44  1.992  \\
            1.45  1.985  \\
            1.46  1.9780000000000002  \\
            1.47  1.971  \\
            1.48  1.9640000000000002  \\
            1.49  1.957  \\
            1.5  1.9500000000000002  \\
            1.51  1.943  \\
            1.52  1.9360000000000002  \\
            1.53  1.929  \\
            1.54  1.9220000000000002  \\
            1.55  1.915  \\
            1.56  1.9080000000000001  \\
            1.57  1.901  \\
            1.58  1.8940000000000001  \\
            1.59  1.887  \\
            1.6  1.8800000000000001  \\
            1.61  1.8730000000000002  \\
            1.62  1.866  \\
            1.63  1.8590000000000002  \\
            1.64  1.852  \\
            1.65  1.8450000000000002  \\
            1.66  1.838  \\
            1.67  1.8310000000000002  \\
            1.68  1.824  \\
            1.69  1.8170000000000002  \\
            1.7  1.81  \\
            1.71  1.8030000000000002  \\
            1.72  1.796  \\
            1.73  1.7890000000000001  \\
            1.74  1.782  \\
            1.75  1.7750000000000001  \\
            1.76  1.7680000000000002  \\
            1.77  1.7610000000000001  \\
            1.78  1.7540000000000002  \\
            1.79  1.747  \\
            1.8  1.7400000000000002  \\
            1.81  1.733  \\
            1.82  1.7260000000000002  \\
            1.83  1.719  \\
            1.84  1.7120000000000002  \\
            1.85  1.705  \\
            1.86  1.6980000000000002  \\
            1.87  1.691  \\
            1.88  1.6840000000000002  \\
            1.89  1.677  \\
            1.9  1.6700000000000002  \\
            1.91  1.663  \\
            1.92  1.6560000000000001  \\
            1.93  1.6490000000000002  \\
            1.94  1.6420000000000001  \\
            1.95  1.6350000000000002  \\
            1.96  1.6280000000000001  \\
            1.97  1.6210000000000002  \\
            1.98  1.614  \\
            1.99  1.6070000000000002  \\
            2.0  1.6  \\
            2.01  1.5930000000000002  \\
            2.02  1.586  \\
            2.03  1.5790000000000002  \\
            2.04  1.572  \\
            2.05  1.5650000000000002  \\
            2.06  1.558  \\
            2.07  1.5510000000000002  \\
            2.08  1.544  \\
            2.09  1.5370000000000001  \\
            2.1  1.5300000000000002  \\
            2.11  1.5230000000000001  \\
            2.12  1.5160000000000002  \\
            2.13  1.5090000000000001  \\
            2.14  1.5020000000000002  \\
            2.15  1.495  \\
            2.16  1.4880000000000002  \\
            2.17  1.481  \\
            2.18  1.4740000000000002  \\
            2.19  1.467  \\
            2.2  1.4600000000000002  \\
            2.21  1.453  \\
            2.22  1.4460000000000002  \\
            2.23  1.439  \\
            2.24  1.4320000000000002  \\
            2.25  1.4250000000000003  \\
            2.26  1.4180000000000001  \\
            2.27  1.4110000000000003  \\
            2.28  1.4040000000000001  \\
            2.29  1.3970000000000002  \\
            2.3  1.3900000000000001  \\
            2.31  1.3830000000000002  \\
            2.32  1.3760000000000001  \\
            2.33  1.3690000000000002  \\
            2.34  1.362  \\
            2.35  1.3550000000000002  \\
            2.36  1.348  \\
            2.37  1.3410000000000002  \\
            2.38  1.334  \\
            2.39  1.3270000000000002  \\
            2.4  1.3200000000000003  \\
            2.41  1.3130000000000002  \\
            2.42  1.3060000000000003  \\
            2.43  1.2990000000000002  \\
            2.44  1.2920000000000003  \\
            2.45  1.2850000000000001  \\
            2.46  1.2780000000000002  \\
            2.47  1.2710000000000001  \\
            2.48  1.2640000000000002  \\
            2.49  1.2570000000000001  \\
            2.5  1.2500000000000002  \\
            2.51  1.243  \\
            2.52  1.2360000000000002  \\
            2.53  1.229  \\
            2.54  1.2220000000000002  \\
            2.55  1.215  \\
            2.56  1.2080000000000002  \\
            2.57  1.2010000000000003  \\
            2.58  1.1940000000000002  \\
            2.59  1.1870000000000003  \\
            2.6  1.1800000000000002  \\
            2.61  1.1730000000000003  \\
            2.62  1.1660000000000001  \\
            2.63  1.1590000000000003  \\
            2.64  1.1520000000000001  \\
            2.65  1.1450000000000002  \\
            2.66  1.1380000000000001  \\
            2.67  1.1310000000000002  \\
            2.68  1.124  \\
            2.69  1.1170000000000002  \\
            2.7  1.11  \\
            2.71  1.1030000000000002  \\
            2.72  1.096  \\
            2.73  1.0890000000000002  \\
            2.74  1.0820000000000003  \\
            2.75  1.0750000000000002  \\
            2.76  1.0680000000000003  \\
            2.77  1.0610000000000002  \\
            2.78  1.0540000000000003  \\
            2.79  1.0470000000000002  \\
            2.8  1.0400000000000003  \\
            2.81  1.0330000000000001  \\
            2.82  1.0260000000000002  \\
            2.83  1.0190000000000001  \\
            2.84  1.0120000000000002  \\
            2.85  1.0050000000000001  \\
            2.86  0.9980000000000001  \\
            2.87  0.991  \\
            2.88  0.9839999999999999  \\
            2.89  0.9770000000000002  \\
            2.9  0.9700000000000001  \\
            2.91  0.963  \\
            2.92  0.9560000000000003  \\
            2.93  0.9490000000000002  \\
            2.94  0.9420000000000001  \\
            2.95  0.9349999999999999  \\
            2.96  0.9280000000000003  \\
            2.97  0.9210000000000002  \\
            2.98  0.914  \\
            2.99  0.9069999999999999  \\
            3.0  0.9000000000000002  \\
        }
        ;
    \addlegendentry {Demand}
    \addplot[color={rgb,1:red,0.0;green,0.4667;blue,0.7333}, name path={4a127d44-8397-4af4-8f99-f766b786d4a9}, draw opacity={1.0}, line width={1.2}, dashed]
        table[row sep={\\}]
        {
            \\
            0.0  3.0  \\
            0.01  2.986  \\
            0.02  2.972  \\
            0.03  2.958  \\
            0.04  2.944  \\
            0.05  2.93  \\
            0.06  2.916  \\
            0.07  2.902  \\
            0.08  2.888  \\
            0.09  2.874  \\
            0.1  2.86  \\
            0.11  2.846  \\
            0.12  2.832  \\
            0.13  2.818  \\
            0.14  2.804  \\
            0.15  2.79  \\
            0.16  2.776  \\
            0.17  2.762  \\
            0.18  2.7479999999999998  \\
            0.19  2.734  \\
            0.2  2.72  \\
            0.21  2.706  \\
            0.22  2.692  \\
            0.23  2.678  \\
            0.24  2.664  \\
            0.25  2.65  \\
            0.26  2.636  \\
            0.27  2.622  \\
            0.28  2.608  \\
            0.29  2.594  \\
            0.3  2.58  \\
            0.31  2.566  \\
            0.32  2.552  \\
            0.33  2.538  \\
            0.34  2.524  \\
            0.35  2.5100000000000002  \\
            0.36  2.496  \\
            0.37  2.482  \\
            0.38  2.468  \\
            0.39  2.454  \\
            0.4  2.44  \\
            0.41  2.426  \\
            0.42  2.412  \\
            0.43  2.398  \\
            0.44  2.384  \\
            0.45  2.37  \\
            0.46  2.356  \\
            0.47  2.342  \\
            0.48  2.328  \\
            0.49  2.314  \\
            0.5  2.3  \\
            0.51  2.286  \\
            0.52  2.272  \\
            0.53  2.258  \\
            0.54  2.244  \\
            0.55  2.23  \\
            0.56  2.216  \\
            0.57  2.202  \\
            0.58  2.188  \\
            0.59  2.174  \\
            0.6  2.16  \\
            0.61  2.146  \\
            0.62  2.132  \\
            0.63  2.118  \\
            0.64  2.104  \\
            0.65  2.09  \\
            0.66  2.076  \\
            0.67  2.062  \\
            0.68  2.048  \\
            0.69  2.0340000000000003  \\
            0.7  2.02  \\
            0.71  2.0060000000000002  \\
            0.72  1.992  \\
            0.73  1.9780000000000002  \\
            0.74  1.9640000000000002  \\
            0.75  1.9500000000000002  \\
            0.76  1.9360000000000002  \\
            0.77  1.9220000000000002  \\
            0.78  1.9080000000000001  \\
            0.79  1.8940000000000001  \\
            0.8  1.8800000000000001  \\
            0.81  1.866  \\
            0.82  1.852  \\
            0.83  1.838  \\
            0.84  1.824  \\
            0.85  1.81  \\
            0.86  1.796  \\
            0.87  1.782  \\
            0.88  1.7680000000000002  \\
            0.89  1.7540000000000002  \\
            0.9  1.7400000000000002  \\
            0.91  1.7260000000000002  \\
            0.92  1.7120000000000002  \\
            0.93  1.6980000000000002  \\
            0.94  1.6840000000000002  \\
            0.95  1.6700000000000002  \\
            0.96  1.6560000000000001  \\
            0.97  1.6420000000000001  \\
            0.98  1.6280000000000001  \\
            0.99  1.614  \\
            1.0  1.6  \\
            1.01  1.586  \\
            1.02  1.572  \\
            1.03  1.558  \\
            1.04  1.544  \\
            1.05  1.5300000000000002  \\
            1.06  1.5160000000000002  \\
            1.07  1.5020000000000002  \\
            1.08  1.4880000000000002  \\
            1.09  1.4740000000000002  \\
            1.1  1.4600000000000002  \\
            1.11  1.4460000000000002  \\
            1.12  1.4320000000000002  \\
            1.13  1.4180000000000001  \\
            1.14  1.4040000000000001  \\
            1.15  1.3900000000000001  \\
            1.16  1.3760000000000001  \\
            1.17  1.362  \\
            1.18  1.348  \\
            1.19  1.334  \\
            1.2  1.3200000000000003  \\
            1.21  1.3060000000000003  \\
            1.22  1.2920000000000003  \\
            1.23  1.2780000000000002  \\
            1.24  1.2640000000000002  \\
            1.25  1.2500000000000002  \\
            1.26  1.2360000000000002  \\
            1.27  1.2220000000000002  \\
            1.28  1.2080000000000002  \\
            1.29  1.1940000000000002  \\
            1.3  1.1800000000000002  \\
            1.31  1.1660000000000001  \\
            1.32  1.1520000000000001  \\
            1.33  1.1380000000000001  \\
            1.34  1.124  \\
            1.35  1.11  \\
            1.36  1.096  \\
            1.37  1.0820000000000003  \\
            1.38  1.0680000000000003  \\
            1.39  1.0540000000000003  \\
            1.4  1.0400000000000003  \\
            1.41  1.0260000000000002  \\
            1.42  1.0120000000000002  \\
            1.43  0.9980000000000001  \\
            1.44  0.9839999999999999  \\
            1.45  0.9700000000000001  \\
            1.46  0.9560000000000003  \\
            1.47  0.9420000000000001  \\
            1.48  0.9280000000000003  \\
            1.49  0.914  \\
            1.5  0.9000000000000002  \\
            1.51  0.886  \\
            1.52  0.8720000000000002  \\
            1.53  0.858  \\
            1.54  0.8440000000000002  \\
            1.55  0.83  \\
            1.56  0.8160000000000002  \\
            1.57  0.8019999999999999  \\
            1.58  0.7880000000000001  \\
            1.59  0.7739999999999999  \\
            1.6  0.7600000000000001  \\
            1.61  0.7460000000000003  \\
            1.62  0.7320000000000001  \\
            1.63  0.7180000000000003  \\
            1.64  0.7040000000000001  \\
            1.65  0.6900000000000003  \\
            1.66  0.676  \\
            1.67  0.6620000000000003  \\
            1.68  0.648  \\
            1.69  0.6340000000000002  \\
            1.7  0.62  \\
            1.71  0.6060000000000002  \\
            1.72  0.592  \\
            1.73  0.5780000000000002  \\
            1.74  0.564  \\
            1.75  0.5500000000000002  \\
            1.76  0.5360000000000004  \\
            1.77  0.5220000000000001  \\
            1.78  0.5080000000000003  \\
            1.79  0.4940000000000001  \\
            1.8  0.4800000000000003  \\
            1.81  0.4660000000000001  \\
            1.82  0.4520000000000003  \\
            1.83  0.43800000000000006  \\
            1.84  0.42400000000000027  \\
            1.85  0.41000000000000003  \\
            1.86  0.39600000000000024  \\
            1.87  0.382  \\
            1.88  0.3680000000000002  \\
            1.89  0.354  \\
            1.9  0.3400000000000002  \\
            1.91  0.32599999999999996  \\
            1.92  0.31200000000000017  \\
            1.93  0.2980000000000004  \\
            1.94  0.28400000000000014  \\
            1.95  0.27000000000000035  \\
            1.96  0.2560000000000001  \\
            1.97  0.24200000000000033  \\
            1.98  0.2280000000000001  \\
            1.99  0.2140000000000003  \\
            2.0  0.20000000000000007  \\
            2.01  0.18600000000000028  \\
            2.02  0.17200000000000004  \\
            2.03  0.15800000000000025  \\
            2.04  0.14400000000000002  \\
            2.05  0.13000000000000023  \\
            2.06  0.11599999999999999  \\
            2.07  0.1020000000000002  \\
            2.08  0.08799999999999997  \\
            2.09  0.07400000000000018  \\
            2.1  0.06000000000000038  \\
            2.11  0.046000000000000145  \\
            2.12  0.032000000000000355  \\
            2.13  0.01800000000000012  \\
            2.14  0.004000000000000329  \\
            2.15  -0.009999999999999907  \\
            2.16  -0.0239999999999997  \\
            2.17  -0.03799999999999993  \\
            2.18  -0.05199999999999972  \\
            2.19  -0.06599999999999996  \\
            2.2  -0.07999999999999975  \\
            2.21  -0.09399999999999999  \\
            2.22  -0.10799999999999978  \\
            2.23  -0.12200000000000001  \\
            2.24  -0.1359999999999998  \\
            2.25  -0.14999999999999958  \\
            2.26  -0.16399999999999984  \\
            2.27  -0.17799999999999963  \\
            2.28  -0.19199999999999987  \\
            2.29  -0.20599999999999966  \\
            2.3  -0.2199999999999999  \\
            2.31  -0.23399999999999968  \\
            2.32  -0.24799999999999991  \\
            2.33  -0.2619999999999997  \\
            2.34  -0.2759999999999999  \\
            2.35  -0.2899999999999997  \\
            2.36  -0.30399999999999994  \\
            2.37  -0.3179999999999997  \\
            2.38  -0.33199999999999996  \\
            2.39  -0.34599999999999975  \\
            2.4  -0.35999999999999954  \\
            2.41  -0.3739999999999998  \\
            2.42  -0.38799999999999957  \\
            2.43  -0.4019999999999998  \\
            2.44  -0.4159999999999996  \\
            2.45  -0.4299999999999998  \\
            2.46  -0.4439999999999996  \\
            2.47  -0.45799999999999985  \\
            2.48  -0.47199999999999964  \\
            2.49  -0.4859999999999999  \\
            2.5  -0.49999999999999967  \\
            2.51  -0.5139999999999999  \\
            2.52  -0.5279999999999997  \\
            2.53  -0.5419999999999999  \\
            2.54  -0.5559999999999997  \\
            2.55  -0.57  \\
            2.56  -0.5839999999999997  \\
            2.57  -0.5979999999999995  \\
            2.58  -0.6119999999999998  \\
            2.59  -0.6259999999999996  \\
            2.6  -0.6399999999999998  \\
            2.61  -0.6539999999999996  \\
            2.62  -0.6679999999999998  \\
            2.63  -0.6819999999999996  \\
            2.64  -0.6959999999999998  \\
            2.65  -0.7099999999999996  \\
            2.66  -0.7239999999999999  \\
            2.67  -0.7379999999999997  \\
            2.68  -0.7519999999999999  \\
            2.69  -0.7659999999999997  \\
            2.7  -0.7799999999999999  \\
            2.71  -0.7939999999999997  \\
            2.72  -0.8079999999999999  \\
            2.73  -0.8219999999999997  \\
            2.74  -0.8359999999999995  \\
            2.75  -0.8499999999999998  \\
            2.76  -0.8639999999999995  \\
            2.77  -0.8779999999999998  \\
            2.78  -0.8919999999999996  \\
            2.79  -0.9059999999999998  \\
            2.8  -0.9199999999999996  \\
            2.81  -0.9339999999999998  \\
            2.82  -0.9479999999999996  \\
            2.83  -0.9619999999999999  \\
            2.84  -0.9759999999999996  \\
            2.85  -0.9899999999999999  \\
            2.86  -1.0039999999999998  \\
            2.87  -1.018  \\
            2.88  -1.0320000000000003  \\
            2.89  -1.0459999999999996  \\
            2.9  -1.0599999999999998  \\
            2.91  -1.074  \\
            2.92  -1.0879999999999994  \\
            2.93  -1.1019999999999996  \\
            2.94  -1.1159999999999999  \\
            2.95  -1.1300000000000001  \\
            2.96  -1.1439999999999995  \\
            2.97  -1.1579999999999997  \\
            2.98  -1.172  \\
            2.99  -1.1860000000000002  \\
            3.0  -1.1999999999999995  \\
        }
        ;
    \addlegendentry {Marginal Ravenue}
\end{axis}
\end{tikzpicture}



\begin{subproblem}
Draw a supply curve with positive and rising elasticity.

\end{subproblem}

\begin{tikzpicture}[/tikz/background rectangle/.style={fill={rgb,1:red,1.0;green,1.0;blue,1.0}, draw opacity={1.0}}, show background rectangle]
\begin{axis}[point meta max={nan}, point meta min={nan}, legend cell align={left}, title={}, title style={at={{(0.5,1)}}, anchor={south}, font={{\fontsize{14 pt}{18.2 pt}\selectfont}}, color={rgb,1:red,0.0;green,0.0;blue,0.0}, draw opacity={1.0}, rotate={0.0}}, legend style={color={rgb,1:red,0.1333;green,0.1333;blue,0.3333}, draw opacity={0.1}, line width={1}, solid, fill={rgb,1:red,1.0;green,1.0;blue,1.0}, fill opacity={0.9}, text opacity={1.0}, font={{\fontsize{8 pt}{10.4 pt}\selectfont}}, text={rgb,1:red,0.0;green,0.0;blue,0.0}, at={(0.02, 0.98)}, anchor={north west}}, axis background/.style={fill={rgb,1:red,1.0;green,1.0;blue,1.0}, opacity={1.0}}, anchor={north west}, xshift={1.0mm}, yshift={-1.0mm}, width={150.4mm}, height={99.6mm}, scaled x ticks={false}, xlabel={}, x tick style={draw={none}}, x tick label style={color={rgb,1:red,0.0;green,0.0;blue,0.0}, opacity={1.0}, rotate={0}}, xlabel style={at={(ticklabel cs:0.5)}, anchor=near ticklabel, font={{\fontsize{11 pt}{14.3 pt}\selectfont}}, color={rgb,1:red,0.0;green,0.0;blue,0.0}, draw opacity={1.0}, rotate={0.0}}, xmajorgrids={true}, xmin={-0.019699999999999995}, xmax={1.0297}, xtick={{0.0,0.25,0.5,0.75,1.0}}, xticklabels={{$0.00$,$0.25$,$0.50$,$0.75$,$1.00$}}, xtick align={inside}, xticklabel style={font={{\fontsize{8 pt}{10.4 pt}\selectfont}}, color={rgb,1:red,0.0;green,0.0;blue,0.0}, draw opacity={1.0}, rotate={0.0}}, x grid style={color={rgb,1:red,0.1333;green,0.1333;blue,0.3333}, draw opacity={0.1}, line width={0.5}, solid}, extra x ticks={{0.05,0.1,0.15,0.2,0.3,0.35,0.4,0.45,0.55,0.6,0.65,0.7,0.8,0.85,0.9,0.95}}, extra x tick labels={}, extra x tick style={grid={major}, x grid style={color={rgb,1:red,0.1333;green,0.1333;blue,0.3333}, draw opacity={0.05}, line width={0.5}, solid}, major tick length={0}}, x axis line style={{draw opacity = 0}}, scaled y ticks={false}, ylabel={}, y tick style={draw={none}}, y tick label style={color={rgb,1:red,0.0;green,0.0;blue,0.0}, opacity={1.0}, rotate={0}}, ylabel style={at={(ticklabel cs:0.5)}, anchor=near ticklabel, font={{\fontsize{11 pt}{14.3 pt}\selectfont}}, color={rgb,1:red,0.0;green,0.0;blue,0.0}, draw opacity={1.0}, rotate={0.0}}, ymajorgrids={true}, ymin={0.12543418159105651}, ymax={0.33938864843262295}, ytick={{0.15000000000000002,0.2,0.25,0.30000000000000004}}, yticklabels={{$0.15$,$0.20$,$0.25$,$0.30$}}, ytick align={inside}, yticklabel style={font={{\fontsize{8 pt}{10.4 pt}\selectfont}}, color={rgb,1:red,0.0;green,0.0;blue,0.0}, draw opacity={1.0}, rotate={0.0}}, y grid style={color={rgb,1:red,0.1333;green,0.1333;blue,0.3333}, draw opacity={0.1}, line width={0.5}, solid}, extra y ticks={{0.13000000000000003,0.14,0.16000000000000003,0.17000000000000004,0.18000000000000002,0.19000000000000003,0.21000000000000002,0.22000000000000003,0.23,0.24000000000000002,0.26,0.27,0.28,0.29000000000000004,0.31000000000000005,0.32000000000000006,0.33000000000000007}}, extra y tick labels={}, extra y tick style={grid={major}, y grid style={color={rgb,1:red,0.1333;green,0.1333;blue,0.3333}, draw opacity={0.05}, line width={0.5}, solid}, major tick length={0}}, y axis line style={{draw opacity = 0}}]
    \addplot[color={rgb,1:red,0.9333;green,0.4667;blue,0.2}, name path={b2150bdf-9f1c-4403-b0be-c470890163bd}, draw opacity={1.0}, line width={1.2}, solid]
        table[row sep={\\}]
        {
            \\
            0.01  0.13148949669034612  \\
            0.02  0.14467544439806992  \\
            0.03  0.1536910937827657  \\
            0.04  0.16080076659533452  \\
            0.05  0.16678529900522837  \\
            0.06  0.17201605885457302  \\
            0.07  0.1767015463989852  \\
            0.08  0.18097160833803233  \\
            0.09  0.18491310238034855  \\
            0.1  0.18858726120609245  \\
            0.11  0.19203902553081245  \\
            0.12  0.1953024474092073  \\
            0.13  0.19840400530473135  \\
            0.14  0.20136473514022868  \\
            0.15  0.20420165384681774  \\
            0.16  0.2069287413986741  \\
            0.17  0.20955763707099265  \\
            0.18  0.2120981448627272  \\
            0.19  0.21455860800046914  \\
            0.2  0.2169461914873545  \\
            0.21  0.2192670987099368  \\
            0.22  0.22152673987748112  \\
            0.23  0.22372986469236447  \\
            0.24  0.22588066806195417  \\
            0.25  0.22798287521786023  \\
            0.26  0.23003981091228137  \\
            0.27  0.2320544561644092  \\
            0.28  0.23402949517252097  \\
            0.29  0.23596735438459007  \\
            0.3  0.23787023526198567  \\
            0.31  0.23974014192963644  \\
            0.32  0.24157890464920204  \\
            0.33  0.24338819985650056  \\
            0.34  0.24516956735452192  \\
            0.35  0.24692442513726173  \\
            0.36  0.24865408222895583  \\
            0.37  0.25035974985196174  \\
            0.38  0.25204255117999586  \\
            0.39  0.2537035298883113  \\
            0.4  0.25534365767616196  \\
            0.41  0.25696384090761193  \\
            0.42  0.2585649264929464  \\
            0.43  0.2601477071134893  \\
            0.44  0.2617129258766571  \\
            0.45  0.26326128047488606  \\
            0.46  0.2647934269111351  \\
            0.47  0.26630998284455515  \\
            0.48  0.2678115306022904  \\
            0.49  0.2692986198969782  \\
            0.5  0.27077177028411376  \\
            0.51  0.2722314733888767  \\
            0.52  0.273678194928137  \\
            0.53  0.2751123765500501  \\
            0.54  0.2765344375108231  \\
            0.55  0.277944776205809  \\
            0.56  0.2793437715700003  \\
            0.57  0.2807317843611913  \\
            0.58  0.2821091583375243  \\
            0.59  0.2834762213397807  \\
            0.6  0.2848332862876056  \\
            0.61  0.2861806520978293  \\
            0.62  0.2875186045321529  \\
            0.63  0.28884741698068367  \\
            0.64  0.2901673511871134  \\
            0.65  0.2914786579207309  \\
            0.66  0.2927815775999236  \\
            0.67  0.2940763408713515  \\
            0.68  0.2953631691485589  \\
            0.69  0.2966422751134207  \\
            0.7  0.2979138631834863  \\
            0.71  0.29917812994799725  \\
            0.72  0.30043526457508796  \\
            0.73  0.30168544919244833  \\
            0.74  0.3029288592435177  \\
            0.75  0.30416566382109217  \\
            0.76  0.3053960259800598  \\
            0.77  0.3066201030308267  \\
            0.78  0.30783804681486143  \\
            0.79  0.30905000396366317  \\
            0.8  0.31025611614234755  \\
            0.81  0.3114565202789454  \\
            0.82  0.31265134878041734  \\
            0.83  0.3138407297363078  \\
            0.84  0.3150247871108849  \\
            0.85  0.3162036409245479  \\
            0.86  0.31737740742521936  \\
            0.87  0.3185461992503866  \\
            0.88  0.3197101255804028  \\
            0.89  0.32086929228361466  \\
            0.9  0.32202380205383796  \\
            0.91  0.3231737545406659  \\
            0.92  0.3243192464730586  \\
            0.93  0.32546037177662895  \\
            0.94  0.3265972216850114  \\
            0.95  0.327729884845671  \\
            0.96  0.3288584474204871  \\
            0.97  0.32998299318142066  \\
            0.98  0.33110360360155394  \\
            0.99  0.33222035794177157  \\
            1.0  0.3333333333333333  \\
        }
        ;
    \addlegendentry {Supply}
\end{axis}
\end{tikzpicture}



Comes from solving the differential equation:
$$\varepsilon = \frac{\partial Q}{\partial P} \frac{P}{Q} = Q$$

\end{proof}

\begin{subproblem}
Draw the marginal revenue curve for a monopolist facing a downward sloping demand curve that is continuous and linear everywhere except for an interval of quantities where it is perfectly elastic.
\end{subproblem}

\begin{tikzpicture}[/tikz/background rectangle/.style={fill={rgb,1:red,1.0;green,1.0;blue,1.0}, draw opacity={1.0}}, show background rectangle]
\begin{axis}[point meta max={nan}, point meta min={nan}, legend cell align={left}, title={}, title style={at={{(0.5,1)}}, anchor={south}, font={{\fontsize{14 pt}{18.2 pt}\selectfont}}, color={rgb,1:red,0.0;green,0.0;blue,0.0}, draw opacity={1.0}, rotate={0.0}}, legend style={color={rgb,1:red,0.1333;green,0.1333;blue,0.3333}, draw opacity={0.1}, line width={1}, solid, fill={rgb,1:red,1.0;green,1.0;blue,1.0}, fill opacity={0.9}, text opacity={1.0}, font={{\fontsize{8 pt}{10.4 pt}\selectfont}}, text={rgb,1:red,0.0;green,0.0;blue,0.0}, at={(1.02, 1)}, anchor={north west}}, axis background/.style={fill={rgb,1:red,1.0;green,1.0;blue,1.0}, opacity={1.0}}, anchor={north west}, xshift={1.0mm}, yshift={-1.0mm}, width={145.4mm}, height={99.6mm}, scaled x ticks={false}, xlabel={}, x tick style={draw={none}}, x tick label style={color={rgb,1:red,0.0;green,0.0;blue,0.0}, opacity={1.0}, rotate={0}}, xlabel style={at={(ticklabel cs:0.5)}, anchor=near ticklabel, font={{\fontsize{11 pt}{14.3 pt}\selectfont}}, color={rgb,1:red,0.0;green,0.0;blue,0.0}, draw opacity={1.0}, rotate={0.0}}, xmajorgrids={true}, xmin={0}, xmax={3}, xtick={{1,2}}, xticklabels={{$a$,$b$}}, xtick align={inside}, xticklabel style={font={{\fontsize{8 pt}{10.4 pt}\selectfont}}, color={rgb,1:red,0.0;green,0.0;blue,0.0}, draw opacity={1.0}, rotate={0.0}}, x grid style={color={rgb,1:red,0.1333;green,0.1333;blue,0.3333}, draw opacity={0.1}, line width={0.5}, solid}, extra x ticks={{0.2,0.4,0.6,0.8,1.2,1.4,1.6,1.8,2.2,2.4,2.6,2.8}}, extra x tick labels={}, extra x tick style={grid={major}, x grid style={color={rgb,1:red,0.1333;green,0.1333;blue,0.3333}, draw opacity={0.05}, line width={0.5}, solid}, major tick length={0}}, x axis line style={{draw opacity = 0}}, scaled y ticks={false}, ylabel={}, y tick style={draw={none}}, y tick label style={color={rgb,1:red,0.0;green,0.0;blue,0.0}, opacity={1.0}, rotate={0}}, ylabel style={at={(ticklabel cs:0.5)}, anchor=near ticklabel, font={{\fontsize{11 pt}{14.3 pt}\selectfont}}, color={rgb,1:red,0.0;green,0.0;blue,0.0}, draw opacity={1.0}, rotate={0.0}}, ymajorgrids={true}, ymin={0}, ymax={3}, ytick={{0.0}}, yticklabels={{$0$}}, ytick align={inside}, yticklabel style={font={{\fontsize{8 pt}{10.4 pt}\selectfont}}, color={rgb,1:red,0.0;green,0.0;blue,0.0}, draw opacity={1.0}, rotate={0.0}}, y grid style={color={rgb,1:red,0.1333;green,0.1333;blue,0.3333}, draw opacity={0.1}, line width={0.5}, solid}, y axis line style={{draw opacity = 0}}]
    \addplot[color={rgb,1:red,0.9333;green,0.4667;blue,0.2}, name path={edcd69d1-f48b-412c-b75a-489c479a717a}, draw opacity={1.0}, line width={1.2}, solid]
        table[row sep={\\}]
        {
            \\
            2.0  2.0  \\
            2.01  1.995  \\
            2.02  1.99  \\
            2.03  1.985  \\
            2.04  1.98  \\
            2.05  1.975  \\
            2.06  1.97  \\
            2.07  1.965  \\
            2.08  1.96  \\
            2.09  1.955  \\
            2.1  1.95  \\
            2.11  1.945  \\
            2.12  1.94  \\
            2.13  1.935  \\
            2.14  1.93  \\
            2.15  1.925  \\
            2.16  1.92  \\
            2.17  1.915  \\
            2.18  1.91  \\
            2.19  1.905  \\
            2.2  1.9  \\
            2.21  1.895  \\
            2.22  1.89  \\
            2.23  1.885  \\
            2.24  1.88  \\
            2.25  1.875  \\
            2.26  1.87  \\
            2.27  1.865  \\
            2.28  1.86  \\
            2.29  1.855  \\
            2.3  1.85  \\
            2.31  1.845  \\
            2.32  1.84  \\
            2.33  1.835  \\
            2.34  1.83  \\
            2.35  1.825  \\
            2.36  1.82  \\
            2.37  1.815  \\
            2.38  1.81  \\
            2.39  1.805  \\
            2.4  1.8  \\
            2.41  1.795  \\
            2.42  1.79  \\
            2.43  1.785  \\
            2.44  1.78  \\
            2.45  1.775  \\
            2.46  1.77  \\
            2.47  1.765  \\
            2.48  1.76  \\
            2.49  1.755  \\
            2.5  1.75  \\
            2.51  1.745  \\
            2.52  1.74  \\
            2.53  1.735  \\
            2.54  1.73  \\
            2.55  1.725  \\
            2.56  1.72  \\
            2.57  1.715  \\
            2.58  1.71  \\
            2.59  1.705  \\
            2.6  1.7  \\
            2.61  1.695  \\
            2.62  1.69  \\
            2.63  1.685  \\
            2.64  1.68  \\
            2.65  1.675  \\
            2.66  1.67  \\
            2.67  1.665  \\
            2.68  1.66  \\
            2.69  1.655  \\
            2.7  1.65  \\
            2.71  1.645  \\
            2.72  1.64  \\
            2.73  1.635  \\
            2.74  1.63  \\
            2.75  1.625  \\
            2.76  1.62  \\
            2.77  1.615  \\
            2.78  1.61  \\
            2.79  1.605  \\
            2.8  1.6  \\
            2.81  1.595  \\
            2.82  1.59  \\
            2.83  1.585  \\
            2.84  1.58  \\
            2.85  1.575  \\
            2.86  1.57  \\
            2.87  1.565  \\
            2.88  1.56  \\
            2.89  1.555  \\
            2.9  1.55  \\
            2.91  1.545  \\
            2.92  1.54  \\
            2.93  1.535  \\
            2.94  1.53  \\
            2.95  1.525  \\
            2.96  1.52  \\
            2.97  1.515  \\
            2.98  1.51  \\
            2.99  1.505  \\
            3.0  1.5  \\
        }
        ;
    \addlegendentry {Demand}
    \addplot[color={rgb,1:red,0.0;green,0.4667;blue,0.7333}, name path={f7c9601a-2673-4d8b-96bc-9438ad7d7f80}, draw opacity={1.0}, line width={1.2}, dashed]
        table[row sep={\\}]
        {
            \\
            2.0  1.0  \\
            2.01  0.99  \\
            2.02  0.98  \\
            2.03  0.97  \\
            2.04  0.96  \\
            2.05  0.95  \\
            2.06  0.94  \\
            2.07  0.93  \\
            2.08  0.92  \\
            2.09  0.91  \\
            2.1  0.9  \\
            2.11  0.89  \\
            2.12  0.88  \\
            2.13  0.87  \\
            2.14  0.86  \\
            2.15  0.85  \\
            2.16  0.84  \\
            2.17  0.83  \\
            2.18  0.82  \\
            2.19  0.81  \\
            2.2  0.8  \\
            2.21  0.79  \\
            2.22  0.78  \\
            2.23  0.77  \\
            2.24  0.76  \\
            2.25  0.75  \\
            2.26  0.74  \\
            2.27  0.73  \\
            2.28  0.72  \\
            2.29  0.71  \\
            2.3  0.7  \\
            2.31  0.69  \\
            2.32  0.68  \\
            2.33  0.67  \\
            2.34  0.66  \\
            2.35  0.65  \\
            2.36  0.64  \\
            2.37  0.63  \\
            2.38  0.62  \\
            2.39  0.61  \\
            2.4  0.6  \\
            2.41  0.59  \\
            2.42  0.58  \\
            2.43  0.57  \\
            2.44  0.56  \\
            2.45  0.55  \\
            2.46  0.54  \\
            2.47  0.53  \\
            2.48  0.52  \\
            2.49  0.51  \\
            2.5  0.5  \\
            2.51  0.49  \\
            2.52  0.48  \\
            2.53  0.47  \\
            2.54  0.46  \\
            2.55  0.45  \\
            2.56  0.44  \\
            2.57  0.43  \\
            2.58  0.42  \\
            2.59  0.41  \\
            2.6  0.4  \\
            2.61  0.39  \\
            2.62  0.38  \\
            2.63  0.37  \\
            2.64  0.36  \\
            2.65  0.35  \\
            2.66  0.34  \\
            2.67  0.33  \\
            2.68  0.32  \\
            2.69  0.31  \\
            2.7  0.3  \\
            2.71  0.29  \\
            2.72  0.28  \\
            2.73  0.27  \\
            2.74  0.26  \\
            2.75  0.25  \\
            2.76  0.24  \\
            2.77  0.23  \\
            2.78  0.22  \\
            2.79  0.21  \\
            2.8  0.2  \\
            2.81  0.19  \\
            2.82  0.18  \\
            2.83  0.17  \\
            2.84  0.16  \\
            2.85  0.15  \\
            2.86  0.14  \\
            2.87  0.13  \\
            2.88  0.12  \\
            2.89  0.11  \\
            2.9  0.1  \\
            2.91  0.09  \\
            2.92  0.08  \\
            2.93  0.07  \\
            2.94  0.06  \\
            2.95  0.05  \\
            2.96  0.04  \\
            2.97  0.03  \\
            2.98  0.02  \\
            2.99  0.01  \\
            3.0  0.0  \\
        }
        ;
    \addlegendentry {Marginal Ravenue}
    \addplot[color={rgb,1:red,0.9333;green,0.4667;blue,0.2}, name path={3dcf0270-60ee-42b3-aab0-902dec0b2f69}, draw opacity={1.0}, line width={1.2}, solid, forget plot]
        table[row sep={\\}]
        {
            \\
            1.0  2.0  \\
            2.0  2.0  \\
        }
        ;
    \addplot[color={rgb,1:red,0.9333;green,0.4667;blue,0.2}, name path={b3a98f27-fcd2-444c-9e26-4540f1740ceb}, draw opacity={1.0}, line width={1.2}, solid, forget plot]
        table[row sep={\\}]
        {
            \\
            0.0  3.0  \\
            0.01  2.99  \\
            0.02  2.98  \\
            0.03  2.97  \\
            0.04  2.96  \\
            0.05  2.95  \\
            0.06  2.94  \\
            0.07  2.93  \\
            0.08  2.92  \\
            0.09  2.91  \\
            0.1  2.9  \\
            0.11  2.89  \\
            0.12  2.88  \\
            0.13  2.87  \\
            0.14  2.86  \\
            0.15  2.85  \\
            0.16  2.84  \\
            0.17  2.83  \\
            0.18  2.82  \\
            0.19  2.81  \\
            0.2  2.8  \\
            0.21  2.79  \\
            0.22  2.78  \\
            0.23  2.77  \\
            0.24  2.76  \\
            0.25  2.75  \\
            0.26  2.74  \\
            0.27  2.73  \\
            0.28  2.72  \\
            0.29  2.71  \\
            0.3  2.7  \\
            0.31  2.69  \\
            0.32  2.68  \\
            0.33  2.67  \\
            0.34  2.66  \\
            0.35  2.65  \\
            0.36  2.64  \\
            0.37  2.63  \\
            0.38  2.62  \\
            0.39  2.61  \\
            0.4  2.6  \\
            0.41  2.59  \\
            0.42  2.58  \\
            0.43  2.57  \\
            0.44  2.56  \\
            0.45  2.55  \\
            0.46  2.54  \\
            0.47  2.53  \\
            0.48  2.52  \\
            0.49  2.51  \\
            0.5  2.5  \\
            0.51  2.49  \\
            0.52  2.48  \\
            0.53  2.47  \\
            0.54  2.46  \\
            0.55  2.45  \\
            0.56  2.44  \\
            0.57  2.43  \\
            0.58  2.42  \\
            0.59  2.41  \\
            0.6  2.4  \\
            0.61  2.39  \\
            0.62  2.38  \\
            0.63  2.37  \\
            0.64  2.36  \\
            0.65  2.35  \\
            0.66  2.34  \\
            0.67  2.33  \\
            0.68  2.32  \\
            0.69  2.31  \\
            0.7  2.3  \\
            0.71  2.29  \\
            0.72  2.28  \\
            0.73  2.27  \\
            0.74  2.26  \\
            0.75  2.25  \\
            0.76  2.24  \\
            0.77  2.23  \\
            0.78  2.22  \\
            0.79  2.21  \\
            0.8  2.2  \\
            0.81  2.19  \\
            0.82  2.18  \\
            0.83  2.17  \\
            0.84  2.16  \\
            0.85  2.15  \\
            0.86  2.14  \\
            0.87  2.13  \\
            0.88  2.12  \\
            0.89  2.11  \\
            0.9  2.1  \\
            0.91  2.09  \\
            0.92  2.08  \\
            0.93  2.07  \\
            0.94  2.06  \\
            0.95  2.05  \\
            0.96  2.04  \\
            0.97  2.03  \\
            0.98  2.02  \\
            0.99  2.01  \\
            1.0  2.0  \\
        }
        ;
    \addplot[color={rgb,1:red,0.0;green,0.4667;blue,0.7333}, name path={16fff635-1aac-4b36-9eb0-7efe0ca1db6d}, draw opacity={1.0}, line width={1.2}, dashed, forget plot]
        table[row sep={\\}]
        {
            \\
            0.0  3.0  \\
            0.01  2.98  \\
            0.02  2.96  \\
            0.03  2.94  \\
            0.04  2.92  \\
            0.05  2.9  \\
            0.06  2.88  \\
            0.07  2.86  \\
            0.08  2.84  \\
            0.09  2.82  \\
            0.1  2.8  \\
            0.11  2.78  \\
            0.12  2.76  \\
            0.13  2.74  \\
            0.14  2.72  \\
            0.15  2.7  \\
            0.16  2.68  \\
            0.17  2.66  \\
            0.18  2.64  \\
            0.19  2.62  \\
            0.2  2.6  \\
            0.21  2.58  \\
            0.22  2.56  \\
            0.23  2.54  \\
            0.24  2.52  \\
            0.25  2.5  \\
            0.26  2.48  \\
            0.27  2.46  \\
            0.28  2.44  \\
            0.29  2.42  \\
            0.3  2.4  \\
            0.31  2.38  \\
            0.32  2.36  \\
            0.33  2.34  \\
            0.34  2.32  \\
            0.35  2.3  \\
            0.36  2.28  \\
            0.37  2.26  \\
            0.38  2.24  \\
            0.39  2.22  \\
            0.4  2.2  \\
            0.41  2.18  \\
            0.42  2.16  \\
            0.43  2.14  \\
            0.44  2.12  \\
            0.45  2.1  \\
            0.46  2.08  \\
            0.47  2.06  \\
            0.48  2.04  \\
            0.49  2.02  \\
            0.5  2.0  \\
            0.51  1.98  \\
            0.52  1.96  \\
            0.53  1.94  \\
            0.54  1.92  \\
            0.55  1.9  \\
            0.56  1.88  \\
            0.57  1.86  \\
            0.58  1.84  \\
            0.59  1.82  \\
            0.6  1.8  \\
            0.61  1.78  \\
            0.62  1.76  \\
            0.63  1.74  \\
            0.64  1.72  \\
            0.65  1.7  \\
            0.66  1.68  \\
            0.67  1.66  \\
            0.68  1.64  \\
            0.69  1.62  \\
            0.7  1.6  \\
            0.71  1.58  \\
            0.72  1.56  \\
            0.73  1.54  \\
            0.74  1.52  \\
            0.75  1.5  \\
            0.76  1.48  \\
            0.77  1.46  \\
            0.78  1.44  \\
            0.79  1.42  \\
            0.8  1.4  \\
            0.81  1.38  \\
            0.82  1.36  \\
            0.83  1.34  \\
            0.84  1.32  \\
            0.85  1.3  \\
            0.86  1.28  \\
            0.87  1.26  \\
            0.88  1.24  \\
            0.89  1.22  \\
            0.9  1.2  \\
            0.91  1.18  \\
            0.92  1.16  \\
            0.93  1.14  \\
            0.94  1.12  \\
            0.95  1.1  \\
            0.96  1.08  \\
            0.97  1.06  \\
            0.98  1.04  \\
            0.99  1.02  \\
            1.0  1.0  \\
        }
        ;
    \addplot[color={rgb,1:red,0.0;green,0.4667;blue,0.7333}, name path={ab396421-ea3f-4446-9963-4be4a97cc922}, draw opacity={1.0}, line width={1.2}, dashed, forget plot]
        table[row sep={\\}]
        {
            \\
            1.0  2.0  \\
            2.0  2.0  \\
        }
        ;
\end{axis}
\end{tikzpicture}




\begin{problem}
There are two groups of Christmas shoppers. Group 2 is gun-shy and hates standing in lines. Group 1 has a lower linear demand, but are willing to take a bullet or stand in line for an hour shopping. Suppose the demands of the two groups are $P_{1}=3-Q_{1}$ and $P_{2}=5-Q_{2}$ respectively, and let $M C=1$ be the firms marginal cost of production. What price should a monopolist charge each group, and how? Suppose, instead that the marginal cost was increasing: $M C=Q$ where $Q=Q_{1}+Q_{2}$, is this problem still separable into two independent optimizations? What price should the monopolist charge to each group?
\end{problem}

\begin{proof}[Answer]

If the monopolist can charge separate prices to each group, she will solve the following maximization problems:

$$\max_{Q_1}\:(3-Q_1)Q_1 - Q_1 \qaq \max_{Q_2}\:(5-Q_2)Q_2 - Q_2$$

first order conditions gives:

$$3-Q_1 = 1 \qiq Q_1 = 1 \qaq 5-Q_2 = 2 \qiq Q_2 = 2$$

plugging in to the demand equations we get:

$$\boxed{P_1 = 2} \qaq \boxed{P_2 = 3}$$

Now, if marginal cost is increasing and depends on the total output of the firm, the problem is no longer separable and the monopolist solves:

$$\max_{Q_1,Q_2} \quad :(3-Q_1)Q_1 + (5-Q_2)Q_2  - \frac{(Q_1 + Q_2)^2}{2}$$

first order conditions gives:

\begin{align*}
    Q_1&:\qquad -3 Q_1-Q_2+3 = 0 \qiq Q_1 =\frac{1}{3} \left(3-Q_2\right)\\ Q_2&:\qquad -Q_1-3 Q_2+5 = 0 \qiq Q_2 = \frac{1}{3} \left(5-Q_1\right)
\end{align*}
Solving the system of equations:

$$Q_1 = \frac{1}{2} \qquad Q_2 = \frac{3}{2}$$

plugging in to the demand equations we get:

$$\boxed{P_1 = \frac{5}{2}} \qquad \boxed{P_2 = \frac{7}{2}}$$

\end{proof}

\end{document}