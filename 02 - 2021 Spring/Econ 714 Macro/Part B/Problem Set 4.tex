
\documentclass[12pt]{article}
\usepackage[utf8]{inputenc}
%\usepackage[left=3cm, right=2.5cm, top=2.5cm, bottom=2.5cm]{geometry}e}
\usepackage[utf8]{inputenc}
\usepackage[spanish,english]{babel}
\usepackage{apacite}
\usepackage[round]{natbib}
\usepackage{hyperref}
\usepackage{float}
\usepackage{svg}
\usepackage[margin = 1in, top=2cm]{geometry}% Margins
\setlength{\parindent}{2em}
\setlength{\parskip}{0.2em}
\usepackage{setspace} % Setting the spacing between lines
\usepackage{amsthm, amsmath, amsfonts, mathtools, amssymb, bm} % Math packages 
\usepackage{svg}
\usepackage{graphicx}
\usepackage{pgfplots}
\usepackage{epstopdf}
\usepackage{subfig} % Manipulation and reference of small or sub figures and tables
\usepackage{hyperref} % To create hyperlinks within the document
\spacing{1.15}
\usepackage{appendix}
\usepackage{xcolor}
\usepackage{cancel}
\usepackage{enumerate}
\usepackage[shortlabels]{enumitem}
\usepackage{optidef}
\usepackage{etoolbox}

\usepackage[round]{natbib}
%\bibliographystyle{plainnat}
\bibliographystyle{apacite}


\newtheorem{defin}{Definition.}
\newtheorem{teo}{Theorem. }
\newtheorem{lema}{Lemma. }
\newtheorem{coro}{Corolary. }
\newtheorem{prop}{Proposition. }
\theoremstyle{definition}
\newtheorem{examp}{Example. }
\newtheorem{problem}{Problem}
\newtheorem{subproblem}{}[problem]
% \numberwithin{problem}{subsection} 

% \AtBeginEnvironment{problem}{\color{gray}}
% \AtBeginEnvironment{subproblem}{\color{gray}}
\AtBeginEnvironment{proof}{\color{red}}

\newcommand{\card}{\operatorname{card}}
\newcommand{\qiq}{\qquad \implies \qquad}
\newcommand{\qiffq}{\qquad \iff \qquad}
\newcommand{\qaq}{\qquad \textbf{and} \qquad}
\newcommand{\qoq}{\qquad \textbf{or} \qquad}
\newcommand{\settf}{\text{ \emph{:} }}
\newcommand{\chbox}{\makebox[0pt][l]{$\square$}\raisebox{.15ex}{\hspace{.9em}}}
\newcommand{\cchbox}{\makebox[0pt][l]{$\square$}\raisebox{.15ex}{\hspace{0.1em}$\checkmark$}}

\title{Problem Set 4}
\author{Mitchell Valdés-Bobes}
\date{\today}
\title{Problem Set 4}
\author{Mitchell Valdés-Bobes}
\date{\today}

\begin{document}

\maketitle

\begin{problem}[50 points] Suppose that an infinitely lived government has to finance a fixed stream of expenditures, $\left\{{g}_{{t}}\right\}_{{t} \geqslant 0}$ and can only use consumption taxes for this purpose. Assume that the representative consumer has the utility function:
$$
\sum_{t=0}^{\infty} \beta^{t}\left[\frac{c_{t}^{1-\sigma}}{1-\sigma}+v\left(l_{t}\right)\right]
$$
where $c_{t}$ is the consumption in period $t$ and $l_{t}$ is leisure in period $t$. Assume that $\sigma \geqslant 0$ and $v$ is an increasing function. Also assume that the production function, $F(K, L)$ satisfies all the standard assumptions (i.e., CRS, etc.), that the representative household has an initial endowment of the capital stock, $k_{0}, l_{t} \leqslant 1$ and that capital is subject to the usual law of motion, $k_{t+1}=(1-\delta) k_{t}+x_{t}$.Set up the Ramsey Problem for this economy, and show that the optimal policy is to set the consumption tax at a constant rate from period one onwards (i.e., show that $\tau_{{ct}}^{{RP}}=\tau_{{ct}+1}^{{RP}}$ for all ${t} \geqslant 1$.
\end{problem}
\begin{proof}[Answer]
\begin{itemize}
    Denote $n_t = 1-l_t$.
    \item Households preferences $$\sum_{t=1}^{\infty}\beta^t \left[\frac{c_t^{1-\sigma}}{1-\sigma} + v(1-n_t)\right]$$
    \item Households Budget Constraints:
    \begin{equation}\tag{HHBC}\label{hhbc}(1+\tau^c_t)+k_t + b_t = w_tn_t+(1-\delta+r_t)k_{t-1}+R^b_t b_{t-1}\end{equation}
    \item Firm's problem:$$\max_{k_{t-1},n_t}\quad F(k_{t-1}, n_t) + w_tn_t+r_tk_{t-1}$$
    \item Government's Budged Constraints:\begin{equation}\tag{GBC}\label{gbc}b_t + \tau_{t}^c c_t = R^b_t b_{t-1} -g_t \end{equation}
\end{itemize}

From the Firm's Problem we know that in equilibrium:

\begin{align*}
    r_t &= F_k(k_{t-1}, n_t)\\
    w_t &= F_n(k_{t-1}, n_t)
\end{align*}

Substituting \eqref{gbc} in to \eqref{hhbc} we obtain:

 \begin{equation}\tag{RC}\label{rc}(1+\tau^c_t)c_t+k_t + g_t = \underbrace{w_tn_t+r_t k_{t-1}}_{=F(k_{t-1}, n_t)\footnotemark} + (1-\delta)k_{t-1}\end{equation}
\footnotetext{ by Euler's Theorem for homogeneous functions}

Denote $R^k_t = 1+r_t-\delta)$. From the agent problem's first order conditions we get:

\begin{align*}
    \beta^t c_t^{-\sigma} &= p_t(1+\tau_t^c)\\
    \beta^t v'(1-n_t) &= p_tw_t\\
    0& = (p_t - p_{t+1}R^k_t)k_t\\
    0& = (p_t - p_{t+1}R^b_t)b_t
\end{align*}

where $p_t$ is the Lagrange multiplier associated with the $t-th$ restriction. Multiplying \eqref{hhbc} by $p_t$, summing over $t$ and using the first order conditions we obtain the following implementability constraint:

\begin{equation}\tag{IC}\label{impconst}
    \sum_{t=1}^\infty \beta^t\Big(c^{1-\sigma} - n_t v'(1-n_t)\Big) = C_0
\end{equation}

where $C_0$ is a constants that depends on the values of the allocation, the prices and the consumption capital at time $t=0$, note that I will drop this constant from the optimization problem that characterize the Ramsey Problem.

\begin{defin}[Ramsey Problem] The Ramsey Problem solves:

$$\max_{c_t, n_t}\quad \sum_{t=1}^\infty \beta^t\left[\frac{c^{1-\sigma}}{1-\sigma}+v(1-n_t)+\lambda((c^{1-\sigma} - n_t v'(1-n_t))\right]$$

Subject to \eqref{rc}.
\end{defin}

Taking the first order conditions associated to consumption and capital on the Ramsey Problem we have:

\begin{align*}
    c_t:&\qquad \beta^tc_t^{-\sigma}(1+\lambda(1-\sigma)) = \mu_t(1+\tau_t^c)\\
    k_t:&\qquad \mu_t = \mu_{t+1}(1-\delta r_t)
\end{align*}

Where $\mu_t$ is the Lagrange multiplier associated with the \eqref{rc} constraint. Using both conditions we arrive at the following Euler Equation:

\begin{equation}\tag{EE}\label{ee}
    \frac{1+\tau_{t+1}^c}{1+\tau_{t}^c}=\beta\frac{c_{t+1}^{-\sigma}}{c_{t}^{-\sigma}}(1-\delta+r_t)
\end{equation}

Note that if the government sets $\tau_t^c = \tau_{t+1}^1$ then the \eqref{ee} becomes:

$$c_{t}^{-\sigma}=\frac{1+\tau_{t+1}^c}{1+\tau_{t}^c}=\beta c_{t+1}^{-\sigma}(1-\delta+r_t)$$

which is the Euler Equation of the problem without taxes.

This means that if the government chooses constant taxes on capital those taxes wont distort agents decisions, therefore this is the optimal policy.  

\end{proof}

\begin{problem}
Problem 2 (50 points) Consider a cash-credit goods economy with preferences given by
$$
\log c_{1 {t}}+\alpha \log c_{2 {t}}+\gamma \log \left(1-n_{{t}}\right)
$$
where $n_{t}$ is the time spent in market activities. The resource constraint is
$$
{c}_{1 {t}}+{c}_{2 {t}}=n_{{t}}
$$
The cash-in-advance constraint is
$$
p_{t} c_{1 t} \leqslant M_{t}
$$
The budget constraint for the hhld at the beginning of the period is
$$
M_{{t}}+{B}_{{t}} \leqslant\left(M_{{t}-1}-{p}_{{t}-1} {c}_{1 {t}-1}\right)-{p}_{{t}-1} {c}_{2 {t}-1}+w_{{t}-1} {n}_{{t}-1}+{R}_{{t}-1} {~B}_{{t}-1}-{T}_{{t}}
$$

where $T_{t}$ denotes lump-sum taxes and all the terms are as we discussed in class. The government conducts monetary policy to keep the interest rate fixed at some level ${R}$ in all periods.

\end{problem}

\begin{subproblem}
    Define a competitive equilibrium.
\end{subproblem}
\begin{proof}[Answer]
\begin{defin} A Competitive Equilibrium is an allocation $\left(c_{1,t}, c_{2,t}, l_t, M_t, B\_t\right)$ and a price system $\left(p_t, R_t\right)$ such that

\begin{enumerate}
    \item Given prices the allocation solves the following household problem:
    \begin{align*}
        \max&\quad \sum_{t=1}^\infty{\log c_{1 {t}}+\alpha \log c_{2 {t}}+\gamma \log \left(1-n_{{t}}\right)}\\
        \text{s.t:}&\quad M_{{t}}+{B}_{{t}} \leqslant\left(M_{{t}-1}-{p}_{{t}-1} {c}_{1 {t}-1}\right)-{p}_{{t}-1} {c}_{2 {t}-1}+w_{{t}-1} {n}_{{t}-1}+{R}_{{t}-1} {~B}_{{t}-1}-{T}_{{t}}\\
        &\quad p_{t} c_{1 t} \leqslant M_{t}
    \end{align*}
    \item The following Budget Constraint for the Government Holds:
    $$M_t-M_{t-1}+B_t=R_{t-1}B_{t-1}+p_{t-1}g_{t-1} + T_t$$
    
    \item The resource constraint holds:
    $$c_{1,t}+c_{2,t} = n_t$$
\end{enumerate}

\end{defin}
\end{proof}


\begin{subproblem}
    What happens to $n_{t}$ as $R$ increases. Prove your result.
\end{subproblem}
\begin{proof}[Answer]
Consider the Lagrangian of the agent's problem:
\begin{align*}
\mathcal{L}(\cdot) &= \sum_{t=1}^\infty{\log c_{1 {t}}+\alpha \log c_{2 {t}}+\gamma \log \left(1-n_{{t}}\right)} \\&+ \lambda_t\left( M_{{t}}+{B}_{{t}}- \left(M_{{t}-1}-{p}_{{t}-1} {c}_{1 {t}-1}\right)-{p}_{{t}-1} {c}_{2 {t}-1}+w_{{t}-1} {n}_{{t}-1}+{R}_{{t}-1} {~B}_{{t}-1}-{T}_{{t}}\right)\\
&+\mu_t( p_{t} c_{1 t} - M_{t})
\end{align*}
 This gives the following conditions for agent optimization:
 
 \begin{align*}
     c_{1, t}:&\quad \frac{\beta^{t}}{c_{1 t}}=\lambda_{\epsilon+1} p_{t}+\mu_{t} p_{t}\\
     c_{2 t}:&\quad  \frac{\beta^{t} \alpha}{c_{2t}}=\lambda_{t+1} p_{t}\\
     B_{t}:&\quad \lambda_{t}=\lambda_{t+1} R_{t}\\
     M_{t}:&\quad \lambda_{t}-\mu_{t}=\lambda_{t+1}\\
     n_{t}:&\quad -\frac{\beta^{t} \gamma}{1-n_{t}}=\lambda_{t+1} w_{t}
 \end{align*}

Note that the interest rate is constant, $R_t = R$. Using the FOC's we can obtain the following:

$$c_{1t} = \frac{c_{2t}}{\alpha R} \qaq c_{2t} = \frac{\alpha (1-n_t)w_t}{\gamma}$$

Then plug this in to the resource constraint and we get:

$$ \frac{(1-n_t)w_t}{\gamma R} +  \frac{\alpha (1-n_t)w_t}{\gamma} = n_t$$

Thus we get the following expression for $n_t$:

$$n_t = \frac{w_t (\alpha  R+1)}{\gamma  R+\alpha  R w_t+w_t} \qiq \frac{\partial n_t}{\partial R} = -\frac{\gamma  w_t}{(\gamma  R+\alpha  R w_t+w_t)^2}<0$$

Then we can conclude that as $R$ increases $n_t$ will decrease. The interpretation is that if $R$ then the household's savings in bonds will return more in the next period then the household can work less.


\end{proof}




% \bibliography{references}
\end{document}
