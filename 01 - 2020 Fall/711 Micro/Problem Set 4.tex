\documentclass{article}
\usepackage[utf8]{inputenc}
\documentclass[12pt]{article}
%\usepackage[left=3cm, right=2.5cm, top=2.5cm, bottom=2.5cm]{geometry}e}
\usepackage[utf8]{inputenc}
\usepackage[spanish,english]{babel}
\usepackage{apacite}
\usepackage[round]{natbib}
\usepackage{hyperref}
\usepackage{float}
\usepackage{svg}
\usepackage[margin = 1in, top=2cm]{geometry}% Margins
\setlength{\parindent}{2em}
\setlength{\parskip}{0.2em}
\usepackage{setspace} % Setting the spacing between lines
\usepackage{amsthm, amsmath, amsfonts, mathtools, amssymb, bm} % Math packages 
\usepackage{svg}
\usepackage{graphicx}
\usepackage{pgfplots}
\usepackage{epstopdf}
%\usepackage{subfig} % Manipulation and reference of small or sub figures and tables
\usepackage{hyperref} % To create hyperlinks within the document
\spacing{1.15}
\usepackage{appendix}
\usepackage{xcolor}
\usepackage{cancel}
\usepackage{enumerate}
\usepackage{subcaption}
\usepackage[shortlabels]{enumitem}


\usepackage[round]{natbib}
%\bibliographystyle{plainnat}
\bibliographystyle{apacite}


\newtheorem{defin}{Definition.}
\newtheorem{teo}{Theorem. }
\newtheorem{lema}{Lemma. }
\newtheorem{coro}{Corolary. }
\newtheorem{prop}{Proposition. }
\theoremstyle{definition}
\newtheorem{examp}{Example. }
\newtheorem{problem}{Problem}
% \numberwithin{problem}{subsection} 

\newcommand{\card}{\operatorname{card}}
\newcommand{\qiq}{\qquad \implies \qquad}
\newcommand{\qiffq}{\qquad \iff \qquad}
\newcommand{\qaq}{\qquad \textbf{and} \qquad}
\newcommand{\qoq}{\qquad \textbf{or} \qquad}
\newcommand{\settf}{\text{ \emph{:} }}
\newcommand{\chbox}{\makebox[0pt][l]{$\square$}\raisebox{.15ex}{\hspace{.9em}}}
\newcommand{\cchbox}{\makebox[0pt][l]{$\square$}\raisebox{.15ex}{\hspace{0.1em}$\checkmark$}}

\title{Problem Set 4}
\author{Mitchell Valdés-Bobes}
\date{October 5, 2020}

\begin{document}

\maketitle
\begin{problem}[Choice rules from preferences]
Let $X$ be a choice set and $\succsim$ a complete and transitive preference relation on $X .$ Show that the
choice rule induced by $\succsim$
$$
C(A, \succsim) \quad=\quad\{x \in A: x \succsim y \quad \forall y \in A\}
$$
must satisfy the Weak Axiom of Revealed Preference (WARP).
\end{problem}

\begin{proof}[Answer]
Let  $A,B \subseteq X$ and $x,y\in A\cap B$. If:

$$x\in C(A, \succsim) \qaq  y\in C(B, \succsim)$$

Then since $\succsim$ is complete:

$$x\succsim z \: \forall z\in A \qaq y\succsim z \: \forall z\in B$$

since $A\cup B \subseteq A$$ and $A\cup B \subseteq B$$ then

$$x\succsim z \: \forall z\in A\cup B \qaq y\succsim z \: \forall z\in A \cup B$$

in particular

$$x\succsim y  \qaq y\succsim x$$

By transitivity:

$$x\succsim z \: \forall z\in B \qaq y\succsim z \: \forall z\in A$$

Therefore

$$x\in C(B, \succsim) \qaq  y\in C(A, \succsim)$$

Thus proving that the choice rule $C(\cdot, \succsim)$ satisfies the Weak Axiom of Revealed Preference (WARP).

\end{proof}

\begin{problem}[Preferences from choice rules]
Let $X$ be a choice set and $C: \mathcal{P}(X) \rightarrow \mathcal{P}(X)$ a nonempty choice rule. Show that if $C$ satisfies WARP, then the preference relation $\succsim_C$ defined on $X$ by
$$x \succsim_C y \quad \text{ if and only if there exists a set } A \subseteq X \text{ such that } x, y \in A \text{ and } x \in C(A)$
is complete and transitive, and that the choice rule it induces, $C(\cdot, \succsim_C)$, is equal to $C$.
\end{problem}

\begin{proof}[Answer]
We start by proving that $\succsim_C$ is \textbf{complete:}

Pick any $x,y\in X$ let $A=\{x,y\}$ since $C(A)\neq \emptyset$ and $C(A)\subseteq A$ then:

$$ x\in C(A) \qoq y\in C(A)$$

Therefore

$$x \succsim_C y \qoq y \succsim_C x$$

Then $\succsim_C$ is \textbf{complete}.

To prove that $\succsim_C$ is \textbf{transitive}, pick any $x,y,z \in X$ such that $x\succsim_C y$ and $y\succsim_C z$ then:

$$\exists A \in  \mathcal{P}(X) \text{ such that } x,y\in A \qaq x \in C(A)$$
$$\exists B \in  \mathcal{P}(X) \text{ such that } y,z\in A \qaq y \in C(B)$$

Let $D=\{x,y,z\}$, we know that $C(D)\neq \emptyset$ and $C(D)\subseteq D$ then at least one of the following must be true:
\begin{enumerate}[(a)]
    \item $x \in C(D)$ then $x\succsim_C z$.
    \item $y \in C(D)$ which implies $x,y \in D\cup A$ and by WARP, then $x \in C(D)$ then $x\succsim_C z$.
    \item $z \in C(D)$ which implies $y,z \in D\cup B$ by WARP, then $y \in C(D)$ then $y\succsim_C z$ by (b), then $x \in C(D)$ then $x\succsim_C z$..
\end{enumerate}

therefore $\succsim_C$ is \textbf{transitive}.

Finally we need to prove that the choice rule $C(\cdot, \succsim_C)$, is equal to $C$, this is:

$$\forall A\in\mathcal{P}(x) \qquad C(A, \succsim_C)=C(A)$$

Pick any $x\in C(A)$ then
$$\forall y\in A \quad x\succsim_C y \qiq x \in C(A, \succsim_C) \qiq \boxed{C(A)\subseteq C(A, \succsim_C)}$$

Pick any $x\in C(A, \succsim_C)\subseteq A$ suppose that $x\not \in C(A)$ since $C(A)\neq \emptyset$ there is $y\in C(A)$ then but by definition of $C(A, \succsim_C) $ then $x\succsim_C y$, then:

$$\exists B \in \mathcal{P}(X) \text{ such that } x,y\in B\qaq x\in C(B)$

Since $x,y\in A\cup B $ then by WARP $x \in C(A)$ therefore $\boxed{C(A, \succsim_C)\subseteq C(A)}$

Then 
$$\boxed{C(A, \succsim_C) = C(A)}$$

\end{proof}

\begin{problem}[Choice over finite sets]
Let $X$ be a finite set, and $\succsim$ a complete and transitive preference relation on $X$.
\begin{enumerate}[(a)]
    \item Show that the induced choice rule $C(\cdot, \succsim)$ is nonempty $-$ that $C(A, \succsim) \neq \emptyset$ if $A \neq \emptyset$
    \item Show that a utility representation exists.
\end{enumerate}
\end{problem}
\begin{proof}[Answer]
\textbf{Part (a)}
Consider $X$ finite and $A \subseteq X$, $A=\{x\}$ since preferences are complete then $x\succsim x$ therefore $x\in C(A, \succsim)$ therefore $C(A, \succsim)\neq \emptyset$.
 Now suppose that for every set $A_n$ with $n$ elements $C(A_n, \succsim)\neq \emptyset$ we will prove that this implies that for every set $A_{n+1}$ with $n+1$ elements $C(A_{n+1}, \succsim)\neq \emptyset$. Let
 
 $$A_{n+1} = \{x_1, ...,x_n, x_{n+1}\} = \underbrace{\{x_1, ...,x_n\}}_{A_n} \cup \{x_{n+1}\}

Since $C(A_n, \succsim)\neq \emptyset$ there is $x\in  A_{n}$ such that $x\succsim X_i$ for $i =1,...,n$ since $\succsim$ is complete, then either 
$$x\succsim x_{n+1} \qiq x \in C(A_{n+1}, \succsim) \qiq C(A_{n+1}, \succsim) \neq \emptyset$$
or
$$x_{n+1}\succsim x \qiq x_{n+1}\succsim\footnote{by transitivity}x_i \:\forall i=1,..,n\qiq  x_{n+1}\in C(A_{n+1}, \succsim) \qiq C(A_{n+1}, \succsim)\neq \emptyset$$
 
\textbf{Part (b)}
\begin{itemize}
    \item If $|X| = 1$ this is $X=\{x\}$ and let $u:X \to \{1\}$ then $u(x)=1$ it is trivial that $u$ is a utility representation of $\succsim$.
    \item Assume that if $|X| < n$ then there is $u:X \to \{1,..,|X|\}$ that is a utility representation of $\succsim$.
    \item Consider $X$ a set with $|X|=n+1$ By \textbf{Part (a)} we know that $|C(X, \succsim)|\geq 1$ therefore the complement of $C(X, \succsim)$ has at most $n$ elements. We can guarantee then that there is a utility representation $u$ of $\succsim$ on $C(X, \succsim)^c$ then we can define the function:
    $$\hat{u}(x)=\left\{ \begin{array}{cc}
        |C(X, \succsim)^c| + 1 & x\in C(X, \succsim) \\
        u(x) & \text{otherwise}
    \end{array}$$
    We need to prove that this is in fact a utility representation of $\succsim$ i.e:
    $$x\succsim y \qiffq \hat{u}(x)\geq \hat{u}(y)$$
    Suppose that $x,y \in C(X, \succsim)^c$ then since $u$ is a utility representation of $\succsim$ in $C(X, \succsim)^c$:
    $$x\succsim y \qiffq u(x)\geq u(y) \qiffq \hat{u}(x)\geq \hat{u}(y)$$
    Suppose that $x\in C(X, \succsim)$ and $,y \in C(X, \succsim)^c$ then $x\succsim y$ and $\hat{u}(x)=|C(X, \succsim)^c|>u(y)=\hat{u}(y)$.
    Finally suppose that $x,y \in C(X, \succsim)$
    $$\hat{u}(x)=|C(X, \succsim)^c|=\hat{u}(y) \qiffq x\succsim y \qaq y \succsim x$$
\end{itemize}

Then using \textit{strong} induction we have proved that if $X$ is a finite set and $\succsim$ is complete and transitive then the utility representation always exist. 






\end{proof}

\end{document}
