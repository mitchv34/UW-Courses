\documentclass{article}
\usepackage[utf8]{inputenc}
\documentclass[12pt]{article}
%\usepackage[left=3cm, right=2.5cm, top=2.5cm, bottom=2.5cm]{geometry}e}
\usepackage[utf8]{inputenc}
\usepackage[spanish,english]{babel}
\usepackage{apacite}
\usepackage[round]{natbib}
\usepackage{hyperref}
\usepackage{float}
\usepackage{svg}
\usepackage[margin = 1in, top=2cm]{geometry}% Margins
\setlength{\parindent}{2em}
\setlength{\parskip}{0.2em}
\usepackage{setspace} % Setting the spacing between lines
\usepackage{amsthm, amsmath, amsfonts, mathtools, amssymb, bm} % Math packages 
\usepackage{svg}
\usepackage{graphicx}
\usepackage{pgfplots}
\usepackage{epstopdf}
\usepackage{subfig} % Manipulation and reference of small or sub figures and tables
\usepackage{hyperref} % To create hyperlinks within the document
\spacing{1.15}
\usepackage{appendix}
\usepackage{xcolor}
\usepackage{cancel}
\usepackage{enumerate}
\usepackage[shortlabels]{enumitem}


\usepackage[round]{natbib}
%\bibliographystyle{plainnat}
\bibliographystyle{apacite}


\newtheorem{defin}{Definition.}
\newtheorem{teo}{Theorem. }
\newtheorem{lema}{Lemma. }
\newtheorem{coro}{Corolary. }
\newtheorem{prop}{Proposition. }
\theoremstyle{definition}
\newtheorem{examp}{Example. }
\newtheorem{problem}{Problem}
% \numberwithin{problem}{subsection} 

\newcommand{\card}{\operatorname{card}}
\newcommand{\qiq}{\qquad \implies \qquad}
\newcommand{\qiffq}{\qquad \iff \qquad}
\newcommand{\qaq}{\qquad \textbf{and} \qquad}
\newcommand{\qoq}{\qquad \textbf{or} \qquad}
\newcommand{\settf}{\text{ \emph{:} }}
\newcommand{\chbox}{\makebox[0pt][l]{$\square$}\raisebox{.15ex}{\hspace{.9em}}}
\newcommand{\cchbox}{\makebox[0pt][l]{$\square$}\raisebox{.15ex}{\hspace{0.1em}$\checkmark$}}

\title{Problem Set 1}
\author{Mitchell Valdés-Bobes}
\date{September 14, 2020}

\begin{document}

\maketitle

\begin{problem}
For two events $A, B \subseteq S,$ prove that $A \cup B=(A \cap B) \cup\left(\left(A \cap B^{c}\right) \cup\left(B \cap A^{c}\right)\right)$.

\end{problem}

\begin{proof}[Answer]

By associativity of the union of sets we know that:

\begin{equation}\label{eq1}
(A \cap B) \cup\left(\left(A \cap B^{c}\right) \cup\left(B \cap A^{c}\right)\right) = ((A \cap B) \cup \left(A \cap B^{c}\right)) \cup\left(B \cap A^{c}\right)    
\end{equation}

we also know from class that:
\begin{equation}\label{eq2}
A = (A\cap B) \cup (A\cap B^c)    
\end{equation}

plugin \eqref{eq2} in \eqref{eq1} we get:

\begin{equation}\label{eq3}
    (A \cap B) \cup\left(\left(A \cap B^{c}\right) \cup\left(B \cap A^{c}\right)\right) = A \cup\left(B \cap A^{c}\right)
\end{equation}

then applying the distributive property of unions and intersections to \eqref{eq3} we get:

$$(A \cap B) \cup\left(\left(A \cap B^{c}\right) \cup\left(B \cap A^{c}\right)\right) = (A \cup B) \cap( A \cup A^{c}) = (A\cup B) \cap (A\cup A^c)=(A\cup B) \cap S = A\cup B$$

\end{proof}

\begin{problem}
Prove that $P(A \cup B)=P(A)+P(B)-P(A \cap B)$

\end{problem}

\begin{proof}[Answer]
By \textbf{Problem 1} we know:
 $A \cup B=(A \cap B) \cup\left(\left(A \cap B^{c}\right) \cup\left(B \cap A^{c}\right)\right)$ and we also know that $(A \cap B),\: (\left(A \cap B^{c}\right)$ and $\left(B \cap A^{c}\right)\right)$ are disjoint, therefore:
 \begin{equation}\label{eq4}
     P(A \cup B)=P(A \cap B) +P\left(A \cap B^{c}\right) + P\left(B \cap A^{c}\right)
 \end{equation}
 
 Using the same idea as in \eqref{eq2} we have:
 
 \begin{align*}\label{eq5}\tag{5}
     A = (A\cap B) \cup (A\cap B^c)  &\qiq  P(A) = P(A\cap B) + P(A\cap B^c) \\ &\qiq   P(A\cap B^c)  = P(A) - P(A\cap B)
\end{align*}

and

\begin{align*}\label{eq6}\tag{6}
     B = (B\cap A) \cup (B\cap A^c)  &\qiq  P(B) = P(B\cap A) + P(B\cap A^c) \\ &\qiq   P(B\cap A^c)  = P(B) - P(B\cap A)
\end{align*}

then, plugin \eqref{eq5} and \eqref{eq6} in \eqref{eq4} we get:
\begin{equation*}
    P(A \cup B)= \cancel{P(A \cap B)} +P(A) - \cancel{P(A\cap B)} + P(B) - P(B\cap A) = P(A) + P(B) - P(B\cap A)
\end{equation}


\end{proof}
\begin{problem}
Suppose that the unconditional probability of a disease is $0.0025 .$ A screening
test for this disease has a detection rate of $0.9,$ and has a false positive rate
of $0.01 .$ Given that the screening test returns positive, what is the conditional
probability of having the disease?

\end{problem}

\begin{proof}[Answer]

Let event $A$ be have the disease and event $B$ be screening test is positive. We know the following about the probabilities of those events:

$$P(A) = 0.0025 \qquad P(A^c) = 1-0.0025=0.9975$$
$$P(B|A) = 0.9 \qquad P(B|A^c) = 0.01$$

We are interested in $P(A|B)$, we can use the Bayes rule:

$$P(A \mid B)=\frac{P(B \mid A) P(A)}{P(B \mid A) P(A)+P\left(B \mid A^{c}\right) P\left(A^{c}\right)} = \frac{0.9 \times 0.0025}{0.9 \times 0.0025 + 0.01\times0.9975}=0.184049$$

\end{proof}

\begin{problem}
Suppose that a pair of events $A$ and $B$ are mutually exclusive, i.e., $A \cap B=\emptyset$,
and that $P(A)>0$ and $P(B)>0 .$ Prove that $A$ and $B$ are not independent.
\end{problem}

\begin{proof}[Answer]
We will prove the result by way of contradiction; suppose that $A$ and $B$ are independent, then:

$$P(A\cap B) = P(A)P(B)$$

but we have that:

$$A\cap B=\emptyset \qiq P(A\cap B)=P(\emptyset)=0$$

then

$$P(A)P(B)=0 \qiif P(A)=0 \: \lor \: P(B)=0$$

but this is a contradiction with the hypothesis that $P(A)>0$ and $P(B)>0$.

\end{proof}

\begin{problem}
(Conditional Independence) Sometimes, we may also use the concept of conditional
independence. The definition is as follows: let $A, B, C$ be three events with positive probabilities. Then $A$ and $B$ are independent given $C$ if $P(A \cap B \mid C)=$
$P(A \mid C) P(B \mid C) . \quad$ Consider the experiment of tossing two dice. Let $A=$
$\{$ First die is 6$\}, B=\{$ Second die is 6$\},$ and $C=\{$ Both dice are the same $\} .$
\begin{enumerate}[(a)]
    \item  Show that $A$ and $B$ are independent (unconditionally), but $A$ and $B$ are dependent given $C$.
    \item Consider the following experiment: let there be two urns, one with 9
black balls and 1 white balls and the other with 1 black ball and 9 white
balls. First randomly (with equal probability) select one urn. Then then
take two draws with replacement from the selected urn. Let $A$ and $B$ be
drawing a black ball in the first and the second draw, respectively, and
let $C$ be the event that urn 1 is selected. Show that $A$ and $B$ are not
independent, but are conditionally independent given $C$.
\end{enumerate}

\end{problem}

\begin{proof}[Answer]
\textbf{Part (a)} We start by writing down explicitly the events:
$$A = \{(6, 1), (6, 2), (6, 3), (6, 4), (6, 5), (6, 6)\}$$
$$B = \{(1, 6), (2, 6), (3, 6), (4, 6), (5, 6), (6, 6)\}$$
$$C = \{(1, 1), (2, 2), (3, 3), (4, 4), (5, 5), (6, 6)\}$$

then

$$P(A\cap B) = P(\{6,6\}) = \frac{1}{6^2}$$

$$P(A) = P(B) = P(C) = \frac{6}{6^2}=\frac{1}{6} \qiq P(A)P(B) = \frac{1}{6^2} = P(A\cap B)$$

then $A$ and $B$ must be independent events.

Now,
$$P(A\cap B|C) = \frac{P((A\cap B)\cap C)}{C}$$

$$P((A\cap B)\cap C) = P(\{(6,6)\}) \qiq P((A\cap B)\cap C)=\frac{1}{6^2} \qiq P(A\cap B|C) = \frac{\frac{1}{6^2}}{\frac{1}{6}}=\frac{1}{6}$$

and 

$$P(A|C)= P(B|C) = P(\{(6,6)\})=\frac{1}{6^2} \qiq P(A|C)P(B|C)\neq P((A\cap B)|C)$$

thus proving that $A$  and $B$ are conditionally dependent given $C$.

\textbf{Part (b)}
We start by saying that 

$$P(A|C) = \frac{9}{10} = P(B|C)$$
$$P(A|C^c) = \frac{1}{10} = P(B|C^c)$$

and since draws are taken with replacement

$$P(A\cap B|C) = \frac{81}{100} \qaq P(A\cap B|C^c) = \frac{1}{100}$$

then

$$P(A) = P(A|C)P(C) + P(A|C^c)P(C^c)=\frac{1}{2}(P(A|C)P(C) + P(A|C^c))=\frac{1}{2}$$

$$P(B) = P(B|C)P(C) + P(B|C^c)P(C^c)=\frac{1}{2}(P(B|C)P(C) + P(B|C^c))=\frac{1}{2}$$

and

$$P(A\cap B) = P(A\cap B|C)P(C) + P(A\cap B|C^c)P(C^c)=\frac{1}{2}(P(A\cap B|C)P(C) + P(A\cap B|C^c))=\frac{41}{100}$$

then

$$P(A\cap B )\neq P(A)P(B) \qqad A\text{and }B\text{ are not independent.}$$

To see that $A$ and $B$ are independent given $C$:

$$P(A\cap B | C) = \frac{81}{100} = \frac{9}{10}\frac{9}{10} = P(A)P(B)$$



\end{proof}


\begin{problem}
A CDF $F_{X}$ is stochastically greater than a CDF $F_{Y}$ if $F_{X}(t) \leq F_{Y}(t)$ for all $t$ and $F_{X}(t)<$ $F_{Y}(t)$ for some $t .$ Prove that if $X \sim F_{X}$ and $Y \sim F_{Y},$ then
$$
P(X>t) \geq P(Y>t) \text { for every } t
$$
and
$$
P(X>t)>P(Y>t) \text { for some } t
$$
that is, $X$ tends to be bigger than $Y$.
\end{problem}

\begin{proof}[Answer]

\begin{align*}
    P(X>t) \geq P(Y>t) &\qiffq 1 - P(X\leq t) \geq 1 -  P(Y\leq t)\\
    &\qiffq - P(X\leq t) \geq -  P(Y\leq t)\\
    &\qiffq  P(X\leq t) \leq  P(Y\leq t)\\
    &\qiffq F_{X}(t) \geq F_{Y}(t)
\end{align*}

Note that this have to happen for every $t$ and that for some $t$ the inequality is strict, thus proving the result. 

\end{proof}

\begin{problem}
Show that the function $F_{X}(x)=\left\{\begin{array}{ll}0 & \text { if } x<0 \\ 1-\exp (-x) & \text { if } x \geq 0\end{array}\right.$ is a CDF, and find $f_{X}(x)$ and $F_{X}^{-1}(y)$
\end{problem}

\begin{proof}[Answer]
To show that $F_X$ is a CDF we will pro the following properties:
\begin{enumerate}[(i)]
    \item $\lim_{x\to -\infty}F_X(x) = \lim_{x\to -\infty}0 = 0$
    \item $\lim_{x\to -\infty}F_X(x) = \lim_{x\to -\infty} 1-\exp (-x) = 1 - \lim_{x\to -\infty}\exp (-x) = 1-0 = 1$
    \item Take $x<y$ then:
    \begin{align*}
        \text{if } x<y< 0 &\qiq F_X(x)=F_X(y)\\
        \text{if } x< 0 \leq y &\qiq F_X(x) =  0 \leq 1 - \exp(-y) = F_X(y)\\
        \text{if }  0 \leq x < y &\qiq F_X(x)-F_X(y) = \exp(-x)-\exp(-y) < 0 \qiq F_X(x)<F_X(y)
    \end{align*}
    therefore $F_X$ is not decreasing.
    \item For $x<0$ and $x>0$ $F_X$ is a continuous function, then it must be right-continuous, the only point that we need to check is $x=0$:
    $$\lim_{x\to -0^+}F_X(x) = \lim_{x\to 0^+} 1-\exp (-x) = 1 - \lim_{x\to -0^+}\exp (-x) = 1-1 = 0=F_X(0)$$
    then $F_X(x)$ is right-continuous.
\end{enumerate}

To fin the PDF $f_X$ we use the fact that in general, when $F_X$ is continuous, $f_X(x)$ is the function which satisfies
$$
F_X(x)=\int_{\infty}^{x} f_X(t) d t
$$
using calculus fundamental theorem

$$
f_X = \frac{\partial}{\partial x}F_X(x) = \begin{cases} 0 & \text{if } x<0\\\exp(-x) &\text{if } x\geq0
\end{cases}$$

To obtain the inverse CDF $F_X^{-1}(x)$ we need to find a function that satisfies that

$$x = F_X(F_X^{-1}(x))$$

The problem is that $F_X(x)$ has flat portions then we need to restrict $F_X$ to it's support set:

$$\mathcal{X} = \{x \settf f(x)> 0 \} = [0,\infty)$$

in this set the function is inyective and surjective therefore invertible, with inverse:

$$F_X^{-1}(x) = -\log{(1-x)} \quad \forall x \in [0,1)$$

\end{proof}

\!\(TraditionalForm\`
\*FractionBox[\(\[DifferentialD]\), \("\[DifferentialD]" x\)] \((\*
TagBox[
RowBox[{
RowBox[{"cos", "(", "x", ")"}], " ", 
RowBox[{"exp", "(", "x", ")"}]}],
HoldForm])\) == 
\*SuperscriptBox[\(E\), \(x\)]\ \((cos(x) - sin(x))\)\)


\end{document}
