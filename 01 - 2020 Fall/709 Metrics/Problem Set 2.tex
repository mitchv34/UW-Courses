\documentclass{article}
\usepackage[utf8]{inputenc}
\documentclass[12pt]{article}
%\usepackage[left=3cm, right=2.5cm, top=2.5cm, bottom=2.5cm]{geometry}e}
\usepackage[utf8]{inputenc}
\usepackage[spanish,english]{babel}
\usepackage{apacite}
\usepackage[round]{natbib}
\usepackage{hyperref}
\usepackage{float}
\usepackage{svg}
\usepackage[margin = 1in, top=2cm]{geometry}% Margins
\setlength{\parindent}{2em}
\setlength{\parskip}{0.2em}
\usepackage{setspace} % Setting the spacing between lines
\usepackage{amsthm, amsmath, amsfonts, mathtools, amssymb, bm} % Math packages 
\usepackage{svg}
\usepackage{graphicx}
\usepackage{pgfplots}
\usepackage{epstopdf}
\usepackage{subfig} % Manipulation and reference of small or sub figures and tables
\usepackage{hyperref} % To create hyperlinks within the document
\spacing{1.15}
\usepackage{appendix}
\usepackage{xcolor}
\usepackage{cancel}
\usepackage{enumerate}
\usepackage[shortlabels]{enumitem}


\usepackage[round]{natbib}
%\bibliographystyle{plainnat}
\bibliographystyle{apacite}


\newtheorem{defin}{Definition.}
\newtheorem{teo}{Theorem. }
\newtheorem{lema}{Lemma. }
\newtheorem{coro}{Corolary. }
\newtheorem{prop}{Proposition. }
\theoremstyle{definition}
\newtheorem{examp}{Example. }
\newtheorem{problem}{Problem}
% \numberwithin{problem}{subsection} 

\newcommand{\card}{\operatorname{card}}
\newcommand{\qiq}{\qquad \implies \qquad}
\newcommand{\qiffq}{\qquad \iff \qquad}
\newcommand{\qaq}{\qquad \textbf{and} \qquad}
\newcommand{\qoq}{\qquad \textbf{or} \qquad}
\newcommand{\settf}{\text{ \emph{:} }}
\newcommand{\chbox}{\makebox[0pt][l]{$\square$}\raisebox{.15ex}{\hspace{.9em}}}
\newcommand{\cchbox}{\makebox[0pt][l]{$\square$}\raisebox{.15ex}{\hspace{0.1em}$\checkmark$}}

\title{Problem Set 2}
\author{Mitchell Valdés-Bobes}
\date{September 14, 2020}

\begin{document}

\maketitle

\begin{problem}
Suppose that $Y=X^{3}$ and $f_{X}(x)=42 x^{5}(1-x), x \in(0,1) .$ Find the PDF of $Y,$ and show
that the PDF integrates to $1 .$
\end{problem}

\begin{proof}[Answer]
$$Y = g(X) \qquad \text{ where } g(x) = x^3$$
Then

$$g^{-1}(y) = \sqrt[3]{y} \qaq \frac{d}{dy}g^{-1}(y)=\frac{1}{3}y^{-2/3}$$

 Since $g:(0,1)\to (0,1)$ and $g^{-1}$ has continuous derivative in $(0,1)$ then
 
 $$f_Y(y)= \left\{ \begin{array}{cc}
     f_X(g^{-1}(y))\left|\frac{d}{dy} g^{-1}(y) \right| &  y \in (0,1)\\
      0 & \text{otherwise} 
 \end{array}

This gives us

$$f_Y(y)=14y(1-y^{1/3})$$

Tanking the integral of $f_Y(y)$

    $$\int_{-\infty}^{\infty}{f_Y(y)} = \int_{0}^1{14y(1-y^{1/3})} = 14\left(\int_{0}^1 y dy - \int_{0}^1 y ^{4/3}dy\right) = 14\left(\frac{y^2}{2} \Big|_0^1 - \frac{3y^{7/3}}{6}\Big|_0^1  \right) = 14\left(\frac{1}{2} - \frac{3}{7}\right) = 1$$


\end{proof}

\begin{problem}
Consider the CDF $F_{X}(x)=\left\{\begin{array}{ll}1.2 x & \text { if } x \in[0,0.5) \\ 0.2+0.8 x & \text { if } x \in[0.5,1]\end{array},\right.$ and the function
$$
f_{X}(x)=\left\{\begin{array}{ll}
1.2 & \text { if } x \in[0,0.5) \\
a & \text { if } x=0.5 \\
0.8 & \text { if } x \in(0.5,1]
\end{array}\right.
$$
Show that $f_{X}$ is the density function of $F_{X}$ as long as $a \geq 0 .$ 
\end{problem}

\begin{proof}[Answer]

$f_X$ is PDF of $F_X$ if and only if for all $x$

$$F_X(x) = \int_{-\infty}^{x}f_X(t)dt$$
 
By the definition of PDF we know that $a\geq 0$. 

For any $x\in [0,0.5)$ 

$$ \int_{-\infty}^{x}f_X(t)dt = \int_{0}^{x}1.2dt = 1.2x = F_X(x)$$

and for any $x\in [0.5, 1]$

\begin{align*}
\int_{-\infty}^{x}f_X(t)dt &= \int_{0}^{0.5}1.2dt + \int_{0.5}^{0.5}adt + \int_{0.5}^{x}0.8dt \\& = 1.2x = 1.2\times 0.5 -0.8\times 0.5 + 0.8 x \\&= 0.2+ 0.8 x =F_X(x)
\end{align*}

\end{proof}


\begin{problem}
Let $X$ have PDF $f_{X}(x)=\frac{2}{9}(x+1), x \in[-1,2] .$ Find the PDF of $Y=X^{2} .$ Note that this is
a bit different from the exercise in the lecture note.
\end{problem}

\begin{proof}[Answer]
We start by finding the CDF of $X$:
$$F_X(x) =\frac{1}{9}(x^2+2x+1) = \frac{1}{9}(x+1)^2$$

Then
$$P(Y\leq y) = P(X^2\leq y) = P(|X|\leq\sqrt{y}) = P(-\sqrt{y}\leq X \leq \sqrt{y})=P(X\leq \sqrt{y})-P(X\leq -\sqrt{y})$$

Note first that for all $y\in[0,1}$ and that for all $y>1$ $P(X\leq \sqrt{y})=0$ therefore

\begin{align*}
y\in[1,4] &\qiq P(Y\leq y) = F_X(\sqrt{y}) \\    
y\in[0,1] &\qiq  P(Y\leq y) = F_X(\sqrt{y}) - F_X(-\sqrt{y})
\end{align*}

Putting all this together we get the CDF of $Y$

$$F_Y(y) = \left\{\begin{array}{cc}
   \frac{4 \sqrt{y}}{9}   & \text{ when } y\in[0,1] \\
   \frac{1}{9} \left(\sqrt{y}+1\right)^2  & \text{ when } y\in[1,4] 
\end{array}$$

To obtain the PDF of $Y$ we need to differentiate $F_Y(y)$ obtaining

$$f_Y(y) = \left\{\begin{array}{cc}
  \frac{2}{9 \sqrt{y}}  & \text{ when } y\in[0,1] \\
  \frac{\sqrt{y}+1}{9 \sqrt{y}}  & \text{ when } y\in[1,4] 
\end{array}$$

\end{proof}

\begin{problem}

A median of a distribution is a value $m$ such that $P(X \leq m) \geq 1 / 2$ and $P(X \geq m) \geq 1 / 2$ Find the median of the distribution $f(x)=\frac{1}{\pi\left(1+x^{2}\right)} \quad x \in R$
\end{problem}


\begin{proof}[Answer]

$$P(X\leq m) = \int_{-\infty}^m f(x)dx = \int_{-\infty}^m \frac{1}{\pi\left(1+x^{2}\right)} = \frac{1}{\pi}(\tan^{-1}(x)\big|_\infty^m) = \frac{\tan^-1(m)}{\pi} - \frac{1}{2}$$
and 
$$P(X\geq m) = 1 - P(X\leq m) 1 - \left(\frac{\tan^-1(m)}{\pi} - \frac{1}{2}\right) =  \frac{1}{2} - \frac{\tan^-1(m)}{\pi}$$
Then
$$P(X\leq m)\geq \frac{1}{2} \qiq \tan^{-1}(m)\geq 0$$
$$P(X\geq m)\geq \frac{1}{2} \qiq \tan^{-1}(m)\leq 0$$
Which means
$$\tan^{-1}(m)=0 \qiq \tan(\tan^{-1}(m)) = \tan(0) \qiq \boxed{m = 0}$$
\end{proof}

\begin{problem}
Show that if $X$ is a continuous random variable, then
$$
\min _{a} E|X-a|=E|X-m|
$$
where $m$ is the median of $X$.

\textit{(Hint: workout the integral expression of $E|X-a|$ and notice that it is differentiable.)}

\end{problem}

\begin{proof}[Answer]

Consider that $$f_X(x)$ is the PDF of X, then

$$E|X-a| = \int_{-\infty}^\infty|x-a|f_X(x)dx = \int_{-\infty}^a(a-x)f_X(x)dx  + \int_{a}^\infty(x-a)f_X(x)dx$$

This is the same as to say:

$$E|X-a| = \lim_{n \to \infty}\left(  \int_{-n}^a(a-x)f_X(x)dx  + \int_{a}^n(x-a)f_X(x)dx\right)$$

We wish to maximize the value of $E|X-a|$ so it is natural to think of taking derivatives with respect to $a$ in the above expression. This is:

$$\frac{\partial}{\partial a}E|X-a| = \frac{\partial}{\partial a}\left(\lim_{n \to \infty}\left(  \int_{-n}^a(a-x)f_X(x)dx  + \int_{a}^n(x-a)f_X(x)dx\right)\right)$$

We can check that the above function expression satisfies the hypothesis of theorem 7.17 of  \cite{rudin1964principles}
therefore we can change the limits and the derivative to obtain

$$\frac{\partial}{\partial a}E|X-a| = \lim_{n \to \infty}\frac{\partial}{\partial a}\left(  \int_{-n}^a(a-x)f_X(x)dx  + \int_{a}^n(x-a)f_X(x)dx\right)$$

Using Leibniz rule we can obtain the derivative of the integral and we get

$$\frac{\partial}{\partial a}E|X-a| = \lim_{n \to \infty}\left(  \int_{-n}^af_X(x)dx  + \int_{a}^nf_X(x)dx\right) = \int_{-\infty}^af_X(x)dx  + \int_{a}^{\infty}f_X(x)dx\right)$$

Therefore

$$\frac{\partial}{\partial a}E|X-a| = -P(X\geq a) + P(X\leq a)$$

The first order condition give us:

$$\frac{\partial}{\partial a}E|X-a| = 0 \qiq P(X\geq a) = P(X\leq a) \qiq \boxed{a=m}$$

\end{proof}

\begin{problem}
Let $\mu_{n}$ denote the $n$ th central moment of a random variable $X .$ Two quantities of interest, in addition to the mean and variance are
$$
\alpha_{3}=\frac{\mu_{3}}{\mu_{2}^{3 / 2}} \text { and } \alpha_{4}=\frac{\mu_{4}}{\mu_{2}^{2}}
$$
The values $\alpha_{3}$ is called the skewness and $\alpha_{4}$ is called the kurtosis. The skewness measures the
lack of symmetry in the density function. The kurtosis, althugh harder to interpret, measures
the peakedness or flatness of the density function.
\begin{enumerate}[(a)]
    \item Show that if a density function is symmetric about a point $a,$ then $\alpha_{3}=0$
    \item Calculate $\alpha_{3}$ for $f(x)=\exp (-x), x \geq 0,$ a density function that is skewed to the right.
    \item Calculate $\alpha_{4}$ for the following density functions and comment on the peakedness of each:
    $$
    \begin{array}{l}
        f(x)=\frac{1}{\sqrt{2 \pi}} \exp \left(-x^{2} / 2\right), x \in R \\
        f(x)=1 / 2, x \in(-1,1) \\
        f(x)=\frac{1}{2} \exp (-|x|), x \in R
    \end{array}
    $$
\end{enumerate}  
\end{problem}

\begin{proof}[Answer]
\textbf{Part (a)} 
Define $Y = X-a$, we will prove that $Y$ is symmetric about $0$:

$$P(-y\leq Y\leq y) = P(-y+a\leq Y\leq y+a) = 2 P(X\leq a+y) = 2 P(Y\leq y)$$

Now consider $n\in \mathbb{Z}_+$ and $n$ odd:

$$E(Y^3) =\int_{-\infty}^{\infty}y^n dF_Y = \int_{-\infty}^{0}y^n dF_Y + \int_{0}^{\infty}y^n dF_Y = \int_{0}^{\infty}(-y)^n dF_Y + \int_{0}^{\infty}y^n dF_Y = 0$$

Note that for $n=1$ we get $E(Y)=0$ then $E(X)=a$:

Therefore

$$E(Y^3) = E(X-a)^3 = E(X-E(X))^3 = \mu_3 \qiq \boxed{\alpha_3 = 0}$$

\textbf{Part (b)}

Let  $f(x)=\exp (-x), x \geq 0,$ then

$$\mu = \int_{0}^{\infty}xe^{-x}dx = 1$$

$$\mu_3 = \int_{0}^{\infty}(x-1)^3e^{-x}dx = 2$$

$$\mu_2 = \int_{0}^{\infty}(x-1)^2e^{-x}dx = 1$$

Then

$$\boxed{\alpha_3 = \frac{\mu_3}{m_2^{3/2}}  = 2$}$

\textbf{Part (c.1)}
$$f(x)=\frac{1}{\sqrt{2 \pi}} \exp \left(-x^{2} / 2\right), x \in R \qiq X\sim \mathcal{N}(0,1)$$

Moment generating function for this distribution is

$$M(t) = e^{t^2/2}$$

Then 

$$\mu_2 = \frac{\partial^2 }{\partial t^2}M\big |_{t=0}= e^{\frac{t^2}{2}} t^2+e^{\frac{t^2}{2}}\big |_{t=0} = 1$$

$$\mu_4 = \frac{\partial^4 }{\partial t^4}M\big |_{t=0}= e^{\frac{t^2}{2}} t^4+6 e^{\frac{t^2}{2}} t^2+3 e^{\frac{t^2}{2}}\big |_{t=0} = 3$$

Therefore:

$$\boxed{\alpha_4 = \frac{\mu_4}{\mu_2^2}=3}$$

\textbf{Part (c.2)}
$$f(x)=1/2, x\in(-1,1) \qiq X\sim \mathcal{U}(-1,1)$$

$$E(X) = \frac{1}{2}(1+(-1))=0$$

Then the $n$-th moment is:

$$\int_{-1}^1\frac{x^n}{2}dx = \frac{(-1)^n+1}{2 (n+1)}$$

Which give us


$$\mu_2 = \frac{1}{3}$$

$$\mu_4 = \frac{1}{5}$$

Therefore:

$$\boxed{\alpha_4 = \frac{9}{5}}$$

\textbf{Part (c.3)}
$$f(x)=\frac{1}{2} \exp (-|x|), x \in R$$

To obtain the moment generating function for this distribution we need to compute the  integral:
$$\int_{-\infty}^{\infty} e^{t-|x|} d x$$

Since  $e^{t-| x |}$  is an even function and the interval $(-\infty, \infty)$  is symmetric about  $0$, $$\int_{-\infty}^{\infty} e^{t-|x|} d x=2 \int_{0}^{\infty} e^{t-x} d x = 2e^t$$

Then 

$$\mu_2 = \frac{\partial^2 }{\partial t^2}M\big |_{t=0}= 2e^t  \big |_{t=0} = 1$$

$$\mu_4 = \frac{\partial^4 }{\partial t^4}M\big |_{t=0}= 2e^t \big |_{t=0} = 1$$


Therefore:

$$\boxed{\alpha_4 = \frac{\mu_4}{\mu_2^2}=1}$$

\end{proof}

\bibliography{references.bib}

\end{document}