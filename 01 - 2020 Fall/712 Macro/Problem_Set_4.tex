\documentclass{article}
\usepackage[utf8]{inputenc}
\documentclass[12pt]{article}
%\usepackage[left=3cm, right=2.5cm, top=2.5cm, bottom=2.5cm]{geometry}e}
\usepackage[utf8]{inputenc}
\usepackage[spanish,english]{babel}
\usepackage{apacite}
\usepackage[round]{natbib}
\usepackage{hyperref}
\usepackage{float}
\usepackage{svg}
\usepackage[margin = 1in, top=2cm]{geometry}% Margins
\setlength{\parindent}{2em}
\setlength{\parskip}{0.2em}
\usepackage{setspace} % Setting the spacing between lines
\usepackage{amsthm, amsmath, amsfonts, mathtools, amssymb, bm} % Math packages 
\usepackage{svg}
\usepackage{graphicx}
\usepackage{pgfplots}
\usepackage{epstopdf}
\usepackage{subfig} % Manipulation and reference of small or sub figures and tables
\usepackage{hyperref} % To create hyperlinks within the document
\spacing{1.15}
\usepackage{appendix}
\usepackage{xcolor}
\usepackage{cancel}
\usepackage{enumerate}
\usepackage[shortlabels]{enumitem}


\usepackage[round]{natbib}
%\bibliographystyle{plainnat}
\bibliographystyle{apacite}


\newtheorem{defin}{Definition.}
\newtheorem{teo}{Theorem. }
\newtheorem{lema}{Lemma. }
\newtheorem{coro}{Corolary. }
\newtheorem{prop}{Proposition. }
\theoremstyle{definition}
\newtheorem{examp}{Example. }
\newtheorem{problem}{Problem}
\newtheorem{subproblem}{}[problem]
% \numberwithin{problem}{subsection} 

\newcommand{\card}{\operatorname{card}}
\newcommand{\qiq}{\qquad \implies \qquad}
\newcommand{\qiffq}{\qquad \iff \qquad}
\newcommand{\qaq}{\qquad \textbf{and} \qquad}
\newcommand{\qoq}{\qquad \textbf{or} \qquad}
\newcommand{\settf}{\text{ \emph{:} }}
\newcommand{\chbox}{\makebox[0pt][l]{$\square$}\raisebox{.15ex}{\hspace{.9em}}}
\newcommand{\cchbox}{\makebox[0pt][l]{$\square$}\raisebox{.15ex}{\hspace{0.1em}$\checkmark$}}

\title{Problem Set 4}
\author{Mitchell Valdés-Bobes}
\date{September 30, 2020}

\begin{document}

\maketitle
\begin{problem}
Consider the following overlapping generations problem. Each period $t=1,2,3, \ldots$ a
new generation of 2 period lived households are born. The measure of identical households
born in any period grows by $1+n .$ That is, we assume population growth of rate $n \geq 0$.
There is a unit measure of initial old who are endowed with $\bar{M}_{1}$ units of fiat money as
well as $w_{2}$ units of consumption goods. Instead of a fixed money supply, now assume that
the money supply increases at the rate $z \geq 0 .$ The increase in money supply is handed out
each period to old agents in direct proportion to the amount of money that they chose when
young. In other words, if a young agent chooses $M_{t+1}^{t} \geq 0,$ they will receive $(1+z) M_{t+1}^{t}$
units of money when old.

Each generation is endowed with $w_{1}$ in youth and $w_{2}$ in old age of nonstorable con-
sumption goods where $w_{1}>w_{2}$. The utility function of a household of generation $t \geq 1$
is
$$
U\left(c_{t}^{t}, c_{t+1}^{t}\right)=\ln \left(c_{t}^{t}\right)+\ln \left(c_{t+1}^{t}\right)
$$
where $\left(c_{t}^{t}, c_{t+1}^{t}\right)$ is consumption of a household of generation $t$ in youth (i.e. in period $t$ ) and
old age (i.e in period $t+1$ ). The preferences of the initial old are given by $U\left(c_{1}^{0}\right)=\ln \left(c_{1}^{0}\right)$,
where $c_{1}^{0}$ is consumption by a household of the initial old.
\end{problem}

\begin{subproblem}
State and solve the planner's problem under the assumption of equal weights on
each generation (i.e., on representative agents from each generation) so that the
the objective of the social planner is to maximize $U\left(c_{1}^{0}\right)+\sum_{t=1}^{\infty} U\left(c_{t}^{t}, c_{t+1}^{t}\right) .$ since
this objective may not be well defined (i.e. add up to infinity), we can apply the "overtaking" criterion to determine optimality. $^{1}$ Hence just go ahead and maximize
away as usual.
\end{subproblem}

\begin{proof}[Answer]
Call $N_0$ the measure of the initial generation. In each period $t$ the resource constraint that the social planer faces is
\begin{align*}
    (1+n)^{t-1}N_0c_{t}^{t-1}+(1+n)^{t}N_0c_{t}^{t}&\leq(1+n)^{t-1}N_0w_2+(1+n)^{t}N_0w_1\\
    (1+n)c_t^t +c_{t}^{t-1}&\leq (1+n)w_1 + w_2
\end{align*}

By the overtaking criterion, the social planner can maximize consumption of every ``alive'' agent in time $t$ subject to the resource constraint which we can assume that binds since utilities are increasing in consumption. Therefore the social planner's problem is

\begin{align}
    \max_{(c_{t}^{t-1},c_{t}^t)\in \mathbb{R}_+^2}&\quad \ln{c_{t}^{t-1}} + \ln{c_{t}^t}\\
    \text{subject to:} &\quad (1+n)c_t^t +c_{t}^{t-1} = (1+n)w_1 + w_2 \tag{RC}
\end{align*}

From \textbf{(RC)} we get that

\begin{equation}\label{ct}
c_{t}^{t-1}=(1+n)(w_1)-c_t^t) - w_2    
\end{equation}

The the problem becomes a problem in one variable:

$$\max_{c_{t}^t\in \mathbb{R}_+}\quad \ln{(1+n)(w_1)-c_t^t) - w_2} + \ln{c_{t}^t}$$

First order condition give us:

\begin{equation}\label{ctstar}
\frac{1}{c_t^t} -\frac{n+1}{(n+1)(w_1+c_t^t) + w_2} \qiq \boxed{c_t^t=\frac{(1+n)w_1+w_2}{2(1+n)}}    
\end{equation}

Plugging \eqref{ctstar} in \eqref{ct} we get:

\begin{equation}\label{ct}
    \boxed{c_t^{t-1}=\frac{(1+n)w_1+w_2}{2}} 
\end{equation}
\end{proof}

\begin{subproblem}
Let $p_{t}$ be the price of consumption goods in terms of money at time $t .$ Define a competitive
equilibrium.
\end{subproblem}

\begin{proof}[Answer]

First we define the optimization problem for the initial old:

\begin{align}\label{initoldprob}
    \max_{c_1^0\in\mathbb{R}_+}&\quad \ln{c^0_1} \\
    \text{subject to:}&\quad p_1c^0_1 \leq \overline{M}_1 + p_1w_2
\end{align}

and every subsequent generation:

\begin{align}\label{subseqgen}
    \max_{(c_t^t,c_t^{t+1}, M^{t}_{t+1})\in\mathbb{R}_+^3}&\quad \ln{c_t^t} + \ln{c_t^{t+1}}\\
    \text{subject to:}&\quad p_tc^t_t + M^{t}_{t+1} \leq p_t w_1\\
    &\quad p_{t+1}c^t_{t+1} \leq p_{t+1} w_2 +  (1+z)M^{t}_{t+1}
\end{align}

Then we can define the competitive equilibrium as a sequence of consumption allocations $\{c^{t-1}_t, c_t^t\}_{t=1}^\infty$ a sequence of money holdings $\{\overline{M}_{t+1}^t\}$ and a sequence of prices $\{p_t\}_{t=1}^\infty$ such that the initial old solve their optimization problem, all subsequent generations solve theirs and markets clear.

Market for money:
$$(1+n)^tM_t^{t+1} = \overline{M}_t \qquad \forall t\geq 1$$ 
Market for goods:
$$c_t^{t-1}+(1+n)c_{t}^t = (1+n)w_1 + w_2$$
\end{proof}

\begin{subproblem}
Solve for an autarkic equilibrium.
\end{subproblem}

\begin{proof}[Answer]
We want an allocation $\{c_t^t, c_t^{t-1},M_{t+1}^t \}_{t=1}^\infty$ and prices $\{p_t \}_{t=1}^\infty$, such that 
 agents of every generation choose to consume their endowments in each period $t$, i.e., 
        $$c_t^{t} = w_1
        \qquad c_{t+1}^{t} = w_2 $$
for all $t\leq1$, and market clears.

Using the market clearing condition of the money market we solve for $M_{t+1}^t$:
\begin{equation}
   {M_{t+1}^t} = \frac{(1+z)^{t-1} \bar{M}_1 }{(1+n)^{t} }
\end{equation}

We also know that at the optimum $MRS(c_t^t, c^{t}_{t+1})=(1+z)p_{t}/p_{t+1}$. This implies
\begin{equation}
    \frac{(1+z)p_{t}}{p_{t+1}} = \frac{w_2}{w_1} 
\end{equation}

Now, using the budgets constrains we get
\begin{align*}
w_2 = \frac{\bar{M}_1}{p_{1}} +  w_2 \quad &\Rightarrow \quad 0 = \frac{1}{p_{t}} \qquad \forall t\geq 0
\end{align*}
We can interpret this situation like this: The only price system that can sustain an autarkic equilibrium is when prices are so high that money has no value, therefore there is no gains from iner-generational trade.

\end{proof}

\begin{subproblem}
Solve for a steady state (non-autarkic) monetary equilibrium. As $w_{1}>w_{2},$ we know this
corresponds to the Samuelson case with $\beta=1 .$ Verify the non-negativity constraint on
money is not binding in the equilibrium. What is the rate of return on money in the monetary
equilibrium? Give intuition why the rate of return is at the level you find.
\end{subproblem}

\begin{proof}[Answer]
Remember that the initial old solve their maximization problem in $t=1$ subject to their budget constraint (problem in \eqref{initoldprob}):

Since the utility function is increasing in consumption the initial old will consume all their endowment and the worth of the fiat money in consumption therefore:

\begin{equation}
\boxed{c^0_1 = \frac{\bar{M}_1}{p_1}+w_2}
\end{equation}

To solve the problem of the following generations we can transform both budget constraints and make them binding to get:

\begin{align}\label{generalproblem}
\max_{(c^t_t,c^t_{t+1}, M^{t}_{t+1})\in \mathbb{R}_+^3 }&\quad \ln c^t_t + \ln c^t_{t+1} \\
\text{subject to} & \quad  c^t_t = w_1 - \frac{M^{t}_{t+1}}{p_t}\tag{BC}\\
 & \quad c^t_{t+1} = w_2 + \frac{(1+z)M^{t}_{t+1}}{p_{t+1}}\tag{BC}
\end{align}
Plugging the budget constraint in to the objective function we get:
\begin{align}\label{simprob}
\max_{M^{t}_{t+1} \in\mathbb{R}_+ }&\quad \ln\left(w_1 - \frac{M^{t}_{t+1}}{p_t}\right) + \ln \left( w_2 + \frac{(1+z)M^{t}_{t+1}}{p_{t+1}} \right)
\end{align}

The FOC of this problem give us:

\begin{equation}\label{money}
-\frac{z+1}{p_{t+1} \left(w_2-\frac{M^{t}_{t+1} (z+1)}{p_{t+1}}\right)}-\frac{1}{p_t \left(w_1-\frac{M^{t}_{t+1}}{p_t}\right)} = 0 \qiq  M^{t}_{t+1} =  \frac{w_1  p_t(1+z) +w_2 p_{t+1}}{2 (z+1)}
\end{equation}

Plugin \eqref{money} in the budget constraint of \eqref{generalproblem} we obtain:

\begin{equation}\label{ct1w20}
   c^t_{t} =\frac{1}{2} \left(w_1+\frac{w_2 p_{t+1}}{(z+1) p_t}\right) \qquad c^t_{t+1} = \frac{1}{2} \left(w_2+\frac{w_1 (z+1) p_t}{p_{t+1}}\right)
\end{equation}

Denote 

$$q_t = \frac{p_t}{p_{t+1}}$$

Plugging in \eqref{ct1w20} we get

\begin{equation}\label{ct1w21}
   c^{t+1}_{t+1} =\frac{1}{2} \left(w_1+\frac{w_2}{(z+1) q_{t+1}}\right) \qquad c^t_{t+1} = \frac{1}{2} \left(w_2-w_1 (z+1) q_t\right)
\end{equation}

Market clearing condition is:

$$c_{t+1}^{t}+(1+n)c_{t+1}^{t+1} = (1+n)w_1 + w_2$$

Plugging in the values from \eqref{ct1w21} we get:

$$\frac{1+n}{2} \left(w_1-\frac{w_2}{(z+1) q_{t+1}}\right) +  \frac{1}{2} \left(w_2-w_1 (z+1) q_t\right)=(1+n)w_1 + w_2$$

Substituting the steady state $\overline{q})q_t = q_{t+1}$ and rearranging we get the quadratic equation:

$$w_1(z+1)\overline{q}^2+(w_2-(1+n)w_1)\overline{q}+\frac{1+n}{1+z}w_2 = 0$$

with discriminant:

$$D = \left((n+1) w_1+w_2\right){}^2$$

the two possible solution of the quadratic equation are:

$$\overline{q}_{1,2} = \frac{(n+1) w_1-w_2 \pm\sqrt{D}}{2a} \qiq \overline{q}_{1} = \frac{n+1}{z+1} \qaq \overline{q}_{2}=  \frac{w_2}{w_1 z+w_1}$$

Since the second solution relates to the autarkic equilibrium we will use the first solution in \eqref{ct1w20} to obtain:

\begin{equation}\label{ct1w20}
  \boxed{c_{t}^{t}=\frac{(1+n) w_{1}+w_{2}}{2(1+n)}} \qquad\boxed{ c^t_{t+1} =\frac{(1+n) w_{1}+w_{2}}{2}}
\end{equation}

Note that in this equilibrium agents do not consume all their endowment in time $t$, instead they use money to smooth their consumption from $t$ to $t+1$.

From the solution of the initial old we have:

$$\frac{\bar{M}_1}{p_1}+w_2} = \frac{(1+n) w_{1}+w_{2}}{2}$$

Therefore we know that the initial price in this equilibrium must be:

$$p_1 = \frac{2\bar{M}_1}{(1+n) w_{1}+w_{2}}$$

Prices evolve according to:

$$p_2 = \left(\frac{n+1}{z+1}\right)p_1 \quad p_3 = \left(\frac{n+1}{z+1}\right)p_2 = \left(\frac{n+1}{z+1}\right)^2 p_1 \quad \hdots \quad p_t = \left(\frac{n+1}{z+1}\right)^{t-1} p_1$$

Which is:

$$\boxed{p_t =\left(\frac{n+1}{z+1}\right)^{t-1} \frac{2\bar{M}_1}{(1+n) w_{1}+w_{2}}}$$

The demand for money evolves according to:

$$\boxed{M_t = \frac{(1+z)^{t-1}}{(n+1)^t}\bar{M}_1}$$
\end{proof}


\begin{subproblem}
Does the stationary monetary equilibrium Pareto dominate autarky? Can you use your
answer in part 1 to establish that? If so, how can the government implement it?
\end{subproblem}

\begin{proof}[Answer]
We can compare consumption:

\textbf{Social Planner:}
$$c_t^t=\frac{(1+n)w_1+w_2}{2(1+n)} \qquad c^t_{t+1}=\frac{(1+n)w_1+w_2}{2}$$

\textbf{Autarkic}

$$c_t^{t} = w_1 \qquad c^t_{t+1} = w_2$$

\textbf{Stationary Monetary Equilibrium}
$$ c_{t}^{t}=\frac{(1+n) w_{1}+w_{2}}{2(1+n)} \qquad c^t_{t+1} =\frac{(1+n) w_{1}+w_{2}}{2} $$

Note that in the consumption allocations in the Stationary Monetary Equilibrium are the same than in the Social Planner Problem and since the benevolent Social Planner achieves the first-best solution, any other allocation will be Pareto dominated by this one, in particular the Autarkic Equilibrium will.

\end{proof}

\begin{subproblem}
Does money exhibit super-neutrality (i.e. the level of inflation does not change equilibrium
consumption allocations)?
\end{subproblem}

\begin{proof}[Answer]
From the 
\end{proof}
Checking the optimal consumption expressions::

$$ c^t_{t} =\frac{1}{2} \left(w_1+\frac{w_2 p_{t+1}}{(z+1) p_t}\right) \qquad c^t_{t+1} = \frac{1}{2} \left(w_2+\frac{w_1 (z+1) p_t}{p_{t+1}}\right)$$

We see that consumption does not depends on prices per se but on the gross real rate of return on money $(1+z)p_{t+1}/p_{t}$ which in turn does not depend on money growth. So the answers is yes, the level of inflation does not change equilibrium consumption allocations therefore money exhibit super-neutrality.
\end{document}