
    




    
\documentclass[10pt,notitlepage,onecolumn,aps,pra]{revtex4-1}

    
    
\usepackage[T1]{fontenc}
\usepackage{graphicx}
% We will generate all images so they have a width \maxwidth. This means
% that they will get their normal width if they fit onto the page, but
% are scaled down if they would overflow the margins.
\makeatletter
\def\maxwidth{\ifdim\Gin@nat@width>\linewidth\linewidth
\else\Gin@nat@width\fi}
\makeatother
\let\Oldincludegraphics\includegraphics
% Set max figure width to be 80% of text width, for now hardcoded.
\renewcommand{\includegraphics}[1]{\Oldincludegraphics[width=.8\maxwidth]{#1}}
% Ensure that by default, figures have no caption (until we provide a
% proper Figure object with a Caption API and a way to capture that
% in the conversion process - todo).
\usepackage{caption}
\DeclareCaptionLabelFormat{nolabel}{}
\captionsetup{labelformat=nolabel}

\usepackage{adjustbox} % Used to constrain images to a maximum size
\usepackage{xcolor} % Allow colors to be defined
\usepackage{enumerate} % Needed for markdown enumerations to work
\usepackage{geometry} % Used to adjust the document margins
\usepackage{amsmath} % Equations
\usepackage{amssymb} % Equations
\usepackage{textcomp} % defines textquotesingle
% Hack from http://tex.stackexchange.com/a/47451/13684:
\AtBeginDocument{%
    \def\PYZsq{\textquotesingle}% Upright quotes in Pygmentized code
}
\usepackage{upquote} % Upright quotes for verbatim code
\usepackage{eurosym} % defines \euro
\usepackage[mathletters]{ucs} % Extended unicode (utf-8) support
\usepackage[utf8x]{inputenc} % Allow utf-8 characters in the tex document
\usepackage{fancyvrb} % verbatim replacement that allows latex
\usepackage{grffile} % extends the file name processing of package graphics
                     % to support a larger range
% The hyperref package gives us a pdf with properly built
% internal navigation ('pdf bookmarks' for the table of contents,
% internal cross-reference links, web links for URLs, etc.)
\usepackage{hyperref}
\usepackage{booktabs}  % table support for pandoc > 1.12.2
\usepackage[inline]{enumitem} % IRkernel/repr support (it uses the enumerate* environment)
\usepackage[normalem]{ulem} % ulem is needed to support strikethroughs (\sout)
                            % normalem makes italics be italics, not underlines
\usepackage{braket}


    
    % Colors for the hyperref package
    \definecolor{urlcolor}{rgb}{0,.145,.698}
    \definecolor{linkcolor}{rgb}{.71,0.21,0.01}
    \definecolor{citecolor}{rgb}{.12,.54,.11}

    % ANSI colors
    \definecolor{ansi-black}{HTML}{3E424D}
    \definecolor{ansi-black-intense}{HTML}{282C36}
    \definecolor{ansi-red}{HTML}{E75C58}
    \definecolor{ansi-red-intense}{HTML}{B22B31}
    \definecolor{ansi-green}{HTML}{00A250}
    \definecolor{ansi-green-intense}{HTML}{007427}
    \definecolor{ansi-yellow}{HTML}{DDB62B}
    \definecolor{ansi-yellow-intense}{HTML}{B27D12}
    \definecolor{ansi-blue}{HTML}{208FFB}
    \definecolor{ansi-blue-intense}{HTML}{0065CA}
    \definecolor{ansi-magenta}{HTML}{D160C4}
    \definecolor{ansi-magenta-intense}{HTML}{A03196}
    \definecolor{ansi-cyan}{HTML}{60C6C8}
    \definecolor{ansi-cyan-intense}{HTML}{258F8F}
    \definecolor{ansi-white}{HTML}{C5C1B4}
    \definecolor{ansi-white-intense}{HTML}{A1A6B2}
    \definecolor{ansi-default-inverse-fg}{HTML}{FFFFFF}
    \definecolor{ansi-default-inverse-bg}{HTML}{000000}

    % commands and environments needed by pandoc snippets
    % extracted from the output of `pandoc -s`
    \providecommand{\tightlist}{%
      \setlength{\itemsep}{0pt}\setlength{\parskip}{0pt}}
    \DefineVerbatimEnvironment{Highlighting}{Verbatim}{commandchars=\\\{\}}
    % Add ',fontsize=\small' for more characters per line
    \newenvironment{Shaded}{}{}
    \newcommand{\KeywordTok}[1]{\textcolor[rgb]{0.00,0.44,0.13}{\textbf{{#1}}}}
    \newcommand{\DataTypeTok}[1]{\textcolor[rgb]{0.56,0.13,0.00}{{#1}}}
    \newcommand{\DecValTok}[1]{\textcolor[rgb]{0.25,0.63,0.44}{{#1}}}
    \newcommand{\BaseNTok}[1]{\textcolor[rgb]{0.25,0.63,0.44}{{#1}}}
    \newcommand{\FloatTok}[1]{\textcolor[rgb]{0.25,0.63,0.44}{{#1}}}
    \newcommand{\CharTok}[1]{\textcolor[rgb]{0.25,0.44,0.63}{{#1}}}
    \newcommand{\StringTok}[1]{\textcolor[rgb]{0.25,0.44,0.63}{{#1}}}
    \newcommand{\CommentTok}[1]{\textcolor[rgb]{0.38,0.63,0.69}{\textit{{#1}}}}
    \newcommand{\OtherTok}[1]{\textcolor[rgb]{0.00,0.44,0.13}{{#1}}}
    \newcommand{\AlertTok}[1]{\textcolor[rgb]{1.00,0.00,0.00}{\textbf{{#1}}}}
    \newcommand{\FunctionTok}[1]{\textcolor[rgb]{0.02,0.16,0.49}{{#1}}}
    \newcommand{\RegionMarkerTok}[1]{{#1}}
    \newcommand{\ErrorTok}[1]{\textcolor[rgb]{1.00,0.00,0.00}{\textbf{{#1}}}}
    \newcommand{\NormalTok}[1]{{#1}}
    
    % Additional commands for more recent versions of Pandoc
    \newcommand{\ConstantTok}[1]{\textcolor[rgb]{0.53,0.00,0.00}{{#1}}}
    \newcommand{\SpecialCharTok}[1]{\textcolor[rgb]{0.25,0.44,0.63}{{#1}}}
    \newcommand{\VerbatimStringTok}[1]{\textcolor[rgb]{0.25,0.44,0.63}{{#1}}}
    \newcommand{\SpecialStringTok}[1]{\textcolor[rgb]{0.73,0.40,0.53}{{#1}}}
    \newcommand{\ImportTok}[1]{{#1}}
    \newcommand{\DocumentationTok}[1]{\textcolor[rgb]{0.73,0.13,0.13}{\textit{{#1}}}}
    \newcommand{\AnnotationTok}[1]{\textcolor[rgb]{0.38,0.63,0.69}{\textbf{\textit{{#1}}}}}
    \newcommand{\CommentVarTok}[1]{\textcolor[rgb]{0.38,0.63,0.69}{\textbf{\textit{{#1}}}}}
    \newcommand{\VariableTok}[1]{\textcolor[rgb]{0.10,0.09,0.49}{{#1}}}
    \newcommand{\ControlFlowTok}[1]{\textcolor[rgb]{0.00,0.44,0.13}{\textbf{{#1}}}}
    \newcommand{\OperatorTok}[1]{\textcolor[rgb]{0.40,0.40,0.40}{{#1}}}
    \newcommand{\BuiltInTok}[1]{{#1}}
    \newcommand{\ExtensionTok}[1]{{#1}}
    \newcommand{\PreprocessorTok}[1]{\textcolor[rgb]{0.74,0.48,0.00}{{#1}}}
    \newcommand{\AttributeTok}[1]{\textcolor[rgb]{0.49,0.56,0.16}{{#1}}}
    \newcommand{\InformationTok}[1]{\textcolor[rgb]{0.38,0.63,0.69}{\textbf{\textit{{#1}}}}}
    \newcommand{\WarningTok}[1]{\textcolor[rgb]{0.38,0.63,0.69}{\textbf{\textit{{#1}}}}}
    
    
    % Define a nice break command that doesn't care if a line doesn't already
    % exist.
    \def\br{\hspace*{\fill} \\* }
    % Math Jax compatibility definitions
    \def\gt{>}
    \def\lt{<}
    \let\Oldtex\TeX
    \let\Oldlatex\LaTeX
    \renewcommand{\TeX}{\textrm{\Oldtex}}
    \renewcommand{\LaTeX}{\textrm{\Oldlatex}}
    % Document parameters
    % Document title
    
    
    
    
% Pygments definitions
\makeatletter
\def\PY@reset{\let\PY@it=\relax \let\PY@bf=\relax%
    \let\PY@ul=\relax \let\PY@tc=\relax%
    \let\PY@bc=\relax \let\PY@ff=\relax}
\def\PY@tok#1{\csname PY@tok@#1\endcsname}
\def\PY@toks#1+{\ifx\relax#1\empty\else%
    \PY@tok{#1}\expandafter\PY@toks\fi}
\def\PY@do#1{\PY@bc{\PY@tc{\PY@ul{%
    \PY@it{\PY@bf{\PY@ff{#1}}}}}}}
\def\PY#1#2{\PY@reset\PY@toks#1+\relax+\PY@do{#2}}

\expandafter\def\csname PY@tok@w\endcsname{\def\PY@tc##1{\textcolor[rgb]{0.73,0.73,0.73}{##1}}}
\expandafter\def\csname PY@tok@c\endcsname{\let\PY@it=\textit\def\PY@tc##1{\textcolor[rgb]{0.25,0.50,0.50}{##1}}}
\expandafter\def\csname PY@tok@cp\endcsname{\def\PY@tc##1{\textcolor[rgb]{0.74,0.48,0.00}{##1}}}
\expandafter\def\csname PY@tok@k\endcsname{\let\PY@bf=\textbf\def\PY@tc##1{\textcolor[rgb]{0.00,0.50,0.00}{##1}}}
\expandafter\def\csname PY@tok@kp\endcsname{\def\PY@tc##1{\textcolor[rgb]{0.00,0.50,0.00}{##1}}}
\expandafter\def\csname PY@tok@kt\endcsname{\def\PY@tc##1{\textcolor[rgb]{0.69,0.00,0.25}{##1}}}
\expandafter\def\csname PY@tok@o\endcsname{\def\PY@tc##1{\textcolor[rgb]{0.40,0.40,0.40}{##1}}}
\expandafter\def\csname PY@tok@ow\endcsname{\let\PY@bf=\textbf\def\PY@tc##1{\textcolor[rgb]{0.67,0.13,1.00}{##1}}}
\expandafter\def\csname PY@tok@nb\endcsname{\def\PY@tc##1{\textcolor[rgb]{0.00,0.50,0.00}{##1}}}
\expandafter\def\csname PY@tok@nf\endcsname{\def\PY@tc##1{\textcolor[rgb]{0.00,0.00,1.00}{##1}}}
\expandafter\def\csname PY@tok@nc\endcsname{\let\PY@bf=\textbf\def\PY@tc##1{\textcolor[rgb]{0.00,0.00,1.00}{##1}}}
\expandafter\def\csname PY@tok@nn\endcsname{\let\PY@bf=\textbf\def\PY@tc##1{\textcolor[rgb]{0.00,0.00,1.00}{##1}}}
\expandafter\def\csname PY@tok@ne\endcsname{\let\PY@bf=\textbf\def\PY@tc##1{\textcolor[rgb]{0.82,0.25,0.23}{##1}}}
\expandafter\def\csname PY@tok@nv\endcsname{\def\PY@tc##1{\textcolor[rgb]{0.10,0.09,0.49}{##1}}}
\expandafter\def\csname PY@tok@no\endcsname{\def\PY@tc##1{\textcolor[rgb]{0.53,0.00,0.00}{##1}}}
\expandafter\def\csname PY@tok@nl\endcsname{\def\PY@tc##1{\textcolor[rgb]{0.63,0.63,0.00}{##1}}}
\expandafter\def\csname PY@tok@ni\endcsname{\let\PY@bf=\textbf\def\PY@tc##1{\textcolor[rgb]{0.60,0.60,0.60}{##1}}}
\expandafter\def\csname PY@tok@na\endcsname{\def\PY@tc##1{\textcolor[rgb]{0.49,0.56,0.16}{##1}}}
\expandafter\def\csname PY@tok@nt\endcsname{\let\PY@bf=\textbf\def\PY@tc##1{\textcolor[rgb]{0.00,0.50,0.00}{##1}}}
\expandafter\def\csname PY@tok@nd\endcsname{\def\PY@tc##1{\textcolor[rgb]{0.67,0.13,1.00}{##1}}}
\expandafter\def\csname PY@tok@s\endcsname{\def\PY@tc##1{\textcolor[rgb]{0.73,0.13,0.13}{##1}}}
\expandafter\def\csname PY@tok@sd\endcsname{\let\PY@it=\textit\def\PY@tc##1{\textcolor[rgb]{0.73,0.13,0.13}{##1}}}
\expandafter\def\csname PY@tok@si\endcsname{\let\PY@bf=\textbf\def\PY@tc##1{\textcolor[rgb]{0.73,0.40,0.53}{##1}}}
\expandafter\def\csname PY@tok@se\endcsname{\let\PY@bf=\textbf\def\PY@tc##1{\textcolor[rgb]{0.73,0.40,0.13}{##1}}}
\expandafter\def\csname PY@tok@sr\endcsname{\def\PY@tc##1{\textcolor[rgb]{0.73,0.40,0.53}{##1}}}
\expandafter\def\csname PY@tok@ss\endcsname{\def\PY@tc##1{\textcolor[rgb]{0.10,0.09,0.49}{##1}}}
\expandafter\def\csname PY@tok@sx\endcsname{\def\PY@tc##1{\textcolor[rgb]{0.00,0.50,0.00}{##1}}}
\expandafter\def\csname PY@tok@m\endcsname{\def\PY@tc##1{\textcolor[rgb]{0.40,0.40,0.40}{##1}}}
\expandafter\def\csname PY@tok@gh\endcsname{\let\PY@bf=\textbf\def\PY@tc##1{\textcolor[rgb]{0.00,0.00,0.50}{##1}}}
\expandafter\def\csname PY@tok@gu\endcsname{\let\PY@bf=\textbf\def\PY@tc##1{\textcolor[rgb]{0.50,0.00,0.50}{##1}}}
\expandafter\def\csname PY@tok@gd\endcsname{\def\PY@tc##1{\textcolor[rgb]{0.63,0.00,0.00}{##1}}}
\expandafter\def\csname PY@tok@gi\endcsname{\def\PY@tc##1{\textcolor[rgb]{0.00,0.63,0.00}{##1}}}
\expandafter\def\csname PY@tok@gr\endcsname{\def\PY@tc##1{\textcolor[rgb]{1.00,0.00,0.00}{##1}}}
\expandafter\def\csname PY@tok@ge\endcsname{\let\PY@it=\textit}
\expandafter\def\csname PY@tok@gs\endcsname{\let\PY@bf=\textbf}
\expandafter\def\csname PY@tok@gp\endcsname{\let\PY@bf=\textbf\def\PY@tc##1{\textcolor[rgb]{0.00,0.00,0.50}{##1}}}
\expandafter\def\csname PY@tok@go\endcsname{\def\PY@tc##1{\textcolor[rgb]{0.53,0.53,0.53}{##1}}}
\expandafter\def\csname PY@tok@gt\endcsname{\def\PY@tc##1{\textcolor[rgb]{0.00,0.27,0.87}{##1}}}
\expandafter\def\csname PY@tok@err\endcsname{\def\PY@bc##1{\setlength{\fboxsep}{0pt}\fcolorbox[rgb]{1.00,0.00,0.00}{1,1,1}{\strut ##1}}}
\expandafter\def\csname PY@tok@kc\endcsname{\let\PY@bf=\textbf\def\PY@tc##1{\textcolor[rgb]{0.00,0.50,0.00}{##1}}}
\expandafter\def\csname PY@tok@kd\endcsname{\let\PY@bf=\textbf\def\PY@tc##1{\textcolor[rgb]{0.00,0.50,0.00}{##1}}}
\expandafter\def\csname PY@tok@kn\endcsname{\let\PY@bf=\textbf\def\PY@tc##1{\textcolor[rgb]{0.00,0.50,0.00}{##1}}}
\expandafter\def\csname PY@tok@kr\endcsname{\let\PY@bf=\textbf\def\PY@tc##1{\textcolor[rgb]{0.00,0.50,0.00}{##1}}}
\expandafter\def\csname PY@tok@bp\endcsname{\def\PY@tc##1{\textcolor[rgb]{0.00,0.50,0.00}{##1}}}
\expandafter\def\csname PY@tok@fm\endcsname{\def\PY@tc##1{\textcolor[rgb]{0.00,0.00,1.00}{##1}}}
\expandafter\def\csname PY@tok@vc\endcsname{\def\PY@tc##1{\textcolor[rgb]{0.10,0.09,0.49}{##1}}}
\expandafter\def\csname PY@tok@vg\endcsname{\def\PY@tc##1{\textcolor[rgb]{0.10,0.09,0.49}{##1}}}
\expandafter\def\csname PY@tok@vi\endcsname{\def\PY@tc##1{\textcolor[rgb]{0.10,0.09,0.49}{##1}}}
\expandafter\def\csname PY@tok@vm\endcsname{\def\PY@tc##1{\textcolor[rgb]{0.10,0.09,0.49}{##1}}}
\expandafter\def\csname PY@tok@sa\endcsname{\def\PY@tc##1{\textcolor[rgb]{0.73,0.13,0.13}{##1}}}
\expandafter\def\csname PY@tok@sb\endcsname{\def\PY@tc##1{\textcolor[rgb]{0.73,0.13,0.13}{##1}}}
\expandafter\def\csname PY@tok@sc\endcsname{\def\PY@tc##1{\textcolor[rgb]{0.73,0.13,0.13}{##1}}}
\expandafter\def\csname PY@tok@dl\endcsname{\def\PY@tc##1{\textcolor[rgb]{0.73,0.13,0.13}{##1}}}
\expandafter\def\csname PY@tok@s2\endcsname{\def\PY@tc##1{\textcolor[rgb]{0.73,0.13,0.13}{##1}}}
\expandafter\def\csname PY@tok@sh\endcsname{\def\PY@tc##1{\textcolor[rgb]{0.73,0.13,0.13}{##1}}}
\expandafter\def\csname PY@tok@s1\endcsname{\def\PY@tc##1{\textcolor[rgb]{0.73,0.13,0.13}{##1}}}
\expandafter\def\csname PY@tok@mb\endcsname{\def\PY@tc##1{\textcolor[rgb]{0.40,0.40,0.40}{##1}}}
\expandafter\def\csname PY@tok@mf\endcsname{\def\PY@tc##1{\textcolor[rgb]{0.40,0.40,0.40}{##1}}}
\expandafter\def\csname PY@tok@mh\endcsname{\def\PY@tc##1{\textcolor[rgb]{0.40,0.40,0.40}{##1}}}
\expandafter\def\csname PY@tok@mi\endcsname{\def\PY@tc##1{\textcolor[rgb]{0.40,0.40,0.40}{##1}}}
\expandafter\def\csname PY@tok@il\endcsname{\def\PY@tc##1{\textcolor[rgb]{0.40,0.40,0.40}{##1}}}
\expandafter\def\csname PY@tok@mo\endcsname{\def\PY@tc##1{\textcolor[rgb]{0.40,0.40,0.40}{##1}}}
\expandafter\def\csname PY@tok@ch\endcsname{\let\PY@it=\textit\def\PY@tc##1{\textcolor[rgb]{0.25,0.50,0.50}{##1}}}
\expandafter\def\csname PY@tok@cm\endcsname{\let\PY@it=\textit\def\PY@tc##1{\textcolor[rgb]{0.25,0.50,0.50}{##1}}}
\expandafter\def\csname PY@tok@cpf\endcsname{\let\PY@it=\textit\def\PY@tc##1{\textcolor[rgb]{0.25,0.50,0.50}{##1}}}
\expandafter\def\csname PY@tok@c1\endcsname{\let\PY@it=\textit\def\PY@tc##1{\textcolor[rgb]{0.25,0.50,0.50}{##1}}}
\expandafter\def\csname PY@tok@cs\endcsname{\let\PY@it=\textit\def\PY@tc##1{\textcolor[rgb]{0.25,0.50,0.50}{##1}}}

\def\PYZbs{\char`\\}
\def\PYZus{\char`\_}
\def\PYZob{\char`\{}
\def\PYZcb{\char`\}}
\def\PYZca{\char`\^}
\def\PYZam{\char`\&}
\def\PYZlt{\char`\<}
\def\PYZgt{\char`\>}
\def\PYZsh{\char`\#}
\def\PYZpc{\char`\%}
\def\PYZdl{\char`\$}
\def\PYZhy{\char`\-}
\def\PYZsq{\char`\'}
\def\PYZdq{\char`\"}
\def\PYZti{\char`\~}
% for compatibility with earlier versions
\def\PYZat{@}
\def\PYZlb{[}
\def\PYZrb{]}
\makeatother


    % For linebreaks inside Verbatim environment from package fancyvrb. 
    \makeatletter
        \newbox\Wrappedcontinuationbox 
        \newbox\Wrappedvisiblespacebox 
        \newcommand*\Wrappedvisiblespace {\textcolor{red}{\textvisiblespace}} 
        \newcommand*\Wrappedcontinuationsymbol {\textcolor{red}{\llap{\tiny$\m@th\hookrightarrow$}}} 
        \newcommand*\Wrappedcontinuationindent {3ex } 
        \newcommand*\Wrappedafterbreak {\kern\Wrappedcontinuationindent\copy\Wrappedcontinuationbox} 
        % Take advantage of the already applied Pygments mark-up to insert 
        % potential linebreaks for TeX processing. 
        %        {, <, #, %, $, ' and ": go to next line. 
        %        _, }, ^, &, >, - and ~: stay at end of broken line. 
        % Use of \textquotesingle for straight quote. 
        \newcommand*\Wrappedbreaksatspecials {% 
            \def\PYGZus{\discretionary{\char`\_}{\Wrappedafterbreak}{\char`\_}}% 
            \def\PYGZob{\discretionary{}{\Wrappedafterbreak\char`\{}{\char`\{}}% 
            \def\PYGZcb{\discretionary{\char`\}}{\Wrappedafterbreak}{\char`\}}}% 
            \def\PYGZca{\discretionary{\char`\^}{\Wrappedafterbreak}{\char`\^}}% 
            \def\PYGZam{\discretionary{\char`\&}{\Wrappedafterbreak}{\char`\&}}% 
            \def\PYGZlt{\discretionary{}{\Wrappedafterbreak\char`\<}{\char`\<}}% 
            \def\PYGZgt{\discretionary{\char`\>}{\Wrappedafterbreak}{\char`\>}}% 
            \def\PYGZsh{\discretionary{}{\Wrappedafterbreak\char`\#}{\char`\#}}% 
            \def\PYGZpc{\discretionary{}{\Wrappedafterbreak\char`\%}{\char`\%}}% 
            \def\PYGZdl{\discretionary{}{\Wrappedafterbreak\char`\$}{\char`\$}}% 
            \def\PYGZhy{\discretionary{\char`\-}{\Wrappedafterbreak}{\char`\-}}% 
            \def\PYGZsq{\discretionary{}{\Wrappedafterbreak\textquotesingle}{\textquotesingle}}% 
            \def\PYGZdq{\discretionary{}{\Wrappedafterbreak\char`\"}{\char`\"}}% 
            \def\PYGZti{\discretionary{\char`\~}{\Wrappedafterbreak}{\char`\~}}% 
        } 
        % Some characters . , ; ? ! / are not pygmentized. 
        % This macro makes them "active" and they will insert potential linebreaks 
        \newcommand*\Wrappedbreaksatpunct {% 
            \lccode`\~`\.\lowercase{\def~}{\discretionary{\hbox{\char`\.}}{\Wrappedafterbreak}{\hbox{\char`\.}}}% 
            \lccode`\~`\,\lowercase{\def~}{\discretionary{\hbox{\char`\,}}{\Wrappedafterbreak}{\hbox{\char`\,}}}% 
            \lccode`\~`\;\lowercase{\def~}{\discretionary{\hbox{\char`\;}}{\Wrappedafterbreak}{\hbox{\char`\;}}}% 
            \lccode`\~`\:\lowercase{\def~}{\discretionary{\hbox{\char`\:}}{\Wrappedafterbreak}{\hbox{\char`\:}}}% 
            \lccode`\~`\?\lowercase{\def~}{\discretionary{\hbox{\char`\?}}{\Wrappedafterbreak}{\hbox{\char`\?}}}% 
            \lccode`\~`\!\lowercase{\def~}{\discretionary{\hbox{\char`\!}}{\Wrappedafterbreak}{\hbox{\char`\!}}}% 
            \lccode`\~`\/\lowercase{\def~}{\discretionary{\hbox{\char`\/}}{\Wrappedafterbreak}{\hbox{\char`\/}}}% 
            \catcode`\.\active
            \catcode`\,\active 
            \catcode`\;\active
            \catcode`\:\active
            \catcode`\?\active
            \catcode`\!\active
            \catcode`\/\active 
            \lccode`\~`\~ 	
        }
    \makeatother

    \let\OriginalVerbatim=\Verbatim
    \makeatletter
    \renewcommand{\Verbatim}[1][1]{%
        %\parskip\z@skip
        \sbox\Wrappedcontinuationbox {\Wrappedcontinuationsymbol}%
        \sbox\Wrappedvisiblespacebox {\FV@SetupFont\Wrappedvisiblespace}%
        \def\FancyVerbFormatLine ##1{\hsize\linewidth
            \vtop{\raggedright\hyphenpenalty\z@\exhyphenpenalty\z@
                \doublehyphendemerits\z@\finalhyphendemerits\z@
                \strut ##1\strut}%
        }%
        % If the linebreak is at a space, the latter will be displayed as visible
        % space at end of first line, and a continuation symbol starts next line.
        % Stretch/shrink are however usually zero for typewriter font.
        \def\FV@Space {%
            \nobreak\hskip\z@ plus\fontdimen3\font minus\fontdimen4\font
            \discretionary{\copy\Wrappedvisiblespacebox}{\Wrappedafterbreak}
            {\kern\fontdimen2\font}%
        }%
        
        % Allow breaks at special characters using \PYG... macros.
        \Wrappedbreaksatspecials
        % Breaks at punctuation characters . , ; ? ! and / need catcode=\active 	
        \OriginalVerbatim[#1,codes*=\Wrappedbreaksatpunct]%
    }
    \makeatother

    % Exact colors from NB
    \definecolor{incolor}{HTML}{303F9F}
    \definecolor{outcolor}{HTML}{D84315}
    \definecolor{cellborder}{HTML}{CFCFCF}
    \definecolor{cellbackground}{HTML}{F7F7F7}
    
    % prompt
    \newcommand{\prompt}[4]{
        \llap{{\color{#2}[#3]: #4}}\vspace{-1.25em}
    }
    

    
    % Prevent overflowing lines due to hard-to-break entities
    \sloppy 
    % Setup hyperref package
    \hypersetup{
      breaklinks=true,  % so long urls are correctly broken across lines
      colorlinks=true,
      urlcolor=urlcolor,
      linkcolor=linkcolor,
      citecolor=citecolor,
      }
    % Slightly bigger margins than the latex defaults
    
    \geometry{verbose,tmargin=1in,bmargin=1in,lmargin=1in,rmargin=1in}
    
    

    \begin{document}
    
    
    \title{Problem Set 1 - ECON 712}\author{Mitchell Valdés-Bobes}\affiliation{University of Wisconsin Madison}

\date{\today}
\maketitle


    
    

    \begin{Verbatim}[commandchars=\\\{\}]
{\color{incolor}In [{\color{incolor}2}]:} \PY{n}{addpath}\PY{p}{(}\PY{l+s}{\PYZsq{}}\PY{l+s}{./Utils\PYZsq{}}\PY{p}{)}
        \PY{n}{format} \PY{l+s}{compact}
\end{Verbatim}

    \begin{Verbatim}[commandchars=\\\{\}]


    \end{Verbatim}

    Here we are interested in how the stock market may react to a future
FOMC announcement of an increase in short term interest rates. To do so,
consider the following simple perfect foresight model of stock price
dynamics given by the following equation \[
p_{t}=\frac{d+p_{t+1}}{(1+r)}
\] where \(p_{t}\) is the price of a share at the beginning of period
\(t\) before constant dividend \(d\) is paid out, and \(r\) is the short
term risk free interest rate. The left hand side of (1) is the cost of
buying a share while the right hand side is the benefit of buying the
share (the owner receives a dividend and capital gain or loss from the
sale of the share). Assume \(r>0\)

    \hypertarget{problem-1}{%
\section{Problem 1}\label{problem-1}}

Solve for the steady state stock price \(p^{*}=p_{t}=p_{t+1}\).

    \hypertarget{answer}{%
\subsection{Answer}\label{answer}}

The steady state is a solution in which prices don't change between
periods:

\begin{align*}
    p^{*}=p_{t}=p_{t+1} &\quad \iff \quad  p^* = \frac{d + p^*}{1+r}\\
    &\quad \iff \quad (1+r)p^* =d + p^*\\
    &\quad \iff \quad rp^* =d\\
    &\quad \iff \quad \boxed{p^* =\frac{d}{r}}\\
\end{align*}

    \hypertarget{problem-2}{%
\section{Problem 2}\label{problem-2}}

Assume the initial price, \(p_{0},\) is given. Solve the closed form
solution to the first order linear difference equation in (1). Explain
how price evolves over time (i.e.~what if \(p_{0}>p^{*}, p_{0}<p^{*}\),
\(\left.p_{0}=p^{*}\right)\) using both a phase diagram (i.e \(p_{t+1}\)
against \(p_{t}\) ) as well as a graph of \(p_{t}\) against time If the
initial stock price is away from the steady state, does it converge or
diverge from the steady state. Explain why.

    \hypertarget{answer}{%
\subsection{Answer}\label{answer}}

Re-writing the difference equation we have:

\[p_{t}=\frac{d+p_{t+1}}{(1+r)} \quad \implies \quad p_{t+1}=\underbrace{(1+r)}_{a}p_t - \underbrace{d}_{b} \]

First we solve the homogenous equation:

\[p_{t+1} = a p_{t} \quad \implies p^c_t = \mathbb{c} a^t \]

as a particular solution we can use the steady state we found in
\textbf{Problem 1}:

\[p_t^{p} = p^* =\frac{d}{r}\]

then the general solution (after we plug \(a\) back in) is:

\[p^g_t = (1+r)^t \mathbb{c} + \frac{d}{r}\]

using the initial condition \(p_0\) we can obtain the value of
\(\mathbb{c}\):

\[p_0 = (1+r)^0 \mathbb{c} + \frac{d}{r} \quad \implies \mathbb{c} = p_0 - \frac{d}{r}\]

this gives us the general solution:

\begin{equation}\label{eq:solution}\tag{1.1}
p_t = (1+r)^t \left(  p_0 - \frac{d}{r} \right) + \frac{d}{r}
\end{equation}

    If we plot the phase diagram of this equation and initial price
\(p_0>p^*\) we get:
\begin{Verbatim}[commandchars=\\\{\}]
{\color{incolor}In [{\color{incolor}3}]:} \PY{c}{\PYZpc{}plot inline \PYZhy{}f=svg}
        \PY{n}{p1} \PY{p}{=} \PY{n}{figure}\PY{p}{(}\PY{p}{)}\PY{p}{;}
        
        \PY{n}{r} \PY{p}{=} \PY{l+m+mf}{0.8}\PY{p}{;} \PY{c}{\PYZpc{} interest rate}
        \PY{n}{d} \PY{p}{=} \PY{l+m+mi}{100}\PY{p}{;} \PY{c}{\PYZpc{} constant dividend}
        
        \PY{n}{p\PYZus{}star} \PY{p}{=} \PY{n}{d}\PY{o}{/}\PY{n}{r}\PY{p}{;} \PY{c}{\PYZpc{} steady state}
        
        \PY{n}{hold} \PY{l+s}{on}
        \PY{n}{p} \PY{p}{=} \PY{l+m+mi}{0}\PY{p}{:}\PY{l+m+mf}{0.001}\PY{p}{:}\PY{l+m+mi}{300}\PY{p}{;}
        \PY{n}{plot}\PY{p}{(}\PY{n}{p}\PY{p}{,} \PY{n}{p}\PY{p}{,} \PY{l+s}{\PYZdq{}LineWidth\PYZdq{}}\PY{p}{,} \PY{l+m+mf}{1.5}\PY{p}{)}
        \PY{n}{plot}\PY{p}{(}\PY{n}{p}\PY{p}{,} \PY{p}{(}\PY{l+m+mi}{1}\PY{o}{+}\PY{n}{r}\PY{p}{)}\PY{o}{*}\PY{n}{p}\PY{o}{\PYZhy{}}\PY{n}{d}\PY{p}{,} \PY{l+s}{\PYZdq{}LineWidth\PYZdq{}}\PY{p}{,} \PY{l+m+mf}{1.5}\PY{p}{)}
        
        \PY{n}{scatter}\PY{p}{(}\PY{n}{p\PYZus{}star}\PY{p}{,} \PY{n}{p\PYZus{}star}\PY{p}{,} \PY{l+m+mi}{55}\PY{p}{,} \PY{l+s}{\PYZsq{}}\PY{l+s}{filled\PYZsq{}}\PY{p}{)}
        
        
        \PY{n}{xticks}\PY{p}{(}\PY{p}{[}\PY{n}{p\PYZus{}star}\PY{p}{]}\PY{p}{)}\PY{p}{;} \PY{n}{xticklabels}\PY{p}{(}\PY{p}{[}\PY{l+s}{\PYZsq{}}\PY{l+s}{\PYZdl{}p\PYZca{}*\PYZdl{}\PYZsq{}}\PY{p}{]}\PY{p}{)}
        \PY{n}{yticks}\PY{p}{(}\PY{p}{[}\PY{n}{p\PYZus{}star}\PY{p}{]}\PY{p}{)}\PY{p}{;} \PY{n}{yticklabels}\PY{p}{(}\PY{p}{[}\PY{l+s}{\PYZsq{}}\PY{l+s}{\PYZdl{}p\PYZca{}*\PYZdl{}\PYZsq{}}\PY{p}{]}\PY{p}{)}
        
        \PY{n}{h} \PY{p}{=} \PY{n}{title}\PY{p}{(}\PY{l+s}{\PYZsq{}}\PY{l+s}{Phase Diagram\PYZsq{}}\PY{p}{,} \PY{l+s}{\PYZsq{}}\PY{l+s}{interpreter\PYZsq{}}\PY{p}{,} \PY{l+s}{\PYZsq{}}\PY{l+s}{latex\PYZsq{}}\PY{p}{)}\PY{p}{;}
        \PY{n}{h}\PY{p}{.}\PY{n}{FontSize}\PY{p}{=}\PY{l+m+mi}{15}\PY{p}{;}
        
        \PY{n}{ax} \PY{p}{=} \PY{n}{gca}\PY{p}{;}
        \PY{n}{ax}\PY{p}{.}\PY{n}{XAxisLocation} \PY{p}{=} \PY{l+s}{\PYZsq{}}\PY{l+s}{origin\PYZsq{}}\PY{p}{;}
        \PY{n}{ax}\PY{p}{.}\PY{n}{YAxisLocation} \PY{p}{=} \PY{l+s}{\PYZsq{}}\PY{l+s}{origin\PYZsq{}}\PY{p}{;}
        \PY{n}{pbaspect}\PY{p}{(}\PY{p}{[}\PY{l+m+mi}{1} \PY{l+m+mi}{1} \PY{l+m+mi}{1}\PY{p}{]}\PY{p}{)}
        
        \PY{n}{xh} \PY{p}{=} \PY{n}{xlabel}\PY{p}{(}\PY{l+s}{\PYZsq{}}\PY{l+s}{\PYZdl{}p\PYZus{}t\PYZdl{}\PYZsq{}}\PY{p}{,} \PY{l+s}{\PYZsq{}}\PY{l+s}{interpreter\PYZsq{}}\PY{p}{,} \PY{l+s}{\PYZsq{}}\PY{l+s}{latex\PYZsq{}}\PY{p}{)}\PY{p}{;} \PY{n}{xh}\PY{p}{.}\PY{n}{FontSize} \PY{p}{=} \PY{l+m+mi}{13}\PY{p}{;}
        \PY{n}{yh} \PY{p}{=} \PY{n}{ylabel}\PY{p}{(}\PY{l+s}{\PYZsq{}}\PY{l+s}{\PYZdl{}p\PYZus{}\PYZob{}t+1\PYZcb{}\PYZdl{}\PYZsq{}}\PY{p}{,} \PY{l+s}{\PYZsq{}}\PY{l+s}{interpreter\PYZsq{}}\PY{p}{,} \PY{l+s}{\PYZsq{}}\PY{l+s}{latex\PYZsq{}}\PY{p}{)}\PY{p}{;} \PY{n}{yh}\PY{p}{.}\PY{n}{FontSize} \PY{p}{=} \PY{l+m+mi}{13}\PY{p}{;}
        \PY{n}{yh}\PY{p}{.}\PY{n}{Position}\PY{p}{(}\PY{l+m+mi}{2}\PY{p}{)} \PY{p}{=} \PY{n}{yh}\PY{p}{.}\PY{n}{Position}\PY{p}{(}\PY{l+m+mi}{2}\PY{p}{)} \PY{o}{\PYZhy{}} \PY{l+m+mi}{12}\PY{p}{;}
        
        \PY{n}{lgd} \PY{p}{=} \PY{n}{legend}\PY{p}{(}\PY{p}{[}\PY{l+s}{\PYZdq{}\PYZdl{}p\PYZus{}t = p\PYZus{}\PYZob{}t+1\PYZcb{} = p\PYZca{}\PYZob{}*\PYZcb{}\PYZdl{}\PYZdq{}}\PY{p}{,} \PY{c}{...}
                      \PY{l+s}{\PYZdq{}\PYZdl{}p\PYZus{}t = (1+r)\PYZca{}t \PYZbs{}left(  p\PYZus{}0 \PYZhy{} p\PYZca{}* \PYZbs{}right) + p\PYZca{}*\PYZdl{}\PYZdq{}}\PY{p}{]}\PY{p}{,} \PY{c}{...}
                      \PY{l+s}{\PYZdq{}location\PYZdq{}}\PY{p}{,} \PY{l+s}{\PYZdq{}Northwest\PYZdq{}}\PY{p}{,} \PY{l+s}{\PYZdq{}interpreter\PYZdq{}}\PY{p}{,} \PY{l+s}{\PYZdq{}latex\PYZdq{}}\PY{p}{)}\PY{p}{;}
        \PY{n}{lgd}\PY{p}{.}\PY{n}{FontSize} \PY{p}{=} \PY{l+m+mi}{12}\PY{p}{;}
        
        \PY{n}{set}\PY{p}{(}\PY{n}{gca}\PY{p}{,}\PY{l+s}{\PYZsq{}}\PY{l+s}{TickLabelInterpreter\PYZsq{}}\PY{p}{,}\PY{l+s}{\PYZsq{}}\PY{l+s}{latex\PYZsq{}}\PY{p}{)}
\end{Verbatim}

    \begin{Verbatim}[commandchars=\\\{\}]


    \end{Verbatim}

    \begin{center}
    \adjustimage{max size={0.9\linewidth}{0.9\paperheight}}{Problem_Set_1_files/Problem_Set_1_7_1.png}
    \end{center}
    { \hspace*{\fill} \\}
    
    We could also analyze the price dynamics to observe how prices evolve
over time depending on the initial price \(p_0\).
\begin{Verbatim}[commandchars=\\\{\}]
{\color{incolor}In [{\color{incolor}4}]:} \PY{n}{figure}\PY{p}{(}\PY{p}{)}\PY{p}{;}
        
        \PY{n}{r} \PY{p}{=} \PY{l+m+mf}{0.01}\PY{p}{;} \PY{c}{\PYZpc{} interest rate}
        \PY{n}{d} \PY{p}{=} \PY{l+m+mi}{1}\PY{p}{;} \PY{c}{\PYZpc{} constant dividend}
        
        \PY{n}{p\PYZus{}star} \PY{p}{=} \PY{n}{d}\PY{o}{/}\PY{n}{r}\PY{p}{;} \PY{c}{\PYZpc{} steady state}
        
        \PY{n}{P0} \PY{p}{=} \PY{p}{[}\PY{n}{p\PYZus{}star} \PY{n}{p\PYZus{}star} \PY{n}{p\PYZus{}star}\PY{p}{]} \PY{o}{+} \PY{l+m+mi}{30}\PY{o}{*}\PY{p}{[}\PY{o}{\PYZhy{}}\PY{l+m+mi}{1} \PY{l+m+mi}{0} \PY{l+m+mi}{1}\PY{p}{]}\PY{p}{;}
        
        \PY{n}{hold} \PY{l+s}{on}
        
        \PY{n}{t} \PY{p}{=} \PY{l+m+mi}{0}\PY{p}{:}\PY{l+m+mi}{200}\PY{p}{;}
        
        \PY{n}{for} \PY{l+s}{p0} \PY{l+s}{=} \PY{l+s}{P0}
            \PY{n}{plot}\PY{p}{(}\PY{n}{t}\PY{p}{,} \PY{p}{(}\PY{n}{p0} \PY{o}{\PYZhy{}} \PY{n}{p\PYZus{}star}\PY{p}{)}\PY{o}{*}\PY{p}{(}\PY{l+m+mi}{1}\PY{o}{+}\PY{n}{r}\PY{p}{)}\PY{o}{.\PYZca{}}\PY{n}{t} \PY{o}{+} \PY{n}{p\PYZus{}star}\PY{p}{,} \PY{l+s}{\PYZdq{}LineWidth\PYZdq{}}\PY{p}{,} \PY{l+m+mi}{3}\PY{p}{)}
        \PY{k}{end}
        
        \PY{n}{h} \PY{p}{=} \PY{n}{title}\PY{p}{(}\PY{l+s}{\PYZsq{}}\PY{l+s}{Price Dynamics\PYZsq{}}\PY{p}{,} \PY{l+s}{\PYZsq{}}\PY{l+s}{interpreter\PYZsq{}}\PY{p}{,} \PY{l+s}{\PYZsq{}}\PY{l+s}{latex\PYZsq{}}\PY{p}{)}\PY{p}{;}
        \PY{n}{h}\PY{p}{.}\PY{n}{FontSize}\PY{p}{=}\PY{l+m+mi}{15}\PY{p}{;}
        \PY{n}{xticks}\PY{p}{(}\PY{p}{[}\PY{l+m+mi}{0}\PY{p}{:}\PY{l+m+mi}{25}\PY{p}{:}\PY{l+m+mi}{200}\PY{p}{]}\PY{p}{)}\PY{p}{;}
        \PY{n}{xticklabels}\PY{p}{(}\PY{n+nb}{repmat}\PY{p}{(}\PY{l+s}{\PYZdq{}\PYZdl{}t=\PYZdq{}}\PY{p}{,} \PY{l+m+mi}{1}\PY{p}{,}\PY{l+m+mi}{9}\PY{p}{)} \PY{o}{+} \PY{n}{string}\PY{p}{(}\PY{p}{[}\PY{l+m+mi}{0}\PY{p}{:}\PY{l+m+mi}{25}\PY{p}{:}\PY{l+m+mi}{200}\PY{p}{]}\PY{p}{)} \PY{o}{+} \PY{n+nb}{repmat}\PY{p}{(}\PY{l+s}{\PYZdq{}\PYZdl{}\PYZdq{}}\PY{p}{,} \PY{l+m+mi}{1}\PY{p}{,}\PY{l+m+mi}{9}\PY{p}{)}\PY{p}{)}
        \PY{n}{yticks}\PY{p}{(}\PY{n}{P0}\PY{p}{)}\PY{p}{;} \PY{n}{yticklabels}\PY{p}{(}\PY{n+nb}{repmat}\PY{p}{(}\PY{l+s}{\PYZdq{}\PYZdl{}P\PYZus{}0=\PYZdq{}}\PY{p}{,} \PY{l+m+mi}{1}\PY{p}{,}\PY{l+m+mi}{3}\PY{p}{)}\PY{o}{+} \PY{n}{P0}\PY{o}{+}\PY{l+s}{\PYZdq{}\PYZdl{}\PYZdq{}}\PY{p}{)}
        \PY{n}{xh} \PY{p}{=} \PY{n}{xlabel}\PY{p}{(}\PY{l+s}{\PYZsq{}}\PY{l+s}{Time \PYZdl{}t\PYZdl{}\PYZsq{}}\PY{p}{,} \PY{l+s}{\PYZsq{}}\PY{l+s}{interpreter\PYZsq{}}\PY{p}{,} \PY{l+s}{\PYZsq{}}\PY{l+s}{latex\PYZsq{}}\PY{p}{)}\PY{p}{;} \PY{n}{xh}\PY{p}{.}\PY{n}{FontSize} \PY{p}{=} \PY{l+m+mi}{13}\PY{p}{;}
        \PY{n}{yh} \PY{p}{=} \PY{n}{ylabel}\PY{p}{(}\PY{l+s}{\PYZsq{}}\PY{l+s}{Price \PYZdl{}P\PYZus{}t\PYZdl{}\PYZsq{}}\PY{p}{,} \PY{l+s}{\PYZsq{}}\PY{l+s}{interpreter\PYZsq{}}\PY{p}{,} \PY{l+s}{\PYZsq{}}\PY{l+s}{latex\PYZsq{}}\PY{p}{)}\PY{p}{;} \PY{n}{yh}\PY{p}{.}\PY{n}{FontSize} \PY{p}{=} \PY{l+m+mi}{13}\PY{p}{;}
        \PY{n}{set}\PY{p}{(}\PY{n}{gca}\PY{p}{,}\PY{l+s}{\PYZsq{}}\PY{l+s}{TickLabelInterpreter\PYZsq{}}\PY{p}{,}\PY{l+s}{\PYZsq{}}\PY{l+s}{latex\PYZsq{}}\PY{p}{)}
\end{Verbatim}

    \begin{Verbatim}[commandchars=\\\{\}]


    \end{Verbatim}

    \begin{center}
    \adjustimage{max size={0.9\linewidth}{0.9\paperheight}}{Problem_Set_1_files/Problem_Set_1_9_1.png}
    \end{center}
    { \hspace*{\fill} \\}
    
    To finally answer the question we can take a look at the solution of the
difference equation that describes the dynamics of prices:

\[p_t = (1+r)^t \left(  p_0 - p^* \right) + p^*\]

If we start with \(p_0<p^*\) then:

\[p_t = (1+r)^t  \underbrace{ \left(p_0 - p^*  \right)}_{<0} + p^* \qquad \implies p_t \to 0\]

starting at \(p_0>p^*\)

\[p_t = (1+r)^t  \underbrace{ \left(p_0 - p^*  \right)}_{>0} + p^* \qquad \implies p_t \to \infty\]

and at \(p_0 = p^*\):

\[p_t = (1+r)^t  \underbrace{ \left(p_0 - p^*  \right)}_{=0} + p^* \qquad \implies p_t = p^* \forall t\]

    \hypertarget{problem-3}{%
\section{Problem 3}\label{problem-3}}

\textbf{Matlab} Suppose the risk free rate is \(r=1 \%\) and the stock
pays constant dividend \(d=1\) per share per period. Open the Matlab
code we provided. The code generates and plots the price dynamics given
the first-order difference equation in part 2 given initial share price
\(p_{0}=100\) at time \(t=0 .\) Modify the code (i.e.~simply replace
\(\left.100\right)\) with three different initial prices which
respectively are below, at, and above the stead

    \hypertarget{answer}{%
\subsection{Answer}\label{answer}}

To show the price dynamics I modified the code that was provided to make
a function that accepts a vector of initial prices and plot how prices
evolve in each case. This is saved in the file \texttt{ps1\_func1.m}
that is attached to this document.
\begin{Verbatim}[commandchars=\\\{\}]
{\color{incolor}In [{\color{incolor}5}]:} \PY{c}{\PYZpc{}\PYZpc{}file Utils/ps1\PYZus{}func1.m}
        
        \PY{k}{function}\PY{+w}{ }p \PY{p}{=}\PY{+w}{ }\PY{n+nf}{ps1\PYZus{}func1}\PY{p}{(}P0\PY{p}{)}
        
        \PY{+w}{    }\PY{c}{\PYZpc{} PARAMETERS}
        
            \PY{c}{\PYZpc{} If no P0 vector is provided use the default P = [70 100 130]}
        
            \PY{n}{r} \PY{p}{=} \PY{l+m+mf}{0.01}\PY{p}{;} \PY{c}{\PYZpc{} interest rate}
            \PY{n}{d} \PY{p}{=} \PY{l+m+mi}{1}\PY{p}{;} \PY{c}{\PYZpc{} constant dividend}
            \PY{n}{dim} \PY{p}{=} \PY{l+m+mi}{99}\PY{p}{;} \PY{c}{\PYZpc{} terminal period t = 99}
        
            \PY{n}{p\PYZus{}star} \PY{p}{=} \PY{n}{d}\PY{o}{/}\PY{n}{r}\PY{p}{;} \PY{c}{\PYZpc{} steady state}
        
            \PY{n}{if} \PY{l+s}{nargin} \PY{l+s}{==} \PY{l+s}{0}
                \PY{n}{P0} \PY{p}{=} \PY{p}{[}\PY{n}{p\PYZus{}star} \PY{n}{p\PYZus{}star} \PY{n}{p\PYZus{}star}\PY{p}{]} \PY{o}{+} \PY{l+m+mi}{30}\PY{o}{*}\PY{p}{[}\PY{o}{\PYZhy{}}\PY{l+m+mi}{1} \PY{l+m+mi}{0} \PY{l+m+mi}{1}\PY{p}{]}\PY{p}{;} \PY{c}{\PYZpc{} initial prices}
            \PY{n}{end}
        
            \PY{l+s}{figure()}\PY{p}{;}
            \PY{n}{hold} \PY{l+s}{on}
            \PY{n}{legend\PYZus{}str} \PY{p}{=} \PY{n+nb}{repmat}\PY{p}{(}\PY{l+s}{\PYZdq{}\PYZdl{}P\PYZus{}0 = \PYZdq{}}\PY{p}{,} \PY{l+m+mi}{1}\PY{p}{,} \PY{n+nb}{numel}\PY{p}{(}\PY{n}{P0}\PY{p}{)}\PY{p}{)}\PY{p}{;}
            \PY{n}{for} \PY{l+s}{i} \PY{l+s}{=} \PY{l+s}{1:numel(P0)}
                \PY{n}{p0} \PY{p}{=} \PY{n}{P0}\PY{p}{(}\PY{n+nb}{i}\PY{p}{)}\PY{p}{;}
                \PY{c}{\PYZpc{} INITIALIZATION}
                \PY{c}{\PYZpc{} creating a vector of price from t=0 to t=99}
                \PY{n}{pvector} \PY{p}{=} \PY{n+nb}{zeros}\PY{p}{(}\PY{n}{dim}\PY{o}{+}\PY{l+m+mi}{1}\PY{p}{,}\PY{l+m+mi}{1}\PY{p}{)}\PY{p}{;}
                \PY{n}{tvector} \PY{p}{=} \PY{n+nb}{linspace}\PY{p}{(}\PY{l+m+mi}{0}\PY{p}{,}\PY{n}{dim}\PY{p}{,}\PY{n}{dim}\PY{o}{+}\PY{l+m+mi}{1}\PY{p}{)}\PY{o}{\PYZsq{}}\PY{p}{;} \PY{c}{\PYZpc{} creating a vector for time from 0 to 99}
                \PY{n}{pvector}\PY{p}{(}\PY{l+m+mi}{1}\PY{p}{)} \PY{p}{=} \PY{n}{p0}\PY{p}{;} \PY{c}{\PYZpc{} giving value to the first element of the price vector}
                                 \PY{c}{\PYZpc{} with the initial price}
        
        
               \PY{c}{\PYZpc{} DYNAMICS}
                \PY{n}{for} \PY{l+s}{n} \PY{l+s}{=} \PY{l+s}{2:dim+1} \PY{l+s}{\PYZpc{}} \PY{l+s}{starting} \PY{l+s}{from} \PY{l+s}{t} \PY{l+s}{=} \PY{l+s}{1} \PY{l+s}{to} \PY{l+s}{t} \PY{l+s}{=} \PY{l+s}{100}
                   \PY{c}{\PYZpc{} updating the price in the next period}
                   \PY{n}{pvector}\PY{p}{(}\PY{n}{n}\PY{p}{)} \PY{p}{=} \PY{p}{(}\PY{l+m+mi}{1}\PY{o}{+}\PY{n}{r}\PY{p}{)}\PY{o}{*}\PY{n}{pvector}\PY{p}{(}\PY{n}{n}\PY{o}{\PYZhy{}}\PY{l+m+mi}{1}\PY{p}{)}\PY{o}{\PYZhy{}}\PY{n}{d}\PY{p}{;}
                                                   \PY{c}{\PYZpc{} with the first\PYZhy{}order difference}
                                                   \PY{c}{\PYZpc{} equation}
                \PY{n}{end}
        
        
                \PY{l+s}{\PYZpc{}} \PY{l+s}{PLOTS}
        
                \PY{n}{plot}\PY{p}{(}\PY{n}{tvector}\PY{p}{(}\PY{p}{:}\PY{p}{)}\PY{p}{,}\PY{n}{pvector}\PY{p}{(}\PY{p}{:}\PY{p}{)}\PY{p}{,} \PY{l+s}{\PYZdq{}LineWidth\PYZdq{}}\PY{p}{,} \PY{l+m+mi}{2}\PY{p}{)}\PY{p}{;}
                \PY{n}{legend\PYZus{}str}\PY{p}{(}\PY{n+nb}{i}\PY{p}{)} \PY{p}{=} \PY{n}{legend\PYZus{}str}\PY{p}{(}\PY{n+nb}{i}\PY{p}{)} \PY{o}{+} \PY{n}{p0} \PY{o}{+}\PY{l+s}{\PYZdq{}\PYZdl{}\PYZdq{}}\PY{p}{;}
            \PY{n}{end}
        
            \PY{l+s}{h} \PY{l+s}{=} \PY{l+s}{title(}\PY{l+s}{\PYZsq{}Price Dynamics\PYZsq{}}\PY{l+s}{,} \PY{l+s}{\PYZsq{}interpreter\PYZsq{}}\PY{l+s}{,} \PY{l+s}{\PYZsq{}latex\PYZsq{}}\PY{l+s}{)}\PY{p}{;}
            \PY{n}{h}\PY{p}{.}\PY{n}{FontSize}\PY{p}{=}\PY{l+m+mi}{15}\PY{p}{;}
            \PY{n}{xh} \PY{p}{=} \PY{n}{xlabel}\PY{p}{(}\PY{l+s}{\PYZsq{}}\PY{l+s}{Time \PYZdl{}t\PYZdl{}\PYZsq{}}\PY{p}{,} \PY{l+s}{\PYZsq{}}\PY{l+s}{interpreter\PYZsq{}}\PY{p}{,} \PY{l+s}{\PYZsq{}}\PY{l+s}{latex\PYZsq{}}\PY{p}{)}\PY{p}{;} \PY{n}{xh}\PY{p}{.}\PY{n}{FontSize} \PY{p}{=} \PY{l+m+mi}{13}\PY{p}{;}
            \PY{n}{yh} \PY{p}{=} \PY{n}{ylabel}\PY{p}{(}\PY{l+s}{\PYZsq{}}\PY{l+s}{Price \PYZdl{}P\PYZus{}t\PYZdl{}\PYZsq{}}\PY{p}{,} \PY{l+s}{\PYZsq{}}\PY{l+s}{interpreter\PYZsq{}}\PY{p}{,} \PY{l+s}{\PYZsq{}}\PY{l+s}{latex\PYZsq{}}\PY{p}{)}\PY{p}{;} \PY{n}{yh}\PY{p}{.}\PY{n}{FontSize} \PY{p}{=} \PY{l+m+mi}{13}\PY{p}{;}
            \PY{n}{legend}\PY{p}{(}\PY{n}{legend\PYZus{}str}\PY{p}{,} \PY{l+s}{\PYZsq{}}\PY{l+s}{Location\PYZsq{}}\PY{p}{,} \PY{l+s}{\PYZsq{}}\PY{l+s}{Northeast\PYZsq{}}\PY{p}{,} \PY{l+s}{\PYZdq{}interpreter\PYZdq{}}\PY{p}{,} \PY{l+s}{\PYZdq{}latex\PYZdq{}}\PY{p}{)}\PY{p}{;}
            \PY{n}{set}\PY{p}{(}\PY{n}{gca}\PY{p}{,}\PY{l+s}{\PYZsq{}}\PY{l+s}{TickLabelInterpreter\PYZsq{}}\PY{p}{,}\PY{l+s}{\PYZsq{}}\PY{l+s}{latex\PYZsq{}}\PY{p}{)}
            \PY{n}{axis}\PY{p}{(}\PY{p}{[}\PY{l+m+mi}{0} \PY{n}{dim} \PY{l+m+mi}{0} \PY{l+m+mi}{150}\PY{p}{]}\PY{p}{)}
        \PY{k}{end}
\end{Verbatim}

    \begin{Verbatim}[commandchars=\\\{\}]
Created file 'P:\textbackslash{}PHD\textbackslash{}Courses\textbackslash{}ECON 712 Macro\textbackslash{}Notebooks (Mathematica Jupyter
Sage)\textbackslash{}Utils\textbackslash{}ps1\_func1.m'.

    \end{Verbatim}

    Starting with \(p^*=100\) which is the steady state price resulting from
the parameters given we could plot the evolution of prices, using as the
vector of initial prices the default \texttt{{[}70\ 100\ 130{]}}:
\begin{Verbatim}[commandchars=\\\{\}]
{\color{incolor}In [{\color{incolor}6}]:} \PY{n}{ps1\PYZus{}func1}\PY{p}{(}\PY{p}{)}
\end{Verbatim}

    \begin{Verbatim}[commandchars=\\\{\}]


    \end{Verbatim}

    \begin{center}
    \adjustimage{max size={0.9\linewidth}{0.9\paperheight}}{Problem_Set_1_files/Problem_Set_1_15_1.png}
    \end{center}
    { \hspace*{\fill} \\}
    
    \hypertarget{problem-4}{%
\section{Problem 4}\label{problem-4}}

Suppose the Federal Reserve announces at \(t=20\) to raise the federal
funds rate from \(1 \%\) to \(2 \%\) at \(t=50\) and remain at the new
level forever. Using ( 1\()\) with \(d=1,\) how does the price respond
to the policy announcement and the interest rate change over time? Plot
the price dynamics from \(t=0\) to \(t=99\). Please assume that we rule
out rational bubbles (i.e., agents know and believe that prices will be
at its fundamental value, the steady state, at \(t=50\) ).

    \hypertarget{answer}{%
\subsection{Answer}\label{answer}}

We need to find how the market will respont at time \(t=20\) to the
anouncement that the interest rate will change at time \(t=50\).

We can define three disctinct time periods:

\[t<20, \enspace\enspace\enspace 20\leq t < 50, \enspace\enspace\enspace  50 < t\]

When \(t<20\) we asume that the market is in a steady state defined by:

\[p_t = p^* = \frac{d}{r_0}\]

where \(r_0=0.01\) is the initial value of the interest rate.

Since we can rule out rational bubbles, that in the period \(t>50\) the
prices will remain in the steady state defined at time \(t=50\). The
prices will stay in the steady state if and only if:

\[p_{50} = p^{**} = \frac{d}{r_1} \label{prob41}\tag{4.1}\]

where \(r_1 = 0.02\) is the new interest rate.

Finally, in \(t=20,...,50\) prices will update following the process
described in \(\eqref{eq:solution}\) but the initial price will be the
price of the stock at time \(t=20\) this is:

\[p_{50} = (1+r_0)^{50-20}\left(p_{20} -\frac{d}{r_0} \right) + \frac{d}{r_0}\label{prob42}\tag{4.2}\]

then by combining \(\eqref{prob41}\) and \(\eqref{prob42}\) we get:

\[(1+r_0)^{30}\left(p_{20} -\frac{d}{r_0} \right) + \frac{d}{r_0} = \frac{d}{r_1} \qquad \implies P_{20}=\frac{(1+r_0)^{-30}d(r_0-r_1)}{r_1 r_0}+\frac{d}{r_0}\]
\begin{Verbatim}[commandchars=\\\{\}]
{\color{incolor}In [{\color{incolor}7}]:} \PY{n}{figure}\PY{p}{(}\PY{p}{)}\PY{p}{;}
        
        \PY{n}{r\PYZus{}0} \PY{p}{=} \PY{l+m+mf}{0.01}\PY{p}{;} \PY{c}{\PYZpc{} old interest rate}
        \PY{n}{r\PYZus{}1} \PY{p}{=} \PY{l+m+mf}{0.02}\PY{p}{;} \PY{c}{\PYZpc{} new interest rate}
        \PY{n}{d} \PY{p}{=} \PY{l+m+mi}{1}\PY{p}{;} \PY{c}{\PYZpc{} constant dividend}
        
        \PY{n}{p\PYZus{}star} \PY{p}{=} \PY{n}{d}\PY{o}{/}\PY{n}{r\PYZus{}0}\PY{p}{;} \PY{c}{\PYZpc{} old steady state}
        \PY{n}{p\PYZus{}sstar} \PY{p}{=} \PY{n}{d}\PY{o}{/}\PY{n}{r\PYZus{}1}\PY{p}{;} \PY{c}{\PYZpc{} new steady state}
        
        
        \PY{n}{p\PYZus{}20} \PY{p}{=} \PY{p}{(}\PY{l+m+mi}{1}\PY{o}{+}\PY{n}{r\PYZus{}0}\PY{p}{)}\PYZca{}\PY{p}{(}\PY{o}{\PYZhy{}}\PY{l+m+mi}{30}\PY{p}{)}\PY{o}{*}\PY{n}{d}\PY{o}{*}\PY{p}{(}\PY{n}{r\PYZus{}0}\PY{o}{\PYZhy{}}\PY{n}{r\PYZus{}1}\PY{p}{)}\PY{o}{/}\PY{p}{(}\PY{n}{r\PYZus{}0}\PY{o}{*}\PY{n}{r\PYZus{}1}\PY{p}{)} \PY{o}{+} \PY{n}{d}\PY{o}{/}\PY{n}{r\PYZus{}0}\PY{p}{;}
        
        \PY{n}{hold} \PY{l+s}{on}
        
        \PY{n}{t\PYZus{}1} \PY{p}{=} \PY{l+m+mi}{0}\PY{p}{:}\PY{l+m+mi}{19}\PY{p}{;}  \PY{n}{p\PYZus{}1} \PY{p}{=} \PY{n+nb}{repmat}\PY{p}{(}\PY{n}{p\PYZus{}star}\PY{p}{,} \PY{l+m+mi}{1}\PY{p}{,} \PY{n+nb}{numel}\PY{p}{(}\PY{n}{t\PYZus{}1}\PY{p}{)}\PY{p}{)}\PY{p}{;}
        \PY{n}{t\PYZus{}2} \PY{p}{=} \PY{l+m+mi}{20}\PY{p}{:}\PY{l+m+mi}{50}\PY{p}{;} \PY{n}{p\PYZus{}2} \PY{p}{=} \PY{p}{(}\PY{n}{p\PYZus{}20} \PY{o}{\PYZhy{}} \PY{n}{p\PYZus{}star}\PY{p}{)}\PY{o}{*}\PY{p}{(}\PY{l+m+mi}{1}\PY{o}{+}\PY{n}{r\PYZus{}0}\PY{p}{)}\PY{o}{.\PYZca{}}\PY{p}{(}\PY{n}{t\PYZus{}2}\PY{o}{\PYZhy{}}\PY{l+m+mi}{20}\PY{p}{)} \PY{o}{+} \PY{n}{p\PYZus{}star}\PY{p}{;}
        \PY{n}{t\PYZus{}3} \PY{p}{=} \PY{l+m+mi}{51}\PY{p}{:}\PY{l+m+mi}{99}\PY{p}{;} \PY{n}{p\PYZus{}3} \PY{p}{=} \PY{n+nb}{repmat}\PY{p}{(}\PY{n}{p\PYZus{}sstar}\PY{p}{,} \PY{l+m+mi}{1}\PY{p}{,} \PY{n+nb}{numel}\PY{p}{(}\PY{n}{t\PYZus{}3}\PY{p}{)}\PY{p}{)}\PY{p}{;}
        
        \PY{n}{plot}\PY{p}{(}\PY{p}{[}\PY{n}{t\PYZus{}1} \PY{n}{t\PYZus{}2} \PY{n}{t\PYZus{}3}\PY{p}{]}\PY{p}{,} \PY{p}{[}\PY{n}{p\PYZus{}1} \PY{n}{p\PYZus{}2} \PY{n}{p\PYZus{}3}\PY{p}{]}\PY{p}{,} \PY{l+s}{\PYZdq{}LineWidth\PYZdq{}}\PY{p}{,} \PY{l+m+mi}{3}\PY{p}{)}
        
        \PY{n}{xticks}\PY{p}{(}\PY{p}{[}\PY{l+m+mi}{0}\PY{p}{:}\PY{l+m+mi}{10}\PY{p}{:}\PY{l+m+mi}{100}\PY{p}{]}\PY{p}{)}\PY{p}{;} \PY{n}{xticklabels}\PY{p}{(}\PY{n+nb}{repmat}\PY{p}{(}\PY{l+s}{\PYZdq{}t=\PYZdq{}}\PY{p}{,} \PY{l+m+mi}{1}\PY{p}{,}\PY{l+m+mi}{11}\PY{p}{)} \PY{o}{+} \PY{n}{string}\PY{p}{(}\PY{p}{[}\PY{l+m+mi}{0}\PY{p}{:}\PY{l+m+mi}{10}\PY{p}{:}\PY{l+m+mi}{100}\PY{p}{]}\PY{p}{)} \PY{p}{)}
        \PY{n}{yticks}\PY{p}{(}\PY{p}{[}\PY{n}{p\PYZus{}sstar} \PY{n+nb}{round}\PY{p}{(}\PY{n}{p\PYZus{}20}\PY{p}{)} \PY{n}{p\PYZus{}star}\PY{p}{]}\PY{p}{)}\PY{p}{;}
        \PY{n}{yticklabels}\PY{p}{(}\PY{p}{[}\PY{l+s}{\PYZdq{}\PYZdl{}p\PYZca{}\PYZob{}**\PYZcb{}= \PYZdq{}} \PY{l+s}{\PYZdq{}\PYZdl{}p\PYZus{}\PYZob{}20\PYZcb{}= \PYZdq{}} \PY{l+s}{\PYZdq{}\PYZdl{}p\PYZca{}*= \PYZdq{}}\PY{p}{]} \PY{c}{...}
                \PY{o}{+}\PY{p}{[}\PY{n}{p\PYZus{}sstar} \PY{n+nb}{round}\PY{p}{(}\PY{n}{p\PYZus{}20}\PY{p}{)} \PY{n}{p\PYZus{}star}\PY{p}{]}\PY{o}{+}\PY{p}{[}\PY{l+s}{\PYZdq{}\PYZdl{}\PYZdq{}} \PY{l+s}{\PYZdq{}\PYZdl{}\PYZdq{}} \PY{l+s}{\PYZdq{}\PYZdl{}\PYZdq{}}\PY{p}{]}\PY{p}{)}
        
        \PY{n}{plot}\PY{p}{(}\PY{p}{[}\PY{l+m+mi}{20} \PY{l+m+mi}{20}\PY{p}{]}\PY{p}{,}\PY{p}{[}\PY{l+m+mi}{0} \PY{n}{p\PYZus{}20}\PY{p}{]}\PY{p}{,} \PY{l+s}{\PYZsq{}}\PY{l+s}{:\PYZsq{}}\PY{p}{,} \PY{l+s}{\PYZsq{}}\PY{l+s}{color\PYZsq{}}\PY{p}{,} \PY{p}{[}\PY{l+m+mf}{0.8}\PY{p}{,}\PY{l+m+mf}{0.8}\PY{p}{,}\PY{l+m+mf}{0.8}\PY{p}{]}\PY{p}{,} \PY{l+s}{\PYZsq{}}\PY{l+s}{linewidth\PYZsq{}}\PY{p}{,}\PY{l+m+mi}{1}\PY{p}{)}
        \PY{n}{plot}\PY{p}{(}\PY{p}{[}\PY{l+m+mi}{0} \PY{l+m+mi}{20}\PY{p}{]}\PY{p}{,}\PY{p}{[}\PY{n}{p\PYZus{}20} \PY{n}{p\PYZus{}20}\PY{p}{]}\PY{p}{,} \PY{l+s}{\PYZsq{}}\PY{l+s}{:\PYZsq{}}\PY{p}{,} \PY{l+s}{\PYZsq{}}\PY{l+s}{color\PYZsq{}}\PY{p}{,} \PY{p}{[}\PY{l+m+mf}{0.8}\PY{p}{,}\PY{l+m+mf}{0.8}\PY{p}{,}\PY{l+m+mf}{0.8}\PY{p}{]}\PY{p}{,} \PY{l+s}{\PYZsq{}}\PY{l+s}{linewidth\PYZsq{}}\PY{p}{,}\PY{l+m+mi}{1}\PY{p}{)}
        
        \PY{n}{plot}\PY{p}{(}\PY{p}{[}\PY{l+m+mi}{50} \PY{l+m+mi}{50}\PY{p}{]}\PY{p}{,}\PY{p}{[}\PY{l+m+mi}{0} \PY{n}{p\PYZus{}sstar}\PY{p}{]}\PY{p}{,} \PY{l+s}{\PYZsq{}}\PY{l+s}{:\PYZsq{}}\PY{p}{,} \PY{l+s}{\PYZsq{}}\PY{l+s}{color\PYZsq{}}\PY{p}{,} \PY{p}{[}\PY{l+m+mf}{0.8}\PY{p}{,}\PY{l+m+mf}{0.8}\PY{p}{,}\PY{l+m+mf}{0.8}\PY{p}{]}\PY{p}{,} \PY{l+s}{\PYZsq{}}\PY{l+s}{linewidth\PYZsq{}}\PY{p}{,}\PY{l+m+mi}{1}\PY{p}{)}
        \PY{n}{plot}\PY{p}{(}\PY{p}{[}\PY{l+m+mi}{0} \PY{l+m+mi}{50}\PY{p}{]}\PY{p}{,}\PY{p}{[}\PY{n}{p\PYZus{}sstar} \PY{n}{p\PYZus{}sstar}\PY{p}{]}\PY{p}{,} \PY{l+s}{\PYZsq{}}\PY{l+s}{:\PYZsq{}}\PY{p}{,} \PY{l+s}{\PYZsq{}}\PY{l+s}{color\PYZsq{}}\PY{p}{,} \PY{p}{[}\PY{l+m+mf}{0.8}\PY{p}{,}\PY{l+m+mf}{0.8}\PY{p}{,}\PY{l+m+mf}{0.8}\PY{p}{]}\PY{p}{,} \PY{l+s}{\PYZsq{}}\PY{l+s}{linewidth\PYZsq{}}\PY{p}{,}\PY{l+m+mi}{1}\PY{p}{)}
        
        
        \PY{n}{h} \PY{p}{=} \PY{n}{title}\PY{p}{(}\PY{l+s}{\PYZsq{}}\PY{l+s}{Price Dynamics\PYZsq{}}\PY{p}{,} \PY{l+s}{\PYZsq{}}\PY{l+s}{interpreter\PYZsq{}}\PY{p}{,} \PY{l+s}{\PYZsq{}}\PY{l+s}{latex\PYZsq{}}\PY{p}{)}\PY{p}{;}
        \PY{n}{h}\PY{p}{.}\PY{n}{FontSize}\PY{p}{=}\PY{l+m+mi}{15}\PY{p}{;}
        \PY{n}{xh} \PY{p}{=} \PY{n}{xlabel}\PY{p}{(}\PY{l+s}{\PYZsq{}}\PY{l+s}{Time \PYZdl{}t\PYZdl{}\PYZsq{}}\PY{p}{,} \PY{l+s}{\PYZsq{}}\PY{l+s}{interpreter\PYZsq{}}\PY{p}{,} \PY{l+s}{\PYZsq{}}\PY{l+s}{latex\PYZsq{}}\PY{p}{)}\PY{p}{;} \PY{n}{xh}\PY{p}{.}\PY{n}{FontSize} \PY{p}{=} \PY{l+m+mi}{13}\PY{p}{;}
        \PY{n}{yh} \PY{p}{=} \PY{n}{ylabel}\PY{p}{(}\PY{l+s}{\PYZsq{}}\PY{l+s}{Price \PYZdl{}P\PYZus{}t\PYZdl{}\PYZsq{}}\PY{p}{,} \PY{l+s}{\PYZsq{}}\PY{l+s}{interpreter\PYZsq{}}\PY{p}{,} \PY{l+s}{\PYZsq{}}\PY{l+s}{latex\PYZsq{}}\PY{p}{)}\PY{p}{;} \PY{n}{yh}\PY{p}{.}\PY{n}{FontSize} \PY{p}{=} \PY{l+m+mi}{13}\PY{p}{;}
        \PY{n}{set}\PY{p}{(}\PY{n}{gca}\PY{p}{,}\PY{l+s}{\PYZsq{}}\PY{l+s}{TickLabelInterpreter\PYZsq{}}\PY{p}{,}\PY{l+s}{\PYZsq{}}\PY{l+s}{latex\PYZsq{}}\PY{p}{)}
        \PY{n}{axis}\PY{p}{(}\PY{p}{[}\PY{l+m+mi}{0} \PY{l+m+mi}{100} \PY{l+m+mi}{0} \PY{l+m+mi}{110}\PY{p}{]}\PY{p}{)}
\end{Verbatim}

    \begin{Verbatim}[commandchars=\\\{\}]


    \end{Verbatim}

    \begin{center}
    \adjustimage{max size={0.9\linewidth}{0.9\paperheight}}{Problem_Set_1_files/Problem_Set_1_18_1.png}
    \end{center}
    { \hspace*{\fill} \\}
    

    % Add a bibliography block to the postdoc
    
    
\bibliography{references}

    
    \end{document}
