
    




    
\documentclass[10pt,notitlepage,onecolumn,aps,pra]{revtex4-1}

    
    
\usepackage[T1]{fontenc}
\usepackage{graphicx}
% We will generate all images so they have a width \maxwidth. This means
% that they will get their normal width if they fit onto the page, but
% are scaled down if they would overflow the margins.
\makeatletter
\def\maxwidth{\ifdim\Gin@nat@width>\linewidth\linewidth
\else\Gin@nat@width\fi}
\makeatother
\let\Oldincludegraphics\includegraphics
% Set max figure width to be 80% of text width, for now hardcoded.
\renewcommand{\includegraphics}[1]{\Oldincludegraphics[width=.8\maxwidth]{#1}}
% Ensure that by default, figures have no caption (until we provide a
% proper Figure object with a Caption API and a way to capture that
% in the conversion process - todo).
\usepackage{caption}
\DeclareCaptionLabelFormat{nolabel}{}
\captionsetup{labelformat=nolabel}

\usepackage{adjustbox} % Used to constrain images to a maximum size
\usepackage{xcolor} % Allow colors to be defined
\usepackage{enumerate} % Needed for markdown enumerations to work
\usepackage{geometry} % Used to adjust the document margins
\usepackage{amsmath} % Equations
\usepackage{amssymb} % Equations
\usepackage{textcomp} % defines textquotesingle
% Hack from http://tex.stackexchange.com/a/47451/13684:
\AtBeginDocument{%
    \def\PYZsq{\textquotesingle}% Upright quotes in Pygmentized code
}
\usepackage{upquote} % Upright quotes for verbatim code
\usepackage{eurosym} % defines \euro
\usepackage[mathletters]{ucs} % Extended unicode (utf-8) support
\usepackage[utf8x]{inputenc} % Allow utf-8 characters in the tex document
\usepackage{fancyvrb} % verbatim replacement that allows latex
\usepackage{grffile} % extends the file name processing of package graphics
                     % to support a larger range
% The hyperref package gives us a pdf with properly built
% internal navigation ('pdf bookmarks' for the table of contents,
% internal cross-reference links, web links for URLs, etc.)
\usepackage{hyperref}
\usepackage{booktabs}  % table support for pandoc > 1.12.2
\usepackage[inline]{enumitem} % IRkernel/repr support (it uses the enumerate* environment)
\usepackage[normalem]{ulem} % ulem is needed to support strikethroughs (\sout)
                            % normalem makes italics be italics, not underlines
\usepackage{braket}


    
    % Colors for the hyperref package
    \definecolor{urlcolor}{rgb}{0,.145,.698}
    \definecolor{linkcolor}{rgb}{.71,0.21,0.01}
    \definecolor{citecolor}{rgb}{.12,.54,.11}

    % ANSI colors
    \definecolor{ansi-black}{HTML}{3E424D}
    \definecolor{ansi-black-intense}{HTML}{282C36}
    \definecolor{ansi-red}{HTML}{E75C58}
    \definecolor{ansi-red-intense}{HTML}{B22B31}
    \definecolor{ansi-green}{HTML}{00A250}
    \definecolor{ansi-green-intense}{HTML}{007427}
    \definecolor{ansi-yellow}{HTML}{DDB62B}
    \definecolor{ansi-yellow-intense}{HTML}{B27D12}
    \definecolor{ansi-blue}{HTML}{208FFB}
    \definecolor{ansi-blue-intense}{HTML}{0065CA}
    \definecolor{ansi-magenta}{HTML}{D160C4}
    \definecolor{ansi-magenta-intense}{HTML}{A03196}
    \definecolor{ansi-cyan}{HTML}{60C6C8}
    \definecolor{ansi-cyan-intense}{HTML}{258F8F}
    \definecolor{ansi-white}{HTML}{C5C1B4}
    \definecolor{ansi-white-intense}{HTML}{A1A6B2}
    \definecolor{ansi-default-inverse-fg}{HTML}{FFFFFF}
    \definecolor{ansi-default-inverse-bg}{HTML}{000000}

    % commands and environments needed by pandoc snippets
    % extracted from the output of `pandoc -s`
    \providecommand{\tightlist}{%
      \setlength{\itemsep}{0pt}\setlength{\parskip}{0pt}}
    \DefineVerbatimEnvironment{Highlighting}{Verbatim}{commandchars=\\\{\}}
    % Add ',fontsize=\small' for more characters per line
    \newenvironment{Shaded}{}{}
    \newcommand{\KeywordTok}[1]{\textcolor[rgb]{0.00,0.44,0.13}{\textbf{{#1}}}}
    \newcommand{\DataTypeTok}[1]{\textcolor[rgb]{0.56,0.13,0.00}{{#1}}}
    \newcommand{\DecValTok}[1]{\textcolor[rgb]{0.25,0.63,0.44}{{#1}}}
    \newcommand{\BaseNTok}[1]{\textcolor[rgb]{0.25,0.63,0.44}{{#1}}}
    \newcommand{\FloatTok}[1]{\textcolor[rgb]{0.25,0.63,0.44}{{#1}}}
    \newcommand{\CharTok}[1]{\textcolor[rgb]{0.25,0.44,0.63}{{#1}}}
    \newcommand{\StringTok}[1]{\textcolor[rgb]{0.25,0.44,0.63}{{#1}}}
    \newcommand{\CommentTok}[1]{\textcolor[rgb]{0.38,0.63,0.69}{\textit{{#1}}}}
    \newcommand{\OtherTok}[1]{\textcolor[rgb]{0.00,0.44,0.13}{{#1}}}
    \newcommand{\AlertTok}[1]{\textcolor[rgb]{1.00,0.00,0.00}{\textbf{{#1}}}}
    \newcommand{\FunctionTok}[1]{\textcolor[rgb]{0.02,0.16,0.49}{{#1}}}
    \newcommand{\RegionMarkerTok}[1]{{#1}}
    \newcommand{\ErrorTok}[1]{\textcolor[rgb]{1.00,0.00,0.00}{\textbf{{#1}}}}
    \newcommand{\NormalTok}[1]{{#1}}
    
    % Additional commands for more recent versions of Pandoc
    \newcommand{\ConstantTok}[1]{\textcolor[rgb]{0.53,0.00,0.00}{{#1}}}
    \newcommand{\SpecialCharTok}[1]{\textcolor[rgb]{0.25,0.44,0.63}{{#1}}}
    \newcommand{\VerbatimStringTok}[1]{\textcolor[rgb]{0.25,0.44,0.63}{{#1}}}
    \newcommand{\SpecialStringTok}[1]{\textcolor[rgb]{0.73,0.40,0.53}{{#1}}}
    \newcommand{\ImportTok}[1]{{#1}}
    \newcommand{\DocumentationTok}[1]{\textcolor[rgb]{0.73,0.13,0.13}{\textit{{#1}}}}
    \newcommand{\AnnotationTok}[1]{\textcolor[rgb]{0.38,0.63,0.69}{\textbf{\textit{{#1}}}}}
    \newcommand{\CommentVarTok}[1]{\textcolor[rgb]{0.38,0.63,0.69}{\textbf{\textit{{#1}}}}}
    \newcommand{\VariableTok}[1]{\textcolor[rgb]{0.10,0.09,0.49}{{#1}}}
    \newcommand{\ControlFlowTok}[1]{\textcolor[rgb]{0.00,0.44,0.13}{\textbf{{#1}}}}
    \newcommand{\OperatorTok}[1]{\textcolor[rgb]{0.40,0.40,0.40}{{#1}}}
    \newcommand{\BuiltInTok}[1]{{#1}}
    \newcommand{\ExtensionTok}[1]{{#1}}
    \newcommand{\PreprocessorTok}[1]{\textcolor[rgb]{0.74,0.48,0.00}{{#1}}}
    \newcommand{\AttributeTok}[1]{\textcolor[rgb]{0.49,0.56,0.16}{{#1}}}
    \newcommand{\InformationTok}[1]{\textcolor[rgb]{0.38,0.63,0.69}{\textbf{\textit{{#1}}}}}
    \newcommand{\WarningTok}[1]{\textcolor[rgb]{0.38,0.63,0.69}{\textbf{\textit{{#1}}}}}
    
    
    % Define a nice break command that doesn't care if a line doesn't already
    % exist.
    \def\br{\hspace*{\fill} \\* }
    % Math Jax compatibility definitions
    \def\gt{>}
    \def\lt{<}
    \let\Oldtex\TeX
    \let\Oldlatex\LaTeX
    \renewcommand{\TeX}{\textrm{\Oldtex}}
    \renewcommand{\LaTeX}{\textrm{\Oldlatex}}
    % Document parameters
    % Document title
    
    
    
    
% Pygments definitions
\makeatletter
\def\PY@reset{\let\PY@it=\relax \let\PY@bf=\relax%
    \let\PY@ul=\relax \let\PY@tc=\relax%
    \let\PY@bc=\relax \let\PY@ff=\relax}
\def\PY@tok#1{\csname PY@tok@#1\endcsname}
\def\PY@toks#1+{\ifx\relax#1\empty\else%
    \PY@tok{#1}\expandafter\PY@toks\fi}
\def\PY@do#1{\PY@bc{\PY@tc{\PY@ul{%
    \PY@it{\PY@bf{\PY@ff{#1}}}}}}}
\def\PY#1#2{\PY@reset\PY@toks#1+\relax+\PY@do{#2}}

\expandafter\def\csname PY@tok@w\endcsname{\def\PY@tc##1{\textcolor[rgb]{0.73,0.73,0.73}{##1}}}
\expandafter\def\csname PY@tok@c\endcsname{\let\PY@it=\textit\def\PY@tc##1{\textcolor[rgb]{0.25,0.50,0.50}{##1}}}
\expandafter\def\csname PY@tok@cp\endcsname{\def\PY@tc##1{\textcolor[rgb]{0.74,0.48,0.00}{##1}}}
\expandafter\def\csname PY@tok@k\endcsname{\let\PY@bf=\textbf\def\PY@tc##1{\textcolor[rgb]{0.00,0.50,0.00}{##1}}}
\expandafter\def\csname PY@tok@kp\endcsname{\def\PY@tc##1{\textcolor[rgb]{0.00,0.50,0.00}{##1}}}
\expandafter\def\csname PY@tok@kt\endcsname{\def\PY@tc##1{\textcolor[rgb]{0.69,0.00,0.25}{##1}}}
\expandafter\def\csname PY@tok@o\endcsname{\def\PY@tc##1{\textcolor[rgb]{0.40,0.40,0.40}{##1}}}
\expandafter\def\csname PY@tok@ow\endcsname{\let\PY@bf=\textbf\def\PY@tc##1{\textcolor[rgb]{0.67,0.13,1.00}{##1}}}
\expandafter\def\csname PY@tok@nb\endcsname{\def\PY@tc##1{\textcolor[rgb]{0.00,0.50,0.00}{##1}}}
\expandafter\def\csname PY@tok@nf\endcsname{\def\PY@tc##1{\textcolor[rgb]{0.00,0.00,1.00}{##1}}}
\expandafter\def\csname PY@tok@nc\endcsname{\let\PY@bf=\textbf\def\PY@tc##1{\textcolor[rgb]{0.00,0.00,1.00}{##1}}}
\expandafter\def\csname PY@tok@nn\endcsname{\let\PY@bf=\textbf\def\PY@tc##1{\textcolor[rgb]{0.00,0.00,1.00}{##1}}}
\expandafter\def\csname PY@tok@ne\endcsname{\let\PY@bf=\textbf\def\PY@tc##1{\textcolor[rgb]{0.82,0.25,0.23}{##1}}}
\expandafter\def\csname PY@tok@nv\endcsname{\def\PY@tc##1{\textcolor[rgb]{0.10,0.09,0.49}{##1}}}
\expandafter\def\csname PY@tok@no\endcsname{\def\PY@tc##1{\textcolor[rgb]{0.53,0.00,0.00}{##1}}}
\expandafter\def\csname PY@tok@nl\endcsname{\def\PY@tc##1{\textcolor[rgb]{0.63,0.63,0.00}{##1}}}
\expandafter\def\csname PY@tok@ni\endcsname{\let\PY@bf=\textbf\def\PY@tc##1{\textcolor[rgb]{0.60,0.60,0.60}{##1}}}
\expandafter\def\csname PY@tok@na\endcsname{\def\PY@tc##1{\textcolor[rgb]{0.49,0.56,0.16}{##1}}}
\expandafter\def\csname PY@tok@nt\endcsname{\let\PY@bf=\textbf\def\PY@tc##1{\textcolor[rgb]{0.00,0.50,0.00}{##1}}}
\expandafter\def\csname PY@tok@nd\endcsname{\def\PY@tc##1{\textcolor[rgb]{0.67,0.13,1.00}{##1}}}
\expandafter\def\csname PY@tok@s\endcsname{\def\PY@tc##1{\textcolor[rgb]{0.73,0.13,0.13}{##1}}}
\expandafter\def\csname PY@tok@sd\endcsname{\let\PY@it=\textit\def\PY@tc##1{\textcolor[rgb]{0.73,0.13,0.13}{##1}}}
\expandafter\def\csname PY@tok@si\endcsname{\let\PY@bf=\textbf\def\PY@tc##1{\textcolor[rgb]{0.73,0.40,0.53}{##1}}}
\expandafter\def\csname PY@tok@se\endcsname{\let\PY@bf=\textbf\def\PY@tc##1{\textcolor[rgb]{0.73,0.40,0.13}{##1}}}
\expandafter\def\csname PY@tok@sr\endcsname{\def\PY@tc##1{\textcolor[rgb]{0.73,0.40,0.53}{##1}}}
\expandafter\def\csname PY@tok@ss\endcsname{\def\PY@tc##1{\textcolor[rgb]{0.10,0.09,0.49}{##1}}}
\expandafter\def\csname PY@tok@sx\endcsname{\def\PY@tc##1{\textcolor[rgb]{0.00,0.50,0.00}{##1}}}
\expandafter\def\csname PY@tok@m\endcsname{\def\PY@tc##1{\textcolor[rgb]{0.40,0.40,0.40}{##1}}}
\expandafter\def\csname PY@tok@gh\endcsname{\let\PY@bf=\textbf\def\PY@tc##1{\textcolor[rgb]{0.00,0.00,0.50}{##1}}}
\expandafter\def\csname PY@tok@gu\endcsname{\let\PY@bf=\textbf\def\PY@tc##1{\textcolor[rgb]{0.50,0.00,0.50}{##1}}}
\expandafter\def\csname PY@tok@gd\endcsname{\def\PY@tc##1{\textcolor[rgb]{0.63,0.00,0.00}{##1}}}
\expandafter\def\csname PY@tok@gi\endcsname{\def\PY@tc##1{\textcolor[rgb]{0.00,0.63,0.00}{##1}}}
\expandafter\def\csname PY@tok@gr\endcsname{\def\PY@tc##1{\textcolor[rgb]{1.00,0.00,0.00}{##1}}}
\expandafter\def\csname PY@tok@ge\endcsname{\let\PY@it=\textit}
\expandafter\def\csname PY@tok@gs\endcsname{\let\PY@bf=\textbf}
\expandafter\def\csname PY@tok@gp\endcsname{\let\PY@bf=\textbf\def\PY@tc##1{\textcolor[rgb]{0.00,0.00,0.50}{##1}}}
\expandafter\def\csname PY@tok@go\endcsname{\def\PY@tc##1{\textcolor[rgb]{0.53,0.53,0.53}{##1}}}
\expandafter\def\csname PY@tok@gt\endcsname{\def\PY@tc##1{\textcolor[rgb]{0.00,0.27,0.87}{##1}}}
\expandafter\def\csname PY@tok@err\endcsname{\def\PY@bc##1{\setlength{\fboxsep}{0pt}\fcolorbox[rgb]{1.00,0.00,0.00}{1,1,1}{\strut ##1}}}
\expandafter\def\csname PY@tok@kc\endcsname{\let\PY@bf=\textbf\def\PY@tc##1{\textcolor[rgb]{0.00,0.50,0.00}{##1}}}
\expandafter\def\csname PY@tok@kd\endcsname{\let\PY@bf=\textbf\def\PY@tc##1{\textcolor[rgb]{0.00,0.50,0.00}{##1}}}
\expandafter\def\csname PY@tok@kn\endcsname{\let\PY@bf=\textbf\def\PY@tc##1{\textcolor[rgb]{0.00,0.50,0.00}{##1}}}
\expandafter\def\csname PY@tok@kr\endcsname{\let\PY@bf=\textbf\def\PY@tc##1{\textcolor[rgb]{0.00,0.50,0.00}{##1}}}
\expandafter\def\csname PY@tok@bp\endcsname{\def\PY@tc##1{\textcolor[rgb]{0.00,0.50,0.00}{##1}}}
\expandafter\def\csname PY@tok@fm\endcsname{\def\PY@tc##1{\textcolor[rgb]{0.00,0.00,1.00}{##1}}}
\expandafter\def\csname PY@tok@vc\endcsname{\def\PY@tc##1{\textcolor[rgb]{0.10,0.09,0.49}{##1}}}
\expandafter\def\csname PY@tok@vg\endcsname{\def\PY@tc##1{\textcolor[rgb]{0.10,0.09,0.49}{##1}}}
\expandafter\def\csname PY@tok@vi\endcsname{\def\PY@tc##1{\textcolor[rgb]{0.10,0.09,0.49}{##1}}}
\expandafter\def\csname PY@tok@vm\endcsname{\def\PY@tc##1{\textcolor[rgb]{0.10,0.09,0.49}{##1}}}
\expandafter\def\csname PY@tok@sa\endcsname{\def\PY@tc##1{\textcolor[rgb]{0.73,0.13,0.13}{##1}}}
\expandafter\def\csname PY@tok@sb\endcsname{\def\PY@tc##1{\textcolor[rgb]{0.73,0.13,0.13}{##1}}}
\expandafter\def\csname PY@tok@sc\endcsname{\def\PY@tc##1{\textcolor[rgb]{0.73,0.13,0.13}{##1}}}
\expandafter\def\csname PY@tok@dl\endcsname{\def\PY@tc##1{\textcolor[rgb]{0.73,0.13,0.13}{##1}}}
\expandafter\def\csname PY@tok@s2\endcsname{\def\PY@tc##1{\textcolor[rgb]{0.73,0.13,0.13}{##1}}}
\expandafter\def\csname PY@tok@sh\endcsname{\def\PY@tc##1{\textcolor[rgb]{0.73,0.13,0.13}{##1}}}
\expandafter\def\csname PY@tok@s1\endcsname{\def\PY@tc##1{\textcolor[rgb]{0.73,0.13,0.13}{##1}}}
\expandafter\def\csname PY@tok@mb\endcsname{\def\PY@tc##1{\textcolor[rgb]{0.40,0.40,0.40}{##1}}}
\expandafter\def\csname PY@tok@mf\endcsname{\def\PY@tc##1{\textcolor[rgb]{0.40,0.40,0.40}{##1}}}
\expandafter\def\csname PY@tok@mh\endcsname{\def\PY@tc##1{\textcolor[rgb]{0.40,0.40,0.40}{##1}}}
\expandafter\def\csname PY@tok@mi\endcsname{\def\PY@tc##1{\textcolor[rgb]{0.40,0.40,0.40}{##1}}}
\expandafter\def\csname PY@tok@il\endcsname{\def\PY@tc##1{\textcolor[rgb]{0.40,0.40,0.40}{##1}}}
\expandafter\def\csname PY@tok@mo\endcsname{\def\PY@tc##1{\textcolor[rgb]{0.40,0.40,0.40}{##1}}}
\expandafter\def\csname PY@tok@ch\endcsname{\let\PY@it=\textit\def\PY@tc##1{\textcolor[rgb]{0.25,0.50,0.50}{##1}}}
\expandafter\def\csname PY@tok@cm\endcsname{\let\PY@it=\textit\def\PY@tc##1{\textcolor[rgb]{0.25,0.50,0.50}{##1}}}
\expandafter\def\csname PY@tok@cpf\endcsname{\let\PY@it=\textit\def\PY@tc##1{\textcolor[rgb]{0.25,0.50,0.50}{##1}}}
\expandafter\def\csname PY@tok@c1\endcsname{\let\PY@it=\textit\def\PY@tc##1{\textcolor[rgb]{0.25,0.50,0.50}{##1}}}
\expandafter\def\csname PY@tok@cs\endcsname{\let\PY@it=\textit\def\PY@tc##1{\textcolor[rgb]{0.25,0.50,0.50}{##1}}}

\def\PYZbs{\char`\\}
\def\PYZus{\char`\_}
\def\PYZob{\char`\{}
\def\PYZcb{\char`\}}
\def\PYZca{\char`\^}
\def\PYZam{\char`\&}
\def\PYZlt{\char`\<}
\def\PYZgt{\char`\>}
\def\PYZsh{\char`\#}
\def\PYZpc{\char`\%}
\def\PYZdl{\char`\$}
\def\PYZhy{\char`\-}
\def\PYZsq{\char`\'}
\def\PYZdq{\char`\"}
\def\PYZti{\char`\~}
% for compatibility with earlier versions
\def\PYZat{@}
\def\PYZlb{[}
\def\PYZrb{]}
\makeatother


    % For linebreaks inside Verbatim environment from package fancyvrb. 
    \makeatletter
        \newbox\Wrappedcontinuationbox 
        \newbox\Wrappedvisiblespacebox 
        \newcommand*\Wrappedvisiblespace {\textcolor{red}{\textvisiblespace}} 
        \newcommand*\Wrappedcontinuationsymbol {\textcolor{red}{\llap{\tiny$\m@th\hookrightarrow$}}} 
        \newcommand*\Wrappedcontinuationindent {3ex } 
        \newcommand*\Wrappedafterbreak {\kern\Wrappedcontinuationindent\copy\Wrappedcontinuationbox} 
        % Take advantage of the already applied Pygments mark-up to insert 
        % potential linebreaks for TeX processing. 
        %        {, <, #, %, $, ' and ": go to next line. 
        %        _, }, ^, &, >, - and ~: stay at end of broken line. 
        % Use of \textquotesingle for straight quote. 
        \newcommand*\Wrappedbreaksatspecials {% 
            \def\PYGZus{\discretionary{\char`\_}{\Wrappedafterbreak}{\char`\_}}% 
            \def\PYGZob{\discretionary{}{\Wrappedafterbreak\char`\{}{\char`\{}}% 
            \def\PYGZcb{\discretionary{\char`\}}{\Wrappedafterbreak}{\char`\}}}% 
            \def\PYGZca{\discretionary{\char`\^}{\Wrappedafterbreak}{\char`\^}}% 
            \def\PYGZam{\discretionary{\char`\&}{\Wrappedafterbreak}{\char`\&}}% 
            \def\PYGZlt{\discretionary{}{\Wrappedafterbreak\char`\<}{\char`\<}}% 
            \def\PYGZgt{\discretionary{\char`\>}{\Wrappedafterbreak}{\char`\>}}% 
            \def\PYGZsh{\discretionary{}{\Wrappedafterbreak\char`\#}{\char`\#}}% 
            \def\PYGZpc{\discretionary{}{\Wrappedafterbreak\char`\%}{\char`\%}}% 
            \def\PYGZdl{\discretionary{}{\Wrappedafterbreak\char`\$}{\char`\$}}% 
            \def\PYGZhy{\discretionary{\char`\-}{\Wrappedafterbreak}{\char`\-}}% 
            \def\PYGZsq{\discretionary{}{\Wrappedafterbreak\textquotesingle}{\textquotesingle}}% 
            \def\PYGZdq{\discretionary{}{\Wrappedafterbreak\char`\"}{\char`\"}}% 
            \def\PYGZti{\discretionary{\char`\~}{\Wrappedafterbreak}{\char`\~}}% 
        } 
        % Some characters . , ; ? ! / are not pygmentized. 
        % This macro makes them "active" and they will insert potential linebreaks 
        \newcommand*\Wrappedbreaksatpunct {% 
            \lccode`\~`\.\lowercase{\def~}{\discretionary{\hbox{\char`\.}}{\Wrappedafterbreak}{\hbox{\char`\.}}}% 
            \lccode`\~`\,\lowercase{\def~}{\discretionary{\hbox{\char`\,}}{\Wrappedafterbreak}{\hbox{\char`\,}}}% 
            \lccode`\~`\;\lowercase{\def~}{\discretionary{\hbox{\char`\;}}{\Wrappedafterbreak}{\hbox{\char`\;}}}% 
            \lccode`\~`\:\lowercase{\def~}{\discretionary{\hbox{\char`\:}}{\Wrappedafterbreak}{\hbox{\char`\:}}}% 
            \lccode`\~`\?\lowercase{\def~}{\discretionary{\hbox{\char`\?}}{\Wrappedafterbreak}{\hbox{\char`\?}}}% 
            \lccode`\~`\!\lowercase{\def~}{\discretionary{\hbox{\char`\!}}{\Wrappedafterbreak}{\hbox{\char`\!}}}% 
            \lccode`\~`\/\lowercase{\def~}{\discretionary{\hbox{\char`\/}}{\Wrappedafterbreak}{\hbox{\char`\/}}}% 
            \catcode`\.\active
            \catcode`\,\active 
            \catcode`\;\active
            \catcode`\:\active
            \catcode`\?\active
            \catcode`\!\active
            \catcode`\/\active 
            \lccode`\~`\~ 	
        }
    \makeatother

    \let\OriginalVerbatim=\Verbatim
    \makeatletter
    \renewcommand{\Verbatim}[1][1]{%
        %\parskip\z@skip
        \sbox\Wrappedcontinuationbox {\Wrappedcontinuationsymbol}%
        \sbox\Wrappedvisiblespacebox {\FV@SetupFont\Wrappedvisiblespace}%
        \def\FancyVerbFormatLine ##1{\hsize\linewidth
            \vtop{\raggedright\hyphenpenalty\z@\exhyphenpenalty\z@
                \doublehyphendemerits\z@\finalhyphendemerits\z@
                \strut ##1\strut}%
        }%
        % If the linebreak is at a space, the latter will be displayed as visible
        % space at end of first line, and a continuation symbol starts next line.
        % Stretch/shrink are however usually zero for typewriter font.
        \def\FV@Space {%
            \nobreak\hskip\z@ plus\fontdimen3\font minus\fontdimen4\font
            \discretionary{\copy\Wrappedvisiblespacebox}{\Wrappedafterbreak}
            {\kern\fontdimen2\font}%
        }%
        
        % Allow breaks at special characters using \PYG... macros.
        \Wrappedbreaksatspecials
        % Breaks at punctuation characters . , ; ? ! and / need catcode=\active 	
        \OriginalVerbatim[#1,codes*=\Wrappedbreaksatpunct]%
    }
    \makeatother

    % Exact colors from NB
    \definecolor{incolor}{HTML}{303F9F}
    \definecolor{outcolor}{HTML}{D84315}
    \definecolor{cellborder}{HTML}{CFCFCF}
    \definecolor{cellbackground}{HTML}{F7F7F7}
    
    % prompt
    \newcommand{\prompt}[4]{
        \llap{{\color{#2}[#3]: #4}}\vspace{-1.25em}
    }
    

    
    % Prevent overflowing lines due to hard-to-break entities
    \sloppy 
    % Setup hyperref package
    \hypersetup{
      breaklinks=true,  % so long urls are correctly broken across lines
      colorlinks=true,
      urlcolor=urlcolor,
      linkcolor=linkcolor,
      citecolor=citecolor,
      }
    % Slightly bigger margins than the latex defaults
    
    \geometry{verbose,tmargin=1in,bmargin=1in,lmargin=1in,rmargin=1in}
    
    

    \begin{document}
    
    
    \title{Problem Set 2 - ECON 712}\author{Mitchell Valdés-Bobes}\affiliation{University of Wisconsin Madison}

\date{\today}
\maketitle


    
    

    \begin{Verbatim}[commandchars=\\\{\}]
{\color{incolor}In [{\color{incolor}95}]:} \PY{n}{addpath}\PY{p}{(}\PY{l+s}{\PYZsq{}}\PY{l+s}{./Utils\PYZsq{}}\PY{p}{)}
         \PY{n}{format} \PY{l+s}{compact}
\end{Verbatim}

    \begin{Verbatim}[commandchars=\\\{\}]


    \end{Verbatim}

    \hypertarget{model}{%
\section{Model}\label{model}}

Consider the model presented during Friday's discussion section. Each
period a continuum of agents is born. Agents live for \(J\) periods
after which they die. The population growth rate is \(n\) per year
(which is the model period length). Thus, the relative size of each
cohort of age \(j, \psi_{j},\) is given by: \[
\psi_{i+1}=\frac{\psi_{i}}{1+n}, \text { for } i=1, \ldots, J-1
\] with \(\psi_{1}=\tilde{\psi}>0 .\) It is convenient to normalize
\(\psi,\) so that it sums up to 1 across all age groups.

Newly born agents (i.e.~\(j=1\) ) are endowed with no initial capital
(i.e.~\(k_{j}=0\) ) but can subsequently save in capital which they can
rent to firms at rate \(r .\) A worker of age \(j\) supplies labor
\(\ell_{j} \in[0,1]\) and pays proportional social security taxes on her
labor income \(\tau w e_{j} \ell_{j}\) until she retires at age
\(J^{R}<J,\) where \(e_{j}\) is the age-efficiency profile. Upon
retirement, agent receives pension benefits \(b\) The instantaneous
utility function of a worker at age \(j=1,2, \ldots, J^{R}-1\) is given
by: \[
u^{W}\left(c_{j}, \ell_{j}\right)=\frac{\left(c_{j}^{\gamma}\left(1-\ell_{j}\right)^{1-\gamma}\right)^{1-\sigma}}{1-\sigma}
\] with \(c_{j}\) denoting consumption and \(\ell_{j}\) denoting labor
supply at age \(j .\) The weight on consumption is \(\gamma\) and the
coefficient of relative risk aversion is \(\sigma\). The instantaneous
utility function of a retired agent at age \(j=J^{R}, \ldots, J\) is
given by: \[
u^{R}\left(c_{j}\right)=\frac{c_{j}^{1-\sigma}}{1-\sigma}
\]

Preferences are then given by \[
\sum_{j=1}^{J^{R}-1} \beta^{j-1} u^{W}\left(c_{j}, \ell_{j}\right)+\sum_{j=J^{R}}^{J} \beta^{j-1} u^{R}\left(c_{j}\right)
\] There is a constant returns to scale production technology
\(Y=F(K, L)=\) \(K^{\alpha} L^{1-\alpha}\) with \(\alpha\) denoting
capital share, \(Y\) denoting aggregate output, \(K\) denoting aggregate
capital stock and \(L\) denoting aggregate effective labor supply. The
capital depreciates at rate \(\delta\). Capital and labor markets are
perfectly competitive.

    \hypertarget{parametrization}{%
\subsection{Parametrization}\label{parametrization}}

    \textbf{Demographics}
\begin{Verbatim}[commandchars=\\\{\}]
{\color{incolor}In [{\color{incolor}61}]:} \PY{n}{J}\PY{p}{=}\PY{l+m+mi}{66}\PY{p}{;}                       \PY{c}{\PYZpc{} life\PYZhy{}span}
         \PY{n}{JR}\PY{p}{=}\PY{l+m+mi}{46}\PY{p}{;}                      \PY{c}{\PYZpc{} age of retirement}
         \PY{n}{tR}\PY{p}{=}\PY{n}{J}\PY{o}{\PYZhy{}}\PY{n}{JR}\PY{o}{+}\PY{l+m+mi}{1}\PY{p}{;}                  \PY{c}{\PYZpc{} length of retirement}
         \PY{n}{tW}\PY{p}{=}\PY{n}{JR}\PY{o}{\PYZhy{}}\PY{l+m+mi}{1}\PY{p}{;}                    \PY{c}{\PYZpc{} length of working life}
         \PY{n}{n}\PY{p}{=}\PY{l+m+mf}{0.011}\PY{p}{;}                    \PY{c}{\PYZpc{} Population growth}
\end{Verbatim}

    \begin{Verbatim}[commandchars=\\\{\}]


    \end{Verbatim}

    \textbf{Preferences}
\begin{Verbatim}[commandchars=\\\{\}]
{\color{incolor}In [{\color{incolor}62}]:} \PY{n+nb}{beta}\PY{p}{=}\PY{l+m+mf}{0.97}\PY{p}{;}                  \PY{c}{\PYZpc{} discount factor}
         \PY{n}{sigma}\PY{p}{=}\PY{l+m+mi}{2}\PY{p}{;}                    \PY{c}{\PYZpc{} coefficient of relative risk aversion}
         \PY{n+nb}{gamma}\PY{p}{=}\PY{l+m+mf}{0.42}\PY{p}{;}                 \PY{c}{\PYZpc{} weight on consumption}
\end{Verbatim}

    \begin{Verbatim}[commandchars=\\\{\}]


    \end{Verbatim}

    \textbf{Production}
\begin{Verbatim}[commandchars=\\\{\}]
{\color{incolor}In [{\color{incolor}64}]:} \PY{n}{alpha}\PY{p}{=}\PY{l+m+mf}{0.36}\PY{p}{;}                 \PY{c}{\PYZpc{} production elasticity of capital}
         \PY{n}{delta}\PY{p}{=}\PY{l+m+mf}{0.06}\PY{p}{;}
\end{Verbatim}

    \begin{Verbatim}[commandchars=\\\{\}]


    \end{Verbatim}

    \textbf{Social Security tax rate}
\begin{Verbatim}[commandchars=\\\{\}]
{\color{incolor}In [{\color{incolor}65}]:} \PY{n}{tau}\PY{p}{=}\PY{l+m+mf}{0.11}\PY{p}{;} \PY{c}{\PYZpc{} Use this for model with SS}
\end{Verbatim}

    \begin{Verbatim}[commandchars=\\\{\}]


    \end{Verbatim}

    \textbf{Measure of each generation}
\begin{Verbatim}[commandchars=\\\{\}]
{\color{incolor}In [{\color{incolor}66}]:} \PY{n}{mass}\PY{p}{=}\PY{n+nb}{ones}\PY{p}{(}\PY{n}{J}\PY{p}{,}\PY{l+m+mi}{1}\PY{p}{)}\PY{p}{;}
         \PY{n}{for} \PY{l+s}{ik0=2:J}
             \PY{n}{mass}\PY{p}{(}\PY{n}{ik0}\PY{p}{)}\PY{p}{=}\PY{n}{mass}\PY{p}{(}\PY{n}{ik0}\PY{o}{\PYZhy{}}\PY{l+m+mi}{1}\PY{p}{)}\PY{o}{/}\PY{p}{(}\PY{l+m+mi}{1}\PY{o}{+}\PY{n}{n}\PY{p}{)}\PY{p}{;}
         \PY{n}{end}
\end{Verbatim}

    \begin{Verbatim}[commandchars=\\\{\}]


    \end{Verbatim}

    \textbf{Normalized measure of each generation (sums up to 1)}
\begin{Verbatim}[commandchars=\\\{\}]
{\color{incolor}In [{\color{incolor}67}]:} \PY{n}{mass}\PY{p}{=}\PY{n}{mass}\PY{o}{/}\PY{n}{sum}\PY{p}{(}\PY{n}{mass}\PY{p}{)}\PY{p}{;}
\end{Verbatim}

    \begin{Verbatim}[commandchars=\\\{\}]


    \end{Verbatim}

    \textbf{Age efficiency profile}
\begin{Verbatim}[commandchars=\\\{\}]
{\color{incolor}In [{\color{incolor}68}]:} \PY{n}{e} \PY{p}{=} \PY{n+nb}{zeros}\PY{p}{(}\PY{n}{tW}\PY{p}{,}\PY{l+m+mi}{1}\PY{p}{)}\PY{p}{;}
         \PY{n}{e}\PY{p}{(}\PY{l+m+mi}{1}\PY{p}{:}\PY{l+m+mi}{11}\PY{p}{)} \PY{p}{=} \PY{n+nb}{linspace}\PY{p}{(}\PY{l+m+mf}{0.6}\PY{p}{,} \PY{l+m+mi}{1}\PY{p}{,} \PY{l+m+mi}{11}\PY{p}{)}\PY{p}{;}
         \PY{n}{e}\PY{p}{(}\PY{l+m+mi}{11}\PY{p}{:}\PY{l+m+mi}{21}\PY{p}{)} \PY{p}{=} \PY{n+nb}{linspace}\PY{p}{(}\PY{l+m+mi}{1}\PY{p}{,} \PY{l+m+mf}{1.08}\PY{p}{,} \PY{l+m+mi}{11}\PY{p}{)}\PY{p}{;}
         \PY{n}{e}\PY{p}{(}\PY{l+m+mi}{21}\PY{p}{:}\PY{l+m+mi}{31}\PY{p}{)} \PY{p}{=} \PY{n+nb}{linspace}\PY{p}{(}\PY{l+m+mf}{1.08}\PY{p}{,} \PY{l+m+mf}{1.12}\PY{p}{,} \PY{l+m+mi}{11}\PY{p}{)}\PY{p}{;}
         \PY{n}{e}\PY{p}{(}\PY{l+m+mi}{31}\PY{p}{:}\PY{l+m+mi}{41}\PY{p}{)} \PY{p}{=} \PY{n+nb}{linspace}\PY{p}{(}\PY{l+m+mf}{1.12}\PY{p}{,} \PY{l+m+mf}{1.06}\PY{p}{,} \PY{l+m+mi}{11}\PY{p}{)}\PY{p}{;}
         \PY{n}{e}\PY{p}{(}\PY{l+m+mi}{41}\PY{p}{:}\PY{l+m+mi}{45}\PY{p}{)} \PY{p}{=} \PY{n+nb}{linspace}\PY{p}{(}\PY{l+m+mf}{1.06}\PY{p}{,} \PY{l+m+mf}{1.02}\PY{p}{,} \PY{l+m+mi}{5}\PY{p}{)}\PY{p}{;}
\end{Verbatim}

    \begin{Verbatim}[commandchars=\\\{\}]


    \end{Verbatim}
\begin{Verbatim}[commandchars=\\\{\}]
{\color{incolor}In [{\color{incolor}69}]:} \PY{n}{plot}\PY{p}{(}\PY{n}{e}\PY{p}{)}
         \PY{n}{xlabel}\PY{p}{(}\PY{l+s}{\PYZsq{}}\PY{l+s}{age\PYZsq{}}\PY{p}{,}\PY{l+s}{\PYZsq{}}\PY{l+s}{FontSize\PYZsq{}}\PY{p}{,}\PY{l+m+mi}{14}\PY{p}{,} \PY{l+s}{\PYZsq{}}\PY{l+s}{interpreter\PYZsq{}}\PY{p}{,} \PY{l+s}{\PYZsq{}}\PY{l+s}{latex\PYZsq{}}\PY{p}{)}\PY{p}{;}
         \PY{n}{ylabel}\PY{p}{(}\PY{l+s}{\PYZsq{}}\PY{l+s}{efficiency\PYZsq{}}\PY{p}{,}\PY{l+s}{\PYZsq{}}\PY{l+s}{FontSize\PYZsq{}}\PY{p}{,}\PY{l+m+mi}{14}\PY{p}{,} \PY{l+s}{\PYZsq{}}\PY{l+s}{interpreter\PYZsq{}}\PY{p}{,} \PY{l+s}{\PYZsq{}}\PY{l+s}{latex\PYZsq{}}\PY{p}{)}\PY{p}{;}
         \PY{n}{title}\PY{p}{(}\PY{l+s}{\PYZsq{}}\PY{l+s}{Age\PYZhy{}Efficiency profile\PYZsq{}}\PY{p}{,}\PY{l+s}{\PYZsq{}}\PY{l+s}{FontSize\PYZsq{}}\PY{p}{,}\PY{l+m+mi}{14}\PY{p}{,}  \PY{l+s}{\PYZsq{}}\PY{l+s}{interpreter\PYZsq{}}\PY{p}{,} \PY{l+s}{\PYZsq{}}\PY{l+s}{latex\PYZsq{}}\PY{p}{)}\PY{p}{;}
         \PY{n}{set}\PY{p}{(}\PY{n}{gca}\PY{p}{,}\PY{l+s}{\PYZsq{}}\PY{l+s}{TickLabelInterpreter\PYZsq{}}\PY{p}{,}\PY{l+s}{\PYZsq{}}\PY{l+s}{latex\PYZsq{}}\PY{p}{)}
\end{Verbatim}

    \begin{Verbatim}[commandchars=\\\{\}]


    \end{Verbatim}

    \begin{center}
    \adjustimage{max size={0.9\linewidth}{0.9\paperheight}}{Problem_Set_5_files/Problem_Set_5_17_1.png}
    \end{center}
    { \hspace*{\fill} \\}
    
    \hypertarget{questions}{%
\section{Questions}\label{questions}}

    \hypertarget{quad}{%
\subsection{\texorpdfstring{\(\quad\)}{\textbackslash quad}}\label{quad}}

Derive the below equation for labor supply, used in the solution of
workers' recursive problem (refer to lecture notes for details): \[
\ell=\frac{\gamma(1-\tau) e_{j} w-(1-\gamma)\left[(1+r) k-k^{\prime}\right]}{(1-\tau) e_{j} w}
\]

    \hypertarget{answer}{%
\subsubsection{Answer}\label{answer}}

From the budget constraint we have: \begin{equation}\label{eq:c}
    c=(1-\tau) w e \ell+(1+r) k- k'
\end{equation}

And we can obtain the \textbf{FOC} of the maximization problem:

\[\frac{\partial}{\partial \ell} u(c(\ell),\ell) = \frac{\partial}{\partial c} u \frac{\partial}{\partial \ell} c +\frac{\partial}{\partial \ell} u\]

Which gives:

\[(1-\tau) w e u'(c(\ell),\ell) \gamma c^{\gamma-1}\left(1-\ell\right)^{1-\gamma} =  u'(c(\ell),\ell)(1- \gamma) c^{\gamma}\left(1-\ell\right)^{-\gamma} \]

Symplifiying we get

\begin{equation}\label{eq:l_first}
\ell =\frac{-\gamma  c+c+\gamma  e \tau  w-\gamma e w}{\gamma  e (\tau -1) w} 
\end{equation}

Plug \(\eqref{c}\) in \(\eqref{l_first}\) and simplyfing get we get:

\begin{equation}\label{eq:l_not_final}
\frac{\gamma  e (\tau -1) w-(\gamma -1) k
   (r+1)+(\gamma -1) k'}{e (\tau -1)
   w}
\end{equation}

And after rearranging:

\begin{equation}\label{eq:l_final}
\boxed{\ell = \frac{\gamma   (1-\tau) ew-(1-\gamma)\big(
   (1+r)k-k'\big)}{(1-\tau)e w}}
\end{equation}

    \hypertarget{quad}{%
\subsection{\texorpdfstring{\(\quad\)}{\textbackslash quad}}\label{quad}}

Matlab code for this problem set is provided in file \texttt{ps5.m}.
Look for areas in the code that asks you to fill in the blanks and fill
them in. Remember, the algorithm consists of the following steps: 1.
Make initial guesses of the steady state values of the aggregate capital
stock \(K\) and aggregate labor \(L\) 2. Compute social security
benefits, \(b,\) and the prices, \(w\) and \(r,\) implied by these
guesses. 3. Compute the household's decision functions using dynamic
programming. 4. Compute the optimal path for savings and labor supply
for every cohort \(j\) by forward induction given that the initial
capital stock is \(k_{1}=0\) 5. Compute the aggregate capital stock and
aggregate labor supply. 6. Update \(K\) and \(L\) and return to step 2
until convergence. If the guess was ``far off'' the obtained values, we
update our initial guess with \(K^{1}=0.9 K^{0}+0.1 K^{\text {new }}\)
and \(N^{1}=0.9 N^{0}+0.1 N^{\text {new }}\) and repeat the procedure.
We proceed so until the guess and the updated values for \(K\) and \(N\)
are ``sufficiently close''.

    \hypertarget{answer}{%
\subsubsection{Answer}\label{answer}}

    I've adapted the loop in \texttt{ps5.m} in to a function which I called
\texttt{dynamic\_prog} I will call this function with different
arguments in this homework and the (attched) script
\texttt{dynamic\_prog.m}; contains the code with the blanks filled.

    \hypertarget{qquad}{%
\subsection{\texorpdfstring{\(\qquad\)}{\textbackslash qquad}}\label{qquad}}

Explain in words what each while and for loop does (lines \(102-258\) ),
about \(1-3\) sentences per loop. Thus, write for example: The first
while loop will iterate until both the absolute difference between the
updated gross capital (labor) equals the initial level of capital
(labor). The loop stops once the number if iterations equals the maximum
number allowed. Hint: Do this before you start programming.

    \hypertarget{answer}{%
\subsubsection{Answer}\label{answer}}

The external \texttt{while} runs each iteration of the main methods if
any of the following criteria is reached: + \textbf{Criteria 1} Capital
and Labor are not updated any more (in practice this is that the guess
and the updated value are sufficiently small) + \textbf{Criteria 2}
Total number of iterations is reached.

First ``outer'' \texttt{for} goes from the year of death (\(J\)) to the
year of retirement (\(JR\)) and by backward induction finds the optimal
lever for retired generations.

\begin{itemize}
\tightlist
\item
  Inner \texttt{for} loops over today's (\(k\)) possible capital levels.
  It finds the best value for \(k\) for any age \(j\).

  \begin{itemize}
  \tightlist
  \item
    \texttt{while} loops over tomorrow's (\(k'\)) capital levels. Find
    the best value for \(k'\), given \(k\) in the previous for, for any
    age \(j\).
  \end{itemize}
\end{itemize}

Second ``outer'' \texttt{for} goes from the year of previous to
retirement (\(JR-1\)) to the birth (\(JR\)) and by backward induction
finds the optimal lever for working generations. * Inner \texttt{for}
loops over today's (\(k\)) possible capital levels. It finds the best
value for \(k\) for any age \(j\). * \texttt{while} loops over
tomorrow's (\(k'\)) capital levels. Find the best value for \(k'\),
given \(k\) in the previous for, for any age \(j\).

    \hypertarget{quad}{%
\subsection{\texorpdfstring{\(\quad\)}{\textbackslash quad}}\label{quad}}

Using your code, solve household optimization problem with
\(K=3.1392, L=\) 0.3496 and social security in place (Hint: run just one
iteration of the main loop \()\). Plot the value function over \(k\) for
a retired agent at the model-age \(j=50, V_{50}(k) .\) Is it increasing
and concave? Plot the savings function for a worker at the model-age
\(j=20, k_{20}^{\prime}(k) .\) Is saving increasing in \(k ?\) What
about net saving, \(k^{\prime}-k ?\)

    \hypertarget{answer}{%
\subsubsection{Answer}\label{answer}}

We will pass our \texttt{dynamic\_prog} the parameter \texttt{nq=1} to
specify just one iteration.
\begin{Verbatim}[commandchars=\\\{\}]
{\color{incolor}In [{\color{incolor}70}]:} \PY{c}{\PYZpc{} Social Security}
         \PY{n}{K0}\PY{p}{=}\PY{l+m+mf}{3.1392}\PY{p}{;}
         \PY{n}{L0}\PY{p}{=}\PY{l+m+mf}{0.3496}\PY{p}{;}
\end{Verbatim}

    \begin{Verbatim}[commandchars=\\\{\}]


    \end{Verbatim}
\begin{Verbatim}[commandchars=\\\{\}]
{\color{incolor}In [{\color{incolor}79}]:} \PY{n}{nq} \PY{p}{=} \PY{l+m+mi}{1}
         \PY{p}{[}\PY{n}{K0}\PY{p}{,} \PY{n}{L0}\PY{p}{,} \PY{n}{w0}\PY{p}{,} \PY{n}{r0}\PY{p}{,} \PY{n}{b}\PY{p}{,} \PY{n}{kgen}\PY{p}{,} \PY{n}{ikgen}\PY{p}{,} \PY{n}{kap}\PY{p}{,} \PY{n}{kapopt}\PY{p}{,} \PY{n}{v}\PY{p}{]} \PY{p}{=} \PY{n}{dynamic\PYZus{}prog}\PY{p}{(} \PY{n}{K0}\PY{p}{,} \PY{n}{L0}\PY{p}{,} \PY{n}{alpha}\PY{p}{,} \PY{n}{delta}\PY{p}{,}
         \PY{n}{tau}\PY{p}{,} \PY{n}{J}\PY{p}{,} \PY{n}{JR}\PY{p}{,} \PY{n}{tW}\PY{p}{,} \PY{n}{mass}\PY{p}{,} \PY{n}{sigma}\PY{p}{,} \PY{n+nb}{beta}\PY{p}{,} \PY{n+nb}{gamma}\PY{p}{,} \PY{n}{e}\PY{p}{,} \PY{n}{nq}\PY{p}{)}\PY{p}{;}
\end{Verbatim}

    \begin{Verbatim}[commandchars=\\\{\}]
nq =
     1
      K0         L0       w         r         b
    3.0832    0.3501    1.4048    0.0290    0.2216


    \end{Verbatim}
\begin{Verbatim}[commandchars=\\\{\}]
{\color{incolor}In [{\color{incolor}78}]:} \PY{c}{\PYZpc{}\PYZpc{} Plots}
         \PY{c}{\PYZpc{} Value function for a retired agent}
         
         \PY{n}{age}\PY{p}{=}\PY{l+m+mi}{50}\PY{p}{;}
         \PY{n}{plot}\PY{p}{(}\PY{n}{kap}\PY{p}{,}\PY{n}{v}\PY{p}{(}\PY{p}{:}\PY{p}{,}\PY{n}{age}\PY{p}{)}\PY{p}{)}\PY{p}{;}
         \PY{n}{xlabel}\PY{p}{(}\PY{l+s}{\PYZsq{}}\PY{l+s}{asset holdings, k\PYZsq{}}\PY{p}{,}\PY{l+s}{\PYZsq{}}\PY{l+s}{FontSize\PYZsq{}}\PY{p}{,}\PY{l+m+mi}{14}\PY{p}{,} \PY{l+s}{\PYZsq{}}\PY{l+s}{interpreter\PYZsq{}}\PY{p}{,} \PY{l+s}{\PYZsq{}}\PY{l+s}{latex\PYZsq{}}\PY{p}{)}\PY{p}{;}
         \PY{n}{ylabel}\PY{p}{(}\PY{l+s}{\PYZsq{}}\PY{l+s}{value function , \PYZdl{}V\PYZus{}\PYZob{}50\PYZcb{}(k)\PYZdl{}\PYZsq{}}\PY{p}{,}\PY{l+s}{\PYZsq{}}\PY{l+s}{FontSize\PYZsq{}}\PY{p}{,}\PY{l+m+mi}{14}\PY{p}{,} \PY{l+s}{\PYZsq{}}\PY{l+s}{interpreter\PYZsq{}}\PY{p}{,} \PY{l+s}{\PYZsq{}}\PY{l+s}{latex\PYZsq{}}\PY{p}{)}\PY{p}{;}
         \PY{n}{title}\PY{p}{(}\PY{p}{[}\PY{l+s}{\PYZsq{}}\PY{l+s}{value function of a retired agent at age \PYZsq{}}\PY{p}{,} \PY{n}{num2str}\PY{p}{(}\PY{n}{age}\PY{p}{)}\PY{p}{]}\PY{p}{,}\PY{l+s}{\PYZsq{}}\PY{l+s}{FontSize\PYZsq{}}\PY{p}{,}\PY{l+m+mi}{14}\PY{p}{,}
         \PY{l+s}{\PYZsq{}}\PY{l+s}{interpreter\PYZsq{}}\PY{p}{,} \PY{l+s}{\PYZsq{}}\PY{l+s}{latex\PYZsq{}}\PY{p}{)}\PY{p}{;}
         \PY{n}{set}\PY{p}{(}\PY{n}{gca}\PY{p}{,}\PY{l+s}{\PYZsq{}}\PY{l+s}{TickLabelInterpreter\PYZsq{}}\PY{p}{,}\PY{l+s}{\PYZsq{}}\PY{l+s}{latex\PYZsq{}}\PY{p}{)}
\end{Verbatim}

    \begin{Verbatim}[commandchars=\\\{\}]


    \end{Verbatim}

    \begin{center}
    \adjustimage{max size={0.9\linewidth}{0.9\paperheight}}{Problem_Set_5_files/Problem_Set_5_30_1.png}
    \end{center}
    { \hspace*{\fill} \\}
    \begin{Verbatim}[commandchars=\\\{\}]
{\color{incolor}In [{\color{incolor}73}]:} \PY{c}{\PYZpc{} Savings of a working agent}
         \PY{n}{age}\PY{p}{=}\PY{l+m+mi}{20}\PY{p}{;}
         \PY{n}{plot}\PY{p}{(}\PY{n}{kap}\PY{p}{,}\PY{n}{kap}\PY{p}{(}\PY{n}{kapopt}\PY{p}{(}\PY{p}{:}\PY{p}{,}\PY{n}{age}\PY{p}{)}\PY{p}{)}\PY{p}{,}\PY{l+s}{\PYZsq{}}\PY{l+s}{k\PYZhy{}\PYZsq{}}\PY{p}{,}\PY{n}{kap}\PY{p}{,}\PY{n}{kap}\PY{p}{,}\PY{l+s}{\PYZsq{}}\PY{l+s}{\PYZhy{}\PYZhy{}\PYZsq{}}\PY{p}{)}\PY{p}{;}
         \PY{n}{xlabel}\PY{p}{(}\PY{l+s}{\PYZsq{}}\PY{l+s}{asset holdings, k\PYZsq{}}\PY{p}{,}\PY{l+s}{\PYZsq{}}\PY{l+s}{FontSize\PYZsq{}}\PY{p}{,}\PY{l+m+mi}{14}\PY{p}{,} \PY{l+s}{\PYZsq{}}\PY{l+s}{interpreter\PYZsq{}}\PY{p}{,} \PY{l+s}{\PYZsq{}}\PY{l+s}{latex\PYZsq{}}\PY{p}{)}\PY{p}{;}
         \PY{n}{ylabel}\PY{p}{(}\PY{l+s}{\PYZsq{}}\PY{l+s}{saving, \PYZdl{}k\PYZbs{}prime\PYZdl{}\PYZsq{}}\PY{p}{,}\PY{l+s}{\PYZsq{}}\PY{l+s}{FontSize\PYZsq{}}\PY{p}{,}\PY{l+m+mi}{14}\PY{p}{,} \PY{l+s}{\PYZsq{}}\PY{l+s}{interpreter\PYZsq{}}\PY{p}{,} \PY{l+s}{\PYZsq{}}\PY{l+s}{latex\PYZsq{}}\PY{p}{)}\PY{p}{;}
         \PY{n}{legend}\PY{p}{(}\PY{l+s}{\PYZsq{}}\PY{l+s}{saving\PYZsq{}}\PY{p}{,}\PY{l+s}{\PYZsq{}}\PY{l+s}{45 degree line\PYZsq{}}\PY{p}{,}\PY{l+s}{\PYZsq{}}\PY{l+s}{Location\PYZsq{}}\PY{p}{,}\PY{l+s}{\PYZsq{}}\PY{l+s}{East\PYZsq{}}\PY{p}{,} \PY{l+s}{\PYZsq{}}\PY{l+s}{interpreter\PYZsq{}}\PY{p}{,} \PY{l+s}{\PYZsq{}}\PY{l+s}{latex\PYZsq{}}\PY{p}{)}\PY{p}{;}
         \PY{n}{title}\PY{p}{(}\PY{p}{[}\PY{l+s}{\PYZsq{}}\PY{l+s}{saving \PYZdl{}k\PYZbs{}prime\PYZdl{} of a working agent at age \PYZsq{}}\PY{p}{,} \PY{n}{num2str}\PY{p}{(}\PY{n}{age}\PY{p}{)}\PY{p}{]}\PY{p}{,}\PY{l+s}{\PYZsq{}}\PY{l+s}{FontSize\PYZsq{}}\PY{p}{,}\PY{l+m+mi}{14}\PY{p}{,}
         \PY{l+s}{\PYZsq{}}\PY{l+s}{interpreter\PYZsq{}}\PY{p}{,} \PY{l+s}{\PYZsq{}}\PY{l+s}{latex\PYZsq{}}\PY{p}{)}\PY{p}{;}
         \PY{n}{set}\PY{p}{(}\PY{n}{gca}\PY{p}{,}\PY{l+s}{\PYZsq{}}\PY{l+s}{TickLabelInterpreter\PYZsq{}}\PY{p}{,}\PY{l+s}{\PYZsq{}}\PY{l+s}{latex\PYZsq{}}\PY{p}{)}
\end{Verbatim}

    \begin{Verbatim}[commandchars=\\\{\}]


    \end{Verbatim}

    \begin{center}
    \adjustimage{max size={0.9\linewidth}{0.9\paperheight}}{Problem_Set_5_files/Problem_Set_5_31_1.png}
    \end{center}
    { \hspace*{\fill} \\}
    \begin{Verbatim}[commandchars=\\\{\}]
{\color{incolor}In [{\color{incolor}80}]:} \PY{c}{\PYZpc{} Net Savings of a working agent}
         \PY{n}{age}\PY{p}{=}\PY{l+m+mi}{20}\PY{p}{;}
         \PY{n}{plot}\PY{p}{(}\PY{n}{kap}\PY{p}{,}\PY{n+nb}{abs}\PY{p}{(}\PY{n}{kap}\PY{p}{(}\PY{n}{kapopt}\PY{p}{(}\PY{p}{:}\PY{p}{,}\PY{n}{age}\PY{p}{)}\PY{p}{)} \PY{o}{\PYZhy{}} \PY{n}{kap}\PY{p}{)}\PY{p}{,}\PY{l+s}{\PYZsq{}}\PY{l+s}{k\PYZhy{}\PYZsq{}}\PY{p}{)}\PY{p}{;}
         \PY{n}{xlabel}\PY{p}{(}\PY{l+s}{\PYZsq{}}\PY{l+s}{asset holdings, k\PYZsq{}}\PY{p}{,}\PY{l+s}{\PYZsq{}}\PY{l+s}{FontSize\PYZsq{}}\PY{p}{,}\PY{l+m+mi}{14}\PY{p}{,} \PY{l+s}{\PYZsq{}}\PY{l+s}{interpreter\PYZsq{}}\PY{p}{,} \PY{l+s}{\PYZsq{}}\PY{l+s}{latex\PYZsq{}}\PY{p}{)}\PY{p}{;}
         \PY{n}{ylabel}\PY{p}{(}\PY{l+s}{\PYZsq{}}\PY{l+s}{saving, \PYZdl{}k\PYZbs{}prime \PYZhy{} k\PYZdl{}\PYZsq{}}\PY{p}{,}\PY{l+s}{\PYZsq{}}\PY{l+s}{FontSize\PYZsq{}}\PY{p}{,}\PY{l+m+mi}{14}\PY{p}{,} \PY{l+s}{\PYZsq{}}\PY{l+s}{interpreter\PYZsq{}}\PY{p}{,} \PY{l+s}{\PYZsq{}}\PY{l+s}{latex\PYZsq{}}\PY{p}{)}\PY{p}{;}
         \PY{n}{legend}\PY{p}{(}\PY{l+s}{\PYZsq{}}\PY{l+s}{net saving\PYZsq{}}\PY{p}{,}\PY{l+s}{\PYZsq{}}\PY{l+s}{Location\PYZsq{}}\PY{p}{,}\PY{l+s}{\PYZsq{}}\PY{l+s}{East\PYZsq{}}\PY{p}{,} \PY{l+s}{\PYZsq{}}\PY{l+s}{interpreter\PYZsq{}}\PY{p}{,} \PY{l+s}{\PYZsq{}}\PY{l+s}{latex\PYZsq{}}\PY{p}{)}\PY{p}{;}
         \PY{n}{title}\PY{p}{(}\PY{p}{[}\PY{l+s}{\PYZsq{}}\PY{l+s}{net saving \PYZdl{}k\PYZbs{}prime\PYZhy{}k\PYZdl{} of a working agent at age \PYZsq{}}\PY{p}{,} \PY{n}{num2str}\PY{p}{(}\PY{n}{age}\PY{p}{)}\PY{p}{]}\PY{p}{,}\PY{l+s}{\PYZsq{}}\PY{l+s}{FontSize\PYZsq{}}\PY{p}{,}\PY{l+m+mi}{14}\PY{p}{,}
         \PY{l+s}{\PYZsq{}}\PY{l+s}{interpreter\PYZsq{}}\PY{p}{,} \PY{l+s}{\PYZsq{}}\PY{l+s}{latex\PYZsq{}}\PY{p}{)}\PY{p}{;}
         \PY{n}{set}\PY{p}{(}\PY{n}{gca}\PY{p}{,}\PY{l+s}{\PYZsq{}}\PY{l+s}{TickLabelInterpreter\PYZsq{}}\PY{p}{,}\PY{l+s}{\PYZsq{}}\PY{l+s}{latex\PYZsq{}}\PY{p}{)}
\end{Verbatim}

    \begin{Verbatim}[commandchars=\\\{\}]


    \end{Verbatim}

    \begin{center}
    \adjustimage{max size={0.9\linewidth}{0.9\paperheight}}{Problem_Set_5_files/Problem_Set_5_32_1.png}
    \end{center}
    { \hspace*{\fill} \\}
    
    \hypertarget{qquad}{%
\subsection{\texorpdfstring{\(\qquad\)}{\textbackslash qquad}}\label{qquad}}

Evaluate the macroeconomic consequences of eliminating social security.
You can use Table 1 to support your answers. Note: when initially
solving the model, lines 300 on wards will give you an error. This is
because the last part is used to compare the two models after you have
solved for it already. Either skip or ignore that part and use it
afterwards. 1. Solve for the stationary competitive market equilibrium
of the benchmark model with social security, i.e., solve the whole model
with initial guess of \((K, L)=(3.1392,0.3496) .\) The model should
converge after \(\approx 25\) iterations. Is this economy dynamically
efficient (compare the interest rate with the implicit return from
social security, which is equal to the population growth rate)?

    \begin{enumerate}
\def\labelenumi{\arabic{enumi}.}
\setcounter{enumi}{1}
\item
  Eliminate social security by setting \(\tau=0\) and solve for the new
  stationary equilibrium, i.e., solve the model again, but this time set
  \(\tau=0\) and use initial guess of \((K, L)=(3.9288,0.3663) .\) The
  model should converge after \(\approx 15\) iterations.

  \begin{itemize}
  \item
    The code should have saved the results using lines 291 . Check how
    aggregate capital and labor supply change as a result of the tax
    reform. Fill in the appropriate column of to following table:
  \item
    Plot and compare the profiles of wealth by age groups for the case
    with and without social security. Provide intuition for observed
    differences.
  \item
    Will a newborn generation prefer to start in a steady state with or
    without social security? (Hint: lines \(332 \& 347)\)
  \item
    Assuming that the reform instantly takes every cohort into the new
    steady state, will it be supported by a majority voting? (Hint:
    Compare aggregate welfare
    \(\left.W=\sum_{j=1}^{J} \psi_{j} V_{j}\left(k_{j}\right)\right)\)
  \end{itemize}
\end{enumerate}

\begin{center}
\begin{array}{|l|l|l|}
\hline & \multicolumn{2}{|c|} {\text { Benchmark model }} \\
\hline \hline & \text { with SS } & \text { without SS } \\
\hline \text { capital } K & 2.9778 & 3.9115 \\
\hline \text { labor } L &0.3511 & 0.3691\\
\hline \text { wage } w &1.3818 & 1.4971\\
\hline \text { interest } r &0.0316 & 0.0195 \\
\hline \text { pension benefit } b & 0.2187 & 0 \\
\hline \text { newborn welfare } V_{1}\left(k_{1}\right) & -54.6138 & -52.767 \\
\hline \text { aggregate welfare } W & -35.0678 & -36.0276 \\
\hline
\end{array}
\end{center}

\textbf{Note:} The value in the table are from the following
calculations:

    \hypertarget{answer}{%
\subsubsection{Answer}\label{answer}}
\begin{Verbatim}[commandchars=\\\{\}]
{\color{incolor}In [{\color{incolor}90}]:} \PY{n}{nq} \PY{p}{=} \PY{l+m+mi}{100}\PY{p}{;}
         \PY{p}{[}\PY{n}{K0}\PY{p}{,} \PY{n}{L0}\PY{p}{,} \PY{n}{w0}\PY{p}{,} \PY{n}{r0}\PY{p}{,} \PY{n}{b}\PY{p}{,} \PY{n}{kgen}\PY{p}{,} \PY{n}{ikgen}\PY{p}{,} \PY{n}{kap}\PY{p}{,} \PY{n}{kapopt}\PY{p}{,} \PY{n}{v}\PY{p}{]} \PY{p}{=} \PY{n}{dynamic\PYZus{}prog}\PY{p}{(} \PY{n}{K0}\PY{p}{,} \PY{n}{L0}\PY{p}{,} \PY{n}{alpha}\PY{p}{,} \PY{n}{delta}\PY{p}{,}
         \PY{n}{tau}\PY{p}{,} \PY{n}{J}\PY{p}{,} \PY{n}{JR}\PY{p}{,} \PY{n}{tW}\PY{p}{,} \PY{n}{mass}\PY{p}{,} \PY{n}{sigma}\PY{p}{,} \PY{n+nb}{beta}\PY{p}{,} \PY{n+nb}{gamma}\PY{p}{,} \PY{n}{e}\PY{p}{,} \PY{n}{nq}\PY{p}{)}\PY{p}{;}
         \PY{n}{kgen\PYZus{}ss}\PY{p}{=}\PY{n}{kgen}\PY{p}{;}
         \PY{n}{save}\PY{p}{(}\PY{l+s}{\PYZsq{}}\PY{l+s}{ss.mat\PYZsq{}}\PY{p}{)}\PY{p}{;}
\end{Verbatim}

    \begin{Verbatim}[commandchars=\\\{\}]
      K0         L0       w         r         b
    2.9778    0.3511    1.3818    0.0316    0.2187


    \end{Verbatim}
\begin{Verbatim}[commandchars=\\\{\}]
{\color{incolor}In [{\color{incolor}96}]:} \PY{c}{\PYZpc{} No Social Security}
         \PY{n}{K0\PYZus{}no\PYZus{}ss} \PY{p}{=}\PY{l+m+mf}{3.9288}\PY{p}{;}
         \PY{n}{L0\PYZus{}no\PYZus{}ss} \PY{p}{=}\PY{l+m+mf}{0.3663}\PY{p}{;}
         
         \PY{n}{tau\PYZus{}no\PYZus{}ss}\PY{p}{=}\PY{l+m+mi}{0}\PY{p}{;}
         
         \PY{n}{nq} \PY{p}{=} \PY{l+m+mi}{100}\PY{p}{;}
         \PY{p}{[}\PY{n}{K0}\PY{p}{,} \PY{n}{L0}\PY{p}{,} \PY{n}{w0}\PY{p}{,} \PY{n}{r0}\PY{p}{,} \PY{n}{b}\PY{p}{,} \PY{n}{kgen}\PY{p}{,} \PY{n}{ikgen}\PY{p}{,} \PY{n}{kap}\PY{p}{,} \PY{n}{kapopt}\PY{p}{,} \PY{n}{v}\PY{p}{]} \PY{p}{=} \PY{n}{dynamic\PYZus{}prog}\PY{p}{(} \PY{n}{K0}\PY{p}{,} \PY{n}{L0}\PY{p}{,} \PY{n}{alpha}\PY{p}{,} \PY{n}{delta}\PY{p}{,}
         \PY{n}{tau\PYZus{}no\PYZus{}ss}\PY{p}{,} \PY{n}{J}\PY{p}{,} \PY{n}{JR}\PY{p}{,} \PY{n}{tW}\PY{p}{,} \PY{n}{mass}\PY{p}{,} \PY{n}{sigma}\PY{p}{,} \PY{n+nb}{beta}\PY{p}{,} \PY{n+nb}{gamma}\PY{p}{,} \PY{n}{e}\PY{p}{,} \PY{n}{nq}\PY{p}{)}\PY{p}{;}
         \PY{n}{kgen\PYZus{}noss}\PY{p}{=}\PY{n}{kgen}\PY{p}{;}
         \PY{n}{save}\PY{p}{(}\PY{l+s}{\PYZsq{}}\PY{l+s}{no\PYZus{}ss.mat\PYZsq{}}\PY{p}{)}\PY{p}{;}
\end{Verbatim}

    \begin{Verbatim}[commandchars=\\\{\}]
      K0         L0       w         r         b
    3.9115    0.3691    1.4971    0.0195         0


    \end{Verbatim}
\begin{Verbatim}[commandchars=\\\{\}]
{\color{incolor}In [{\color{incolor}99}]:} \PY{c}{\PYZpc{}\PYZpc{} Welfare comparison}
         \PY{n}{load}\PY{p}{(}\PY{l+s}{\PYZsq{}}\PY{l+s}{ss.mat\PYZsq{}}\PY{p}{,}\PY{l+s}{\PYZsq{}}\PY{l+s}{v\PYZsq{}}\PY{p}{,}\PY{l+s}{\PYZsq{}}\PY{l+s}{mass\PYZsq{}}\PY{p}{,}\PY{l+s}{\PYZsq{}}\PY{l+s}{ikgen\PYZsq{}}\PY{p}{,}\PY{l+s}{\PYZsq{}}\PY{l+s}{J\PYZsq{}}\PY{p}{)}\PY{p}{;}
         
         \PY{c}{\PYZpc{} Newborn generation}
         \PY{n}{V1\PYZus{}ss} \PY{p}{=} \PY{n}{v}\PY{p}{(}\PY{l+m+mi}{1}\PY{p}{,}\PY{l+m+mi}{1}\PY{p}{)}
         
         \PY{c}{\PYZpc{} Aggregate welfare}
         \PY{n}{V\PYZus{}ss} \PY{p}{=} \PY{n+nb}{zeros}\PY{p}{(}\PY{n}{J}\PY{p}{,}\PY{l+m+mi}{1}\PY{p}{)}\PY{p}{;}
         \PY{n}{V\PYZus{}ss}\PY{p}{(}\PY{l+m+mi}{1}\PY{p}{)} \PY{p}{=} \PY{n}{v}\PY{p}{(}\PY{l+m+mi}{1}\PY{p}{,}\PY{l+m+mi}{1}\PY{p}{)}\PY{p}{;}
         \PY{n}{for} \PY{l+s}{j} \PY{l+s}{=} \PY{l+s}{2:J}
             \PY{n}{ik0} \PY{p}{=} \PY{n}{ikgen}\PY{p}{(}\PY{n+nb}{j}\PY{o}{\PYZhy{}}\PY{l+m+mi}{1}\PY{p}{)}\PY{p}{;}
             \PY{n}{V\PYZus{}ss}\PY{p}{(}\PY{n+nb}{j}\PY{p}{)} \PY{p}{=} \PY{n}{v}\PY{p}{(}\PY{n}{ik0}\PY{p}{,}\PY{n+nb}{j}\PY{p}{)}\PY{p}{;}
         \PY{n}{end}
         \PY{l+s}{W\PYZus{}ss} \PY{l+s}{=} \PY{l+s}{V\PYZus{}ss}\PY{o}{\PYZsq{}} \PY{o}{*} \PY{n}{mass}
\end{Verbatim}

    \begin{Verbatim}[commandchars=\\\{\}]
V1\_ss =
  -54.6138
W\_ss =
  -35.0678


    \end{Verbatim}
\begin{Verbatim}[commandchars=\\\{\}]
{\color{incolor}In [{\color{incolor}101}]:} \PY{n}{load}\PY{p}{(}\PY{l+s}{\PYZsq{}}\PY{l+s}{no\PYZus{}ss.mat\PYZsq{}}\PY{p}{,}\PY{l+s}{\PYZsq{}}\PY{l+s}{v\PYZsq{}}\PY{p}{,}\PY{l+s}{\PYZsq{}}\PY{l+s}{mass\PYZsq{}}\PY{p}{,}\PY{l+s}{\PYZsq{}}\PY{l+s}{ikgen\PYZsq{}}\PY{p}{,}\PY{l+s}{\PYZsq{}}\PY{l+s}{J\PYZsq{}}\PY{p}{)}\PY{p}{;}
          
          \PY{c}{\PYZpc{} Newborn generation}
          \PY{n}{V1\PYZus{}no\PYZus{}ss} \PY{p}{=} \PY{n}{v}\PY{p}{(}\PY{l+m+mi}{1}\PY{p}{,}\PY{l+m+mi}{1}\PY{p}{)}
          
          \PY{c}{\PYZpc{} Aggregate welfare}
          \PY{n}{V\PYZus{}no\PYZus{}ss} \PY{p}{=} \PY{n+nb}{zeros}\PY{p}{(}\PY{n}{J}\PY{p}{,}\PY{l+m+mi}{1}\PY{p}{)}\PY{p}{;}
          \PY{n}{V\PYZus{}no\PYZus{}ss}\PY{p}{(}\PY{l+m+mi}{1}\PY{p}{)} \PY{p}{=} \PY{n}{v}\PY{p}{(}\PY{l+m+mi}{1}\PY{p}{,}\PY{l+m+mi}{1}\PY{p}{)}\PY{p}{;}
          \PY{n}{for} \PY{l+s}{j} \PY{l+s}{=} \PY{l+s}{2:J}
              \PY{n}{ik0} \PY{p}{=} \PY{n}{ikgen}\PY{p}{(}\PY{n+nb}{j}\PY{o}{\PYZhy{}}\PY{l+m+mi}{1}\PY{p}{)}\PY{p}{;}
              \PY{n}{V\PYZus{}no\PYZus{}ss}\PY{p}{(}\PY{n+nb}{j}\PY{p}{)} \PY{p}{=} \PY{n}{v}\PY{p}{(}\PY{n}{ik0}\PY{p}{,}\PY{n+nb}{j}\PY{p}{)}\PY{p}{;}
          \PY{n}{end}
          \PY{l+s}{W\PYZus{}no\PYZus{}ss} \PY{l+s}{=} \PY{l+s}{V\PYZus{}no\PYZus{}ss}\PY{o}{\PYZsq{}} \PY{o}{*} \PY{n}{mass}
\end{Verbatim}

    \begin{Verbatim}[commandchars=\\\{\}]
V1\_no\_ss =
  -52.7671
W\_no\_ss =
  -36.0276


    \end{Verbatim}
\begin{Verbatim}[commandchars=\\\{\}]
{\color{incolor}In [{\color{incolor}97}]:} \PY{n}{load}\PY{p}{(}\PY{l+s}{\PYZsq{}}\PY{l+s}{ss.mat\PYZsq{}}\PY{p}{,}\PY{l+s}{\PYZsq{}}\PY{l+s}{kgen\PYZsq{}}\PY{p}{,}\PY{l+s}{\PYZsq{}}\PY{l+s}{J\PYZsq{}}\PY{p}{)}\PY{p}{;}
         \PY{n}{load}\PY{p}{(}\PY{l+s}{\PYZsq{}}\PY{l+s}{no\PYZus{}ss.mat\PYZsq{}}\PY{p}{,}\PY{l+s}{\PYZsq{}}\PY{l+s}{kgen\PYZsq{}}\PY{p}{,}\PY{l+s}{\PYZsq{}}\PY{l+s}{J\PYZsq{}}\PY{p}{)}\PY{p}{;}
\end{Verbatim}

    \begin{Verbatim}[commandchars=\\\{\}]


    \end{Verbatim}
\begin{Verbatim}[commandchars=\\\{\}]
{\color{incolor}In [{\color{incolor}98}]:} \PY{n}{plot1}\PY{p}{=}\PY{n}{plot}\PY{p}{(}\PY{l+m+mi}{20}\PY{o}{+}\PY{l+m+mi}{1}\PY{p}{:}\PY{l+m+mi}{20}\PY{o}{+}\PY{n}{J}\PY{p}{,}\PY{n}{kgen\PYZus{}ss}\PY{p}{,}\PY{l+m+mi}{20}\PY{o}{+}\PY{l+m+mi}{1}\PY{p}{:}\PY{l+m+mi}{20}\PY{o}{+}\PY{n}{J}\PY{p}{,}\PY{n}{kgen\PYZus{}noss}\PY{p}{)}\PY{p}{;}
         \PY{n}{set}\PY{p}{(}\PY{n}{plot1}\PY{p}{,}\PY{l+s}{\PYZsq{}}\PY{l+s}{LineWidth\PYZsq{}}\PY{p}{,}\PY{l+m+mf}{1.5}\PY{p}{)}
         \PY{n}{xlabel}\PY{p}{(}\PY{l+s}{\PYZsq{}}\PY{l+s}{(real\PYZhy{}life) age\PYZsq{}}\PY{p}{,}\PY{l+s}{\PYZsq{}}\PY{l+s}{FontSize\PYZsq{}}\PY{p}{,}\PY{l+m+mi}{14}\PY{p}{,} \PY{l+s}{\PYZsq{}}\PY{l+s}{interpreter\PYZsq{}}\PY{p}{,} \PY{l+s}{\PYZsq{}}\PY{l+s}{latex\PYZsq{}}\PY{p}{)}\PY{p}{;}
         \PY{n}{ylabel}\PY{p}{(}\PY{l+s}{\PYZsq{}}\PY{l+s}{wealth\PYZsq{}}\PY{p}{,}\PY{l+s}{\PYZsq{}}\PY{l+s}{FontSize\PYZsq{}}\PY{p}{,}\PY{l+m+mi}{14}\PY{p}{,} \PY{l+s}{\PYZsq{}}\PY{l+s}{interpreter\PYZsq{}}\PY{p}{,} \PY{l+s}{\PYZsq{}}\PY{l+s}{latex\PYZsq{}}\PY{p}{)}\PY{p}{;}
         \PY{n}{AX}\PY{p}{=}\PY{n}{legend}\PY{p}{(}\PY{l+s}{\PYZsq{}}\PY{l+s}{with Social Security\PYZsq{}}\PY{p}{,}\PY{l+s}{\PYZsq{}}\PY{l+s}{without Social Security\PYZsq{}}\PY{p}{,}\PY{l+s}{\PYZsq{}}\PY{l+s}{Location\PYZsq{}}\PY{p}{,}\PY{l+s}{\PYZsq{}}\PY{l+s}{NorthWest\PYZsq{}}\PY{p}{,}
         \PY{l+s}{\PYZsq{}}\PY{l+s}{interpreter\PYZsq{}}\PY{p}{,} \PY{l+s}{\PYZsq{}}\PY{l+s}{latex\PYZsq{}}\PY{p}{)}\PY{p}{;}
         \PY{n}{LEG}\PY{p}{=}\PY{n}{findobj}\PY{p}{(}\PY{n}{AX}\PY{p}{,}\PY{l+s}{\PYZsq{}}\PY{l+s}{type\PYZsq{}}\PY{p}{,}\PY{l+s}{\PYZsq{}}\PY{l+s}{text\PYZsq{}}\PY{p}{,} \PY{l+s}{\PYZsq{}}\PY{l+s}{interpreter\PYZsq{}}\PY{p}{,} \PY{l+s}{\PYZsq{}}\PY{l+s}{latex\PYZsq{}}\PY{p}{)}\PY{p}{;}
         \PY{n}{set}\PY{p}{(}\PY{n}{LEG}\PY{p}{,}\PY{l+s}{\PYZsq{}}\PY{l+s}{FontSize\PYZsq{}}\PY{p}{,}\PY{l+m+mi}{14}\PY{p}{,} \PY{l+s}{\PYZsq{}}\PY{l+s}{interpreter\PYZsq{}}\PY{p}{,} \PY{l+s}{\PYZsq{}}\PY{l+s}{latex\PYZsq{}}\PY{p}{)}\PY{p}{;}
         \PY{n}{set}\PY{p}{(}\PY{n}{gca}\PY{p}{,}\PY{l+s}{\PYZsq{}}\PY{l+s}{TickLabelInterpreter\PYZsq{}}\PY{p}{,}\PY{l+s}{\PYZsq{}}\PY{l+s}{latex\PYZsq{}}\PY{p}{)}
\end{Verbatim}

    \begin{Verbatim}[commandchars=\\\{\}]


    \end{Verbatim}

    \begin{center}
    \adjustimage{max size={0.9\linewidth}{0.9\paperheight}}{Problem_Set_5_files/Problem_Set_5_41_1.png}
    \end{center}
    { \hspace*{\fill} \\}
    
    \begin{itemize}
\tightlist
\item
  Capital increases because in the absence of pension benefits people
  have tom save more to retirement.
\item
  Labor supply increases because in the absence of taxes, wages (after
  taxes) are higher.
\item
  Wages go up because when capital increases the marginal productivity
  of labor increases and dominates the rise in labor supply.
\item
  Interest \(r\) goes down because the increase in the marginal
  productivity of capital caused an increase in labor is not enough to
  compensate for the increase in capital supply (and the production
  function is concave).
\end{itemize}

\begin{center}
\begin{array}{|l|l|l|}
\hline & \multicolumn{2}{|c|} {\text { Benchmark model }} \\
\hline \hline & \text { with SS } & \text { without SS } \\
\hline \text { capital } K & 2.9778 & 3.9115 \\
\hline \text { labor } L &0.3511 & 0.3691\\
\hline \text { wage } w &1.3818 & 1.4971\\
\hline \text { interest } r &0.0316 & 0.0195 \\
\hline \text { pension benefit } b & 0.2187 & 0 \\
\hline \text { newborn welfare } V_{1}\left(k_{1}\right) & -54.6138 & -52.767 \\
\hline \text { aggregate welfare } W & -35.0678 & -36.0276 \\
\hline
\end{array}
\end{center}



    \begin{itemize}
\tightlist
\item
  The new born generation will be better off so they will prefer the
  economy without social security.
\item
  The population as a whole is worse off in a world without social
  security therefore in a majority voting removing social security will
  not be approved.
\end{itemize}


    % Add a bibliography block to the postdoc
    
    
\bibliography{references}

    
    \end{document}
