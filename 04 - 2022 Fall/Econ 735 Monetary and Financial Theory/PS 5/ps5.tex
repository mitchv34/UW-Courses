\documentclass[12pt]{amsart}
% ------------------------------------------------------------------------------
% Packages
% ------------------------------------------------------------------------------
\usepackage[utf8]{inputenc} % input encoding
\usepackage{framed} % for \framebox
\usepackage[english]{babel} % language support 
\usepackage[colorlinks=true, linkcolor=blue, urlcolor=blue, citecolor=blue, anchorcolor=blue]{hyperref} % hyperref package
\usepackage{amsthm, amsmath, amsfonts, mathtools, amssymb, bm} % Math packages
\usepackage{xcolor} % Color package
\usepackage[shortlabels]{enumitem} % Enumeration package
\usepackage{booktabs}
\usepackage{float}
\usepackage{cancel}

% ------------------------------------------------------------------------------
% Document Style
% ------------------------------------------------------------------------------

% Header ----------------------------------------------------------------------
\addtolength{\hoffset}{-2.25cm}
\addtolength{\textwidth}{4.5cm}
\addtolength{\voffset}{-2.5cm}
\addtolength{\textheight}{5cm}
\setlength{\parskip}{0pt}
\setlength{\parindent}{15pt}

\pagestyle{myheadings}

\setlength{\parindent}{0in}

\pagestyle{empty}
\makeatletter
\def\fps@figure{H}
\def\fps@table{H}
% ------------------------------------------------------------------------------

% ------------------------------------------------------------------------------
% New commands
% ------------------------------------------------------------------------------
\newcommand{\qiq}{\qquad \implies \qquad}
\newcommand{\qiffq}{\qquad \iff \qquad}
\newcommand{\qaq}{\qquad \textbf{and} \qquad}
\newcommand{\qoq}{\qquad \textbf{or} \qquad}

% ------------------------------------------------------------------------------
% New environments
% ------------------------------------------------------------------------------
\newtheorem*{theorem}{\color{red!60!black}Theorem}
\newtheorem{corollary}{\color{blue}Corollary}
\counterwithin*{corollary}{subsection}
\newtheorem{lemma}{\color{blue}Lemma}
\counterwithin*{lemma}{subsection}
\newtheorem{proposition}{\color{blue}Proposition}
\counterwithin*{proposition}{subsection} 
\theoremstyle{definition}
\newtheorem*{definition}{\color{green!60!black}Definition}
\newtheorem{example}{\color{orange!80!black}Example}
\newtheorem*{Obs}{\color{purple!80!white}Observation}
\newtheorem*{As}{\color{red!80!white}Assumptions}
\newtheorem*{answer}{\color{red!60!white}Answer}
\newtheorem*{Cor}{\color{blue!60!white}Corollary}
\newtheorem{exercise}{\color{blue!60!white}Exercise}
\newtheorem{subexercise}{ \color{blue!40!white}  }[exercise]

\begin{document}

\thispagestyle{empty}


{\scshape ECON-735} \hfill {\scshape \Large Problem Set \#4} \hfill {\scshape Spring 2022}\\
{\scshape Author(s): \hfill Mitchell Valdes-Bobes\\
{\scshape Date: \hfill \texttt{\today}
\medskip

\hrule
\bigskip

\bigskip
% In the following questions, we use the same notation as in the lecture.
% Problem 1
\begin{exercise}
We continue to consider the model by Lagos and Wright (2005), but instead of fiat money, study credit. The CM value function satisfies
$$
\begin{aligned}
W(d)=& \max _{x, \ell}\{U(x)-\ell+\beta V\} \\
& \text { subject to } x=\ell-d
\end{aligned}
$$
The DM value function satisfies
$$
\begin{aligned}
V=&(1-2 \alpha \sigma) W(0)+\alpha \sigma[u(q)+W(d)] \\
&+\alpha \sigma[-c(q)+W(-d)]
\end{aligned}
$$
Let $D \geq 0$ be the exogenous debt limit that each agent faces in the DM. For simplicity, to determine the terms of trade, we employ the take-it-or-leave-it offer by buyers.
Assume that there exists an interior solution to the CM problem. Derive the first-order and envelope conditions.
\end{exercise}

\begin{answer}
From the constraint, we have $\ell = x + d$ therefore:
\begin{equation*}
    W(d) = \max _{x}\{U(x)-x-d+\beta V\} = \max _{x}\{U(x)- x+\beta V\}  - d
\end{equation*}
Taking the derivatives, we get:
\begin{empheq}[box=\fbox]{align*}
    \textbf{[FOC:]} &\quad u'(x) - 1 = 0\\
    \textbf{[Env:]} &\quad w'(d) + 1 = 0
\end{empheq}
\end{answer}

% Problem 2
\begin{exercise}
Derive the terms of trade in the DM.
\end{exercise}

\begin{answer}

In the decentralized market, we have the following:

\begin{align*}
    \max_{q}&\: \left\{u(q) - w(d)\right\} \\
    \text{subject to:}&\: -c(q) + w(-d) \geq 0\\
    &d\leq D
\end{align*}
Note that $$w(d) - w(0) = - d = -(w(-d) -w(0)) \qiq w(d) = -w(-d) + 2 w(0)$$
there fore the optimization problem become:
\begin{align*}
    \max_{q}&\: \left\{u(q) + w(-d) - 2w(0)\right\} \\
    \text{subject to:}&\: -c(q) + w(-d) \geq 0\\
    &d\leq D
\end{align*}
Since the term $2w(0)$ is constant, we can ignore it. First we solve a relaxed version of the problem without the constraint $d\leq D$:

$$c(q) = w(-d)$$

and the solution is $q^*$ such that:

$$c'(q^*) = w'(q^*)$$

This would be the optimal solution if feasible, if not then the usual assumptions on $w(\cdot)$ and $c(\cdot)$ guarantee that $d=D$. Therefore, the solution is as follows:
\begin{empheq}[box=\fbox]{align*}
    q &= \min\left\{ c^{-1}(w(D)), q^* \right\} \\
    d &= \min\left\{ D,  w^{-1}(c( q^*)) \right\}
\end{empheq}
\end{answer}



\end{answer}

% Problem 3
\begin{exercise}

Now, we consider a situation where agents are punished with probability $\mu$ if they renege on their debt. Once they are punished, they cannot trade in the DM anymore. Hence we impose the following incentive constraint:
$$
W(d) \geq \mu W^{D}+(1-\mu) W(0)
$$
where
$$
W^{D}=\frac{1}{1-\beta}\left[U\left(x^{*}\right)-x^{*}\right]
$$
By this constraint, endogenize the debt limit $D$.
\end{exercise}

\begin{answer}
We can substitute $W(d)=-W(-d) + 2W(0)$ and cancel terms to get:

$$V  = W(0) + \alpha \sigma (u(q) - c(q))$$

Note that $$W(0) =  \max _{x}\{U(x)- x+\beta V\}$$ substituting in the above expression we get:

\begin{align*}
V  &= \max _{x}\{U(x)- x\}  +\beta V + \alpha \sigma (u(q) - c(q)) \\ &\qiq  (1-\beta)V  = \max _{x}\{U(x)- x\}  + \alpha \sigma (u(q) - c(q)) \\
&\qiq \frac{(1-\beta)}{\beta}(W(0) -  \max_{x}\{U(x)- x\})  = \max_{x}\{U(x)- x\}  + \alpha \sigma (u(q) - c(q)) \\
&\qiq \frac{(1-\beta)}{\beta} W(0) = \frac{1}{\beta} \max_{x}\{U(x)- x\}  + \alpha \sigma (u(q) - c(q)) \\
&\qiq W(0) = \frac{1}{1 - \beta} \max_{x}\{U(x)- x\}  + \alpha \sigma \frac{\beta}{1 - \beta} (u(q) - c(q)) 
\end{align*}

Note that from the definition of $W^D$:
\begin{align*}
    W^D &= W(0) - \alpha \sigma \frac{1}{1 - \beta} (u(q) - c(q)) \\
\end{align*}

Define $$\theta =  \alpha \sigma \frac{1}{1 - \beta} (u(q) - c(q))$$
then 
$$ \mu W^{D}+(1-\mu) W(0) = W(0) - \frac{\mu \beta \theta}{1 - \beta}  $$

Note also that $$W(d) = \max _{x}\{U(x)- x\}  - d +\beta V = \max _{x}\{U(x)- x\}  - d +\beta W(0) + \beta \theta $$

Then the new constraint becomes

$$-d \geq \frac{\beta \theta}{1-\beta}(1-\beta -\mu)$$
or 

$$\boxed{d \leq \underbrace{-\frac{\beta  \alpha \sigma \frac{1}{1 - \beta} (u(q) - c(q))}{1-\beta}(1-\beta -\mu)}_{D}}$$

\end{answer}

\end{document}