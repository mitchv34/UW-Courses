\documentclass[12pt]{amsart}
% ------------------------------------------------------------------------------
% Packages
% ------------------------------------------------------------------------------
\usepackage[utf8]{inputenc} % input encoding
\usepackage{framed} % for \framebox
\usepackage[english]{babel} % language support 
\usepackage[colorlinks=true, linkcolor=blue, urlcolor=blue, citecolor=blue, anchorcolor=blue]{hyperref} % hyperref package
\usepackage{amsthm, amsmath, amsfonts, mathtools, amssymb, bm} % Math packages
\usepackage{xcolor} % Color package
\usepackage[shortlabels]{enumitem} % Enumeration package
\usepackage{booktabs}
\usepackage{float}
\usepackage{cancel}

% ------------------------------------------------------------------------------
% Document Style
% ------------------------------------------------------------------------------

% Header ----------------------------------------------------------------------
\addtolength{\hoffset}{-2.25cm}
\addtolength{\textwidth}{4.5cm}
\addtolength{\voffset}{-2.5cm}
\addtolength{\textheight}{5cm}
\setlength{\parskip}{0pt}
\setlength{\parindent}{15pt}

\pagestyle{myheadings}

\setlength{\parindent}{0in}

\pagestyle{empty}
\makeatletter
\def\fps@figure{H}
\def\fps@table{H}
% ------------------------------------------------------------------------------

% ------------------------------------------------------------------------------
% New commands
% ------------------------------------------------------------------------------
\newcommand{\qiq}{\qquad \implies \qquad}
\newcommand{\qiffq}{\qquad \iff \qquad}
\newcommand{\qaq}{\qquad \textbf{and} \qquad}
\newcommand{\qoq}{\qquad \textbf{or} \qquad}

% ------------------------------------------------------------------------------
% New environments
% ------------------------------------------------------------------------------
\newtheorem*{theorem}{\color{red!60!black}Theorem}
\newtheorem{corollary}{\color{blue}Corollary}
\counterwithin*{corollary}{subsection}
\newtheorem{lemma}{\color{blue}Lemma}
\counterwithin*{lemma}{subsection}
\newtheorem{proposition}{\color{blue}Proposition}
\counterwithin*{proposition}{subsection} 
\theoremstyle{definition}
\newtheorem*{definition}{\color{green!60!black}Definition}
\newtheorem{example}{\color{orange!80!black}Example}
\newtheorem*{Obs}{\color{purple!80!white}Observation}
\newtheorem*{As}{\color{red!80!white}Assumptions}
\newtheorem*{answer}{\color{red!60!white}Answer}
\newtheorem*{Cor}{\color{blue!60!white}Corollary}
\newtheorem{exercise}{\color{blue!60!white}Exercise}
\newtheorem{subexercise}{ \color{blue!40!white}  }[exercise]

\begin{document}

\thispagestyle{empty}


{\scshape ECON-735} \hfill {\scshape \Large Problem Set \#1} \hfill {\scshape Spring 2022}\\
{\scshape Author(s): \hfill Mitchell Valdes-Bobes\\
{\scshape Date: \hfill \texttt{\today}
\medskip

\hrule
\bigskip

\bigskip
In the following questions, we use the same notation as in the lecture.
% Problem 1
\begin{exercise}
    Under autarky, there is no trade:
    $$
        V^{A}=0 .
    $$
    With barter, trade takes place if we have double coincidences:
    $$
        r V^{B}=\alpha \delta(u-c)
    $$
    With credit, trade takes place if we have double or single coincidences:
    $$
        r V^{C}=\alpha \delta(u-c)+\alpha \sigma u-\alpha \sigma c .
    $$
    For credit to be viable, we must have the incentive constraint:
    $$
        -c+V^{C} \geq \mu V^{D}+(1-\mu) V^{C}
    $$
    where $\mu$ is the probability that deviators are caught and punished and $V^{D}$ is the deviation payoff.
\end{exercise}

% Part a)
\begin{subexercise}
    Assume $V^{D}=V^{A}$. Compute the upper bound of $r$ such that credit is viable.
\end{subexercise}

% Answer 1.a)
\begin{answer}
 Substituting  $V^{D}=V^{A}=0$ we get that credit is viable if:
 
 $$ -c+V^{C} \geq (1-\mu) V^{C}  \qiq c  \leq \mu V^C \qiq  r c  \leq r \mu V^C$$ 
 
 multiplying the expression for $rV^C$ by $\mu$ we get:
 
 $$ r \mu V^{C}= \mu (\alpha \delta(u-c)+\alpha \sigma u-\alpha \sigma c ) = \mu\alpha(u-c)(\delta -\sigma ) $$
 
 Therefore the upper bound for $r$ such that credit is viable $(\tilde{r}_C)$ is  
 
 $$r c \leq   \mu\alpha(u-c)(\delta -\sigma ) \qiq \boxed{\tilde{r}_C \equiv \frac{ \mu\alpha(u-c)(\delta -\sigma )}{c}}$$
 
\end{answer}

% Part b)
\begin{subexercise}
    Assume $V^{D}=V^{B}$. Redo part (a) and discuss the difference between parts (a) and (b).
\end{subexercise}

% Answer 1.b)
\begin{answer}
    Start by noting that $$V^C = V^B + \frac{\sigma}{\delta}V^B$$
   therefore $$-c + \cancel{V^C} \geq \mu V^B + (\cancel{1}-\mu) V^C \qiq c\leq \mu\left(V^C - V^B\right)  \mu\left(\cancel{V^B} + \frac{\sigma}{\delta}V^B - \cancel{V^B}\right)$$
   
   Therefore 
   $$ c\leq \frac{\sigma}{\delta}V^B = \frac{\sigma}{\delta} \frac{\alpha \delta(u-c)}{r} \qiq \boxed{\hat{r}_{C} \equiv \frac{\mu \alpha \sigma(u-c)}{c} }$$
\end{answer}

% Problem 2
\begin{exercise}
    Now, we introduce fiat money. The Bellman equations are
    $$
        r V_{0}=\alpha \delta(u-c)+\alpha \sigma M \max _{\tau} \tau\left(V_{1}-V_{0}-c\right)
    $$
    and
    $$
        r V_{1}=\alpha \delta(u-c)+\alpha \sigma(1-M) \bar{\tau}\left(u+V_{0}-V_{1}\right)
    $$
\end{exercise}

% Part a)
\begin{subexercise}
    Characterize the Nash equilibria.
\end{subexercise}

% Answer 2.a)
\begin{answer}
    We will consider three cases:
\begin{enumerate}
    \item If $c < V_1 - V_0$ then $\tau = 1$ and everybody accepts money.
    \item If $c > V_1 - V_0$ then $\tau = 0$ and no one accepts money.
    \item If $c = V_1 - V_0$ then:
    \begin{align*}
        r V_0 &= \alpha \delta (u-c) \\
        r V_1 &= \alpha \delta(u-c)+\alpha \sigma(1-M) \bar{\tau}\left(u-c\right)
    \end{align*}
    subtracting the two expressions:
    $$r(V_1 - V_0) = r c = a(u-c)(\cancel{\delta} + \sigma(1-M)\bar{\tau} - \cancel{\delta})$$
    solving for $\bar{\tau}$ we get that the only equilibrium in mixed strategies is:
    $$\bar{\tau} \equiv \frac{r c}{\sigma (1-M)\alpha (u-c)}$$
\end{enumerate}
\end{answer}

% Part b)
\begin{subexercise}
    Compute the upper bound of $r$ such that a monetary equilibrium exists and discuss the essentiality of money.
\end{subexercise}

% Answer 2.b)
\begin{answer}
    For a monetary equilibrium to exist we need:
    $$\bar{\tau} =  \frac{r c}{\sigma (1-M)\alpha (u-c)} \leq 1 \qiq \boxed{\hat{r}_M \equiv \frac{\sigma (1-M) \alpha (u-c)}{c}}$$
    
    Then if $r<\hat{r}_M$ a monetary equilibrium exits, comparing $\hat{r}_M$ to $\hat{r}_C$ we obtain that 
    $$\hat{r}_M > \hat{r}_M \qiffq \frac{\sigma (1-M) \alpha (u-c)}{c} > \frac{\mu \alpha \sigma(u-c)}{c} \qiffq (1-M) > \mu$$
    
    Therefore money is essential when $\mu < (1-M)$.
\end{answer}







\end{document}